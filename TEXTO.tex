\part{Sol de inverno}

A Marieta de Queiroz Guimarães e

Arthur de Queiroz Guimarães,

meus queridos pais

L.G.

\chapter{Capítulo 1}


\openany

Ao finalizar as suas atividades diurnas, o Rio de Janeiro apresentava o
seu aspecto característico e agradável das noites de verão. Não era a
cidade oprimida, em descanso das horas de trabalho, que se aperta entre
as montanhas e o mar; era a cidade-repouso, saindo das praias e
alcançando a altitude das serras.

Pelas suas avenidas de liso asfalto, os carros, à semelhança de
patinadores, deslizavam mansamente. Entre eles ia um ``Dodge'' cinzento,
em marcha lenta e indecisa. Conduzia-o Bárbara Stuart; uma inglesa de
nascimento que residia em São Paulo e estava a passeio na capital do
país. Mas, à direção do carro, aquelas ruas e avenidas tinham para ela,
que era estranha na cidade, o mesmo aspecto. Na impossibilidade de
orientar-se, acompanhava ora um, ora outro automóvel, no abandono de
quem procura distrair-se.

Um véu úmido distendia-se pela avenida; era uma sombra tênue, volátil,
que o vento trouxera do mar e, qual desenhista invisível, esboçara ali.
Bárbara contemplou a transparência daquela cortina que se alongava num
plano indeterminado e coloria a cidade com esse tom indefinido,
vacilante entre o cinzento e o azul, numa escala de gradações
imperceptíveis.

O carro prosseguia na sua marcha lenta; a cada rua que transpunha
parecia mais indeciso. Embora o traçado da cidade fosse desconhecido
para a moça que o conduzia, ela se aventurara ao passeio, guiada pelas
mãos da sorte, e confiada naquela esperança íntima de que tudo sai bem
numa noite calma de verão. Bárbara só se deteve quando, à sua frente, um
caminhão enorme, cuja carga ultrapassava as grades da carroceria, bateu
no para-choque traseiro de um ``Packard'' preto, parado na esquina à
espera da luz verde. Do ``Packard'' saiu um senhor bem trajado que
dirigiu ao condutor do caminhão, palavras pesadas. A resposta não se fez
esperar. Bárbara ouviu logo a voz áspera do homem de trabalho:

--- O senhor precisa compreender que isto é uma coisa que acontece.

E, como o senhor não se mostrasse satisfeito com a explicação, a voz do
homem se elevou:

--- O senhor não compreende que eu estou no trabalho desde a madrugada?
Trago de São Paulo coisas que vão para o seu estômago; enquanto o senhor
passeia aí para matar o tempo.

Naquela independência habitual da classe trabalhadora do Brasil, o
homem, irritado, crescia na sua grosseria, enfrentando a arrogância do
proprietário do ``Packard''.

A agitação do acontecimento não foi o bastante para desapontar a noite
serena, agradável, que Bárbara vinha já apreciando. No tumulto dessa
cena vulgar, a moça encostou o seu carro junto a uma árvore e caminhou
para fugir da perturbação. Esteve atenta e curiosa nos primeiros
quarteirões vencidos; não tardou, porém, que aos poucos se descuidasse
na sua direção. Aquela noite morna, suave, de nuvens limpas, condizia
com o seu estado de espírito e ela gostava de pensar, caminhando longas
distâncias. Numa esquina deteve-se, porque os outros também pararam ali.
Cercada pela pequenina massa humana, com ela se moveu quando o sinal
abriu o trânsito. Passando para o outro lado, tomou a rua de mais fácil
acesso e continuou a andar...

Trajava-se com simplicidade. Um vestidinho de linho pérola; sapatos sem
saltos, cômodos, próprios para vencer distâncias.

Afastava-se da cidade. Já não viera de lugar mui central e entregue a si
própria e aos seus pensamentos, buscava inconscientemente lugar mais
tranquilo. E aquela noite trazia à sua contemplação a tela mais
perfeita, espontânea e harmoniosa de um artista divino. Ela não pôde ser
indiferente a tanta beleza; pois tinha olhos para ver e sensibilidade
para sentir. As pequeninas coisas não lhe passavam despercebidas.

Andou sem descanso; passou da iluminação ao luar e tudo lhe parecia
belo, admirável. Estendeu longe o olhar, e o vulto das montanhas,
molduras naturais da cidade, confundiu-se à sua frente. A vegetação
ampla da capital, que nascida na praia saltou para as serras, era um
espetáculo repousante a que Bárbara se prendeu, sem dar contas do quanto
avançara. Entrou em um pequenino jardim, espécie de parque em miniatura,
que ficava abaixo do nível comum; e, contemplando-o, notou então que o
lugar era deserto. Sentiu que se afastara demais e pensou em regressar.
Quis retomar o mesmo caminho pelo qual viera; mas, ao rodear, o parque,
percebeu sua desorientação. Entretanto, como lhe era peculiar nas
ocasiões difíceis, conservou o seu habitual sangue frio. Andou para
todos os lados, na esperança de avistar ao longe a iluminação mais
intensa ou de perceber o bulício das aglomerações. Nada! Em tudo
permanecia o deserto.

--- Perdi mesmo o rumo... Como arranjar-me agora? --- comentou
intimamente a moça.

E não chegara a tomar uma resolução quando ouviu ruídos e voltou-se.
Pelo caminho estreito do parque, um desconhecido dirigia-se para ela.
Com o seu temperamento calmo, teve tempo de pensar; tomou deliberações
para enfrentá-lo. O desconhecido aproximou-se e, em tom polido, falou
sem demora, para que a moça percebesse logo a sua intenção:

--- Parece-me que a senhora se perdeu. Se me permite, tomo a liberdade
de oferecer minha companhia.

Bárbara fitou-o atentamente; e notou logo que não era mal-intencionado.
Observou mesmo a finura dos seus modos e a correção da sua linguagem.
Hesitou algum tempo e respondeu em tom grave, porém delicado:

--- Fico-lhe muito grata, se o senhor puder conduzir-me à cidade.
Gostaria, entretanto, de saber onde estou.

--- Dizem que este jardim se chama ``Retiro da Saudade''. Se é mesmo,
não sei, informou o desconhecido.

Pelas suas atitudes confirmara-se a impressão de que tratava com um
cavalheiro. Reparando, porém, nos seus trajes, notou-os surrados. Como
poderia um cavalheiro fino de maneiras, andar vestido assim? Veio-lhe
certa dúvida: seria um boêmio, um poeta, ou um filósofo?

Não era alto; possuía ombros largos e era cheio de corpo. Tinha um porte
firme e andava descuidado; era figura esquisita --- encarnava o paradoxo
de parecer simultaneamente um vencido e um vencedor. Vencido, pela
incúria às aparências pessoais, e vencedor, na atitude de quem supera os
preconceitos comuns.

Para Bárbara a noite não perdera o encanto. O céu claro e pontilhado de
estrelas era uma sinfonia azul de semitons esmaecidos, mas nítidos.
Tomaram atalhos diversos e foram dar em uma estrada mais larga. De
espaço em espaço um poste elétrico erguia-se à beira do caminho, bem
visível, no centro daquele círculo de luz que o focalizava como um
mastro tendo ao alto uma bandeira do progresso. Aquela área iluminada,
qual uma pequenina ilha, facilitava aos transeuntes o caminho nas suas
proximidades; mas, como tudo tem o seu reverso, era de ofuscante
confusão à distância. Assim iam pela estrada; ora na luz, ora na sombra.
Aconteceu então que Bárbara, ao sair do âmbito de luz, não distinguiu um
arvoredo cujos ramos avançavam pela estrada. Num galho mais ousado,
ficaram enroscados os seus cabelos. Trabalhou para soltá-los, mas sentiu
espinhos que dificultavam a tarefa. Levemente irritada pelo
acontecimento, Bárbara perguntou a si mesma --- como vou acabar isto
?...

O desconhecido chegou-se a ela e, parecendo ignorar o seu desaponto,
pediu licença e deixou-lhe em liberdade os cabelos escuros, longos
\emph{e} crespos.

Bárbara sentiu-se embaraçada e procurou palavras para agradecer.

--- Não sei o que dizer... mal me prestou um serviço, e veio logo outro.

--- Não posso qualificar isto como serviço --- tornou o rapaz
polidamente.

--- Em todo caso, agradeço a sua gentileza.

Ele sorriu naturalmente e respondeu:

--- Tive prazer em ser gentil. Não há o que agradecer.

--- Espero, entretanto, que não venha causar-lhe novos embaraços ---
completou a moça.

O rapaz parecia decidido a não deixar uma frase sem resposta. Assim, se
voltou para ela e concluiu:

--- Não causou até agora nenhum.

Bárbara achou graça naquela troca de amabilidade e, sorrindo
intimamente, calou-se.

O rapaz continuava, a andar descuidado; ao passar por um arbusto,
arrancou uma folha e pôs-se a triturá-la entre os dedos. Naquele gesto
mecânico, olhava para a frente ou para o lado oposto ao da moça.
Bárbara, impressionada pelas suas atitudes, observava-o furtivamente. A
escassez de luz não lhe permitia ver muito; distinguia, contudo, que o
rapaz era de tipo moreno e tinha os cabelos escuros, lisos e não muito
bem penteados. Na estatura média, havia uma conformação física bem
masculina; tinha no seu todo aquela aparência máscula que pode revestir
um homem sem embrutecê-lo.

Ao passar por um dos postes, o rapaz dirigiu-se a ela:

--- É por aqui --- disse, indiciando um novo caminho.

Bárbara acompanhou-o e, mal dera alguns passos, ouviu-o falar novamente.

--- Tenciona ir a algum bairro? --- indagou ele.

--- Não senhor; ficarei no centro.

--- Perguntei --- tornou, explicando-se --- porque talvez pudesse lhe
ser útil.

--- Obrigada --- tornou a moça --- estou no centro mesmo.

--- Reside no centro?

--- Não resido; hospedo-me. Depois, eu não abusaria assim da sua
gentileza.

E sorrindo, desviou o olhar.

--- Achou graça em alguma cousa? --- indagou o moço com naturalidade.

--- Lembrei-me, apenas, de um ditado brasileiro que bem se aplica ao
caso.

--- Posso saber?

--- Naturalmente. Diz o povo --- continuou ela e acentuando suas
palavras --- cuidado com o primeiro obséquio, pois este puxa os demais.

--- O povo é interessante; nunca dispensa as suas máximas.

--- ... E que não deixa de ter um fundo de verdade --- ajuntou Bárbara.

--- O povo é como o filósofo, tem sempre alguma cousa a dizer ---
acrescentou o moço num sacudir de ombros.

--- Há de concordar, todavia, que sendo semelhantes, são também
diferentes...

--- Sim; mas pode haver o filósofo pensador e o filósofo popular. Hoje
em dia existem tantas classes de filósofos --- exclamou ele com
acentuada ironia. --- Há os que não ligam à vida, os que se trajam mal,
enfim, uma imensidade. O que está em falta no momento, como todo artigo
bom, é o filósofo pensador.

Bárbara fitou-o curiosa. Notou o seu português correto; predicado esse
pouco encontrado nos brasileiros que conhecera. Observou que ele era
simples nas suas palavras, mas preciso nos seus termos.

--- E prosseguindo no assunto --- volveu o companheiro antes que Bárbara
dissesse alguma coisa --- na falta de filósofos originais surgem os
intérpretes. Aí, que desastre. Veem-se Platão, Aristóteles e toda aquela
gente que pensou na Antiguidade, modernizando-se. Fazem os coitados
dizerem coisas que francamente, não se compreende como não voltam ao
mundo para desmentir.

--- Nesse ponto os pré-socráticos foram mais felizes --- comentou a
moça.

--- Porque se ocuparam menos deles. E, nesse caso, é melhor ser
desconhecido; pelo menos não é deturpado. Já não diz o italiano que
``\emph{tradutor, traditores}''? Imagine então o intérprete...

--- Mas existe hoje e existirá sempre uma porcentagem, embora diminuta,
dos que verdadeiramente interpretam o filósofo --- objetou a moça.

--- O pensador, porém. Convém precisar o adjetivo. Caso contrário, vão
acreditar que a senhora se refere a alguém mal trajado.

A conversa, nascida do simples desejo de criar-se um ambiente mais
desafogado, ia tomando um rumo de interesse para ambos. O assunto os
atraía e eles, involuntariamente, se deixavam levar pelo agradável de
pensar em comum. Não se conheciam ainda, mas este quê --- traço de união
entre almas do mesmo nível --- deixou-os num terreno mais livre e eles
se aventuravam pela estrada do pensamento, sem a consciência plena do
quanto avançavam. Bárbara, pelo seu temperamento altivo e confiante,
prosseguia naturalmente na troca de ideias com o companheiro que
circunstâncias inesperadas lhe trouxeram. Percebera desde o início,
intuitivamente, que podia confiar no rapaz; todavia, uma defesa
instintiva fê-la ter certa reserva. Ela, porém, desprendia-se desta
reserva para prender-se mais às ideias que ele apresentava. O rapaz,
percebendo que sua interlocutora o acompanhava nos seus pensamentos,
procurou continuar o assunto e dirigindo-se a ela, perguntou:

--- A senhora acha considerável essa porcentagem a que se referiu?

--- Que interpreta o filósofo? --- inquiriu a moça.

--- Sim; isto mesmo.

--- Considerável... não. Não poderia achar semelhante coisa.

--- Então, está comigo --- volveu ele. --- Pode-se dizer mesmo que é uma
minoria incrível.

--- Mas por esta minoria vale a pena.

--- Vale --- concordou ele. --- O que irrita, porém, é que todos querem
se incluir nesta minoria.

--- E acabam na maioria --- concluiu Bárbara, sorrindo.

--- Justo! --- exclamou ele com entusiasmo. --- Temos então o medíocre
julgando-se um pensador. E, ainda não é só... quer que os outros o
julguem também.

--- E quando um medíocre se julga sábio!...

--- Por isso mesmo, num dia em que estava pensando, dividi os filósofos
em duas classes: o filósofo pensador e o filósofo prático.

--- Este filósofo prático é bastante original --- ponderou a moça.

--- Não é. Bem ao contrário, é a cópia grosseira do filósofo pensador.

--- Onde o senhor o encontra?

--- No povo.

--- E se o senhor exclui o filósofo pensador do povo, onde o irá
colocar?

--- Perdão; entendeu-me mal. Ao dizer povo, não me referi
intencionalmente à expressão máxima de uma coletividade; mas apenas a
esse conglomerado de inteligências vulgares, onde o filósofo jamais
encontraria o seu nível. Bem, podemos substituir por massa, uma vez que
o termo é assim importante. Este filósofo prático a senhora o encontra
na massa, frisou o rapaz.

--- Algum pragmatismo avançado?

--- A minha divisão não vai a ponto de classificar esta filosofia...
prática --- disse em tom de gracejo. --- Não faço parte destes homens do
Brasil que reduzem os intelectuais para depois se colocarem no meio
deles como preciosidades raras: "nós os quarenta ou cinquenta
intelectuais brasileiros..." E voltando ao assunto, o meu filósofo
prático é um ser coletivo que para mim não precisaria de especificação,
mas que para os outros, eu diria ser a massa.

--- É então um ser coletivo --- repetiu a moça para si mesma.

--- Isto mesmo. Se fôssemos procurar alguma coisa em cada um
separadamente, quase nada se encontraria; mas, se os deixarmos juntos,
formando um todo, originar-se-ão as fórmulas, compondo nova filosofia.

--- Nova?! O senhor disse que são cópias...

--- Empreguei o termo ``nova'' assim como empregaria ``outra''; apenas
para precisar bem a separação --- explicou ele com firmeza. E virando-se
para o lado da moça, concluiu --- Há muito que penso, sem traduzir em
palavras os meus pensamentos. Talvez, por isso, não tenha explicado
convenientemente.

--- O senhor deve ter percebido que eu não quis me prender às suas
palavras. A minha insistência num ou noutro ponto, foi simplesmente,
para chegar às suas ideias pelo trajeto mais curto.

--- Percebi, sim --- disse ele a sorrir.

--- E então? --- interrogou a moça.

--- E então? --- repetiu o rapaz. --- Então... a massa, note-se bem, a
massa, no seu bloco, é também uma espécie de filósofo. --- E apontando
com o dedo indicador para Bárbara, continuou no seu sorriso --- poderia
aprofundar-me mais; prefiro, porém, deixar assim.

--- Por quê?

--- Porque é perigoso explicar ou definir muito. Os latinos, que já
sabiam disto, diziam que ``\emph{Omnis definitio periculosa est}''.

\chapter{Capítulo 2}

Pelo novo atalho que tomaram, Bárbara deixou-se conduzir; não conhecia o
lugar e, distraída como viera, nada lhe ficou do trajeto. Era uma
espécie de estrada que, de espaço em espaço, se ramificava: impossível
reconhecer por onde passara. Observou a extensão desabitada e conquanto
a achasse considerável, não lhe causou grande impressão. O Rio de
Janeiro crescia de forma irregular e, em pleno coração da cidade, havia
ainda terrenos imensos, abandonados. O morro do Largo da Carioca, no
caminho de Sta. Tereza, era um exemplo curioso, frente ao progresso da
lindíssima capital brasileira. Continuaram a andar por algum tempo e o
companheiro nada mais dissera. Distraía-se com uma ou outra coisa sem
importância: jogara aos poucos pela estrada, os retalhos da folha
triturada. Lá a certa altura, chutara, por acaso, uma pedra à beira do
caminho. Prosseguiam... todavia, os passos vacilantes do rapaz começavam
a denotar indecisão. E Bárbara viu-o franzir a testa, numa atitude de
quem luta com preocupações. Aqueles passos vagarosos ecoavam no
silêncio, quase absoluto, do lugar. Iam para a frente, mas... o contágio
do silêncio na alma do rapaz, levava-o à reflexão. Na solidão
ponderam-se os instintos. --- Por que tomara ele esse atalho? perguntou
a si mesmo. Não era um prolongamento desnecessário? Não se oferecera
para acompanhá-la e não era esta a sua firme intenção quando no parque?
E não faltara agora à lealdade do seu compromisso? --- Perturbado por
aquela voz íntima, que tão bem fala aos homens nos momentos de
incerteza, voltou-se para a moça --- Considerava-a ele como devia?
Observou-a atentamente: a confiança com que ela o seguira até ali, a
altivez da sua atitude, a firmeza com que sustentara o seu olhar e
sobretudo a sua expressão, infundiram-lhe respeito. O prazer de uma
companhia agradável de mulher, o desejo de mais conversar com ela, a
vida solitária que levava e a oportunidade inesperada que as
circunstâncias criaram, pesaram menos que o respeito que ela lhe
inspirara. Assim, voltou-se bruscamente:

--- Cometi um engano lastimável. Prolonguei desnecessariamente o
caminho.

Bárbara apenas sorriu com tristeza. No íntimo, porém, compreendera o
gesto natural a que fora levado; e a confissão do rapaz era uma prova
apreciável da posição em que a colocara.

--- Vamos voltar à estrada principal --- disse ainda o moço.

Bárbara, constrangida, acompanhou-o. Ele parecia atento e seus passos,
mais firmes, tomaram-se mais rápidos. Bárbara observou que o percurso
era o mesmo de há pouco, em sentido inverso. Mais adiante viu a pedra
que ele chutara inadvertidamente. Assim continuou o desconhecido,
apressando-se cada vez mais; e, somente ao entrar de novo na estrada
principal, voltou se para a moça, a fim de explicar-se. Notou, então,
que ela se cansara muito; respirava com dificuldade e, embora procurasse
disfarçar, sua fisionomia mostrava o esforço da caminhada. Desapontado,
tentou desculpar-se:

--- É sempre assim --- disse aborrecido --- pior é a emenda que o
soneto.

--- Não se preocupe com isso --- volveu Bárbara solícita, ante o
embaraço do companheiro que continuava desconhecido para ela.

--- Parece que ali está um banco --- disse o rapaz, apontando uma forma
esboçada na penumbra.

Mais para ser delicada, ela aceitou o oferecimento:

--- Podemos descansar um minuto --- respondeu em tom complacente.

Aproximaram-se do banco e Bárbara sentou-se logo. Ao lado, ele
permaneceu de pé.

--- A senhora está bem? --- indagou com acentuada delicadeza.

--- Muito bem.

Bárbara deixou-se ficar ali, pensativa, e ele, pedindo licença,
distanciou-se para fumar um cigarro. Pôde assim contemplá-la.
Pareceu-lhe bem morena e de olhos bem negros; todavia, não lhe estava
sendo possível distinguir exatamente as suas feições. Uma coisa, porém,
era certa: a moça era linda e ele percebeu na sua atitude a convicção
dessa beleza. No seu porte feminino a graça da adolescente se valorizara
com a altivez sublime da mulher formada. Seu corpo tinha formas
acentuadas e era elegante. Havia ainda alguma coisa a mais... não era
somente a mulher bonita. E como Bárbara estivesse distraída, ele
aproximou-se... Notou a expressão no seu olhar --- era algo suave e
enérgico, feminino e profundo, próximo e longínquo, simples e difícil.
As sobrancelhas finas, bem separadas, partiam retas e terminavam
levemente em curva, formando ao centro um ligeiro ângulo. Os frontais
amplos contornavam a sua testa de linhas marcadas. E quando Bárbara ia
dirigir-se a ele, viu que o rapaz a observava atentamente. Apanhado em
flagrante, o rapaz perguntou-lhe:

--- De que cor são os seus olhos?

Não obstante estranhasse a pergunta, Bárbara respondeu com certa
naturalidade, pois outros já a haviam feito antes:

--- Castanhos.

--- Estarei enganado? Vejo-os pretos.

--- Não está enganado --- afirmou a moça.

--- E então?

--- Mas, não são pretos.

--- Não entendo.

--- Apenas contêm elementos para serem definidos ``vagamente'' ---
acentuou ela com certa malícia e vaidade.

--- Vejo que não preguei no deserto --- disse em tom de brincadeira. ---
E agora ---perguntou estendendo-lhe a mão em atitude provocadora ---
pode-se saber como definiria ``vagamente'' a cor dos seus olhos?

--- Castanhos escuros... quase pretos.

--- Este quase... tira um pouco da vacuidade --- comentou ele com
forçada naturalidade.

--- Não se esqueça, porém, que o senhor é que prefere ser vago ---
emendou a moça.

--- Prefiro mesmo --- concordou --- como... como vago foi este acaso,
continuou o rapaz mais para dizer alguma coisa no momento.

--- Vago?

--- Sim. Não sabemos com quem estamos falando e nem como viemos parar
aqui. Não pense que eu deseje saber quem a senhora é; prefiro mesmo
deixar as coisas correrem, sem apressá-las ou retê-las. Encontrá-la-ei
novamente, se um dia eu tiver de sabê-lo.

--- Tem certeza?

--- Confio no acaso.

--- Confia? --- indagou Bárbara admirada.

--- Sou meio fatalista; curvo-me, seria o termo mais apropriado.

--- Mesmo os que não o são, curvam-se ao acaso.

--- A senhora não o é?

--- Depende do seu modo de interpretar esse fatalismo.

--- É bastante prudente para emitir as suas opiniões --- considerou ele.

--- Não o necessário; o senhor está julgando precipitadamente, pois não
me conhece.

E levantando-se do banco, dirigiu-se ao rapaz.

--- Eu desejava saber as horas.

--- Onze horas.

--- É tarde --- tornou a moça com delicadeza. --- Poderíamos continuar?

Puseram-se novamente a caminho. E embora desejasse esquecer o incidente,
Álvaro surpreendeu-se a perguntar no seu íntimo --- terá ela confiança
para ainda me seguir? Procurou devassar-lhe o pensamento; nada, porém,
lhe transparecia na atitude. Teria continuado nas conjecturas, se um
perfume suave não lhe viesse acariciar os sentidos.

--- Que aroma agradável --- exclamou ele.

--- É o Heliotrópio --- respondeu simplesmente a moça.

--- Flor?

--- Não; perfume. Em abril uso também a flor.

--- Não me esquecerei deste perfume. Que suavidade --- disse aspirando o
ar delicioso.

Bárbara pôs no bolso novamente o lencinho de cambraia que espalhava o
perfume de heliotrópio. E quando se voltou, o rapaz olhava para o alto.
Acompanhou-lhe o gesto, e lá em cima... quantos pontos prateados e, cada
um deles, um mundo. De repente, outros pontos luminosos, menos
distantes, tomaram forma no espaço... eram as luzes da cidade.

\chapter{Capítulo 3}

Na manhã seguinte, Bárbara dormia profundamente quando a campainha do
telefone vibrou com insistência. Não muito acordada, estendeu o braço e
tomou o aparelho de sobre o criado-mudo. Uma voz feminina falou do outro
lado:

--- Alô, Bárbara?

--- Sim, Helena. Que há para acordar-me nesta madrugada? --- perguntou
com a voz embaraçada.

--- Madrugada? São oito horas e está uma linda manhã. Que aconteceu com
você?

--- Comigo? Nada. Estou apenas sendo original; dormindo mais do que nos
outros dias. E você, que faz?

--- Já leu os jornais?

--- Para saber da política? Sei que a Europa está na iminência de uma
guerra; porém, o fato de eu dormir menos não adiantaria coisa alguma
para eles.

--- Algo mais. O seu carro foi multado, minha querida amiga.

--- O meu? Gente! Onde eu o teria deixado?

--- Sim, o seu. Levou várias multas e acabou sendo apreendido pela
polícia.

--- E que pretende a polícia?

--- Não o abandonou? Agora, aguente as consequências.

--- Sabe de uma coisa, Helena? Estou meio confusa ainda; devo estar com
sono.

--- Está meio confusa desde ontem. Que se esqueça um guarda-chuva, não
há nada, mas um carro!

--- Bem Helena, estas repreensões logo ao amanhecer... Escute, quer
café?

--- Já não é amanhecer e não quero café.

--- Que quer então?

--- Sair com você hoje à tarde. Quero mostrar-lhe o Rio e convencê-la de
que deve morar aqui.

--- Que pretende, um cinema ou um sorvete?

--- Um chá, algum passeio e uma visita.

Bárbara riu e perguntou assentindo:

--- A que horas, então?

--- Às três, está bem?

--- Esperá-la-ei pronta.

--- Até logo.

--- Até logo, Helena.

Bárbara espreguiçou-se na cama e olhou para o relógio ainda admirada.
Bateram de leve à porta e a governanta entrou, do quarto contíguo,
trazendo o café da manhã.

--- Parece que dormi hoje mais que nos outros dias --- disse Bárbara,
sentando-se na cama.

--- Fiquei um pouco apreensiva --- respondeu a governanta --- mas, vim
vê-la, achei-a calma e voltei para o meu quarto.

A governanta tinha um ar circunspecto e tratava Bárbara com certa
intimidade. Mostrava ter quarenta e cinco anos, e pelos seus cabelos
pretos, só agora, começavam a aparecer os primeiros fios prateados. Era
bastante magra, mas de aspecto sadio e agradável. No seu rosto oval
sobressaíam os olhos cinzentos, perscrutadores, emoldurados por cílios
curtes e escuros; o nariz adunco e a boca pequena de linhas firmes. Mrs.
Patrice olhava para as pessoas sempre de frente e na sua expressão
transparecia a firmeza do seu caráter. Era boa de sentimentos. Não dessa
bondade passiva, a que poderíamos chamar de preguiça mental e que o
vulgo aclama como sendo a mais perfeita; mas, a bondade enérgica que
sabe dizer ``não'' no momento preciso. Bondade somente compreendida
pelos que se elevaram acima do nível comum ou pelos que sentiam em si os
sintomas dessa mesma força. Como simples governanta, exercera na vida de
Bárbara uma influência mais benéfica que muitas mães em seus próprios
lares.

Bárbara tomou o café e levantou-se. De pijama ainda, sentou-se à
escrivaninha; abriu uma gaveta, tirou de lá o seu Diário e, olhando uma
fotografia que tinha sobre a mesa, começou a escrever.

\chapter{Capítulo 4}

A fotografia mostrava um homem de trinta anos. Moreno, feições enérgicas
e um olhar profundamente expressivo. Era somente o busto, mas, por este,
percebia-se um físico forte. A mesma testa ampla de Bárbara, com os
frontais marcados, sem exagero no tamanho. Era a fotografia de John
Stuart, pai de Bárbara. De família inglesa, Bárbara vivera na Inglaterra
até que, pela morte de seus pais, se vira só no mundo, em companhia da
governanta dedicada. Perdera primeiramente a mãe que a deixara com três
meses apenas. Não conhecera, portanto, a progenitora, mas, aprendera a
amá-la através das palavras paternas. Perdendo a esposa, John
dedicara-se à filha com todo o amor e carinho. Via nela um misto da
esposa e de si próprio; certos traços do temperamento fleumático materno
e outros, mais impulsivos, semelhantes ao seu. Sendo Bárbara filha
única, tinha satisfeitos todos os seus caprichos e John, já perturbado
nos sentimentos pela falta da companheira, entregou-se inteiramente à
filha. Fazia-lhe companhia nos brinquedos; contava-lhe histórias;
servia-lhe de cavalinho, levando-a montada, nos seus ombros a percorrer
a casa; passeava com ela pelos bairros mais alegres de Londres. Quando a
criança chegou à idade dos estudos, observou-a, procurando conhecer
desde logo as suas tendências. Depois que aprendera a ler, comentavam
juntos o procedimento das fadas da Carochinha. A menina parecia dotada
de senso crítico desde os primeiros indícios da razão. Sentiu queda pela
música; imediatamente o pai procurou satisfazer-lhe o gosto. Deu-lhe boa
professora e ela iniciou-se no piano, em uma escola reconhecidamente
certa. Percebendo, entretanto, que era um gosto e não uma vocação,
deixou-a estudar para que adquirisse boas noções de música; assim, se
não seguisse a carreira, teria elementos para mais apreciá-la.

O divertimento predileto de Bárbara, quando pequena, era fazer grandes
bolhas de sabão. Passava tempo assoprando pela mão em forma de canudo e
divertia-se, vendo subir as bolhas, até que arrebentassem no alto. Certo
dia, o pai perguntou-lhe o que achava da música. A menina pensativa
respondeu --- quem não estuda música, parece um jardim só de grama ---
Não obstante, quando fez nova pergunta --- De que você gosta mais, de
estudar ou de fazer bolhas de sabão? --- a criança riu e respondeu logo
--- Fazer bolhas de sabão. --- Assim, ia percebendo seus gostos e
encaminhando sua educação.

Mas como tudo o que é bom dura pouco, aquela felicidade, subitamente,
foi cortada mais uma vez. John adoeceu gravemente e, pelo adiantamento
da moléstia, os médicos não tardaram em reconhecer o mal que tanta
preocupação tem dado ao homem moderno. Quis saber o seu estado e, diante
da sua insistência, nada restou senão revelar a verdade. Quando, pelos
médicos, soube que tinha um câncer, pôde ainda considerar toda a
situação. Chamou a governanta, Mrs.~Patrice, e fê-la prometer que não
deixaria nunca a menina. Recomendou-lhe as menores coisas, pois, ele
sabia que Bárbara, mimada como fora e tendo tido sempre um cuidado
especial, sofreria se fosse entregue a parentes como uma criança órfã.
Um procurador providenciou os seus negócios e os rendimentos foram
assegurados com eficiência. Quando piorou mais, foram buscar a menina
que naquele momento se divertia com bolhas de sabão. Vendo chegar a
governanta, apontou-lhe uma:

--- Veja. Mrs.~Patrice, como está subindo... Eia! Arrebentou.

--- Bárbara --- disse em tom carinhoso a governanta --- o papai vai
visitar a mamãe e quer despedir-se da filhinha.

A criança deixou-se conduzir ao leito do pai e, vendo-o, sentiu medo.

--- Para onde é que papai vai, e por que deixa Bárbara sozinha?

--- Minha filha --- respondeu com voz embargada --- papai não vai por
vontade própria. Há uma vontade superior que governa o mundo e quer
assim. Mas minha filha não vai ficar sozinha; Mrs.~Patrice
acompanhá-la-á toda a vida.

--- E o senhor, papai, volta um dia? O senhor não vai morrer? Mamãe não
voltou mais --- disse rompendo em pranto.

--- Escute Bárbara, tenho um presente novo...

--- Para mim? --- gaguejou a menina.

--- É. Mas vou pedir uma coisa.

--- Até uma porção --- disse enxugando as lágrimas.

--- Veja isto.

--- É um livro?

--- Mas ainda não está escrito.

--- E quem vai escrever nele?

--- Você, minha filha.

--- Eu? Por quê?

--- Para papai ler depois.

--- Depois, quando?

--- Quando Deus quiser. Você vai escrever o que acontecer; faça de conta
que está escrevendo uma carta para papai.

--- Vou contar tudo?

--- Tudo.

--- Mas eu não quero que o senhor vá embora. Não sei fazer nada sozinha.

Suas lágrimas caíram sobre o livro e, enxugando-as, viu na sua capa azul
um dístico: \emph{``Cartas para papai'';} a um canto, o seu nome
\emph{``Bárbara''.}

--- Minha filha --- disse ainda o pai --- você só vai escrever o que for
verdade. Nem uma vírgula mentirosa deixe aí.

--- Se for mentirosa, eu tiro com a borracha --- disse com seriedade a
menina.

Ele sorriu ainda das palavras de Bárbara e, passando-lhe a mão
enfraquecida sobre os cabelos, deu o último suspiro com a expressão
transfigurada pela dor, nessa agonia de torturas que enfrenta um
canceroso quando não se entrega a entorpecentes.

Bárbara estava órfã aos nove anos.

\chapter{Capítulo 5}

Sentada ali à escrivaninha, Bárbara lembrava-se dessas cenas como se
tivessem sido ontem. Lembrava-se do seu riso infantil, quando as bolhas
arrebentavam no alto. Tinha brinquedos caríssimos, mas deixava-os por
aquele divertimento tão simples. Quanta puerilidade nessa coisa
insignificante que lhe dava mais gosto que muito brinquedo vistoso.
Lembrava-se da primeira e única vez que fora fazer as bolhas após a
morte do pai; elas arrebentavam logo, e Mrs.~Patrice encontrara-a
pensativa. Viera-lhe à mente a resposta que dera à governante --- Sabe,
Mrs.~Patrice, não vou brincar mais com o sabão sem o papai; estava tão
acostumada com ele. Ficara ali por longo tempo, procurando adivinhar
para onde iam as bolhas ao se desfazerem. A ideia passada voltou-lhe
comparativamente: como as bolhas se pareciam com a vida. Sabia-se da sua
existência após a mudança de estado, mas, uma ignorância absoluta,
quanto ao seu paradeiro. Assim para com seu pai; sentia que ele vivia
ainda... onde, porém?

Tomou a fotografia do pai; fitou-a demoradamente. Parecia tão moço, tão
forte, e a vida o levara tão cedo. Falecera com trinta anos, e aquela
fotografia fora a última, tirada no mesmo ano do seu falecimento.
Poderia, na verdade, extinguir-se com a morte tanta vitalidade? Sentia
uma saudade imensa do pai e teve desejo de vê-lo. Veio-lhe ao pensamento
um verso de Tagore: \emph{Imediata seja a minha volta à sua presença.}

Não era mulher que vivesse do passado; mas havia dias em que esse
passado se reconstruía nitidamente na sua imaginação. Desenhou-se então
o solar inglês, onde passara a infância; viu o seu quarto, a sala de
música. Lembrou-se de que aí o tio Maurício a convidara para visitar o
Brasil, quando tinha apenas treze anos de idade. Viu-se saindo com Mrs.
Patrice e dirigindo-se para o cais. A entrada no grande transatlântico
que as levaria da pátria por alguns anos e, quem sabe, se para sempre...
O tio Maurício, com o seu porte fino, em pé no tombadilho, olhando para
os últimos sinais do porto que desaparecia. Fizera outras vezes essa
mesma viagem; sempre em serviço de inspeção aos interesses dos ingleses
nas terras longínquas do Atlântico Sul. Recordou a sensação de abandono
e arrependimento quando, sob suas vistas, se apagaram, lentamente, os
vestígios da Inglaterra. Depois em viagem, já mais ambientada, ao
conhecer Helena. Lembrava-se com clareza de como se aproximara daquela
menina loira, com os cabelos em cachos e de uma timidez colegial --- uma
ave marinha, num voo cheio de cadências, despertou o interesse comum que
as levou a trocarem impressões sobre as figuras descritas pelo voo da
ave. Helena era brasileira e regressava ao Brasil após seis anos de
ausência. Descrevia o Brasil para Bárbara, mas o Brasil da sua infância
e, em conjecturas com a nova conhecida, imaginava que mudanças
encontraria. A convivência aproximou-as e, em chegando ao Brasil, o
destino reuniu-as na mesma cidade. Uma vez em São Paulo, tornaram-se
amigas para sempre. Quando Helena reiniciou seus estudos, Bárbara, para
acompanhá-la, ingressara com ela no Mackenzie, onde ambas concluíram o
curso secundário pelas diretrizes do ensino brasileiro. E em tudo isso,
Mrs.~Patrice, sempre dedicada, seguindo-a como se fora sua própria
sombra. Em seguida, veio-lhe à mente a partida do tio Maurício, com rumo
a terras mais distantes, e ficando de vir buscá-la mais tarde.
Finalmente, aos dezessete anos, a sua resolução firme de ficar no
Brasil. O aviso ao tio de que não mais regressaria com ele; e a carta
que ele lhe mandou, pondo-se ao seu inteiro dispor e prometendo-lhe uma
visita para breve. Lembrou-se, então, da surpresa ofendida dos seus
parentes distantes; da indecisão após cada carta que chegava da
Inglaterra. Mrs.~Patrice aconselhando-a --- Vamos ficar mais um pouco,
porém não para sempre. Os livros e revistas que recebia da pátria
longínqua, pondo-a em contato com ela, todavia, sem alimentarem a ideia
do regresso. Por outro lado, Helena insistindo, suplicando para que não
fosse. E ainda mais, o ambiente que formara, a camaradagem dos colegas
\emph{mackenzistas}, a direção que tomara na vida e que mais uma vez
seria cortada, a permanência de quase quatro anos no Brasil, a
aprendizagem de outra língua... Todos estes fatos pesando na sua
indecisão e adiando indefinidamente a sua partida.

E como se afeiçoara à terra hospitaleira. Que mágoa não lhe causavam os
colegas ao desprezarem as coisas do Brasil; e como os próprios
brasileiros se tornavam cúmplices de um Brasil espezinhado! Desde os
bancos do ginásio até as cátedras, tornava-se ``difícil e raro'' o
conhecimento de um nacional ilustre. Para se darem a uma importância
fictícia e figurarem um conhecimento mundial, era necessário a elevação
exclusiva de nomes e empreendimentos estrangeiros em função direta do
arrasamento dos nacionais.

Bárbara reclinou a cabeça para trás e o seu olhar pousou ainda na
fotografia do pai. À ausência dele estavam ligados todos estes fatos.
Seria diferente se ele estivesse no mundo. Mas, não fora ele levado para
que estas coisas acontecessem?...

E, desviando o olhar, retomou ao passado involuntariamente. Na tela
reprodutiva da imaginação, outros fatos se desenharam com nitidez.
Lembrou-se de quando aprendia a língua portuguesa --- como se admirava
ao verificar que o pronome ``eu'' não era escrito com maiúscula. E o
professor da matéria, respondendo à sua pergunta: ``Dos idiomas que
conheço, o inglês é o único a fazer tal uso''. Se outros povos,
continuadores da cultura clássica, usavam com simplicidade uma inicial
minúscula, qual seria a razão que levava os ingleses a agirem assim?
Estranhava estas coisas, mas observava-as e esperava compreendê-las um
dia.

Cerrou os olhos e, na abstração do presente, viu-se ainda na faculdade.
Os dias que lá passava estavam vivos na sua imaginação; pois, havia
algumas semanas apenas que se diplomara. Cursara filosofia naquele
estabelecimento; e assim, oportunamente, encontrara elementos insignes
da cultura universal. Professores europeus e autoridades brasileiras nos
diversos ramos do conhecimento; porém, desconhecidos pelos nacionais. Ah
o Brasil, o Brasil, que terra estranha esse Brasil!

Finalmente, após terminar o curso, estava a passeio no Rio, para onde
Helena se mudara um ano antes.

Longe da pátria, crescendo em um meio diferente de raças e costumes,
obrigada a resolver desde cedo os seus problemas e aguentar com as
resoluções tomadas, Bárbara foi acentuando a sua personalidade e tomando
uma posição firme na vida.

\chapter{Capítulo 6}

Três horas.

Alguns minutos depois, Helena entrava no apartamento de Bárbara.

Helena era o tipo oposto de Bárbara. Muito magra, seu corpo quase não
chegava a ter formas; excessivamente loira, tinha traços delicados e um
conjunto agradável; de altura média, o que mais a salientava era uma
espontaneidade de gestos e palavras, e uma sinceridade tipicamente
feminina. Tinha certa cultura, pois, estudara ao lado de Bárbara, e
deixara-se conduzir por ela em vários pontos. Conquanto não chegasse a
acompanhá-la plenamente nas suas divagações, aproveitava muita coisa do
que ela dizia e em ocasiões oportunas vinham-lhe à mente as palavras da
amiga. Helena adorava Bárbara e esta correspondia com amizade intensa.

Helena era falante e estava sempre ao par dos últimos acontecimentos.

Quando avistou Bárbara, dirigiu-se a ela com a sua habitual
espontaneidade e sentiram-se ambas muito alegres por estarem juntas.

--- Bárbara, onde esteve ontem? indagou com interesse.

--- Porque me pergunta?

--- Francamente, estou admirada. Esquecer um carro numa esquina! Minha
querida, um carro é algo grande para ser esquecido, comentou
maliciosamente a amiga.

--- Vê? As coisas grandes também se esquecem. O mesmo não se dá com as
grandes coisas, respondeu Bárbara, olhando-se ao espelho pela última
vez.

--- Não divague; estou interessada no caso. Onde esteve ontem à noite?
--- insistiu Helena. --- Não me venha dizer que visitava um museu ou uma
galeria, de arte...

--- E porque não lhe dizer a verdade? Eu estava mesmo num museu ---
tomou sorrindo.

--- Museu, à noite? Essa é um pouco avançada.

--- O museu que eu visitei está aberto a qualquer hora.

--- Quem o deixa aberto?

--- Pergunte antes, quem o poderia fechar.

--- Quem? --- repetiu ingenuamente Helena.

--- Pessoa alguma.

--- Por quê?!

--- Quem fecharia a natureza!

--- Logo vi que era uma cilada. A natureza é então um museu\emph{?}

--- O mais belo e mais completo que conheço.

Helena, sabendo já a que proporções poderia chegar a conversa, preferiu
tomar outro rumo. E voltando-se para a amiga, disse-lhe em tom decidido:

--- Bem, tratemos então de sair. Que vamos fazer primeiro?

--- Eu gostaria de ir buscar o meu carro.

--- Isso mesmo.

E ambas saíram à rua. Trajavam vestidos leves, pois estavam em janeiro.
Formavam um contraste interessante; Helena muito loira, e Bárbara,
morena de cabelos escuros.

Dirigiram-se à Delegacia. Quando chegaram, o encarregado da Portaria
perguntou-lhes o que desejavam. Bárbara respondeu:

--- Vim retirar o meu carro que foi apreendido ontem à noite.

O encarregado observou-as um tanto surpreso e, passando à frente,
pediu-lhes que o acompanhassem para falar com o delegado.

Entrando em uma sala, no final do corredor, o encarregado dirigiu-se a
um homem que, sentado à escrivaninha, examinava alguns papéis.

--- Dr.~Madeira, aqui estão as proprietárias de um carro apreendido.

Dr.~Madeira, homem prático, de seus cinquenta anos, já meio calvo,
ergueu os olhos dos papéis. Mostrou-se admirado ao vê-las, e Bárbara
preferiu falar logo para encurtar a situação.

--- Dr.~Madeira, eu sou a proprietária do auto 15.152, deixado ontem na
rua e apreendido pela polícia. Gostaria de saber quais as providências
exigidas para a devolução.

--- Apresenta credenciais, minha senhora?... ou senhorita? --- corrigiu
ele.

--- Senhorita --- respondeu Bárbara calmamente.

--- E o seu nome?

--- Bárbara Stuart.

--- Senhorita Bárbara Stuart, pode apresentar seus documentos?

--- Na caixa interior do carro, estão todos os documentos exigidos pela
lei e minha carta de habilitação.

Dr.~Madeira levou-as ao pátio e Bárbara reconheceu logo o seu carro. O
delegado deu a ordem de devolução do veículo, e, antes que ela pensasse
em retirá-lo dali, um guarda atencioso conduziu-o para o portão da rua.
Fez-lhe a entrega, frisando a sua gentileza; a moça recebeu-o com um
``muito agradecida'' e uma gorjetinha disfarçada. O guarda protestando o
seu desinteresse, levou logo a mão ao bolso, e o incidente parecia
esquecido se um repórter social não estivesse ali na ocasião. Vendo-o,
Bárbara virou-se para o lado oposto e chamou Helena que ficara atrás.
Imaginou, assim, ter fugido à impertinência da objetiva fotográfica que
estivera assestada para ela. Sentir-se-ia contrariada se, naquela chapa,
se visse coagida a deixar o permanente registro do seu desleixo. Porque,
perguntou a si mesma, não guardaria consigo, apenas, a responsabilidade
do que fizera?

\chapter{Capítulo 7}

--- Eram cinco horas quando Bárbara e Helena pararam o automóvel ante
uma residência de fino gosto, no bairro de Copacabana. Bárbara,
surpresa, interpelou a amiga:

--- Para onde vai levar-me?

--- Aqui --- respondeu a outra, indicando a casa. --- É a visita
combinada, não se lembra?

--- Helena, Helena --- volveu Bárbara --- você me apronta cada uma.

--- Apronta? Isto é de paulista, minha amiga.

--- E de onde venho eu?

--- De São Paulo, é verdade; eu, porém, que desejo vê-la no Rio... ---
insinuou a amiga.

Bárbara sorriu e Helena bateu no mesmo assunto:

--- Nos seus trajes está bem à carioca. Só falta agora, usar o ``tu'' na
intimidade.

--- Nos primeiros dias de janeiro, que trajes poderia usar senão os mais
leves? E quanto ao tratamento, aqui no Rio usa-se o tu tanto como o você
--- considerou a moça lembrando-se de que observara isto desde a sua
chegada.

--- Aqui há muitas, famílias paulistas --- tornou Helena apertando o
botão da campainha. --- E com o costume de muitos anos do você, não o
deixam assim depressa; como resultado, ouve-se indistintamente a segunda
e a terceira pessoa.

Nesta altura da conversa, o criado abriu a porta e, cumprimentando-as,
pediu-lhes que o acompanhassem. Introduziu-as em uma sala que, pelo
aspecto, parecia ser de prosa. Dois carrinhos de chá, cinzeiros de altos
pés, cigarros e isqueiros. Um rádio e, nos braços-estantes das poltronas
modernas, alguns livros; e, destes, quase todos, impressões de viagens.
Sobre a mesinha de canto, várias revistas. Não tardou que a dona da
casa, a senhora a quem Helena ia visitar, aparecesse na sala. Tinha um
porte altivo; e conquanto apresentasse os cabelos grisalhos, não seriam
estes o indício da sua idade. Rosto largo, comprido, onde seus olhos
apareciam cercados por olheiras profundas. Aquele círculo azulado, em
evidência pelas pestanas curtas, apresentava uma pele rugosa e gasta.
Seus olhos tendiam à cor parda, eram, porém, desse tom castanho quase
indefinível. Uma boca rasgada, terminando sem lábios, figurando apenas
dois cortes na pele. Sua expressão traduzia firmeza, não da mulher que
sabe o que faz, mas, da mulher que se acostumou a mandar. Alta, tendendo
para a gordura da idade, tinha ainda formas esbeltas. Não fora isto e
suas olheiras inchadas, comparar-se-ia à mulher balzaquiana; havia nela,
entretanto, indícios de quem já dobrava os quarenta. Suas maneiras
fidalgas, e ainda estudadas, contavam que d\textsuperscript{a}. Alda
pertencia à classe aristocrata e esforçava-se por permanecer nela.

Helena apresentou Bárbara à d\textsuperscript{a}. Alda Macedo.
Cumprimentaram-se delicadamente e d\textsuperscript{a}. Alda convidou-as
para o terraço, onde estavam, pessoas de suas, relações; pois, aquele
era o dia de receber da senhora Macedo.

--- São meus amigos --- dirigiu-se d\textsuperscript{a}. Alda às duas
moças que a seguiam.

Helena conhecia-os todos; ao vê-los, sorriu-lhes prazenteiramente e
esperou que a dona da casa apresentasse a sua amiga para então
cumprimentá-los. D\textsuperscript{a}. Alda estendendo a mão para
Bárbara, num gesto amável, colocou-a em destaque e, a seguir,
numerou-lhe os presentes:

--- Quero apresentá-la ao Dr.~Martins, ao Paulo e Elvira Cunha, e à
condessa de Serra Azul. Oh! Desculpe... e Alice, a senhora do Dr.
Martins.

--- Bárbara Stuart. Muito prazer --- respondeu a moça, curvando-se
levemente.

Os homens levantaram-se logo para cumprimentar as recém-chegadas; as
senhoras seguiram o seu exemplo, e só a condessa, fazendo uso das
regalias da idade, permaneceu sentada.

--- É amiga de Helena --- contou d\textsuperscript{a}. Alda aos que
estavam ali.

Dr.~Martins cedeu lugar à Bárbara e procurou acomodar-se a seu lado.
Helena sentou-se em frente à amiga, e não tardou que a conversa se
reiniciasse.

--- A senhora, é Stuart? --- indagou o Dr.~Martins. Americana ou
inglesa?

--- Inglesa.

Bárbara olhou para o Dr.~Martins. Era figura saliente pelo seu físico;
extremamente alto, tinha os ombros para trás numa posição
requintadamente aprumada. Os cabelos encanecidos contrastavam com a
fisionomia enérgica que ele aparentava. As sobrancelhas e cílios
conservavam ainda o tom negro primitivo realçando os olhos verdes e
grandes. Os lábios um tanto grossos e arroxeados, completavam aquele
semblante esquisito. Podia ter cinquenta anos. Sua esposa parecia um
modelo vivo da moda, tão bem se trajava. Não era bonita, porém, elegante
e vistosa. Traços e maneiras comuns, verdadeiro contraste ao lado do
marido. Um rosto de feições vulgares; tinha esse ar indiferente que
caracteriza a mulher grã-fina, inacessível pelos seus dons materiais.
Alice era isso mesmo; distanciava-se da criatura fina para ser
notadamente grã-fina. Até mesmo quando fazia perguntas, ouvia as
respostas com o olhar disperso, largado ao acaso.

--- Pelo seu sotaque --- continuou o Dr.~Martins admirado --- eu não
diria que é estrangeira.

--- Estou no Brasil há alguns anos --- informou a moça.

--- E estudou aqui --- acrescentou ainda Helena.

--- E seus pais não quiseram voltar à Inglaterra?

--- Não tenho pais; perdi-os quando criança ­--- respondeu em tom sério.

--- Era muito criança?

--- Com três meses, faleceu mamãe e, aos nove anos, perdi também meu
pai.

Um murmúrio de consternação correu entre os presentes; murmúrio dos que
se penalizam ante a desgraça alheia quando esta não lhes pesa
diretamente.

A condessa de Serra Azul, interessada em saber como uma moça, de hoje,
deixa a pátria e os parentes, perguntou a Bárbara:

--- E como decidiu vir ao Brasil?

A condessa era uma senhora envelhecida que se aproximava dos oitenta
anos; o espírito estava ainda lúcido, mas as feições gastas e sem vida.
Pelo seu rosto espalhavam-se rugas de formas e tamanhos diversos.
Mostrava a gordura das senhoras abastadas que, avançando na idade, se
entregam a uma indolência completa. As principais atribuições da
condessa eram receber e fazer visitas; de quando em quando uma obra
caridosa lhe trazia a promessa de um céu que, embora atemorizasse a
condessa, se aproximava cada vez mais. Bárbara dirigiu-se a ela.

--- Vim a passeio, sra. condessa, e acabei por ficar no Brasil.

--- E não tem parentes na Inglaterra? --- indagou admirada a velha
senhora.

Para a condessa, só existiam os laços do sangue; falar sobre laços de
espírito seria algo insensato e incompreensível. Para a senhora
condessa, como para a maioria das pessoas, a palavra espírito estava
irremediavelmente ligada ao espiritismo. E ser espírita no Brasil, era
fazer parte das macumbas, ter contato com o bode preto e mil coisas mais
que a boa imaginação do povo pudesse inventar.

Bárbara, percebendo, de início, a limitação inconsciente dos seus
interlocutores, limitou conscientemente suas respostas. Assim voltou-se
delicadamente para a condessa:

--- Sim. Ainda tenho parentes na Inglaterra.

--- E não pretende voltar? --- insistiu, com visível admiração.

--- Tenciono viajar ainda, mas, as coisas não parecem bem na Europa ---
considerou a moça normalmente.

Paulo Cunha que ainda não falara a Bárbara, esperava uma oportunidade
para fazê-lo. Era ele o mais moço da roda; tinha trinta e cinco anos e
admirava com intensidade o belo sexo, mesmo quando fosse para uma
simples conversa. Nada o separava do comum dos homens: a estatura média,
mais para gordo, traços vulgares; apenas uma elegância cultivada e uma
arte peculiar em tratar as mulheres. Era fino de maneiras e, mais que
isso... era maneiroso nos seus galanteios. A seu lado, a esposa, uma
figura sem brilho; aparentava ser mais velha que o marido, embora os
cremes e cosméticos lhe trouxessem impecável a ``maquillage''. Não tinha
beleza e nem mostrava traços de quem a tivera; era apenas uma mulher
cuidada, de loiros cabelos oxigenados, penteados com toda a arte, e um
corpo normal, sem gordura e de linhas quase retas. Como estas coisas que
acontecem e que ninguém explicava, Paulo Cunha casara-se com uma mulher
sem encantos. Logo que Bárbara apareceu, não pôde ser indiferente à sua
beleza; todavia, percebeu logo que ela não era dessas conquistas de um
sorriso ou de uma orquídea bem oferecida. Sabia distinguir os terrenos e
era cauteloso no avançar; mas a ideia de trocar umas palavras com aquela
mulher bonita e atraente era-lhe em extremo agradável. Paulo Cunha
procurou, então, desviar a conversa insistente da Condessa. Tornar-se-ia
aprazível à visitante, além do seu desejo de conversar com ela.

--- Agora que se ambientou no Brasil, convém ficar mais um pouco. Os
alemães parecem dispostos à guerra, e quando a Alemanha entra em guerra,
a Europa costuma fazer-lhe frente.

--- Bárbara não pensa em deixar o Brasil --- interrompeu Helena com
veemência. --- E justo agora que conhece bem a nossa língua; afinal de
contas, ela cresceu no Brasil.

--- É verdade --- tornou a amiga sorrindo --- aumentei bem alguns
centímetros.

Todos riram da resposta e Paulo Cunha voltou ansioso:

--- E teve dificuldades para ambientar-se nesta terra que todos dizem
hospitaleira?

--- Naturalmente. Quem não as teria em um país estranho?

--- Sentiu dificuldades para comunicar-se com os brasileiros?

--- Com os brasileiros em geral, não; pois no Brasil estuda-se muito a
língua dos outros. Para mim o mais difícil foi aprender o português.

--- A senhora achou o português, uma língua difícil? --- perguntou
Alice, espantada.

--- Você não sabia Alice? --- disse a condessa entrando novamente na
palestra. --- A nossa língua é a mais difícil do mundo --- afirmou com
uma arrogância ingênua.

--- E por ser uma língua difícil, porque não conservar a sua tradição?
--- avançou o Dr.~Martins. --- Isto ainda vai dar barulho.

A discussão foi num crescendo. Pois se os velhos aprenderam assim,
porque não persistir nos mesmos costumes? Não usavam eles o \emph{ph}, o
\emph{ch} com seus diversos sons? Qual a razão para reduzirem assim os
grupos diacríticos, enfeites preciosos de uma língua? E as consoantes
mudas? E o Y?... Oh, o Y! Como deixar o Y?! Dava aos nomes um quê de
aristocrático; o pessoal do povo não escreve com Y!!! Era a mania
moderna de misturar aristocracia à plebe. Como se não chegassem os
plebeus enriquecidos que compraram títulos e invadiram a alta sociedade.
Gente que ontem recebeu nossas gorjetas e hoje as vem dar ao nosso lado.
Coisas de comunismo. Ah, isso era demais! E o brasileiro não se
levantava para reclamar os seus direitos!

Bárbara ouvia com delicada paciência tudo o que se dizia ali. Helena,
conhecendo-a bem, perguntou a si mesma que estaria a amiga pensando da
inconsciência daquela gente. O resto do pessoal, porém, interpretou o
silêncio de Bárbara como uma falta de argumentos contrários às ideias
expostas, ou como uma concordância implícita ao que se dizia. Certo é o
ditado ``a união faz a força''; frente àqueles cérebros unidos, qual a
broca que perfuraria semelhante massa? Por isso, Bárbara resolvera
calar; todavia, as circunstâncias trouxeram-lhe à mente um texto de
Maupassant --- ``As palavras têm uma alma; a maioria, porém, só lhes
pede um sentido''. E Paulo Cunha, vendo que Bárbara se mantinha afastada
da conversa, dirigiu-se a ela em tom aprazível:

--- Mas afinal, que ia dizer sobre os brasileiros e a língua que se fala
no Brasil?

Os outros também se calaram. Não para ouvir o que uma estrangeira iria
dizer, mas simplesmente porque todos falavam e já ninguém mais ouvia.
Bárbara olhou com naturalidade para todos e respondeu:

--- Achei dificuldades porque os brasileiros que encontrei, e falavam o
inglês, desconheciam o português, a língua que se fala no Brasil.

\chapter{Capítulo 8}

No dia seguinte Bárbara retornou aos seus hábitos. Acordou logo cedo e
foi contemplar a manhã. Fazia um tempo admirável e, como o tempo passa,
era preciso aproveitá-lo. Verificou as horas: sete, ainda. Sentou-se ao
penteador e começou a escovar os cabelos quando se lembrou de convidar
Helena para a praia. Telefonou à amiga e, combinando com ela de
esperá-la no hotel, vestiu-se a toda pressa. No seu traje esportivo,
resolveu tomar o café no salão do hotel; despedindo-se de Mrs.~Patrice,
deixou o quarto. Dirigiu-se ao elevador e quando o menino lhe abriu a
porta, sorriu de modo peculiar. Bárbara cumprimentou-o como de costume
e, enquanto desciam, o garoto contou algumas novidades do hotel. Ela o
ouvia complacentemente; ao deixá-lo, desejou-lhe um bom dia. Foi para o
salão, onde o servente não se fez esperar; todavia, estranhou a maneira
insistente com que os hóspedes olhavam para ela. Estava para terminar
quando a amiga, entrando no salão, aproximou-se da mesa.
Cumprimentaram-se e Helena convidou-a para sair logo.

--- Vamos; tenho uma coisa para você ver --- disse Helena, percebendo
que ela ainda não sabia da última.

--- Vamos; mas, você não quer antes um cafezinho?

--- Obrigada, recusou a outra.

--- Nem mais parece paulista --- comentou Bárbara levantando-se.

Saíram do salão. No corredor, sentado numa cadeira de vime, estava um
senhor lendo comodamente o seu jornal. Bárbara olhou de longe para ver
os cabeçalhos.

--- Oh! --- exclamou sem parar. --- Vi ali uma fotografia que, de
relance, me pareceu familiar.

--- E sem relance também --- volveu Helena, abrindo já para a amiga, a
porta do seu carro.

E antes de ligar o motor, Helena, passou-lhe um jornal:

--- Conhece esta moça? --- perguntou.

--- Oh! E ainda pareceu familiar --- comentou Bárbara examinando a
fotografia. --- Mas, sabe Helena, eu pensei que havia fugido ao
fotógrafo ontem.

--- Não se aborreça por isso; nada há de mais.

E Helena, à direção, deu saída ao carro.

---Então, porque me fotografaram? --- continuou Bárbara

--- Apenas por ter sido o seu automóvel apreendido pela polícia.

--- Nada mais?

--- Está bastante humorístico. Diz que, após várias multas, o carro
esquecido foi dormir na Delegacia.

Bárbara correu os olhos pelo jornal. Viu a sua fotografia ao lado do
carro, e o dístico: ``Uma linda jovem descuida do que é seu''.

--- Não vou dizer que isto me agrada; enfim, parece não haver má
intenção.

--- Aqui no Rio costumam fazer dessas coisas.

--- Não quero dizer que em São Paulo não o façam também; mas, este
excesso de publicidade sem razão alguma\ldots{}

--- Que quer? É falta de assunto.

--- Bem, e com isto parece que chegamos --- disse Helena diminuindo a
marcha do veículo. --- Prefere este posto?

--- Este é o três? Acho que aqui ficaremos mais à vontade.

--- Então desçamos --- concordou Helena --- este é menos concorrido.

Encostaram o carro numa rua transversal à avenida Atlântica e
dirigiram-se à praia. Como traziam trajes apropriados, não lhes foi
difícil, chegando ali, verem-se prontas para nadar.

Não havia muita gente, mas a animação era grande e de todos os lados
ouvia-se o barulho das petecas, bolas e demais folguedos esportivos.
Bárbara sentia-se bem e iria direta à agua se Helena não a detivesse.
Estavam ali alguns conhecidos de Helena e ela insistiu em apresentados à
amiga. Entre eles, aproximou-se um rapaz de vinte anos que Bárbara diria
ser o Robert Taylor se o visse em fotografia. Tinha um físico notável e
seu rosto era de uma beleza de chamar a atenção. Bem moreno, cabelos
pretos, ondulados e rigorosamente em ordem, até mesmo ali na praia. Seus
olhos verdes pareciam sorrir sempre. Usava bigodes e, como os rapazes da
alta roda, era cavalheiro por educação; completava-o essa expressão de
vivacidade, própria dos vinte anos. Mas, nessa vivacidade percebia-se
uma indiferença cultivada, como se, cônscio de seus dotes pessoais, o
rapaz se precavesse constantemente contra os riscos donjuanescos dos
namoros fáceis. Talvez se mostrasse mais indiferente que os
companheiros, porque a sua beleza física atraía o olhar das mulheres em
geral. Ele aproximou-se de Helena com intimidade e esta o apresentou à
Bárbara:

--- É um amigo nosso; filho da d\textsuperscript{a}. Alda, a senhora a
quem visitamos ontem --- informou Helena.

--- Carlos de Macedo --- disse o rapaz estendendo-lhe a mão.

Bárbara retribuiu-lhe o gesto atenciosamente.

--- Na intimidade o chamamos de Carlito --- disse ainda Helena.

--- E terei muito prazer que me trate assim --- completou o rapaz.

Bárbara apenas sorriu. Carlito, impressionado pela beleza da moça,
observou-a disfarçadamente. A plástica era impecável; que formas lindas
se desenhavam sob aquele maiô vermelho! A cintura fina punha em destaque
a perfeição de suas linhas. As mulheres bonitas de rosto, pensou Carlito
consigo mesmo, têm sempre um corpo tão feio; como fora a natureza reunir
assim os dois elementos? Era uma excessiva prodigalidade! E a tudo isso,
ainda um quê inexplicável, elevando-a acima do nível comum. Conquanto
não estivesse à altura da moça, para apreciá-la integralmente, Carlito
sentiu-se atraído por ela. Nesse ínterim, Bárbara notou que outras
pessoas se aproximavam e, julgando que os cumprimentos se prolongariam
indefinidamente, preferiu afastar-se. Pediu licença, e, retirando-se
depressa, atirou-se à agua.

Estava com disposição para nadar e era-lhe agradável o contato com a
natureza. Entrou pelo mar. Satisfeita, entregou-se ao esporte. Nadou
algum tempo; quando o exercício a cansou, virou-se de costas e parada,
permaneceu flutuando. As ondas vinham e levavam-na para o alto; ela se
deixava arrastar naquele balanço perigoso. De repente, pareceu-lhe ouvir
uma voz; voltou-se e reconheceu Carlito, o rapaz que, havia pouco, lhe
fora apresentado.

--- Avançou demais, não tem medo do mar?

--- Não foi demais; nem o posto avisou...

--- Já não ouviu dizer que a praia de Copacabana é traiçoeira?

--- Vou então descansar na areia que é lugar mais seguro --- volveu a
moça.

Dizendo isto, pôs se a nadar outra vez. Ao chegar à terra firme, viu que
Carlito a seguira.

--- Onde aprendeu a nadar? --- perguntou o rapaz admirado.

--- Na Inglaterra.

--- Na Inglaterra?!

--- Por que se admira? Sou inglesa.

--- Inglesa?!!!

Bárbara riu da crescente escala exclamativa do rapaz e objetou:

--- Afinal... há tanto estrangeiro no Brasil.

--- É verdade. Você, porém, é uma inglesa original.

--- Acha? Ainda não me conhece; como pode afirmar?

--- Só o fato de vê-la falando a nossa língua, é bastante estranho para
nós.

--- Satisfaz-me ouvir isto. E, doravante, ninguém mais poderá dizer que
os ingleses não aprendem o português.

Perceberam que alguém os chamava e Bárbara conheceu logo a voz de
Helena. Levantaram-se da praia e, sacudindo a areia do corpo foram-lhe
ao encontro.

--- Onde esteve? --- indagou Helena.

--- Na água. Pois não viemos ao banho? --- volveu a outra com
naturalidade.

--- Escute Bárbara --- informou Helena em tom de reserva --- estão aí
outros rapazes que desejam conhecê-la.

--- Fica para outro dia. Dê uma desculpa qualquer e vamos.

\chapter{Capítulo 9}

Na tarde do mesmo dia, Bárbara e Helena, em uma confeitaria central,
tomavam um sorvete nas mesinhas da calçada. O cair da tarde no Rio tinha
um encanto próprio, e as duas amigas, não obstante os temperamentos
diversos, tinham o gosto quase sempre em comum. Ficaram longo tempo em
silêncio, frente ao contraste da grande cidade--- os transeuntes
apressados e a tarde parada, morrendo aos poucos.

Um cachorro magro, desprezível vira-latas, passou vagarosamente por ali;
farejou em redor, olhou desanimado para as pessoas e, como não lhe
dessem nada, continuou no seu andar preguiçoso, aproximando-se da mesa
das duas amigas. Perto de Bárbara parou e olhou para ela, submisso,
naquela expressão triste, suplicante, dos cães maltratados. Ela tomou
uma bolacha da mesa e deu-a ao animal; ele a abocanhou e engoliu-a
apressadamente. Bárbara deu-lhe outra; e de uma em uma, o cão faminto
devorou as bolachas todas. Vendo a bandeja vazia, Bárbara chamou a
atenção de Helena:

--- Foram-se as suas bolachas...

Helena pareceu não ouvir e Bárbara, observando-a, percebeu que ela
estava longe dali. No seu olhar distante, triste, espelhava-se a
amargura da sua alma torturada. Ela estivera alheia a tudo que a
cercara. Seu pensamento deixara o presente, para reviver, em sonho,
cenas do passado. Bárbara calou-se até que a amiga retornasse do seu
mundo imaginário.

--- Sabe em quem eu pensava? --- perguntou Helena com voz sumida.

--- Sei...

--- Onde estaria agora? --- perguntou Helena vencida pelo desânimo.

--- Nunca mais tive notícias dele.

--- E eu, Bárbara, nunca pude esquecê-lo.

Bárbara contemplou a fisionomia abatida de Helena e continuou:

--- Já há algum tempo que não me falava nele:

--- Falta de oportunidade... nunca, por esquecimento.

Um silêncio triste caiu entre as duas amigas. Helena, ouvindo os seus
sentimentos e Bárbara, solidária à sua dor. Ao fim de longo tempo,
Helena dirigiu-se a ela:

--- Bárbara... você não imagina o que lhe possa ter acontecido?

--- Há seis anos, como você mesma, que eu ignoro qualquer referência
sobre o rapaz.

--- Repito para você Bárbara, se ele não houvesse desaparecido, eu não o
teria deixado. Ah, os tempos do Mackenzie, em que estávamos juntos todos
os dias! Daria a minha vida para voltar ao passado.

--- Mesmo sabendo o futuro?

Relutando, disse com a voz dura:

--- Mesmo assim.

--- Este ano de vida e lugar diferentes não cicatrizaram a sua ferida?
--- indagou Bárbara carinhosamente.

--- Você se lembra como Paulo era poeta? O poeta vive em toda a parte.
Vê este cair de tarde? Parece que eu estou no passado, vendo-o parado na
rua, contemplando o horizonte e dizendo: ``Vê Helena, estes arranha-céus
não deixam livre a beleza do crepúsculo. Ah, quando você estiver comigo
na minha terra!''

Aos quinze anos de idade, Helena cursava ainda o ginásio quando
conhecera Paulo Machado. Seis anos mais velho do que ela, era também
estudante no curso de Engenharia do mesmo estabelecimento. Viam-se
diariamente e aquela convivência continuada, por três anos, fez brotar
entre eles um sentimento que criou raízes profundas. Aos vinte e quatro
anos, Paulo formava-se e Helena, já na Faculdade da Praça tirava um
curso de línguas. Dr.~Luiz Barreto, pai de Helena, soube então do
romance da filha. Ciente desde o início de que a família do rapaz era
simples e desconhecida, tomou providências enérgicas para cortar a
ligação que ele julgava desonrosa. Percebendo que a filha já não o
atenderia, foi diretamente ao rapaz. Assim, o filho mais velho do Dr.
Luiz Barreto procurou Paulo Machado para lembrar que o nível social e
familiar de sua irmã era incompatível com o dele. Em nome do pai e por
si mesmo, o irmão foi inflexível e injusto nas suas atitudes. Terminou
então, bruscamente, um romance que Helena, desde adolescente, povoara de
sonhos e ilusões. Paulo fora até à casa de Bárbara e, despedindo-se
dela, deixou o seu adeus a Helena. --- Diga-lhe que o meu nome foi o
machado que desfechou o golpe no meu destino.

--- Eu me lembro bem de Paulo --- respondeu Bárbara às palavras da
amiga.

--- Será que ele fez alguma diferença?

Bárbara sorriu em atitude complacente.

--- Tem trinta anos agora --- continuou Helena. --- Talvez já tenha
algum cabelo branco... É sim, Bárbara, deve estar envelhecido. Uma
notícia de longe em longe, diz que se afundou no álcool. Que lhe teria
acontecido? --- repetiu, pensativa.

Não adiantava aconselhar Helena que procurasse esquecer Paulo Machado.
Se até agora, seis anos depois, não o fizera, é porque não lhe fora
possível. Aquele romance, cortado de maneira tão injusta, deixara no
espírito da moça uma impressão indissipável e uma desilusão profunda que
muita vez a desorientava. Helena tinha medo da sua própria desilusão; um
temor supersticioso dizia-lhe que ela ainda seria vítima desse
desprendimento pela vida. Hoje, o irmão estava casado numa família mais
modesta que a de Paulo Machado e completamente esquecido de que, por
essa mesma razão, arruinara para sempre a vida de sua irmã mais moça.
Fora difícil para o pai controlar o sentimento do filho; e como o homem
é mais livre, o rapaz casou-se, alheio aos preconceitos de família a que
ele tanto se apegara ao tratar-se da irmã.

--- Bárbara --- insistiu Helena --- como são raros os rapazes como
Paulo. Comparo-o aos outros que conheço; nenhum é como Paulo.

--- Cuidado Helena, o amor deforma a visão.

--- Mas diga a verdade, ele não era diferente?...

--- Era um ótimo rapaz, minha amiga; porém, não o único.

--- Você acha que ele vive ainda?

--- Por que me pergunta isso?

--- Porque tenho quase certeza que sim.

--- Alguém lhe falou a respeito?

--- Não; a não ser quando você está a meu lado, este assunto nasce e
morre comigo.

--- E seus pais?

--- Procedem como se nada houvesse ocorrido. Se Paulo não lhe dissesse
tudo, até hoje eu não saberia.

Bárbara, num gesto de solidariedade, apertou a mão de Helena. Custou-lhe
crer no que ouvia e suas palavras foram o eco de seus pensamentos.

--- Como você está só... Helena!

--- Você a dizer-me isto?!

--- Eu estou só --- disse Bárbara com firmeza --- mas sem ilusões. Você
está só na sua própria família!

Helena sabia, mais do que isso... sentia a verdade das palavras de
Bárbara; contudo, a lembrança dos que viviam com ela despertou o seu
afeto e ela procurou desculpá-los:

--- É um costume brasileiro, Bárbara. Os pais não se aproximam dos
filhos.

--- O carinho, Helena, não é artigo estrangeiro.

--- Quis dizer que esta educação em que se permite maior intimidade
entre pais e filhos, é mais do americano do norte --- concertou Helena.

Bárbara deixou passar as palavras da amiga. Sabia que, no íntimo, ela
reconhecia a verdade; contudo, um pudor tribal, comum ao espírito
feminino, fazia-a defender os membros da família. Bárbara lembrou-se do
seu lar... como fora feliz nele e que trilho aberto para a vida lhe
deixara o pai. Entrara por ele com firmeza e trazia consigo a lembrança
imorredoura do ser que lhe estendera os braços num gesto carinhoso e
seguro. Como era esquisita a vida! À Bárbara pareceu dar tudo para tirar
depois; à Helena, nada tirara... também nada dera. O seu lar era apenas
um abrigo social, sem alento e sem amparo.

\chapter{Capítulo 10}

À noite, Bárbara, no hotel, dirigiu-se ao salão de leitura. Desde que
chegara, não tivera oportunidade para ouvir boa música; naquela
necessidade imperiosa do seu espírito, foi ao rádio que sabia existir
ali. Não estava pessoa alguma no salão; era a hora do cinema ou das
confeitarias. Ligou a primeira estação, veio um samba; experimentou
outra... outro samba. Seguiu o processo das tentativas--- aqui o fox,
ali a marcha e o mais que ela apanhou foi um tango choroso que expirava
nos seus últimos acordes. Procurou uma estação de S. Paulo, quem sabe...
mas o rádio não acusou uma sequer. Desanimada, caminhou vagarosamente
até a porta. Na incerteza do seu itinerário, atravessou a praça e foi
debruçar-se, do outro lado, no paredão que separava a praia da calçada.
Ficou ali, olhando para o mar, desprendida de si e de todos. Depois,
resolveu entrar para o hotel. Sem prestar atenção ao movimento da
cidade, deixou a calçada repentinamente. Um carro surpreendeu-a,
chegando a tocar-lhe de leve. Não fora o volume do veículo e o grito do
pneumático arrastando-se no solo, ela não teria percebido o perigo que a
ameaçara. Quis afastar-se, quando ouviu uma voz conhecida:

--- Que é isso? Não tem medo da morte?

Bárbara aproximou-se e reconheceu Carlito, o rapaz que lhe fora
apresentado pela manhã e que, por curiosa coincidência, lhe fizera esta
mesma pergunta no mar. Procurou desculpar a sua distração; mas Carlito
parecia encantado por lhe ter poupado a vida e mal deixou que ela
dissesse uma palavra.

--- Assustei-a, Bárbara? Não foi por minha culpa; nem de leve eu
pensaria em tal.

--- Não tive tempo para assustar-me; foi tudo tão rápido.

--- Quer dar uma volta? Conhecer o Rio, ofereceu com delicadeza.

--- Num outro dia talvez --- respondeu a moça.

--- E por que não hoje? --- insistiu o rapaz.

--- Obrigada --- disse Bárbara com firmeza.

Ele compreendeu que ela não iria e, irritado pela oportunidade perdida,
procurou esconder o seu desaponto.

--- Espero que outra não lhe aconteça. Nesta esperança, me despeço de
você; boa noite.

--- Boa noite. Carlito.

Bárbara viu-o desaparecer ao longe. Não distinguiu a marca do seu carro;
mas, as linhas modernas contavam que era novo e de alto preço: uma
barata vermelho-escuro, com um filete prateado em toda a volta e de
formas alongadas.

Bárbara retornou ao hotel que começava a animar-se outra vez. Terminara
uma sessão cinematográfica e os hóspedes reuniam-se no salão de leitura
para o cafezinho das dez e meia. Formavam-se diversos grupos e os
assuntos mais variados surgiam. Como não desejasse conversar no momento,
procurou acomodar-se entre dois grupos sem fazer parte de nenhum.
Apanhou uma revista e olhou em volta, observando os hóspedes--- quase
todos, estrangeiros. Tipos e fisionomias diferentes; mas de uma
diferença marcada, ostensiva, de estrangeiros que não se integravam ao
novo meio. Recusavam, e com um desprezo visível, os costumes sociais da
terra que lhes deu abrigo. Ao longe, no outro lado, estava a senhora
loira que falara com Bárbara no elevador. Conversava com alguns senhores
que, pelos traços, pareciam ser judeus. Trazia no dedo os dois grandes
brilhantes que ela usava até no banho de mar.

Bárbara teria passado os olhos em toda a revista, se os apartes
circunvizinhos não lhe chamassem a atenção. No grupo à direita estavam
alguns ingleses conversando; nenhum deles falava em português, mas,
Bárbara, que cultivara a sua língua natal, ouviu toda a conversa. O
assunto era a possibilidade de uma guerra europeia; os pactos de não
agressão entre es nações, a teoria de Bismarck sobre os tratados e a boa
sorte dos que estivessem fora da Europa no momento. Alguém do grupo
lembrara a tendência da política totalitária do Brasil; o assunto,
porém, fora tratado como coisa simples. Pelo modo que conversavam,
parecia que eles mesmos resolveriam o caso no momento oportuno. A
declaração do Estado Novo, no final do ano anterior, fora recebida por
eles como o teria sido por Bismarck. Logo depois os ingleses se
levantaram e alguns dos raros brasileiros ocuparam os mesmos lugares.
Como se fosse o segundo ato de uma comédia, o assunto prosseguiu. O
mesmo ódio à Alemanha, as tendências nazistas do governo brasileiro, o
poderio de Hitler e o perigo que este representava para o mundo. Por que
haveriam os alemães de tentar a conquista do mundo? Já não pertencia à
Inglaterra que o descobrira primeiro? A mania de fazer frente aos outros
era um predicado desprezível. O assunto caiu logo no comércio, e a ideia
de estoque preventivo foi discutida com as minúcias necessárias.

No grupo à esquerda comentavam a fita do cinema; Charles Boyer estava no
apogeu, e o seu olhar melancólico era, por si mesmo, um romance. Os
serventes passavam trazendo o café e Bárbara, enquanto o saboreava,
pensava nas coisas que ouvira. Um hotel era um pequeno mundo: reunia
pessoas, cujos caracteres e interesses eram os mais diversos. Assim,
aquele mundo em miniatura congregava desde os tipos mais estranhos até
os mais vulgares.

\chapter{Capítulo 11}

Depois de um passeio demorado à Cascatinha da Tijuca, cantada nos versos
de Castro Alves, Bárbara e Helena jantavam no salão de refeições do
hotel. Helena lembrou-se de ir ao cassino, aquela noite; Bárbara,
considerando que seria uma atividade diferente das que tivera até então,
aceitou a ideia. Não era frequentadora assídua de cassinos; porém,
conhecia-os, e lá ia distrair-se alguma vez. O servente veio atendê-las
e trouxe para Bárbara o prato de verduras escolhidas; Helena acompanhou
a amiga na refeição vegetariana. Para a sobremesa veio a laranja; era a
fruta que ela mais apreciara no Brasil; e não tardou que preferisse a
laranja brasileira às frutas europeias. Para terminar, não dispensou o
cafezinho; o costume paulista fora-lhe agradável ao paladar e, em breve,
passou a fazer parte dos seus hábitos diários. Depois de terem jantado
como vegetarianas brasileiras, levantaram-se da mesa, bem-dispostas e
prontas para aproveitarem a noite. O servente recuou-lhes a cadeira.
Elas agradeceram e retiraram-se.

Quando passavam pelo corredor, Helena ouviu o seu nome; voltou-se, e
duas meninas loiras -aproximaram-se. Uma delas, irrequieta, viva, foi
contando logo que iriam ao cassino da Urca. Traía um contentamento sem
limites pela noitada que aguardava. A outra não disse coisa alguma.
Helena apresentou-as à amiga; e embora fossem ditos os nomes de cada uma
delas, Bárbara só ouviu pronunciar Macedo.

--- Macedo? --- perguntou, dirigindo-se às meninas. --- Conheci uma
senhora Macedo aqui no Rio; foi-me apresentada por Helena.

--- É a mãe das meninas --- ajuntou Helena sorrindo.

--- O nome de vocês, eu não entendi, disse Bárbara.

--- Eu sou Ivete; e ela chama-se Lia --- continuou a menina que falara
antes.

Eram gêmeas e tinham agora dezesseis anos completos. O rostinho jovem,
de traços imperfeitos, tornava-se interessante pelo contraste dos
cabelos loiros com os olhos escuros. Não tinham beleza, mas estavam
nessa idade em que tudo é graça e encanto. Um corpinho adolescente, já
de formas acentuadas e bonitas, dava-lhes este quê elegante,
característico do porte feminino. Eram magrinhas; os trajes, escolhidos
com fino gosto, assentavam sempre com a graça encantadora da idade.
Bárbara mal as distinguira pelas feições; não fora a exuberância de
Ivete e o silêncio de Lia, ela não teria feito a menor diferença.

--- Você é a amiga de Helena que esteve lá em casa? --- perguntou Ivete
sorridente.

--- Sou eu mesma. Vocês não estavam em casa naquele dia?

--- Tínhamos ido ao colégio retirar o certificado. Concluímos o ginásio
--- explicou a menina.

--- E você também está contente por ir ao cassino? --- perguntou Bárbara
à outra.

--- Não me incomodo muito com festas --- tomou Lia. --- Hoje, porém,
parece que vai ser agradável.

Bárbara reparou-lhe a expressão dos olhos ternos e a meiguice com que a
menina respondera à sua pergunta; Lia era um encanto.

--- Então --- voltou Ivete --- vamos nos encontrar no cassino;
formaremos uma roda grande. Oh! Como eu estou contente. Só mesmo você,
sua boba --- proferiu dirigindo-se à irmã --- é que não sabe apreciar as
coisas boas.

Lia não respondeu e Ivete, pretextando procurar uns parentes de sua mãe
que as acompanhava, afastou-se chamando a irmã.

--- São iguaizinhas --- comparou Bárbara, dirigindo-se a Helena.

--- A diferença está no espírito --- disse Helena, que as conhecia
melhor.

\chapter{Capítulo 12}

O Cassino da Urca estava repleto. Um vai e vem de pessoas, entre a sala
de jogo e a de bebidas. Representava-se ali a alta sociedade; não só da
que ainda tinha dinheiro, como da que já não o tinha, ou da que o
desejava ter algum dia. Ao lado da sala de jogo havia outra, espécie de
refúgio, onde as pessoas agitadas iam acender um cigarro ou descansar do
movimento jogatino. Ali estavam sempre alguns homens, senhoras e mesmo
moças. Sentavam-se nas cômodas poltronas, andavam de um lado para outro,
procurando disfarçar o nervosismo. E na sala adjacente, as mesas do
tapete verde, o barulho das fichas, a loucura do jogo. Em frente,
ficava, o outro salão, onde revistas caprichosamente encenadas e
conhecidos artistas divertiam os frequentadores. Neste mesmo salão havia
dança nos intervalos do programa de variedades, escolhido e organizado
pelo Cassino.

Assim, as senhoras, que prudentemente acompanhavam as filhas, tentavam a
sorte na roleta, bacará, estrada de ferro, etc. enquanto as moças se
entretinham no salão de danças. E as mãos femininas, com anéis
gigantescos e uma aliança discreta, iam puxar nervosamente as fichas,
portadoras da sorte ou da desgraça. Algumas vezes, quando vinham as
fichas, ficavam os anéis; e quando as fichas voltavam, os anéis não
vinham mais.

Na sala contígua à do jogo, algumas senhoras fumavam e conversavam
descuidadamente. A palestra parecia cordial, e ouviam-na todos os que
ali se achavam.

--- Ercília --- perguntava uma delas --- acompanha também a filha?

--- Sim, minha cara Julieta. As filhas de hoje são exigentes, não ficam
em casa como no nosso tempo.

Veio o aparte de uma terceira:

--- É bem verdade; os tempos mudaram muito. E com essas reuniões,
algumas vezes, em traje a rigor\ldots{} olha, é uma correria louca.

--- Gente de hoje não tem sossego, Ercília.

--- É mesmo, Julieta. Por isso não duram muito. Aos trinta anos, estão
aí completamente gastas.

--- Mas, seja como for; ao menos, hoje, a civilização procura
divertimentos para as mães. Calculem agora, como acontecia antes, se
viéssemos aos bailes e ficássemos sentadas, olhando os outros dançarem
--- tornou Julieta.

--- É verdade --- confirmou d.\textsuperscript{a} Ercília. --- A vida
traz compensações. Não acha, Zulmira?

Como não se ouvisse resposta, d.\textsuperscript{a} Ercília insistiu:

--- Está tão quieta, Zulmira. Que lhe aconteceu?

--- Estou procurando ver se perdi muito hoje --- falou
d.\textsuperscript{a} Zulmira pela primeira vez.

--- Oh, eu nunca perco! Tenho uma sorte incrível! Ou ganho ou saio em
casa; o cacife está sempre intacto! --- exclamou Ercília convicta.

--- Nenhuma de nós é tola para arriscar somas elevadas --- tornou
Julieta. --- Temos filhos! Além do exemplo, precisamos pensar no futuro
deles.

--- Então desculpe o meu engano de há pouco --- volveu maliciosamente a
senhora que calculava os seus prejuízos.

--- Que engano, Zulmira?!

--- Eu a vi arriscando uma quantia enorme!...

--- Ah! Era mesmo um engano --- explicou a outra.

--- Pois é, mas, o mal-entendido está no eu saber o valor das fichas.

--- Espere um pouco... há uns dez minutos... sim, eu sei o que foi. Foi
dinheiro do próprio jogo. Acertei três vezes no número quinze e... como
dizem que o bom filho à casa torna, já estou sem um níquel daquela
parada...

--- Vê, Zulmira? Julieta é prudente, não arrisca o dinheiro dos filhos
--- comentou a Ercília.

D.\textsuperscript{a} Zulmira contemplou a amiga com uma ironia educada:

--- Que boa mãe...

\chapter{Capítulo 13}

D.\textsuperscript{a} Julieta olhou indignada para a amiga e, num
trejeito nervoso, contraiu os lábios no canto esquerdo, mordendo-os
incontidamente. Nisso entraram as gêmeas de d.\textsuperscript{a} Alda e
ao vê-las, d.\textsuperscript{a} Julieta, tendo os lábios ainda
enrijecidos num esgar, sorriu com artifício.

--- Mamãe não veio? --- perguntou a eles.

--- Não senhora --- respondeu Lia atenciosamente. --- Mamãe ficou para
esperar o papai que chega ainda esta noite.

--- Que boa esposa! Esta é das antigas --- ponderou com ironia a mesma
senhora.

D.\textsuperscript{a} Ercília deu logo o seu aparte:

--- E verá que o marido não lhe dá valor. Os homens são assim; nunca
apreciam o que realmente têm! Mas se acontece de perderem, aí é um
queixar sem fim.

E virando-se para as meninas:

--- Só agora chegaram? Já é bem tarde.

--- Só agora --- tornou Ivete --- e por pouco que não vínhamos. Mamãe é
tão antiga! Quer que vivamos como viveu há trinta anos atrás. No tempo
dela não havia o cinema lançando a moda no traje e nos costumes ---
explicou a menina revoltada. --- É preciso acompanhar a época; pois, se
até Lavoisier disse que no mundo tudo se transforma...

--- Ele não disse mundo...

--- Mundo ou natureza, dá tudo na mesma. Não seja pedante, Lia, querendo
corrigir o sábio.

Lia calou-se. Pelas palavras da irmã, sabia o que poderia vir ainda:
para evitar uma situação desagradável, deu o fato por terminado. Ivete
sorriu satisfeita; como sempre, vencera.

--- As duas estão muito bonitinhas --- disse d.\textsuperscript{a}
Julieta. --- Trajam se com gosto apurado. Mas, não vão dançar? A minha
Luíza deve lá estar de velho.

Ivete, sagaz, irrequieta, respondeu logo:

--- O Dr.~Martins, que nos acompanha, foi providenciar uma mesa. E ainda
faltam pessoas da nossa turma.

Sem ouvir as palavras de Ivete, d.\textsuperscript{a} Julieta exclamou,
olhando para a porta:

--- Oh, gente! Quem será esta criatura tão fascinante?!

As senhoras ali presentes voltaram-se para ver. Era Bárbara que entrava
no cassino. Os homens olharam admirados para aquela criatura de beleza
singular e de atitude inconfundível. Bárbara estava bela; mas de uma
beleza simples e altiva. Sabia que era bonita, e dispensava o artifício
e a afetação. Trajava um vestido de cetim pérola, enviesado, desenhando
suas formas impecáveis. Um medalhão antigo; um anel de valor, porém,
discreto; e um relógio de pulso em fina incrustação de platina e
brilhantes, joia que pertencera à sua mãe. Completava o seu traje uma
capa de cetim plissado, amarrada ao pescoço, e indo até ao chão.

Ivete contemplou aquela beleza harmoniosa. Sentiu ciúmes e saiu sem
dizer palavra. Os presentes não a notaram mais. Helena trajava um
vestido de gaze estampada, de saia farta e leve. Estava graciosa; e
aparecia ao lado da beleza grave de Bárbara como o símbolo da beleza
mimosa, delicada.

Dr.~Martins surgiu, após algum tempo, avisando que arranjara a mesa. A
esta notícia, o pequenino grupo dirigiu-se para a sala de dança, o
``grill-room'' como se chamava então; uma vez que o brasileiro não
achara, ainda, uma palavra sua para substituir o termo estrangeiro.

\chapter{Capítulo 14}

Estavam todos à mesa quando Carlito chegou ao cassino. No grupo onde se
achavam suas irmãs, viu também Bárbara. Perturbado pela surpresa, o moço
parou indeciso; tinha vontade de aproximar-se, mas, a lembrança da
recusa de Bárbara doeu-lhe no íntimo. Em um rapaz rico, bonito, de
família tradicional, a quem nenhuma outra moça contrariara ainda, a
vaidade tem muita força. Ao ver Bárbara, porém, o desejo de aproximar-se
dela pareceu imperioso. Como era bonita! E que estupefação causaria aos
colegas quando o vissem em companhia de uma mulher assim. Carlito não
resistiu; achegou-se à mesa e convidou a moça para dançar. Ela aceitou
naturalmente e foi com o rapaz para o centro do salão.

--- Não esperava encontrá-la aqui --- começou ele já na contradança.

--- Nem eu mesma --- respondeu a sorrir.

--- Nem eu mesma, o quê? Você fala pela metade.

--- Mas, para bom entendedor, meia palavra basta.

--- Enfim, que é que você não esperava? Encontrar-me?

--- Não.

--- A quem, então? --- insistiu admirado.

--- A mim mesma. Eu não esperava vir.

--- Ah...

Continuaram a dançar, e Bárbara percebeu que Carlito ainda lhe guardava
ressentimento. Por outro lado, notou que não fora sem esforço que viera
dançar com ela; complacente, teve para com o rapaz um gesto de simpatia.

--- Conheci suas irmãs --- contou em tom cordial --- e como se parecem!
Vocês não fazem confusão alguma vez?

--- Não --- respondeu ele. --- É porque ainda não as conhece bem. Verá
depois, como será fácil distingui-las

--- Bem, eu as vi pouco; mas, não acho assim fácil distinguir uma da
outra.

--- Com o tempo... tudo é o tempo --- afirmou o rapaz, procurando olhar
para ela.

Bárbara sorriu. Como a música terminasse, Carlito convidou-a para
conhecer as demais dependências do cassino. A moça aceitou. Ao passarem
pela porta da saída, esbarraram com numerosas pessoas que vinham do
salão de jogo. O rapaz procurou ser delicado protegendo-a dos empurrões;
todavia, uma senhora gorda passou à sua frente e Bárbara, afastando-se,
apoiou-se involuntariamente a Carlito. No íntimo, perturbado por aquele
contato momentâneo, Carlito bendisse o incidente. E quando passaram
todos, eles atravessaram a rua e entraram na sala do outro lado. Ali
encontraram d.\textsuperscript{a} Julieta que, um tanto aflita,
perguntou a Carlito;

--- Você não viu minha filha?

--- Não senhora.

--- Mas, não estava com vocês?

--- Esteve algum tempo. Depois, foi dançar e eu não a vi mais. Mas...
olhe lá, aí vem Luíza --- disse o rapaz indicando a moça.

--- Até que enfim! --- exclamou irritada a senhora. --- Onde foi você
minha filha? Seu pai e eu estamos à sua espera.

--- Fui... fui retocar o batom --- tornou a moça aliviada por encontrar
uma resposta.

--- Devia estar escuro por lá; há batom até pelo queixo.

--- Foi a pressa --- tentou ainda explicar Luíza perturbada. --- Eu
sabia que a senhora esperava por mim.

Carlito e Bárbara presenciaram a cena sem comentários. O gesto normal de
Carlito, ao afastar-se dali, seria o de pilheriar sobre o caso; mas,
diante de Bárbara, a ideia nem lhe atravessou o espírito.

Foram então para a sala de jogo. Curiosos, ``sapos'' na gíria jogadora,
andavam pelo salão ou presenciavam os lances do memento. Permaneciam
indiferentes até que alguém os convidasse a tomar parte no jogo. Logo à
entrada, uma jovem, de seus vinte anos, passou por Bárbara e Carlito;
olhou para o rapaz como a fazer-lhe um convite mudo. Tinha os cílios e
os lábios profusamente pintados, os cabelos loiros penteados à Ginger
Rogers, e usava um vestido justo a lhe exagerar as formas provocantes.
Bárbara pareceu ignorar o incidente, e Carlito, constrangido, observou-a
com atenção; nenhum gesto indiciava que ela tivesse notado a atitude
afoita da moça. Procurou afastar-se logo para evitar aborrecimentos
maiores. Conhecera-a na última estação de águas, na qual ela o fizera
jogar até perder todo o dinheiro que levara. Atuara como ``farol'' do
cassino das termas. Era sua profissão atrair rapazes ricos.

Carlito entranhou-se por entre as mesas que, àquela hora da noite,
tinham grande movimento. O ambiente era pesado! Servia-se café aos
jogadores, e o tinir das louças confundia-se com o barulho das fichas.
Os participantes dos diversos jogos fumavam sem cessar; no gesto de
segurar o cigarro havia uma atividade escoadora para os nervos. Todos
queriam aparentar calma; mas, às respostas cantadas pelos banqueiros, a
agitação das emoções contagiava o ambiente. Uma expressão comum
confundia os jogadores--- olhos atentos, fixos, de uma dureza
impenetrável; lábios contraídos em esgares deformantes; mãos que se
apertavam, se machucavam mutuamente; a cupidez em todos os gestos, em
todos os movimentos. Era o sinal que caracterizava as pessoas--- a
avidez marcada a ferro, selecionando a classe. Uma nuvem densa de fumaça
pairava no ar; através dessa nuvem viam-se as fisionomias daqueles que,
ainda há pouco, eram homens ou mulheres. Na roleta, um velho colocava,
fichas.; indeciso na escolha, fazia trocas de números até o último
segundo. E quando o banqueiro gritava ``feito'', parava de repente com a
mão trêmula sobre as fichas, como para arrancá-las dali se o resultado
fosse adverso. O banqueiro prosseguia: ``feito'', ``preto''...

Bárbara andou pela sala, conhecendo os costumes da casa. Junto à mesa do
bacará, porém, ela se deteve. Carlito permaneceu a seu lado, satisfeito
por acompanhá-la. Ficou ali por algum tempo; seu espírito, despreocupado
com o jogo, prendeu-se a coisas diversas. O separador de fichas,
trabalhando com a mão direita e a esquerda, catava-as da mesa com uma
velocidade incrível. Uma ideia comparativa veio à mente de Bárbara;
então, disse baixinho a Carlito:

--- Você já viu um grupo de aves bicando milho no chão? Assemelha-se a
esse rapaz, tirando as fichas da mesa.

Carlito achou graça na comparação, e ambos continuaram a observar o
jogo. Ao centro da mesa, um moço falava sem cessar; por mais que Bárbara
prestasse atenção, não lhe entendia os dizeres. De quando em quando, ele
pronunciava o termo ``bacará'', mais adiante ouviu perfeitamente
``marquem ponto''. Daí para frente, estas palavras não lhe escaparam
mais; todavia, uma outra frase soava ao seu ouvido sempre na mesma
cadência. Qual seria essa frase tão repetida? A muito custo, Bárbara
distinguiu os sons e finalmente pôde ouvir com nitidez ``ganhou a
banca''. No pano verde da mesa havia riscas amarelas e, em lugares
destacados, viam-se nas mesmas cores as duas palavras que o banqueiro
dizia sempre: ``ponto'' e ``banca''. Por uma coincidência esquisita, as
cores usuais das mesas de jogo eram as mesmas da bandeira do Brasil! Uma
enorme espátula de madeira flexível, ia e vinha à frente dos jogadores,
catando cartas, fichas e dinheiro com uma agilidade de acrobata.

Bárbara ia deixar a mesa do Bacará, quando viu ali a senhora loira do
hotel que falara com ela no elevador. Bastante agitada, assinalava, como
os demais, um cartão branco que não verificava a seguir. Suas mãos
nervosas, trêmulas, jogavam fichas para a mesa. Olhando de frente para
Bárbara, não a reconheceu. A certa hora, verificou as fichas restantes,
pensou algum tempo e jogou-as todas na mesa; indecisa, empurrou-as para
onde estava escrito ``ponto''. Fechou as mãos com força; as unhas
pareciam penetrar-lhe a carne. Nas suas mãos, sobre a mesa, apenas
fulgia um dos brilhantes. Em seu ânimo, percebia-se, lutava a ânsia
unida ao medo de perder. Ela permaneceu de olhos baixos; no seu íntimo
ergueu-se a voz da sua impotência--- porque, pensou desesperadoramente,
hão de ganhar somente alguns? A fortuna dos homens será feita apenas à
custa dos que perdem? Deus do céu, que irá gritar o banqueiro?!

Bárbara, já apta para observar o sistema do jogo, sentia-se ligada a
essa luta e fitava ansiosa o banqueiro. Se disser ``ponto'', a jogadora
estará salva; se, porém, ele disser ``banca'', ela estará perdida; pois
percebera, naquele lance supremo e angustioso, a sorte do brilhante que
já lhe faltava. Indiferente a tudo, maquinai e ágil, o banqueiro gritou
``banca''.

A senhora levantou-se com a fisionomia contraída. Os olhos tinham
estrias vermelhas, e parecia andar sem direção. Era difícil reconhecer
naquele amontoado de nervos, a senhora loira de meigos olhos azuis que
cumprimentava sorrindo os companheiros de hotel. Alguém comentou ao lado
que ela perdera uma soma enorme. Um moço ocupou o seu lugar, e os
jogadores continuaram atentos aos novos lances. Os homens da banca
permaneceram indiferentes, acostumados com a desgraça alheia. Assim como
os coveiros se habituam a enterrar os mortos, aqueles homens se haviam
habituado a enterrar os vivos.

\chapter{Capítulo 15}

A luz entrava em cheio pela janela aberta e ia bater no rosto de
Bárbara, adormecida. Não tardou que ela abrisse os olhos; vendo o dia
claro, levantou-se e foi à janela. O ondular das águas, o reflexo dos
raios de sol, as embarcações longínquas, tudo aquilo, cenas de todos os
dias, dava à Bárbara a ideia de uma paisagem imaginária. Teria sido mais
pródiga a natureza em qualquer outro lugar da terra?

Olhou para as ruas e viu gente. A vida do trabalho começa cedo.
Apoiou-se no peitoril da janela e ficou ali, pensativa. O telefone
chamou e Bárbara estendendo a mão, apanhou-o do gancho. Do outro lado
Helena falou:

--- Alô...

--- Alô, Helena. Desta vez não me venha passar um pito! Pois, se me
recordo bem das coisas, não esqueci o carro na rua.

--- Não, minha amiga --- disse Helena rindo --- desta vez, venho
convidá-la para um almoço fora da cidade.

--- Que ideia agradável! --- comentou Bárbara.

--- Mas, não é minha. O convite veio de Carlito.

--- Para você?

--- Para você... e para mim; foi bem explicado --- declarou Helena com
malícia.

Achando graça naquele convite por tabela, Bárbara concordou. A manhã
estava agradável e ela sentia-se disposta a aproveitar o tempo.

* * *

Carlito passou pela casa de Helena e ambos foram buscar Bárbara no
hotel. Quando ela desceu, na simplicidade dos seus trajes esportivos, o
rapaz, que saíra do carro para que ela entrasse, não se conteve e
exclamou numa admiração sincera:

--- Como está bonita, hoje!

Ela agradeceu com um sorriso e sentou-se no meio, perto de Helena;
Carlito entrou a seguir, dando partida ao carro.

--- Sabe? --- disse a amiga --- Tenho uma novidade para você.

--- Para mim? Que seria?

--- Imagine só, que ontem uma senhora veio procurar-me no hotel.

--- Que senhora? --- indagou Helena.

--- Uma senhora que deseja mudar-se para São Paulo. Viu a minha
procedência no registro e veio propor uma troca de residência.

Helena juntou as mãos numa alegria infantil;

--- Bárbara! Que maravilha!!!

Carlito, perturbado, não soube o que dizer; mas, desejou ardentemente
que a troca se realizasse.

--- Bárbara --- repetia Helena ­--- que sonho para mim! Você aceitou?

\emph{---} Estou bem inclinada --- respondeu. --- Agora só falta ver a
casa.

--- Eu a levaria, se me consentisse esse prazer --- ofereceu o rapaz.

--- Obrigada --- tornou a moça. --- Quem sabe, poderíamos ir ainda hoje?

--- Quer, agora?

--- Pensando em escapar do almoço, hein --- gracejou Bárbara. --- Não,
agora queremos é almoçar, não é Helena?

--- Claro. Você quis ser gentil... --- lembrou Helena a Carlito.

Ele ouviu tudo satisfeito e pela estrada boa, apertou o acelerador. O
carro voava e os três, de bom humor, continuaram conversando, até que a
barata diminuiu a marcha e parou à porta de um restaurante campestre.
Desceram ali e ocuparam uma mesinha ao ar livre. Tudo era simples nessa
casa, mas dizia-se que as refeições eram boas. As mesas e cadeiras de
troncos de árvores, apareciam no seu estilo rústico em agradável
conjunto. Plantas por todos os lados e uma pequenina fonte com uma
estátua sem valor. Enquanto esperavam, passou o jornaleiro.

--- Compre Carlito --- gritou uma voz masculina que passava.

--- Alô... para onde vai? --- disse Carlito voltando-se e reconhecendo
um amigo.

--- Volto para a cidade.

--- E que há no jornal?

--- Um artigo notável: ``A Flor, o Perfume e a Mulher''.

--- Obrigado pela informação --- respondeu adquirindo o jornal.

--- Que diz o artigo? --- indagou Helena.

--- Diz... espere... uma flor predileta, um perfume agradável... em
síntese --- falou, correndo os olhos pelo jornal --- uma flor ou um
perfume determinado nos faz lembrar uma determinada mulher.

--- Por quê? --- insistiu Helena.

--- Porque certas circunstâncias da vida associam tais lembranças.

--- E quem é o articulista?

--- Álvaro Prado. Nunca vi nada desse camarada.

--- Qual a flor? --- insistiu Helena.

--- Heliotrópio; também não conheço.

--- Não conhece o heliotrópio? --- perguntou Bárbara.

--- Assim como não conheço o perfume Heliotrópio.

--- Quem é mesmo o articulista? --- tornou Bárbara curiosa.

--- Álvaro Prado. Conhece?

--- Não, mas é esquisito.

--- O quê?

--- Heliotrópio é a flor e o perfume que prefiro.

--- Então, a mulher é você também. Pelo menos para mim --- falou
baixinho.

\emph{---} Não devo ser.

--- Por quê?

--- A flor, Heliotrópio; uma. O perfume, Heliotrópio; duas. Ora, duas
vezes Heliotrópio e uma só vez a mulher... Duas contra uma; perdi.

--- Isso é jogo de palavras --- disse Carlito --- não pega. Não se trata
de mais, comparativamente e sim, adicionalmente. Por isso eu completo: a
flor, o perfume e... você.

\chapter{Capítulo 16}

Na manhã seguinte, Bárbara, ao sair do elevador, tornou a ver Carlito. O
rapaz dirigiu-se a ela e cumprimentou-a alegremente.

--- Já aqui? --- indagou a moça. --- São ainda nove horas.

--- Não disse ontem que ia, às nove, visitar a casa? Quis ter o prazer
de levá-la.

--- Disse isso?

--- Disse, quando voltávamos do almoço.

--- Que boa memória você tem, Carlito. Agora vou pedir o meu carro;
espere um pouco, sim.

--- Não é preciso; tenho o meu à porta do hotel.

--- Quanta gentileza!

--- Preferiria que falasse de outra forma. Enfim... seja, por enquanto,

Bárbara pareceu não ouvir bem todas as palavras. Voltando-se para
Carlito, contou-lhe que ainda estava no jejum da manhã. Convidou-o a
acompanhá-la no café. O rapaz que também saíra em jejum para não chegar
atrasado, aceitou o convite. No refeitório, Bárbara e Carlito foram
atendidos prontamente. Com simplicidade ela se servia, e o moço não se
sentiu constrangido na sua presença. As maneiras finas de Bárbara, seus
gestos delicados e naturais prenderam a atenção do rapaz. Notou que
tratara bem o servente, que o cumprimentara solícita, e que tudo isto
era espontâneo no seu temperamento. Terminada a ligeira refeição,
deixaram o hotel. No carro de Carlito, tomaram o rumo de Santa Tereza.

Sta. Tereza é um bairro nas montanhas. Edificado nas alturas, tem
encanto poético, beleza quase selvagem, onde o morador toma contato com
a natureza, sem perder o da civilização. Árvores frondosas se erguem
soberbas em plena mata; casas ainda distanciadas, como se o problema do
espaço não existisse ali. Veem-se quarteirões inteiros, não edificados;
e ao lado de casas confortáveis, casebres de tabique da mais acentuada
pobreza. Clima agradável, tido mesmo como o mais agradável de toda a
cidade. A vista panorâmica, de vasto descortino--- veem-se as avenidas,
os arranha-céus, o Cristo do Corcovado e a entrada da baía. A cidade
inteira estendia-se diante de Santa Tereza. De longe em longe, uma
palmeira solitária, gigantesca, das que no Rio existem tantas.

--- Como deve ser agradável morar aqui --- disse Bárbara contemplando a
beleza natural da cidade maravilhosa.

--- Decidiu então? --- perguntou alegremente o rapaz.

--- Não sei, ainda.

--- Que falta?

--- Ver a casa; pensar um pouco mais.

--- As mulheres não pensam, imaginam.

--- Acha que sou assim? --- perguntou sorrindo.

--- Que situação difícil, estou realmente embaraçado.

--- Os homens falam demais --- tornou Bárbara em tom levemente irônico.

Carlito andava devagar. Na marcha lenta do carro, os dois conversavam e
iam conhecendo o bairro. Ante a censura de Bárbara, voltou-se para ela,
procurando olhar nos seus olhos e disse:

--- Prometo ser mais cauteloso de outra vez.

--- Bem, uma promessa já é algo em direção determinada.

--- E por falar nisso, qual é a nossa direção? --- perguntou o rapaz.

--- Você é que sabe. Confiando no seu conhecimento, nada indaguei a
respeito.

Carlito sorriu e procurou então o lugar desejado. Tomaram diversas ruas,
mas, andavam vagarosamente, detendo-se, em pontos mais abertos nos
montes, para descortinar a cidade. Bárbara examinava atentamente o
bairro. À certa altura, ouviu a voz de Carlito, avisando que já estavam
na rua indicada.

--- 148 --- disse ele --- estamos chegando.

--- 150, 152; é aqui --- disse a moça.

Estacionaram o carro e Bárbara pediu licença a Carlito para examinar a
casa. Enquanto isso, ele foi até a esquina. Comprou cigarros, olhou bem
a rua, localizou a residência. Um homem passou com um enorme cesto de
flores. O rapaz comprou algumas e desejou, de todo o coração, que a casa
agradasse a Bárbara. Nisto ouviu a buzina do carro e voltou
precipitadamente.

--- Pensei que ia demorar mais --- disse ele, desculpando a sua
ausência.

--- Não foi preciso.

--- Agradou? --- indagou Carlito curioso.

--- Muito.

--- Negócio certo?

--- Parece.

--- Viva! --- exclamou o rapaz.

Bárbara entrou no carro e esperou que ele fizesse o mesmo. Depois de
fechar a porta do lado da moça, o rapaz ofereceu-lhe o pequenino
ramalhete que trouxera:

--- São para você; gosta de violetas?

--- Gosto muito; obrigada.

--- São as suas flores prediletas? --- perguntou ainda.

--- As prediletas, não; mas aprecio as violetas.

--- E quais são as prediletas?

--- Para quê?

--- Eu insisto, Bárbara.

--- Ora Carlito; estou contente com as violetas.

--- Uma vez que não quer dizer, procurarei adivinhar. As violetas estão
próximas?

--- Sim. Está esquentando, como diria no jogo de esconde-esconde.

--- Já sei; deve ser o amor-perfeito.

--- Esfriou de uma vez --- disse a moça.

--- Por quê?

--- Não gosto de amor-perfeito; tem cores muito fortes, desagradáveis
desde a primeira vista. Além disso, tem um nome incoerente.

--- Acha? --- indagou ele admirado.

--- Acho --- respondeu.

--- Como deve ser o amor-perfeito?

--- Suave, delicado e sempre agradável à vista, E quanto a essa flor, eu
a chamaria de paixão-perfeita.

--- Você tem cada ideia. Fazer diferença entre amor e paixão...

--- Talvez seja uma diferença pessoal, tentando justificar o impulso
onde ainda não há o amor.

--- Pois, para mim, a paixão é o mesmo amor em última escala.

--- Mas, se pensar bem --- voltou a dizer Bárbara --- verá que pode
existir paixão sem ter existido o amor.

--- Confesso que me deixa desnorteado. Chega; não quero discutir com
você. Você estudou filosofia.

Bárbara achou interessante a despreocupação de conhecimento por parte de
Carlito, quando os rapazes da época, em geral, queriam passar por
intelectuais; e encerrou ali toda a série de pensamentos que se formara
no seu cérebro. Ele não se deu ao trabalho de tentar impressioná-la; foi
dizendo logo o que pensava:

--- Sabe de uma coisa, Bárbara; sou bastante avesso ao estudo. Estou na
faculdade por um hábito brasileiro.

A moça riu da confissão e ele voltou a perguntar:

--- E a flor, qual é a sua predileta?

--- Heliotrópio, respondeu ela.

--- Que distraído sou! --- comentou num sacudir de cabeça. --- Pois já
me disse isso uma vez.

--- Quando?

--- Por ocasião daquele artigo

--- Que artigo, Carlito?

--- Aquele de que toda a gente fala--- A Flor, o Perfume e a Mulher.

\chapter{Capítulo 17}

O avião levantou voo no aeroporto Santos Dumont, e, em breve, a cidade
diminuiu sob as vistas dos passageiros. Bárbara olhou pelo vidro: lá em
baixo o Rio de Janeiro, onde deveria morar dentro de alguns dias. As
águas represavam-se entre línguas de terra e formavam um cenário
esquisito a se contorcerem entre o casario e o mar. A manhã estava
límpida e, lá das alturas, o descortino alcançava grande distância. A
linda baía ficara para trás e do lugar que Bárbara ocupava no avião, não
pudera ver, em conjunto, o espelho líquido das formas geográficas do
Brasil. --- Voltou-se para Mrs.~Patrice que a acompanhava e contou-lhe
que lera, havia tempo, alguma coisa sobre a baía. Jean de Léry, o
francês que acompanhara Villegaignon, escrevendo sobre as novas terras,
falara ``nesse braço de mar e rio de água salgada chamada Ganabara pelos
selvagens e Geneure pelos portugueses''. Os portugueses não designavam
os lugares por data; entretanto, pelo que se diz, o nome viria da sua
descoberta em 1\textsuperscript{o} de janeiro. A invocação costumeira
seria a do santo do dia; mas o historiador referia-se à data como um dia
sem santo, e aquela faixa líquida dera primeiramente a impressão de um
rio; daí o nome de Rio de Janeiro. Os indígenas chamavam a baía de
Guanapará; Léry registrando foneticamente a língua desconhecida,
escreveu ``Ganabara''. E quando os portugueses o leram, não pronunciaram
como o francês; o galicismo gráfico levou-os ao erro prosódico, e surgiu
então a baía da Guanabara. E não há importância em ter ficado sem o nome
de um santo, concluiu Bárbara; há uma outra baía, na qual todos foram
incluídos--- Baía de todos os Santos.

O aparelho voava na mesma estabilidade; alguns minutos depois, passava
sobre matas intermináveis. Uma extensão grande das terras brasileiras
ainda esperava a mão do trabalhador; em qualquer ponto do país havia
terreno sem cultivo.

A viagem não era longa. Em hora e meia de voo, os sinais de São Paulo
apareceram inconfundíveis.

Acompanhada pela governanta amiga, Bárbara desceu na capital. Teve a
sensação de quem volta para a casa--- tudo lhe era familiar. Passando
pelo centro, atravessou o viaduto. A construção pesada, de linhas
sóbrias, dava ao visitante uma ideia do espírito paulista. As ruas
regurgitavam; apressados, homens e mulheres, seguiam em direções
diversas. Bárbara teve a impressão de que corriam atrás do tempo. A luta
pela vida na cidade de São Paulo era uma tarefa pesada. Ao barulho dos
ônibus e à buzina dos carros juntavam-se os gritos dos jornaleiros; cada
pessoa parecia viver mais de uma vida. Bárbara mandou parar o táxi e
explicou a Mrs.~Patrice que iria depois. Desceu ali na praça do
Patriarca e deixou que a governanta chegasse antes à sua casa.
Contemplou o edifício Matarazzo, cujas linhas ela apreciava. Situado num
extremo do viaduto, o edifício tirara uma vista linda da cidade, mas,
como a civilização vê mais o útil que o belo, naquele espaço aberto para
os horizontes, levantou-se um gigante da arquitetura.

Às portas do mesmo edifício, encostados às suas paredes fortes, alguns
mendigos imploravam a caridade dos transeuntes. Aos pés de tanta
suntuosidade, farrapos humanos escolheram um lugar. Bárbara
contemplou-os com o coração apertado. Ela não dava esmolas nas ruas; era
sócia de inúmeras instituições de benemerência, para não se tornar
cúmplice de vidas inúteis, parasitárias na mendicância. Naquele dia,
porém, em que chegava à sua terra, um sentimento mais humano gritava no
seu íntimo. O mendigo à sua esquerda tossiu nervosamente; sua aparência
de debilitado era o atestado vivo da invalidez. Ela abriu a bolsa
àqueles infelizes e continuando o seu caminho, pensou--- Poderíamos nós,
pessoas de saúde e de recursos financeiros, condenar esta gente por não
terem feito o melhor na vida? Quais as nossas circunstâncias e quais as
deles? Há alguém dentre os homens que possa responder a tal pergunta?

\chapter{Capítulo 18}

Eram cinco e meia da tarde.

O apito rouco do vapor avisou os passageiros que ia levantar ferros.
Todos se reuniram a bordo, e, em alguns minutos, a pequenina cidade
flutuante afastava-se do cais. Primeiro mês de mil novecentos e trinta e
oito. Bárbara ia começar a vida em lugar diferente. Enquanto o navio se
distanciava, permaneceu no tombadilho, olhando as casas que desapareciam
aos poucos. Por coincidência, ela saía de São Paulo, por mar, como
outrora da Inglaterra. Fora feliz na terra paulista e deixava atrás de
si, recordações alegres de um passado que estabelecera a sua vida.
Estendeu o olhar uma vez mais à cidade de Santos, a última do Estado.

Adeus, São Paulo!

O navio apitou de novo.

Em uma semana apenas, tudo ficara resolvido. Contando sempre com a fiel
governanta, Bárbara dispôs as coisas da melhor forma possível; despachou
os móveis, fez as suas despedidas sociais, pôs a casa à disposição dos
amigos íntimos e, concluídas todas as coisas, embarcou no cais de Santos
com destino ao Rio. Mrs.~Patrice seguira antes; fora receber os
despachos já na casa alugada em Santa Tereza. Rememorando estas coisas,
ela ficara ali até que a cidade desaparecesse sob suas vistas; depois,
desceu à cabina. Conferiu as bagagens e, recostando-se à cama, procurou
descansar. Sentiu uma inquietação vaga; esteve a ponto de arrepender-se
do que fizera. Bárbara, porém, contava consigo mesma; sabia que se a
vida em outro lugar lhe fosse adversa, voltaria, ou tomaria as
providências possíveis. Para ela, firmeza de atitude não consistia em
sustentar uma coisa errada; se o caminho não fosse aquele, seguiria em
busca de outro, estranhassem ou não os que a vissem voltar atrás.
Seguia, pois, confiante, mas, no coração ficara uma saudade aflitiva da
cidade onde crescera, estudara e sentira-se feliz. Nestes pensamentos,
adormeceu.

O vapor continuava a sua marcha lenta; para trás, o tremular das águas
partidas, sinal momentâneo da sua passagem. Os viajores já não
calculavam a distância que passara; somente o tempo vencido lhes dava a
ideia do que restava para vencer.

Bárbara abriu os olhos; era já noite fechada. Revistou tudo o que a
cercava, e tudo estava como quando adormecera. Abriu a vigia e espiou.
Fazia uma noite linda e o silêncio convidava a apreciá-la. Vestiu calças
compridas, uma blusinha leve, passou ligeiramente o pente nos cabelos
escuros e subiu ao convés. Estava deserto! Um ou outro passageiro
aparecia de vez em quando; alguns dormiam e outros estavam pelos salões
ouvindo rádio ou jogando cartas.

Bárbara procurou uma espreguiçadeira, acomodando-se, ficou olhando o
horizonte longínquo que aos olhos humanos parecia, às vezes, tão perto.
Lembrou-se de quando era pequenina e dissera, certa vez, ao pai--- Vamos
tomar uma estrada bem grande, que dê para chegar lá onde o céu se
encontra com a terra. Pensara sempre que pelo horizonte se subiria ao
céu; o caminhar era fatigante, mas uma vez atingida a linha de encontro
era só subir. Bárbara não se enganara. Pela terra se caminharia
indefinidamente; sofrendo o cansaço e a fadiga, atormentado pela dor,
até que se atingisse esse ponto último que, na sua infância, era o
horizonte.

Quando seu pai falecera, construíra na sua própria imaginação um quadro
nítido, onde ele ascendia ao céu pelo horizonte que vira com ela tantas
vezes. Bárbara pensava sempre na vida! Julgava-se feliz, apesar dos
transes por que passara. Dominara a situação com o tempo, e a
sensibilidade para as pequeninas coisas desenvolveram nela um gosto,
quase poético, pelo viver. Bárbara gostava da vida!

Olhando, porém em torno de si, percebia a desproporção tremenda do
sofrimento. Indagava para si mesma, de todas as formas, as razões
plausíveis para tanta dor. Uma confiança intuitiva, axiomática,
respondia-lhe sempre que elas existiam; e que o fato de os homens não as
terem encontrado, ainda, não era motivo para se deixarem arrastar pela
correnteza das paixões. A vida solitária que levara fortalecera nela os
impulsos primitivos; índices acentuados de um caráter nato e de uma
individualidade moral. Tinha gênio alegre, estava sempre disposta ao
humor; mas sabia deter-se para pensar. Vivendo só, aprendera a controlar
seus sentimentos; e aquela atitude de reserva, que mantinha fora da
intimidade, parecia um atributo integrante da sua natureza. Conhecia os
seus defeitos, e pensava neles, sem ilusão, enfim... como, em geral, não
pensaria uma mulher.

\chapter{Capítulo 19}

De quando em quando, um marinheiro passava pelo tombadilho; depois, tudo
recaía na solidão.

Bárbara espreguiçou-se, combatendo o entorpecimento de músculos a que
uma noite parada convida tanto. Pousou a mão na cadeira próxima e o
contato com uma superfície lisa fê-la voltar-se. Viu sobre o assento uma
revista e, apanhando-a, pôs-se a folheá-la com interesse. A luz fraca e
distante dificultava-lhe a visão; ela, porém, examinava o que era
possível, sem intenção de mover-se dali. Os títulos, em grandes letras,
noticiavam atividades paulistas--- Vistas de São Paulo---
Empreendimentos Científicos de São Paulo--- A literatura em São Paulo...
Bárbara deteve-se. Uma fotografia, que tomava página inteira,
prendeu-lhe a atenção. Aquele olhar próximo, ligeiramente zombeteiro;
aqueles lábios finos, com uma leve contração no canto esquerdo...
traziam-lhe recordações vagas. --- Conhecia aquele moço, mas de onde? E
qual era o seu nome? Desviando o olhar daquela expressão excêntrica, viu
ao canto uma pequenina notícia. Deveria ser algo com relação ao
fotografado; e, numa ansiedade crescente, esforçou-se por decifrar as
palavras que lhe serviriam de informação. Mas a luz era escassa, e os
tipos quase invisíveis pelo tamanho. Bárbara levou a revista para ambos
os lados; virou-a de todos os jeitos e não conseguiu ler. Uma réstea de
luz vinha do salão mais próximo e refletia-se nos seus joelhos; vendo-a,
quis aproveitá-la, e colocou sob os raios luminosos a revista que
excitara a sua curiosidade. Curvou-se sobre a fotografia, tentando a
leitura, quando uma chama avermelhada fê-la voltar-se repentinamente---
a seu lado acenderam um isqueiro. Quase ao mesmo tempo, ouviu uma voz
masculina:

--- Assustei-a? Desculpe-me; eu a vi em dificuldade e quis auxiliá-la.

Bárbara não conseguiu identificar quem lhe falava; mas aquela voz tinha
um timbre peculiar, e a memória acusava tê-la ouvido antes. Constrangida
pela situação incerta em que se achava, tomou cautelosamente a chama e,
pela sua luz, leu a notícia:

\emph{O escritor Álvaro Prado desperta a atenção do público com o seu
recente artigo}--- \emph{``A Flor, o Perfume e a Mulher''.}

Bárbara voltou-se para devolver o isqueiro ainda aceso. O rapaz, que se
afastara, aproximou-se então e, curvando-se, segurou-o cuidadosamente. A
luz da pequenina chama bateu-lhe em cheio no rosto--- era o mesmo da
fotografia.

Bárbara agradeceu e calou-se. O rapaz foi até a cadeira próxima e tirou
dali um jornal.

--- Esta revista deve ser sua. Estava aí também --- disse,
entregando-lhe.

--- Terei imenso prazer em que a veja --- ofereceu o rapaz.

--- Já a folhei toda, obrigada.

--- Folheou somente --- continuou ele --- mas se quiser ler, há aí bons
artigos.

--- Sobre literatura? --- perguntou Bárbara.

--- E sobre música também. Nas últimas páginas há um belo trabalho sobre
Beethoven.

--- Beethoven?! Eu gostaria de ler.

--- Está às suas ordens. Recomendo-o porque Romain Rolland é quem
escreve, informou o rapaz.

--- Ah, sim! Aliás --- continuou a moça --- eu conheço as biografias de
Beethoven, por ele.

--- Então já conhece a força de Rolland. E a tradução é boa, sente-se
nas suas palavras, o homem indefinível. É um biógrafo de quem não se
pode dizer se escreve como poeta ou como músico --- ponderou o rapaz
como a pensar para si mesmo.

Bárbara ouviu-o com interesse; depois tomou a revista e, pedindo
licença, levantou-se. As palavras do escritor repetiram-se na sua mente:
``como músico ou como poeta?''

--- Bem --- pensou a moça --- e haveria mais belo poema que a própria
música?

\chapter{Capítulo 20}

Bárbara sentou-se perto do rádio, na sala de leitura. Abriu a revista e
correu os olhos pelo ensaio biográfico. Estava já nas últimas linhas
quando viu, sentado em uma poltrona próxima, o rapaz que desse ensaio
lhe falara. Fumando, percorria as notícias do jornal. Como se achava de
costas, não podia a moça identificá-lo.

Quando fechou a revista, ouviu novamente a mesma voz:

--- Gostou?

Muito --- disse Bárbara fitando-o curiosamente. --- Mas, o senhor não
é...?

--- O desconhecido daquela noite --- completou o rapaz.

--- Até que enfim, encontrei a chave do enigma.

--- Espero que não tenha má sorte por isso --- gracejou ele.

--- O enigma não foi proposto por uma esfinge --- tornou Bárbara em
resposta à alusão.

Ele apenas sorriu, e fez-lhe depois um convite:

--- Vamos andar um pouco?

--- Com muito prazer --- acedeu a moça.

--- Então, posso convidá-la para ver céu e mar?

--- Naturalmente --- retorquiu Bárbara levantando-se e caminhando com
ele.

Andaram pelo tombadilho, até que o rapaz, encostando-se à grade,
exclamou entusiasmado:

--- Que noite linda! Quem fez isto é realmente um poeta!

--- O senhor sabe quem a fez?

--- Chame pelo nome. Álvaro Prado, às suas ordens.

--- Álvaro Prado --- repetiu ela.

--- Sim, E posso chamá-la por Bárbara?

--- Como sabe que eu me chamo Bárbara?

--- Sempre sei aquilo que me interessa.

Ela desviou-se do assunto, voltando ao precedente.

--- Repito a pergunta anterior. Sabe quem fez a noite?

--- Não. E você sabe?

--- Sei.

--- Quem contou?

--- Sei intuitivamente.

--- E que lhe diz a intuição? Algum Deus ...

--- Quem mais acha que poderia ser?

--- Não sei. Não estou afirmando nada; não cabe a mim provar o que é
verdade para os outros.

--- E neste ponto, o que é verdade para mim, não o posso provar; pois é
um conhecimento intuitivo. Não obstante seja esta a mais pura forma de
conhecimento, pela maior das ironias, é também a prova mais completa da
fraqueza humana. Sabemos porque sentimos, mas, porque sentimos, não o
sabemos.

--- Respondendo a problemas dessa natureza os homens têm dito tanto... e
nesse tanto, tão pouco de real e incontestável. Bárbara, eu continuo
sendo ateu.

Ele a chamara por Bárbara. No íntimo, ela sentiu uma sensação agradável;
esse quê intraduzível de quando alguém nos chama com maior intimidade
pela primeira vez. O rapaz, que parecia pensativo, voltou-se de repente:

--- É isto mesmo --- concluiu --- não creio em nada. Ateu e cético.

--- Em nada?

--- Nada --- confirmou ele.

--- Nem na sua própria descrença?

--- Que pergunta embaraçosa! Não sei responder.

--- Mas pode responder um dia.

--- Esforçar-me-ei --- respondeu.

--- E os seus livros, como vão? --- perguntou Bárbara.

--- Inacabados. São irmãos gêmeos da sinfonia.

E notando que ela se surpreendera, indagou naturalmente :

--- De que se admira?

--- A revista não o diz escritor? Ou terei visto errado?

--- Não errou. Pessoas com a sua acuidade, raramente cometem erros ---
tornou o moço observando a expressão de Bárbara.

Ela sacudiu a cabeça:

--- Pobres dos meus erros!

--- Por quê?

--- Serão levados em grande conta, quando afinal são como os de toda
gente.

\chapter{Capítulo 21}

Na manhã seguinte, os passageiros, mais acostumados ao mar, deixaram as
suas cabinas. Bárbara dormira algumas horas apenas, mas sentia-se bem.
Nos mesmos trajes da véspera, preparou-se para sair. Abriu a maleta azul
e procurou um livro. Leu o título do que estava por cima e sorriu:
Beethoven-Romain Rolland. Tirou-o dali e saiu com ele na mão. Subiu ao
convés. Sentou-se comodamente, e abriu o livro. Seus olhos tocaram um
trecho assinalado: ``Artista soberano no reino de sua personalidade; com
o que ama e com o que o aborrece, com as alegrias e dores, cria um
universo próprio. É Beethoven, é um homem!'' Ela ergueu os olhos da
leitura; Beethoven empolgava-a pela sua música e pela sua força. ``Kraft
über alles''--- como dizia ele próprio. Ia prosseguir quando viu no
convés a senhora loira que estivera no seu hotel e no cassino. Estava
distraída, olhando algo ao longe... Tinha um ar cansado; parecia ter
envelhecido muitos anos. Involuntariamente Bárbara olhou para as suas
mãos; viu-as estendidas, largadas sobre o colo, como se estivessem
cansadas de uma luta intensa. Na mão esquerda, uma aliança de ouro
aparecia solitária. Bárbara teve um estremecimento; onde estariam os
seus dois anéis? Eram duas pedras iguais, brilhantes de família que
pertenceram a sua bisavó; assim, ela lhe dissera, um dia, ao referir-se
às valiosas joias.

--- Que é que prende tanto a sua atenção? --- perguntaram do lado
oposto.

Bárbara voltou-se e reconheceu o escritor.

--- Estava tão absorta... alguma coisa séria? --- perguntou ele.

--- Apenas suposição.

Percebendo que ela não desejava falar do que vira, perguntou-lhe:

--- Como passou de ontem?

--- Bem. E você?

Ele sacudiu os ombros como a dizer que estava na mesma. Depois, pediu
licença a Bárbara e sentou-se na cadeira ao lado.

--- Prefere ler? Ou gostaria de ir ver o mar?

--- Estava apenas relendo uns trechos --- explicou sem mostrar o livro.
--- Mas, na verdade eu preferiria ver um tubarão --- gracejou a moça ---
ainda não tive tal oportunidade.

--- Não fale isso; eu seria capaz de ir buscá-lo.

--- Por acaso, será você algum Netuno desconhecido?

--- Quem sabe?

--- Então, quero ir ao palácio marinho.

--- É uma viagem incômoda; não se pode trazer nada de lá.

--- É verdade; talvez não haja carregador...

--- E desejaria você --- acentuou o rapaz --- trazer de lá alguma coisa?

--- Não sei --- respondeu simplesmente. --- E você?

--- Toda a riqueza que encontrasse.

--- Ambicioso... então?

Ele não respondeu. Tirou um cigarro do bolso, ofereceu a Bárbara que o
recusou; acendeu-o lentamente e deu uma baforada.

--- Gosto de olhar a fumaça --- falou novamente. --- É como a vida; não
se sabe ao certo para onde vai. Você não fuma?

Ela respondeu negativamente com a cabeça.

--- O cigarro é um lenitivo --- insistiu o rapaz.

--- Mas, um lenitivo momentâneo.

--- De momento em momento forma-se uma existência.

--- Fumando?. . .

Ele tirou outra baforada e perguntou sem olhar para a moça:

--- Desaprova a fraqueza dos fumantes?

--- Absolutamente. Que direitos me arrogaria para fazer tal
desaprovação?

--- Simplesmente por não fumar também.

--- O fumo, para mim, não passa de um vício ou de um hábito que eu não
tenho.

--- Ninguém tem, até que um dia...

--- Pois enquanto não chegar este dia... --- interrompeu Bárbara no
mesmo tom.

--- Que coisa estranha! --- exclamou ele admirado. --- Note bem,
Bárbara, que os homens, e mais ainda as mulheres, não perdoam no próximo
um vício que não têm em si próprios.

--- Esquecem-se de considerar os equivalentes --- ponderou ela com
simplicidade.

Álvaro não escondeu a sua admiração; e a curiosidade em torno daquela
personalidade invulgar cresceu no seu íntimo. Observando-a, não venceu o
desejo de continuar a fazer perguntas:

--- Como vê você os homens?

--- Com defeitos e qualidades. Aos outros também é dado o ser
imperfeito.

--- Mas estes outros não pensam da mesma forma.

--- Paciência. Eu penso por mim.

O rapaz ouviu a sua resposta e permaneceu calado; pensativo, olhava para
o espaço como se nele visse um ponto fixo. Vagarosamente, diminuiu a
extensão do olhar, contemplou o cigarro aceso entre os dedos e
dirigiu-se novamente à moça:

--- Para onde vai?

--- Para o Rio.

--- De onde vem?

--- De São Paulo.

--- As nossas coincidências, um dia tornar-se-ão notáveis. Também venho
de São Paulo e sigo para o Rio.

--- Veio para dar outro passeio no arrabalde silencioso?

--- Não; eu moro no Rio. Fui a São Paulo substituir um amigo num jornal.
Ele já regressou, e eu também estou de volta. Dizem que o bom filho à
casa torna.

--- Foi aí que publicou o seu famoso artigo?

--- Famoso? --- e sorriu com ironia. --- Ele apenas tornou famoso por
causa do assunto; a mulheres leram porque lhes era agradável estar em
destaque. Os homens leram porque se tratava das mulheres. Às ideias,
ninguém se prendeu.

--- Álvaro, se não o contrariasse, eu pediria para ler os seus artigos.

--- Não os apreciaria. Não são para você.

--- Por quê?!

--- Vive muito de si mesma. É diferente de todas.

--- Não me julgue assim. Sou uma mulher como as outras e nada mais ---
replicou com enfado.

--- Até nisso é diferente. As mulheres não querem ser iguais; lutam
sempre pela supremacia.

--- Que entende você de mulher?

--- Entender de mulher?... Oscar Wilde dizia ser isto uma coisa
impossível.

--- Impossível, por quê?

--- Porque a mulher não pode ser entendida.

--- Permitem-me discordar?

--- Naturalmente, se apresentar razões. Isto por mim; quanto a Wilde,
não sei como poderia dar tal permissão.

--- Quem afirmaria que o homem pode compreender a mulher?

--- Quer dizer que a deficiência está na compreensão e não na mulher?
Delicada maneira de atacar o homem.

--- Não estou denunciando deficiência em uma das partes, porém,
lembrando que tanto pode estar aqui como ali. Isto em tese; pois, nos
fatos particulares a coisa é bem diferente.

--- Você me empurrou contra a parede --- disse Álvaro brincando. --- Não
quis pôr em teorema os sentimentos humanos. E penso que a intenção de
Wilde também não foi tão longe, ele apenas disse que a mulher é para ser
amada e não compreendida.

\chapter{Capítulo 22}

O vapor anunciava a sua chegada ao Rio. Os sinais de terra fizeram-se
bem distintos e, em breve, na linha do horizonte os prédios gigantescos
apareceram aureolados pelos raios do sol. Bárbara deixou a cabina; tomou
as providências para o desembarque e subiu. Encostou-se à grade do
tombadilho e começou a notar que à medida que a vista panorâmica
diminuía em extensão, aumentava o vulto dos objetos que continha. O
vapor aproximou-se mais... Emoldurando os edifícios, via-se a poeira de
prata, últimos resíduos que se erguiam das ruas... das avenidas. Ao
longe, formando o fundo do cenário, altos montes escuros, pondo em
primeiro plano os arranha-céus agrupados. Em um proscênio extenso, tendo
a baía por ribalta, se apresentava a Bárbara a Capital do país.
Impulsionada por um sentimento estranho, indizível, contemplava aquele
conjunto de decorações naturais. Era como se estivesse em uma exposição
de quadros, cujos motivos ela conhecesse. Agora, iria morar ali; dentro
em pouco, quem sabe, tudo aquilo lhe seria familiar. Nisto, alguém falou
atrás de si:

--- Chegamos.

--- É verdade --- respondeu voltando-se.

--- Foi uma viagem agradável.

--- Para mim, também.

--- Tornaremos a encontrar?

--- Creio que sim. O acaso nos persegue.

--- Não posso chamar a isto de perseguição --- tornou ele amavelmente.

Bárbara não disse nada.

--- Onde vai morar? --- insistiu ele.

--- Santa Teresa.

--- É um bairro essencialmente agradável.

--- E o lugar de minha casa é lindíssimo. É o topo de uma montanha. A
casa é a primeira; nada me impede a vista. Descortino todo o Rio.

--- Desejo que seja feliz lá.

--- Obrigada.

--- Seria mais acertado dizer, que continue feliz.

--- Por quê?

--- Você tem uma expressão de felicidade. Mas uma felicidade segura; até
mesmo severa.

--- Notou isto?

--- Mais ou menos. Em pessoas como você não se pode notar tudo de uma só
vez.

Bárbara silenciou; mas, pela sua atitude, o rapaz percebeu que
compreendera o que ele dissera.

--- Gostaria de visitá-la, tenho prazer na sua companhia --- continuou
ainda.

--- E eu o receberia com agrado. E com a presunção de que o animaria a
terminar os seus livros.

--- Talvez...

--- Vai tentar?

--- Não custa. A vida não passa de uma tentativa.

--- E com isso, chegamos.

--- Posso ir vê-la hoje?

--- Seria melhor amanhã.

Tirou um cartão da bolsa e escreveu nele o seu endereço, dando-o ao
rapaz, logo a seguir.

--- Então, até amanhã.

--- Até amanhã --- e estendeu-lhe a mão.

Ele apertou aquela mão fina, de pele morena e macia. Sentindo que alguma
coisa lhe vinha através daquele contato, demorou-a na sua.

--- Bárbara...

--- Que é? --- perguntou, tentando afastar-se.

--- Há qualquer coisa de comum em nossas vidas...

\chapter{Capítulo 23}

Bárbara entrou na sua casa, a terceira que possuíra na vida. Tudo lhe
parecia novo. Mrs.~Patrice veio recebê-la e com grande prazer, ela
abraçou a governanta. Olhou os mesmos móveis, já dispostos em seus
lugares e, do que lhe deveria ser familiar, recebia uma sensação de
alheamento. Abriu as janelas todas e a casa deu impressão de vida.
Percorreu-a de cômodo em cômodo, olhou o jardim cultivado, voltou ao
terraço e contemplou o panorama; lá em baixo, estendia-se a cidade.

--- Não sei se aprecia os móveis assim --- veio dizer a governanta. ---
Foi o que arranjei. Livros, quadros e miudezas, estão todos aí.

--- Desses, eu cuidarei, Mrs.~Patrice.

Bárbara deixou o chapéu, bolsa e luvas que ainda segurava distraidamente
e, tomando o fone, ligou para Helena. Outra vez, estavam as duas
reunidas na mesma cidade. Parecia-lhe estar no passado, chegando da
Inglaterra e montando casa em São Paulo. Só que no passado estava com
ela o tio Maurício e agora, apenas a governanta.

--- Alô... --- começou Bárbara.

--- Até que enfim --- gritou Helena entusiasmada. E desandou a fazer
tantas perguntas que Bárbara não sabia qual responder primeiro.

--- Chegou agora? E suas coisas já vieram? Oh, como se demorou em São
Paulo! Cheguei a ficar preocupada. Você já almoçou? Posso providenciar
almoço para você; vou levar de automóvel.

--- Helena... espere até que eu responda alguma coisa.

--- Estou tão contente... você sabe, eu preciso dizer. Olhe aqui,
Bárbara, vou já até aí. Até logo.

Bárbara, sem responder a coisa alguma do que a amiga perguntara, pôs o
fone no gancho. Não chegara a dar um passo, já a campainha soou
novamente:

--- Alô... Bárbara, e sobre o almoço? Você não me disse nada.

Bárbara riu gostosamente e falou à amiga.

--- Obrigada; está tudo providenciado.

Helena bateu o fone e Bárbara sabia que ela chegaria logo. Entrou no
quarto em busca de um traje mais leve; vestiu-se apropriadamente para o
calor de janeiro. Sentada ao penteador, escovava já os cabelos, quando
ouviu a voz de Mrs.~Patrice chamando-a para o almoço.

***

--- Ó de casa --- gritou Helena na porta da rua.

Bárbara veio-lhe ao encontro e as duas amigas abraçaram-se
demoradamente.

--- Agora, sim, eu acredito que vamos ficar juntas outra vez.

Bárbara convidou-a para ver a casa e Helena notou à amiga que esta não
poderia ser melhor para ela. Era confortável sem ser luxuosa; Bárbara
gostava de coisas belas, mas não iria ao ponto de passar o tempo a
cultuar a casa. Tinha gosto para os arranjos, algumas raridades em arte
e vários quadros de pintura apreciáveis pelo trabalho e pelo motivo.

--- Veja Helena --- mostrou Bárbara --- aqui estão meus discos. Não
perdi um sequer e as peças de Beethoven continuam intactas. Ah como vou
ouvir música\ldots{} faz quase um mês que eu não a ouço.

--- Amanhã, não --- retrucou Helena.

--- Por quê?

--- Vão jogar à noite, lá em casa; e eu não quero ficar só.

--- Pois venha você aqui, ouviremos juntas.

--- Não posso; você sabe, mamãe vai jogar e eu preciso tomar as
providências de hospedagem.

--- Que maçada, Helena.

--- Também acho.

--- Bem, então vamos ver o resto.

Na frente, um amplo terraço para a face norte; embora o jardim fosse
pequeno, tinha no canto esquerdo uma árvore copada à cuja sombra estava
um banco de mármore. Bárbara pensou que chamaria logo um jardineiro para
saber o nome daquela árvore; depois deu risada e comentou com a amiga
que quantos jardineiros chamasse, tantos nomes teria a árvore. Restaria,
depois, saber qual o verdadeiro.

\chapter{Capítulo 24}

No dia seguinte, logo após o almoço, Bárbara desencaixotava os livros na
sala quando ouviu a campainha. Adivinhou quem era; todavia, prosseguiu
na tarefa. Avisara Mrs.~Patrice que conduzisse o visitante para ali.

--- Boa tarde.

--- Boa tarde --- respondeu então vindo ao seu encontro.

Ele não parecia alegre como na véspera; mostrara-se, porém, satisfeito
ao ver Bárbara.

--- Muito atarefada? --- perguntou.

--- Um pouco.

--- Há serviço para mim?

--- Há e bastante. A ocasião não é para oferecimentos.

--- Pois disponha; auxiliá-la, seria uma tarefa agradável. --- E olhando
em redor, acrescentou--- Esta casa está tão cuidada! Reparei também no
jardim. Mesmo a grama está em ordem.

--- Esteve habitada até o dia em que mandei os móveis para cá.

--- E a família daqui?

--- Foi para a minha casa, em São Paulo.

--- Estava destinada, então.

--- Parece.

--- É... dizem que tudo na vida é destino --- comentou ele.

--- Depende da forma de encarar este destino.

--- Estou pelo fatalismo. Ninguém escapa à sua sorte, disse com uma
certa revolta.

Bárbara tornava a perceber que algo se arruinara em profundo
ressentimento na vida de Álvaro. Tudo terminava numa expressão
pessimista ou de tédio.

A empregada pediu licença e entrou na sala, trazendo o café. Bárbara
ofereceu-lhe:

--- Uma xicrinha antes do cigarro, aceita?

--- Com muito gosto.

Bárbara serviu-o, e depois a si.

--- É costume de São Paulo. Lá não se dispensa o café.

--- Também morei em São Paulo; sei destes hábitos paulistas.

--- Morou lá muitos anos?

--- Vinte e cinco anos.

--- Vinte e cinco?!

--- Tenho vinte e nove. Estou fora, somente há quatro.

--- Então é paulista? --- perguntou ela contente.

--- Sou; nasci, vivi e sofri naquela bendita terra.

--- Tem saudades de lá?

--- Algumas.

Confirmara-se a impressão de Bárbara; em Álvaro, parecia viver um homem
já meio vencido, mas, que às vezes, tentava lutar. Pelo seu espírito
solidário passou a ideia de levantar-lhe o ânimo. Entretanto, que tinha
ao seu alcance? Ele não lhe pedira auxílio.

--- Trouxe tudo isto de São Paulo? --- indagou, observando os móveis.

--- Há coisas que trago desde a Inglaterra.

--- Por que desde a Inglaterra?

--- Porque sou inglesa --- tornou com naturalidade.

--- Inglesa?!

--- Causo admiração a toda gente --- comentou sorrindo.

--- Mas é incrível! Fala tão bem a nossa língua, conhece os nossos
costumes... está tão dentro do Brasil! --- exclamou Álvaro com certa
emoção.

--- É porque gosto do Brasil, Álvaro. E, depois eu não vim ao Brasil em
busca das suas riquezas; vim a passeio. Não vim explorar as suas terras,
vim conhecê-las.

--- Só em uma coisa, você não parece brasileira; no seu temperamento.
Logo vi que não podia ser latino.

Bárbara riu e respondeu:

--- Pois minha mãe, de descendência francesa, tinha temperamento
fleumático; meu pai, de raça saxônica, era mais impulsivo. Vê como são
as coisas!

--- Hum... --- fez o rapaz. --- Volto à minha teoria: quando se trata
dos homens, é melhor dizer sempre, parece... Mas, você referiu-se a
eles, no passado; são falecidos?

--- São.

--- Há muito tempo?

--- Muito. Eu tinha três meses, quando mamãe faleceu. E papai... eu
tinha nove anos.

--- Lembra-se bem dele?

--- E com tanta saudade... vou mostrá-lo a você.

Apanhou uma caixinha sobre a poltrona e, abrindo-a, tirou de dentro um
relógio de grossas correntes de ouro.

--- Era dele --- disse abrindo a folha posterior do relógio. --- Vê?
Aqui está o retrato de mamãe, posto por ele mesmo quando noivo.

--- Tem alguma coisa dela --- comentou, comparando-a com interesse.

--- Papai dizia sempre isso. E este é ele --- disse, mostrando o retrato
que estivera à sua mesa no hotel.

--- Mais parecido com você. Você, porém, é um misto dos dois.

E depositando o retrato na mão de Bárbara, prosseguiu Álvaro:

--- Você o perdeu com nove anos. Eu tinha o dobro dessa idade quando
perdi o meu. E perdi-o, já órfão de mãe; pois ela morreu quando eu
nasci. Posso assim avaliar seus sentimentos.

--- A morte é inexplicável --- comentou Bárbara. --- Tudo na vida se
forma à determinação de morrer. A ciência nem sequer se preocupa em
resolver o problema da morte. Quando muito, consegue adiá-la.
Entretanto, por que não nos educam os sentimentos a essa determinação?
Por que devemos sentir a morte assim, e permitir que nos destrua, às
vezes, a própria felicidade? Existirá em nós, por acaso, a ilusão de que
poderemos evitá-la?

\chapter{Capítulo 25}

Como prometera à Helena, Bárbara foi, à noite, para a casa da amiga.
Entre as pessoas reunidas para o jogo, encontrou algumas já conhecidas.
D\textsuperscript{a}. Alda ali estava com o marido. Bárbara conheceu
então o Dr.~Renato de Macedo que a distinguiu atenciosamente. Alice e o
Dr.~Martins tinham vindo também para jogar em casa de Helena. Havia
ainda mais umas vinte pessoas, às quais Bárbara foi apresentada
ligeiramente. Serviu-se um ``cocktail'' de entrada, e, entre as
mulheres, Alice Martins iniciou a conversa dizendo que em São Paulo
``fazia um calor louco''. Passara lá alguns dias, encontrara as velhas
conhecidas e matara as saudades do tempo em que morava na capital
paulista. Descreveu o movimento das casas de moda e de chá; falou dos
cavalos favoritos da estação e como o povo se habituava ao jogo
elegante. Jogavam os que iam ao prado e jogavam os que ficavam em casa.
A emissora ``Excelsior'', que o povo dizia ter exclusividade de
irradiação das corridas, transmitia os resultados, de tempo em tempo,
levando assim a boa ou a má sorte aos que não fossem apostar nas pistas.
São Paulo estava diferente. Dava gosto ver como os paulistas se
divertiam mais, e perdiam o hábito estúpido de contar dinheiro nos sete
dias da semana. A seguir, enumerou as senhoras em destaque no momento e
acabou dizendo que fizera um ``tour de force'' para regressar ao Rio.
Uma outra senhora comentou que Alice era uma ``habitué'' de São Paulo. E
ter-se-iam dividido as mesas sem mais novidades, se o Dr.~Martins, em
conversa, não contasse às senhoras ali presentes, que Bárbara era
inglesa. A iminência de uma guerra na Europa levara muita gente a
aprender geografia, pois, a Inglaterra, tão falada nos últimos tempos,
suscitava a curiosidade dos brasileiros. Tudo quanto era inglês ou
americano estava em grande moda no Brasil. Fizeram inúmeras perguntas a
Bárbara, às quais ela foi respondendo na medida do possível. Uma senhora
de cabelos grisalhos perguntou-lhe amavelmente:

--- Você no Brasil, sentiu-se em casa, não foi?

Bárbara não soube o que responder e a senhora continuou:

--- O Brasil é uma terra hospitaleira. Além disso, quase toda a gente
fala inglês.

O Dr.~Martins, que ainda se lembrava das ideias de Bárbara quanto à
linguagem do Brasil, sorriu maliciosamente. Interpelado, porém, teve que
explicar-se. Contou aos presentes o que Bárbara dissera em casa de
d\textsuperscript{a}. Alda.

--- Oh! --- volveu a mesma senhora --- No Brasil as línguas são assim;
de proveito prático, prático --- repetiu ela convicta.

--- O francês não teve a mesma sorte --- objetou Bárbara com delicadeza.
--- Ficou sendo uma língua da classe mais culta, e quem conhece o
francês, conhece os seus poetas, seus filósofos, enfim, a sua
civilização.

--- Isto lá é verdade --- comentou um senhor mais ao longe. --- Mas, não
se aborreça por isso. As traduções do inglês estão aumentando dia a dia;
logo, a sua literatura estará toda no Brasil.

--- E a influência da Inglaterra no Brasil! --- exclamou entusiasmado o
Dr.~Martins. --- Ainda quando estivemos em São Paulo, presenciei um fato
muito interessante. Eu estava no bonde da Lapa; e para o meu banco
subiram quatro operárias de fábrica. Percebi que eram operárias pelas
conversas entre elas. Terminara o expediente e o bonde estava daquele
jeito! Pois bem, quando passamos pelo cinema Santa Cecília, uma delas
viu o cartaz e quis ir ao cinema, à noite. Sabe \emph{como} perguntou às
colegas se poderiam ir ao cinema?

Os presentes voltaram-se curiosos para o Dr.~Martins. Ele tossiu,
certificou-se da atenção dos ouvintes e continuou:

--- Escuita, disse a moça de pé na ponta do banco, ocêis num qué inforcá
a aula de ingrêis, hoje, pa nóis i no cinema? As outras responderam
logo: quar, deixa disso. Nóis tem aula de ingrêis, nóis vai na aula de
ingrêis. Vejam!!! --- comentou o Dr.~Martins --- Não sabem a própria
língua.

--- Que desastre para o Brasil! --- disse Bárbara com tristeza.

--- É, Bárbara --- disse rindo o Dr.~Martins --- não tardará que você
esteja na sua terra outra vez.

Bárbara não acreditando no que ouvira, interpelou o Dr.~Martins:

--- Que disse o senhor?!

--- Se a coisa continua como está, mui breve só se falará o inglês no
Brasil.

Bárbara estremeceu. Olhou para o Dr.~Martins, sem compreender que ali
estivesse um brasileiro! Em redor, a conversa prosseguia: ela, porém,
fora de si pela surpresa, tornou-se alheia aos comentários. Se ouvisse
aquelas palavras de uma mulher, talvez não fosse tão grande sua
estupefação; o ambiente em que elas viviam, não lhes dava para pensar na
Pátria. Mas de um homem! E de um homem formado, como o Dr.~Martins! Um
médico que diziam ser competente, ter posição na sociedade brasileira! E
que, como médico, devia ter passado pelos bancos do ginásio, onde se
estuda, ainda hoje, embora vagamente, a história do Brasil-colônia!!!
Este homem, como tantos outros no Brasil, não aprendera a amar a
liberdade? A decantada liberdade das democracias! Não saberia ele que
ser colônia era ser escravo? E ser escravo, era obedecer em proveito de
outrem? Era o Brasil escapando aos brasileiros e deslizando mansamente
para outras mãos. --- A liberdade mal compreendida permitia tudo, até
mesmo a traição inconsciente de sua própria pátria. Saberiam os
brasileiros o que quer dizer pátria? E, afinal, estava o Brasil em
democracia, após a declaração do Estado Novo? Que se passava nesta
grande terra? --- O tempo o diria.

Quando Bárbara voltou a si, os homens e as mulheres ainda palestravam.
Falavam dos feitos ingleses, da sua marinha, Chamberlain, dos Estados
Unidas, Roosevelt, a organização americana, o grande povo. Nenhum deles
parecera estranhar a atitude do Dr.~Martins; e falou-se por muito tempo
destes dois países, como se eles começassem a existir agora. Era uma
descoberta do povo brasileiro.

Ouviu-se a voz do Dr.~Renato:

--- Ó gente! Hoje não se joga? Viemos ao jogo ou à conversa?

Levantou-se um vozerio na sala e as pessoas procuraram os seus lugares.
Dividiram-se as mesas e, aos poucos, as vozes foram diminuindo. Os
homens silenciosos, de cartas na mão, mostravam-se atentos aos lances.
Bárbara permaneceu pensativa. Helena folheava uma revista, a seu lado.
Nisto ouviram uma senhora dizer--- estou pela boa. Bárbara olhou uma vez
mais para os presentes e recostando-se à cômoda poltrona, por uma
associação de ideias, veio-lhe à mente um texto de Ingenieros: ``O homem
sem ideais faz da sorte um ofício, da ciência um comércio, da filosofia
um instrumento, da virtude uma empresa, da caridade uma festa, do prazer
um sensualismo.''

\chapter{Capítulo 26}

No dia seguinte, Álvaro reaparecia na casa de Bárbara à mesma hora. Como
não a ajudara na véspera, sentia-se culpado e vinha então a tempo de
prestar algum auxílio. Ao entrar na sala, viu a moça carregada com uma
enorme pilha de grandes livros; quase desaparecia por trás dos volumes.
Ele se apressou a tomar o peso das mãos de Bárbara e, sorrindo, ela
passou-lhe o fardo.

--- Vejo que está disposta ao trabalho --- comentou o rapaz, deixando a
pilha no lugar indicado. --- Mas, afinal, que vem a ser isto? Música?

--- Sim; eu tenho muitas.

Distraidamente, ele pôs-se a folhear o caderno de cima.

--- Só agora reparei; dois pianos?!

--- Pois é; dois pianos --- disse ela naturalmente.

--- E ambos de cauda. Como vai arranjar-se com o problema do espaço?

--- Há outra saleta ainda. Aqui, será só de música; e da outra farei uma
sala de leitura.

--- E vai tocar em dois pianos? --- brincou ele.

--- Talvez alguma fada me acompanhe --- respondeu.

--- Pena que eu não seja fada.

--- Estudou música? --- indagou Bárbara.

--- Parei na metade --- disse a rir.

--- Por que ri? --- insistiu a moça.

--- Porque na minha vida, tudo ficou pela metade. E você, toca?

--- Muito pouco. Parei em menos da metade.

Ele a fitou e sorriu:

--- Para quê dois pianos; então?

--- Mamãe era pianista e vovó também. Não obstante mamãe tivesse o piano
da família, papai deu-lhe outro quando se casou.

--- Era apreciador da música?

--- Enamorou-se da mamãe num concerto --- respondeu Bárbara.

Álvaro parou de repente de folhear o caderno e tentou reconhecer a
música da página. Assobiou algumas notas e voltando-se para a moça,
perguntou:

--- Que é isto?

--- Moszkowski; a Valsa do Amor.

--- É um manuscrito? --- indagou curioso o rapaz.

--- Sim; é um arranjo de papai. Ele tocava esta valsa com mamãe a dois
pianos. Mais tarde, simplificou este arranjo para tocar comigo. Não
cheguei a passar do simplificado, pois, ele faleceu e eu não a toquei
mais.

Bárbara contou o fato com simplicidade. Embora se referisse ao pai com
carinho, não tomou aquela atitude sofredora com que usualmente se fala
nos mortos. O rapaz observando-a, comentou:

--- Você fala na morte com tanta naturalidade! Se eu não a conhecesse um
pouco, diria ser isto uma desilusão.

--- Às vezes podemos dar uma impressão errada do que somos --- comentou
ela. --- Veja que coisa: eu gosto da vida.

--- E eu a abomino, Bárbara. Acho a vida um presente de grego --- disse
ele revoltado.

Bárbara assustou-se com aquela resposta brusca; percebendo, porém, que
tocara em alguma ferida aberta, pensou em desviar o rumo da conversa.
Não lhe acudira ainda um assunto apropriado, quando Álvaro, andando pela
sala para disfarçar a emoção, dirigiu-se novamente a ela:

--- Você vai me desculpar; eu sou tão brusco! Tive um repente e não pude
dominá-lo. Será melhor assim --- disse parando em frente à moça ---
conhecer-me-á mais depressa.

Álvaro tomou outra vez a música e continuou a assobiar baixinho as notas
da melodia. Dotado de excelente ouvido e conhecendo o significado
daquela escrita universal, lia a peça antes da execução. Bárbara
observou-o atentamente; que dom recebera ele da natureza! Que excelente
músico não seria se lhe houvessem cultivado a tendência! Estava pensando
nisso, quando Álvaro se voltou para ela:

--- Tudo isso --- disse com certa ironia --- porque não queremos ficar
onde estamos. Já não falou o poeta que a felicidade está onde a pomos,
mas... ora que bobagem, já nem me recordo mais disso. Vamos ficar aqui
mesmo.

--- Por quê, Álvaro? --- perguntou Bárbara com brandura.

--- Porque quando se conhecem --- ponderou com tristeza --- as regiões
mais altas do espírito \emph{e} não se poder viver nelas, é melhor, ou
não viver ou desconhecê-las novamente. Você pode achar engraçado, mas eu
já tive ideais --- disse ele mostrando-se descrente. --- Tive ideais e
não pude viver para eles; tinha o espírito voltado para as coisas altas,
a sorte impediu-me para o caminho oposto. Detesto a mediocridade e, no
entanto, que tenho feito até agora? Debater-me nas suas teias, sem
conseguir libertar-me. E apesar de tudo isso, ainda penso como o
filósofo, que ``o \emph{homem} sem ideais faz parte da sorte um ofício,
da ciência um comércio, da filosofia um instrumento, da virtude uma
empresa, da caridade uma festa, do prazer um sensualismo''.

Bárbara admirou a coincidência. Na véspera, ela recitara mentalmente
todo aquele trecho e hoje, Álvaro lhe repetia em circunstâncias tão
diversas. Seria uma transmissão de pensamento?

--- Afinal --- disse ele --- estou tornando pesada a minha companhia.
Desculpe-me, sim?

--- A reflexão não é um peso, Álvaro. Jamais o seria para mim.

--- Mas eu reflito com pessimismo.

--- Sempre?

--- Quase sempre.

--- Este quase já é animador.

\emph{---} E agora, não quero continuar atrapalhando --- volveu ele. ---
Vamos fazer alguma coisa?

Bárbara sugeriu a colocação dos livros nas estantes embutidas;
começariam lá do alto, assim ele faria o mais difícil.

--- Como iniciaremos? Pode-se usar esta escada? --- perguntou ele,
indicando-a.

--- Está aqui para isso --- respondeu ela.

O rapaz colocou a escada aberta junto da estante. Bárbara, que contava
fazer o serviço sozinha, mostrou satisfação pelo auxílio. Combinaram que
Álvaro ficaria no alto da escada e Bárbara lhe levaria os livros na
ordem em que costumava guardá-los. Graciosa, e leve nos seus movimentos,
elegante no seu traje caseiro, que era sempre o de calças compridas e
blusinha esporte, Bárbara tomou uma pilha de livros e subiu os degraus.
No topo da escada, aberta, Álvaro parecia montado nalgum ginete de
confiança.

--- Vê --- chamou o rapaz --- são as últimas reminiscências da
mitologia. Não pareço um exímio cavaleiro? E este é o meu Pégaso ---
disse a rir.

E foi tirando os livros da pilha que Bárbara trouxera, aliviando-lhe o
peso. Ela desceu em busca de outros:

--- Está bem aí? --- perguntou, empilhando novos livros.

--- Muito --- respondeu ele, e estendeu a mão para apanhar um volume
mais distante.

A escada oscilou.

--- Cuidado --- gritou Bárbara.

--- É... --- disse o rapaz endireitando-se --- se o Pégaso inventa de
voar, chegarei ao chão mais depressa do que desejo.

Bárbara pôs o pé na escada com cautela:

--- Posso subir?

--- Venha sem medo. O tal está manso.

Recomendando cuidado, ela subiu novamente.

--- Desta vez os livros são mais pesados --- comentou.

--- É mesmo; e que livro é este? Parece-me que eu já o vi --- disse
olhando a cor verde da encadernação.

--- É a biografia de Beethoven por Rolland. Estive com ele no vapor.

Álvaro colocou o livro sobre os joelhos e começou a folheá-lo. A seu
lado, de pé na escada, Bárbara acompanhava-lhe os gestos. A empregada
entrou no momento e, vendo-os lá no alto, mostrou-se tão surpreendida
que os dois mal puderam conter o riso.

--- O que é Maria? --- indagou Bárbara, olhando para ela.

--- É... é... esse quadro daqui --- disse a mulher embasbacada.

--- Qual deles?

--- Aquele que a senhora fez.

--- Também pinta, Bárbara?! --- perguntou o rapaz admirado.

--- Um pouco. As pessoas como eu, fazem um pouco de tudo; porém, nada
perfeito.

--- E eu o que faço? --- insistiu a empregada.

--- Deixe-me vê-lo --- pediu o rapaz.

A empregada entregou-lhe o quadro. Álvaro pôde então apreciá-lo--- as
linhas não eram boas, mas as cores bem distribuídas revelavam a
sensibilidade artística de quem as combinara.

--- Onde viu a palmeira? --- indagou ele.

--- É a da Vista Chinesa; pintei-a porque estava solitária.

--- Já foi recompensada --- disse em tom mais carinhoso.

Ela sorriu. Houve um curto silêncio, e Álvaro voltou-se novamente para a
moça:

--- Bárbara... seria muito pedir-lhe uma ilustração?

--- Para quê?

--- Um artigo... --- tentou explicar meio perturbado. --- Um artigo
que... ora, é bobagem. E começou a pegar os livros apressadamente para
colocá-los em seus lugares.

--- Álvaro --- insistiu a moça com brandura --- o artigo, que saiu há
pouco, é um caminho precioso.

--- Isto tudo é bobagem; eu nunca passarei de um idealista\ldots{}

--- E que seria do mundo se não existissem os idealistas? --- indagou
ela com firmeza.

--- Eu sou um vencido. E você verá com certeza o que sou quando me
conhecer mais. Depois, já não escrevo em qualquer parte; preciso andar,
ver a natureza, procurar a emoção. Isto tudo já não sei fazer sozinho;
embora eu tenha tentado. Sim, eu tenho tentado, Bárbara, pode crer ---
disse-lhe com certa amargura.

--- E se eu o acompanhasse, adiantaria alguma coisa? Enquanto você
escrevesse, eu tentaria desenhar os mesmos cenários que o inspirassem.
Eu gostaria, Álvaro, de vê-lo tomar impulso na vida.

Ele fitou-a, entre admirado e enternecido:

--- Eu não teria coragem para dizer não. Bárbara... muito obrigado.

\chapter{Capítulo 27}

Bárbara levantou-se cedo e, após o café, saiu para o jardim. A manhã
esplendia deliciosa. Bárbara não resistiu ao desejo de passear. Avisou a
governanta e saiu, a pé, pelo bairro de Santa Tereza. Logo à esquina, um
menino maltrapilho veio pedir-lhe uma esmola:

--- Moça, eu não tenho pai nem mãe, me dá um tostão para comer alguma
coisa?

Ela achou graça na pronúncia do garoto; notara já como os cariocas
falavam normalmente bem o português. Voltando-se para ele, perguntou:

--- Quantos anos tem?

--- Não sei --- respondeu o menino embaraçado.

--- Olhe --- disse Bárbara --- eu não dou esmola na rua; mas, se você
quiser, pode ir varrer o quintal da minha casa. Ganhará pelo seu serviço
e ainda lhe darei uma xícara de café com leite, pão e manteiga; você não
gosta?

O menino, atrapalhado, olhou para mais longe e Bárbara, acompanhando o
seu olhar, viu uma mulher que o chamava com modos grosseiros. Percebeu
logo a exploração em torno da criança e compadecida, animou-o:

--- Se quiser, o serviço é seu; moro ali adiante, naquela casa branca,
de esquina.

O garoto saiu um tanto amedrontado e Bárbara prosseguiu. Subiu toda a
rua e foi dar num largo ainda desconhecido para ela. Como era difícil o
traçado do bairro; todo cheio de curvas. Algumas crianças pobres
brincavam pelas ruas e nos prédios de apartamentos, mulheres, ainda em
trajes matinais, compravam verdura nos cestos ambulantes. Não raro,
encontrava-se uma mulher de roupão na calçada, tão à vontade como se
estivesse em seu próprio quarto. Bárbara tomava já a segunda rua depois
do largo, quando viu uma voz que a chamava pelo nome:

--- Bárbara.

Voltando-se, reconheceu Carlito.

--- Você? --- perguntou a moça. --- Que faz aqui a estas horas?

--- Eu é que pergunto: que faz aqui a estas horas?

--- Visito o lugar --- tornou ela.

--- Simplesmente?

--- Simplesmente.

--- Não gostaria mais de estar na praia, agora?

O rapaz ia acrescentando ``comigo'', mas resolveu parar. Aqueles dias de
separação não o animaram a maiores intimidades. Bárbara sorriu e
respondeu:

--- Não. Estou aqui e, portanto, gostaria de estar aqui mesmo.

--- Você gosta de Santa Tereza! --- exclamou ele admirado. --- E já
parece tão adaptada ao Rio!

--- Adaptei-me a mim mesma --- respondeu sorrindo.

--- E como se pode fazer tal adaptação?

--- É uma questão puramente pessoal.

--- Quer ser proprietária exclusiva do seu segredo?

--- Acha? --- inquiriu ela.

--- Será possível! --- comentou desalentado o rapaz. --- Eu faço uma
pergunta e você me responde com outra.

--- É porque há perguntas que não têm resposta --- disse Bárbara com
simplicidade.

Carlito ficou meio desconcertado, sem entender bem o que ela queria
dizer. Todavia, quando se achava ao lado da moça, não queria perder a
oportunidade, e arriscava-se a mais. Procurando ser amável, convidou-a
para a noite:

--- Vamos hoje a Urca?

--- Cassino? Não: obrigada. Não gosto do jogo e hoje não sinto vontade
de dançar.

--- Você condena as mulheres que jogam? --- indagou ele curioso.

--- Por que haveria eu de fazer isso? Não sou nenhum juiz.

--- E nem poderia ser --- respondeu numa risada longa. --- A
magistratura não está ao alcance das mulheres.

Bárbara não se atrapalhou; voltou-se para ele e disse com toda
naturalidade:

--- Há muita mulher que não sabe disso.

* * *

Quando Bárbara regressou à casa, tinha esquecido Carlito completamente.
Na esquina, ela comprara flores e, tendo tempo para o almoço, foi vestir
calças à jardineira, afim de examinar as suas plantas. Entrou pelo
canteiro a dentro, e, com uma faquinha de ponta, pôs-se a revolver o
solo. Estava já entretida no trabalho de jardinagem, quando alguém a
chamou:

--- Alô\ldots.

Ergueu os olhos da terra:

--- Álvaro! Que surpresa.

O rapaz aproximou-se e comentou que chegara em boa hora.

--- Por quê? --- indagou a moça.

--- Trouxe aqui algumas sementes de heliotrópio. Sabe semear?

Ela não era muito entendida no ofício. A falta de espaço nas casas,
levara-a sempre a adquirir mudas já formadas; todavia, lembrou-se de um
almanaque orientador em tal trabalho. Foi buscá-lo e, examinando as suas
páginas, deduziu que o tempo para semear seria março e agosto. Estavam
nos primeiros dias de fevereiro; seria arriscado, mas, de bom humor pela
manhã tão agradável, resolveram fazer a experiência. Álvaro tirou o
paletó; em mangas de camisa, perguntou a Bárbara.

--- Tem ferramenta?

Bárbara respondeu enfaticamente que sim. Afastou-se e voltou logo com
uma enxadinha minúscula.

--- Que é isto? --- perguntou ele. --- É de brinquedo?

--- Não; é de verdade --- assegurou a moça.

Álvaro achou graça na afirmativa. Meio desajeitado no ofício, pôs-se a
arrancar o mato do terreno. Deixou a terra revolvida e pediu a Bárbara
que fiscalizasse o trabalho. De almanaque em mão, ela ia dizendo o que
fazer. Espalharam-se as sementinhas pela terra afofada. No almanaque
dizia ser melhor regá-las depois; Bárbara, porém, achou prudente
molhá-las na hora. Álvaro concordou. Puxaram a borracha para perto;
logo, uma chuva fina caiu sobre as sementes. Terminado o serviço, Álvaro
pensou em retirar-se. Bárbara comentou sorrindo que o jardineiro
procederia assim, porque era o técnico contratado; ao artista das
flores, porém, fazia-se uma exceção. E lembrando a valsa que ele
assobiara tão facilmente no outro dia, convidou-o para tocar com ela.
Álvaro não desejava retirar-se; a companhia da moça fazia-lhe bem ao
espírito, portanto, aceitou com prazer o amável convite. Passaram pelo
terraço da frente e entraram na sala de música. Bárbara mostrou-lhe o
arranjo simplificado que tocara em criança e o original que o pai
deixara. Por faltar-lhe sempre um companheiro, desconhecia o efeito do
trabalho paterno.

--- Tentarei --- disse ele --- mas se Moszkowski reconhecer a sua valsa,
certamente, me pegará na esquina.

--- Há muito que não toca? --- indagou Bárbara.

--- Desde que levaram o meu piano.

--- Para onde foi o seu piano?

--- Voltou ao depósito. É a tal história: ``o bom filho à casa torna''
--- comentou sorrindo.

--- Então, vamos recordar o aprendido --- tornou Bárbara, desviando-se
do assunto.

Após rápida leitura, sentaram-se ambos aos pianos que se encontravam
paralelos. A ``Valsa do Amor'' cantou sua melodia; suave, delicada na
primeira parte, tornou-se mais forte e mais grave na segunda. Quando
Bárbara fazia o canto, Álvaro acompanhava; outras vezes, Álvaro fazia o
canto, e Bárbara acompanhava. Nas partes mais lentas, olhavam-se e
acertavam o ritmo. Pelo espírito, viviam, assim, momentos da mais
deliciosa emoção. Transportados pelo prazer de tocarem juntos e na
simplicidade de uma valsa, encontravam eles alimento para as suas almas
de artista. É assim mesmo a arte; depois do Criador, somente ela podia
dizer aos homens: ``sursum corda''.

\chapter{Capítulo 28}

A janela do quarto estava aberta; os primeiros raios da alvorada
invadiam já o recinto. Bárbara abriu os olhos vagarosamente e, vendo que
amanhecia, pôs-se a contemplar a madrugada cinzenta.

A luz intensificava-se e, pouco a pouco, vencia o sombreado escuro da
manhã que começava.

Bárbara deixou as cobertas quentes e aproximou-se da janela. Lá estava o
mesmo cenário que, embora visse todos os dias, parecia sempre novo. As
embarcações esparsas, as tiras de luz e de sombra insinuando-se entre as
ramagens altas das palmeiras, o vulto escuro e aureolado das serras,
muralhas fortes da cidade... tudo isso dava-lhe a impressão de uma
paisagem imaginária.

Alguns homens passaram silenciosos--- era a caminhada do trabalho.
Bárbara contemplou-os até que desaparecessem ao longe. Verdade, que ela
não precisava trabalhar para sustento próprio; mas, conhecia de perto as
necessidades prementes dos que buscam o próprio pão. E Bárbara, sentindo
a responsabilidade dos seus haveres, procurava gastá-los com parcimônia.
Vivia comodamente; mas evitava os excessos, e a caridade para ela não se
colocava entre as sobras e sim entre os deveres principais.

Distraía-se, olhando o pessoal que passava, na rua, quando a despertou o
telefone. Afastou-se vagarosamente da janela.

--- Alô... --- disse uma voz masculina.

Bárbara reconheceu logo a de Carlito.

--- Oh, como se levanta cedo! --- comentou em resposta ao seu alô.

--- E seria capaz de levantar mais cedo ainda\ldots{}

--- Para quê tão cedo assim? --- indagou a moça.

--- Para levá-la à praia. Aceita?

--- O convite é agradável, mas, hoje não posso. Que tal na próxima
semana?

--- Está bem --- disse ele aborrecido --- eu me conformo. Mas você não
vai sair nestes dias?

--- Tenho ocupações para estes dias, Carlito; fico-lhe grata pela
gentileza.

\chapter{Capítulo 29}

Carlito despediu-se contrariado. Que criatura estranha era Bárbara. Já,
uma vez, recusara um convite seu para ver a cidade; depois, aceitou
almoçar num restaurante de estrada, afastado até dos arrabaldes. Verdade
que fora com Helena; mas, em Santa Tereza, não passeara só com ele
quando a surpreendeu andando a pé pelo bairro? E a casa? Não fora ele,
Carlito, quem a levara para realizar o negócio? Não; Bárbara não se
prendia a esses preconceitos. Mas, e a razão de ora aceitar, ora
recusar? Seria pela maneira de fazer o convite? Bem, pensou o rapaz,
coçando a nuca, talvez esteja mesmo ocupada; porém que sorte de
ocupação? Ou seria simplesmente por não querer, nestes dias? Isso também
não seria possível. Com estas considerações, continuava ainda na cama de
onde falara, pelo telefone, com Bárbara. Ia acender o cigarro, quando se
lembrou do café. Tocou a campainha. Pouco depois, entrou no quarto a
empregada com a bandeja e Carlito, tão preocupado, não percebeu que a
empregada era outra. Só mesmo quando ela lhe serviu o café já na xícara
e perguntou se queria com muito ou pouco açúcar, é que notou a voz
estranha. Ele achou tudo bom e dispensou a servente que se retirou a
seguir. Enquanto tomava o café, esqueceu-se momentaneamente de Bárbara e
pensou então que sua mãe teria substituído a empregada por sua causa.
Esta agora, era feia; não haveria perigo de demorar-se nos arranjos do
quarto dele. Deu uma risadinha de caçoada e acendeu o cigarro. Fumou-o
todo e pensou em dormir outra vez; mas, às dez horas, tinha uma aula na
faculdade, e seus pais queriam que ele se formasse em Direito. Para que
tanto trabalho, pensou aborrecido; o ginásio e os prés custaram tanto, e
não houve outro jeito, senão cursá-los em estabelecimento particular...
agora, mais esta maçada. Ah, que vida difícil para quem tem um pai rico!
Carlito virou-se na cama e, antes de pegar no sono, lembrou-se de
Bárbara. --- Que diabinha! Por ela eu teria levantado.

* * *

À hora do almoço, chamaram Carlito no quarto. Quinze minutos depois, o
rapaz descia as escadas e ia beijar a mãe. A família estava reunida para
a refeição. Dr.~Renato, na qualidade de chefe, ocupava a cabeceira da
mesa; na outra extremidade, em frente ao marido, estava
d.\textsuperscript{a} Alda. De um lado, as duas meninas e de outro,
Carlito, perto da mãe. Dr.~Renato chegara de São Paulo pelo noturno.
Embora a viagem o abatesse, mostrava-se pronto para os trabalhos do dia.
Nos seus raros cabelos, penteados para trás, parecia mais intensa a
mecha branca que lhe saía da testa. Um tanto obeso, tinha ainda
movimentos ágeis e esforçava-se por vencer a prostração da idade que
caminhava. Seus traços, de linhas indecisas, formavam conjunto vulgar,
semelhante à maioria dos homens. Nada o diferenciava do tipo comum, a
não ser aquela mecha de cabelos brancos. Contudo, no seu olhar
transparecia, de longe em longe, um brilho diferente, contando que
aquele homem se acomodara às conveniências. Demonstrava certa energia
que se evidenciava ao tratar com seres mais fracos. Advogado, exercera a
profissão com êxito financeiro. Apoiado em parentes e amigos de
influência na política, Dr. Renato aproveitara, sem escrúpulos, as
oportunidades que lhe surgiram. Aceitava convicto, portanto com
sinceridade, o velho ditado: ``Política é para se tirar proveito''. E
como para a maioria, vencer é tomar-se rico, era ele considerado um
vencedor na vida. Como tal, recebia as honras. Sua esposa percebera,
através da sua carreira, que a riqueza desfrutada largamente pela
família, não viera do trabalho honesto e inteligente do advogado. Como,
porém, era agradável ter uma vida farta, deixou à margem este ponto,
para considerar os de maiores vantagens pessoais. Todavia, não
conseguira esconder um certo desprezo pelo marido; tinha consciência de
tal sentimento e das circunstâncias que a levaram a percebê-lo.

Regressando de sua última viagem, dr. Renato almoçava agora com a
família. Logo ao primeiro prato, a nova empregada apareceu. Ao ser
servido, Carlito olhou de soslaio para a mãe. Na maneira como lhe foi
retribuído o olhar, Carlito percebeu a intenção materna. Para mudar a
situação, dirigiu-se ao pai:

--- Como vão os parentes de São Paulo?

--- É mesmo --- concordou Lia --- conte-nos alguma coisa de lá.

--- Minha filha, quem viaja a negócios, não tem tempo para muita coisa
--- respondeu o pai já a servir-se.

--- Mas há algo novo por lá? O senhor não visitou algum parente? ---
insistiu a menina.

--- Os mais chegados; os outros, não tive tempo. Depois, alguns já estão
divorciados e novamente casados; não saberia como encontrá-los.

--- Novos divórcios, papai? --- perguntou Ivete com interesse.

--- Isto há sempre --- disse o Dr.~Renato num sacudir de ombros.

--- Mas há divórcio no Brasil? --- insistiu a filha.

--- Não.

--- E como se dá isto então?

--- Fora do Brasil --- disse secamente o pai.

--- Ah --- exclamou Ivete --- então é questão de dinheiro! E digam que o
dinheiro não é a mola real da vida!!!

--- Os pobres não se divorciam, papai? --- perguntou Lia com
ingenuidade.

Distraidamente, o Dr.~Renato não respondeu.

--- Então é preferível ser pobre, o senhor não acha? --- continuou.

--- Boba --- tornou Ivete --- pensa que é suficiente ser pobre, para
viver com o marido a vida inteira?

--- E como fazem os pobres então?

--- Separam-se; sem o dinheiro, sem viagem ao estrangeiro e sem barulho.

--- Eu gostaria que o senhor me explicasse --- tornou Lia ainda --- onde
é que os divorciados são casados.

--- Que pergunta esquisita! --- comentou o Dr.~Renato, que parecia não
estar apreciando muito o assunto da conversa.

--- Isto tudo é complicado para a sua idade; mais tarde, compreenderá
--- interveio d.\textsuperscript{a} Alda com autoridade.

--- A questão não é minha idade e sim esse casamento fora da lei, mamãe.

--- Lia, você se interessa sempre por coisas além do seu alcance ---
interrompeu d.\textsuperscript{a} Alda.

--- Eu gostaria de compreender todo o mal para não cair nele ---
explicou Lia com a mesma simplicidade.

--- Você é ainda criança, minha irmã --- falou Carlito pela primeira
vez.

--- E esta compreensão minha querida, não vai encontrá-la, pensando no
divórcio --- disse ainda Ivete, provocante.

--- Não estou pensando no divórcio, estou pensando nas leis. Para mim,
as leis são para uma nação o que os pais são para os filhos.

--- Muito bem --- disse logo d.\textsuperscript{a} Alda --- as ordens
dos pais devem ser leis para os filhos.

--- Hoje em dia --- voltou Ivete --- as mães têm pouco tempo para os
filhos; passam as horas a ver o que faz a vizinha, ou em busca do preço
do traje de uma conhecida. Dizem que vão a festas para acompanhar as
filhas, mas se divertem mais que elas. E as filhas percebem isto!
Percebem, mas calam-se. É mais cômodo viver longe de zelos excessivos
--- comentou Ivete indiretamente às palavras da mãe.

--- Isto me faz lembrar Bárbara --- interveio Carlito pela primeira vez.

--- Por quê, Carlito? --- inquiriu d.\textsuperscript{a} Alda.

--- Porque certa vez, mamãe, perguntei à Bárbara, qual seria a diferença
entre as mulheres que frequentam e as que não. Sabe o que ela disse? ``É
muito simples; se todas fizessem a mesma coisa, não haveria diferença,
portanto não caberia crítica. É preciso que façam coisas diferentes,
para haver assunto de parte a parte''.

Foi uma risada geral.

--- Interessante a psicologia desta sua amiguinha, meu filho.

--- O senhor disse a verdade, papai.

--- E é bonita, papai! --- exclamou Lia entusiasmada.

O Dr.~Renato riu e perguntou;

--- Que é para você uma mulher bonita, Lia?

--- Uma mulher que agrada; uma mulher como Bárbara.

Carlito ouviu as palavras da irmã, e, mau grado seu, lembrou-se da
recusa da manhã. E quando Ivete zangou-se por lhe terem cortado o
assunto, falou irritado:

--- Foi oportuno, Ivete; e não se esqueça, é melhor silenciar do que
dizer tolices.

\chapter{Capítulo 30}

Veio a outra semana e Bárbara não teve notícias de Álvaro. Sem dizer por
que, o rapaz não viera mais à sua casa. Lembrou-se dele com saudade---
das observações oportunas, do temperamento artístico, um tanto boêmio...
do modo despreocupado de falar. Quando lia algum poeta ou filósofo, como
se entregava às ideias. Bárbara passaria horas a conversar com ele. Suas
atitudes, como eram corretas! Mas afinal, quem seria Álvaro Prado? Teria
ele uma vida estabelecida? Bárbara não o ouvira falar em trabalho; notou
desde o início que tinha mais tendência para as ideias que para as
realizações. Já aparecera em sua casa durante o dia e de manhã;
naturalmente não prestaria contas de suas horas a ninguém. Pensava nele
quando o telefone tocou; tomando-o, Bárbara ouviu uma voz masculina.
Inconscientemente, pronunciou o nome de Álvaro.

--- Quem?! --- perguntaram com espanto do outro lado.

Bárbara notou o erro e desculpou-se:

--- Nada, Carlito.

--- Você disse um nome; esperava telefonema de alguém?

--- Não: nem mesmo seu.

Um tanto intrigado, ele continuou:

--- E você vai bem, Bárbara?

Notava-se que Carlito frisava o nome da moça.

--- Muito bem, e você?

--- Esta resposta vai depender de você.

--- De mim?

--- De você, somente. Vamos à praia?

--- Que convite agradável!

--- Passarei por aí; nesta meia hora, combinado?

--- E Helena? Gostaria de convidá-la.

--- Avise-a o quanto antes. Queremos aproveitar a manhã.

Na verdade, ele queria aproveitar a companhia da moça.

--- Combinado, então --- concordou Bárbara.

Mrs.~Patrice que estava no quarto, perguntou-lhe.

--- Vai ao banho de mar?

--- Vou sim. Faça-me um favor Mrs.~Patrice, telefone a Helena, dizendo
que vamos buscá-la; passaremos lá nestes vinte minutos.

Enquanto a governanta a atendia, ela se aprontou; e logo veio sentar-se
ao carrinho que lhe trazia a primeira refeição do dia. Não a terminara
ainda e já, à porta, ouviu a buzina do carro de Carlito. Bárbara veio à
janela e cumprimentando-o, perguntou-lhe:

--- Uma xicrinha de café, não é boa ideia.?

Carlito aceitou o oferecimento e o carrinho foi levado para o terraço da
frente. Serviu-lhe o café e o rapaz aceitou também uma bolacha.

--- Eu estava terminando, sabe? --- explicou a moça.

Depois do café, entraram na linda, barata e Carlito tomou o rumo da casa
de Helena. Foi tão rápido que Bárbara nem mesmo avistou, a apenas uns
metros de sua casa, a pessoa em quem ainda pensara, naquela manhã. E
quando o carro partiu, não percebeu o desaponto que deixara na alma de
quem custara a retomar a decisão de procurá-la.

Quando, de longe, Álvaro viu Bárbara sair em trajes de esporte, pensou,
naturalmente vai à praia com algum admirador. E desfez-se a expectativa
do novo encontro.

Ao passarem por ele, Álvaro contemplou a fisionomia jovem e descuidada
de Carlito. De camisa desabotoada, guiando um carro fino, e de uma
alegria comunicativa, o rapaz era um ornamento na manhã clara e
agradável daquele dia. Álvaro quis esquecer o fato, mas, a fisionomia da
Bárbara, aquele olhar docemente penetrante, de quem vai até o fundo da
alma, voltou-lhe com insistência. Olhou para trás: o carro tinha
desaparecido da vista. E ele não pôde conciliar Bárbara com aquele filho
de ``papai rico''. Acendeu um cigarro e passou direito, sem mesmo olhar
para aquela casa branca da esquina.

\chapter{Capítulo 31}

A frescura da manhã, o passeio pela cidade e a expectativa de um banho
divertido davam aos três companheiros um humor alegre. Tagarelando,
chegaram à praia de Copacabana. Desceram do carro. Bárbara lembrou-se
então do chapéu, e pediu-o ao rapaz:

--- Pega o meu chapéu, Carlito?

--- Pois não. E não quer mais alguma coisa --- perguntou-lhe baixinho.

--- Por enquanto, é só o chapéu.

--- Ainda bem que diz por enquanto.

Bárbara sorriu da impetuosidade do rapaz, tão própria dos vinte anos.

Quando deixavam o carro, Carlito contou-lhes que estavam planejando uma
corrida de bicicleta, e o vencedor teria direito a um prêmio.

--- Vai ser divertido --- disse Bárbara, avistando já o grupo de pessoas
ali reunidas.

Carlito segredou-lhe ao ouvido:

--- Contanto que não se afaste de mim. Se o fizer, acharei tudo
aborrecido.

Lia estava na praia. Vendo que Bárbara chegava, aproximou-se para
cumprimentá-la. Informou ao irmão que apenas esperavam por ele para
iniciar o torneio. A animação era grande. Helena falou com os rapazes e
moças ali presentes e quando Carlito apareceu, começaram a gritar com
alvoroço; a alegria era contagiosa. Oito bicicletas estavam dispostas em
linha e os rapazes, de calção, mostravam-se prontos para a partida.
Carlito chegou atrasado, mas já estava no meio deles. Mais além, Bárbara
viu um que a observava com insistência.

--- Quem é? --- perguntou a Helena, indicando-o.

--- Sérgio Augusto de Toledo.

--- Por que me olha assim?

--- Achou-a linda! Sabe o que ele disse? Que se sair vencedor, vai pedir
uma contradança com você.

--- E onde espera encontrar-me?

--- Na Urca; vamos hoje para lá.

--- Eu também?

--- Também; prometi que a levaria.

Bárbara sacudiu a cabeça, duvidando da sua ida. Helena perguntou
baixinho.

--- Vai conhecer mais um rapaz. Este coração de gelo não se derrete?

--- Oh! --- gracejou a outra --- são apenas camaradagens.

--- Vá acreditando nessas camaradagens... Carlito está doidinho por
você.

Deram a partida, e, entre brados e vivas, os ciclistas saíram pela
praia. A atenção geral voltou-se para os concorrentes. Bárbara viu Ivete
pulando do outro lado, e pronunciando um nome, aos gritos, que ela não
pôde compreender. Perto de Bárbara, estava Lia, olhando com interesse e
torcendo pela vitória do irmão. Dois ou três ciclistas apontaram ao
longe; já estavam de volta. Uma fita branca estendeu-se, à. pressa,
transversalmente à avenida, tendo Ivete em uma das extremidades e, em
outra, uma moça que Bárbara não conhecia. Os que se aproximavam recebiam
palmas entusiásticas; e Sérgio Augusto, em primeiro lugar, rompeu a fita
vitorioso.

--- Pronto --- disse Helena com malícia.

--- Pode ser que tenha esquecido --- tornou Bárbara.

--- Duvido.

Ivete aproximou-se com o rapaz:

--- Bárbara, quero apresentar-lhe um amiguinho nosso.

Sérgio Augusto de Toledo --- disse ele, estendendo-lhe a mão.

--- Bárbara Stuart.

--- Não sei se está informada do meu prêmio --- perguntou Sérgio
Augusto, sem intimidar-se.

--- Não muito bem --- respondeu Bárbara com delicadeza.

--- Peço-lhe uma contradança para hoje --- disse com o mesmo
desembaraço.

--- Com muito prazer --- respondeu Bárbara.

--- Estará na Urca, esta noite?

--- Irei com Helena e os demais.

Sérgio era rapaz de estatura média, cabelos castanhos, e um pouco magro.
Tinha vinte e dois anos e estava na época de apreciar as mulheres mais
velhas, para se dar ao valor. Sobrancelhas cerradas e olhos verdes, um
bigode da cor dos cabelos e rosto cheio. Não era bonito; tinha esse ar
um tanto cheio de si, comum aos rapazes da sua idade que pensam
descobrir um mundo novo dentro de outro já velho e gasto. Satisfeito
pela vitória alcançada e fazendo a corte a Bárbara, não deixou de olhar
algumas vezes para Ivete. Tratou gentilmente Helena e só não tomou
conhecimento de Lia que se manteve afastada da conversa.

--- Bem --- tornou Bárbara --- como está ficando tarde e ainda não
entrei na água, vão dar-me licença agora. Quer ir, Helena?

--- Não, obrigada; vou jogar peteca.

--- Lia vem comigo, não é?

--- Vou sim --- respondeu logo a menina; e seguiu-a.

Avançaram pela praia e escolheram um lugar para quebrar as ondas. Quando
estas vinham impulsivas, Bárbara e Lia deixavam-se erguer por elas. Logo
depois, Carlito aproximou-se. De mãos dadas, Bárbara, Lia e Carlito,
ficaram ali divertindo-se por longo tempo. Quebrar as ondas era um
desses brinquedos da fase adulta.

\chapter{Capítulo 32}

Bárbara estava só e tudo indicava que aquele dia seria um dois muitos da
rotina habitual. Helena ausentara-se; fora visitar a avó paterna que
passava o verão em Petrópolis. Ligou o rádio e enquanto esperava por um
bom programa, aproveitou ver a correspondência. Alguns anúncios
comerciais prevenindo liquidações vantajosas; um cartão postal de Helena
focalizando as hortênsias da cidade alta. Estava ali também um envelope
grande que pelo jeito deveria ser uma participação. Bárbara abriu-o. Leu
a primeira vez; teve dúvidas da sua visão e releu atenciosamente. Era
verdade; Ivete estava noiva e convidava-a para o chá. Olhou a data;
seria naquela noite. Que convite apressado, e assim tão de repente!
Ivete, aquela menina fútil, que vira ainda há dias, brincar na praia.
Compreenderia Ivete o significado de um casamento? Bem, concluiu Bárbara
para si, esta é uma das muitas loucuras que o povo, complacentemente,
tachou de ``sociais''.

É certo que Ivete não alcançaria por si mesma o erro de que ela seria
também uma das vítimas; mas, e os pais? Seriam eles indiferentes? Ou
aproveitavam-se da oportunidade para encerrar as suas responsabilidades
familiares?

O locutor da emissora anunciou o ``Quarteto em dó menor'' de Beethoven.
Às primeiras notas do ``Allegro ma non tanto'', Bárbara esqueceu-se de
Ivete, do casamento e demais coisas.

A angústia íntima de Beethoven era inconfundível desde o primeiro tema;
ela aparecia ali, tocando os sentimentos de Bárbara como se a sua alma
irmã, pudesse acompanhá-lo naquela estrada. Embora esse quarteto
figurasse nas primeiras composições do artista, nele Bárbara sentia já o
mesmo Beethoven das sinfonias. Era difícil separá-lo nas suas obras; a
sua personalidade marcada, em tudo deixava um traço próprio. E aquelas
páginas de intenso lirismo traziam à luz os sentimentos do compositor.
Beethoven não seria Beethoven, se em tudo o que ele fosse, não o fosse
intensamente. Terminou o movimento e Bárbara, emocionada, esperou pelo
``Scherzo''. O locutor anunciou um azeite de boa qualidade e prosseguiu
num desfile de bolos e pratos diversos que se podiam fazer com a tal
gordura. Bárbara desanimou. Lembrou-se então de um recital que assistira
em São Paulo, em cuja direção orquestral estava o maestro G. Entre o
primeiro e segundo movimento da peça executada, uma onda de pessoas
atrasadas invadiu a plateia. O maestro descansou a batuta sobre a
partitura e, voltando-se de frente para o público, cruzou os braços em
silêncio. A sua atitude parecia perguntar àquela gente. --- Vocês não
compreendem?

\chapter{Capítulo 33}

Bárbara não fora ao chá de Ivete, mas combinara com Helena, fazer-lhe
uma visita no sábado. Telefonou à amiga e às quatro da tarde, no dia de
sábado, Bárbara e Helena dirigiram-se para a residência dos Macedo.

Bárbara vestiu-se toda de branco. Um vestido justo, discreto, que lhe
desenhava as formas. Tinha mangas compridas e um apanhado farto em
panos, arrematando em pregas no decote justo ao pescoço. Completando,
uma placa não muito grande, mas de pedras de primeira água. Um
chapeuzinho leve, bolsa e luvas brancas.

Helena também se trajava adequadamente. Toda de vermelho, trazia em
realce os seus cabelos loiros.

Pararam o carro na residência de d.\textsuperscript{a} Alda.
Introduzidas cerimoniosamente na sala de visitas, Bárbara pôde ver
algumas telas raras de pintores brasileiros. Estavam ali também cópias
das telas de Pedro Américo, Almeida Júnior e outros. Ao lado, um Goyen,
que a deixou extasiada. Era a representação de uma colina em cujo
primeiro plano se viam alguns cavaleiros na estrada; fazendo fundo ao
quadro, uma cidade distante, deixando ver a parte alta das casas e as
torres das igrejas. E depois, um céu azul, de grande extensão, cuja
escala de tonalidades era a mais bela que Bárbara já observara. A
simplicidade do motivo, a extraordinária beleza daquele firmamento,
fizeram-na esquecer o tempo que esperou ali. Viu ainda umas aquarelas de
Casciaro e uma cópia da ``Flora'' de Ticiano. Porcelanas diversas;
miniaturas antigas; um elefantezinho de marfim, notável pela sua
escultura, e muitos outros objetos de arte que chamaram a sua atenção.
Percebeu que tudo fora arranjado por pessoa de gosto e critério
artístico. Os objetos autênticos estavam completamente separados das
cópias, aliás, valiosas pela perfeição. Bárbara não desprezava as
cópias. Conquanto pusesse em primeiro plano o artista criador, tinha
também o seu culto por aqueles que sabiam reproduzir com arte.

D.\textsuperscript{a} Alda entrou na sala, pedindo desculpas pela
demora, e cumprimentou-as com grande amabilidade.

--- Sentimos tanto a sua ausência Bárbara. Porque não veio? --- disse a
senhora.

--- Estive um pouco cansada com estas coisas de mudança. Mas venho a
tempo de desejar um futuro risonho à sua Ivete.

--- Conhece o rapaz? É o Roberto Bastos, filho do Dr.~Bastos: já não
ouviu falar?

--- Conheço-os de nome.

--- É família conhecida e de tradição no Brasil. Não faz parte dos
grã-finos enriquecidos; traz a fortuna dos ascendentes mais remotos. A
baronesa de Serra Azul é a última descendente que traz o título pelo
Estado.

--- O título da baronesa, ainda foi pelo Estado?! --- perguntou Bárbara.

--- Deve ser... --- respondeu gaguejando, pois sentiu a insegurança do
terreno.

Bárbara calou-se, deixando d.\textsuperscript{a} Alda expandir-se quanto
às qualidades do rapaz. Ouviu nomes ilustres que aureolavam a família e
certificou-se ao mesmo tempo, de que a mãe arranjara a filha, conforme
seus pontos de vista.

--- Bem --- concluiu d.\textsuperscript{a} Alda --- vou mandar
chamá-las.

Tocou a sineta antiga, uma das raridades da sala, e ordenou ao criado
que chamasse as meninas.

Bárbara, que ouvira tudo com atenção delicada, ficou satisfeita ao ver
que o assunto parecia encerrado.

As meninas entraram e cumprimentaram as duas amigas.

--- Vim trazer-lhe os meus votos de felicidade, Ivete --- disse-lhe
Bárbara.

--- Muita gentileza sua. Obrigada.

--- E você, Lia, como está?

--- Muito bem --- respondeu sorridente a mesma.

--- E Clarisse, por que não veio com você, Helena? --- indagou
d.\textsuperscript{a} Alda.

--- Vim para fazer companhia à Bárbara, d.\textsuperscript{a} Alda.
Mamãe não esperava esta visita de hoje. De outra vez, virá conosco;
apesar de que a senhora não pode se queixar, pois, a sua... é a casa em
que mamãe vem mais.

--- Naturalmente. Amizades velhas, desde São Paulo.

--- É sim --- respondeu atenciosamente Helena.

--- E você não pensa em casar-se? --- indagou d.\textsuperscript{a}
Alda.

--- Ainda não chegou a hora --- respondeu Helena.

--- Quando se casar, Clarisse ficará só. Por isso não é bom filha única.

--- Mas com a senhora pode se dar o mesmo, se Lia se casar.

--- É verdade --- ponderou d.\textsuperscript{a} Alda --- aparecendo
alguém para Lia, pronto. Lá se vão as minhas filhas.

Lia, silenciosa, levantou os olhos ternos e pareceu longe dali. Bárbara
desconfiou que na vida de Lia já existisse esse alguém.

--- Acho difícil Lia casar --- tornou Ivete com ironia --- é distinta
demais para isso.

--- Por que fala assim de sua irmã? perguntou --- Helena admirada.

--- Por quê, Helena? É simples. Ela tem lá os seus princípios errados e
teima em segui-los. Quis ensiná-la muitas vezes, mas era tão bobinha que
acabava irritando-me terrivelmente.

--- E o que você sabe para ensinar?

--- Sei agradar os rapazes\ldots. e. também escolher um casamento. Prova
que sei é que me caso à minha altura, mal tendo completado dezessete
anos. Enquanto que ela...

--- Continue Ivete --- disse Lia calmamente.

--- ... não tem um admirador, sequer.

--- Não pode afirmar isto --- tornou Helena cada vez mais admirada.

--- Como não? Vejo os rapazes que se afastam dela. É muito rígida, e
além disso, estuda mais do que deve; logo, sabe mais que os rapazes.
Ora, não é nada agradável aos homens, encontrar alguém que os supere em
alguma coisa. Eles são o sexo forte, e nós, mulheres, devemos fazê-los
cientes de que nos sentimos seguras sob sua proteção.

--- Minha filha --- redarguiu d.\textsuperscript{a} Alda espantada, mas
um tanto orgulhosa pela sagacidade de Ivete --- onde aprendeu tais
coisas?

--- Na experiência, mamãe. Comparava os acontecimentos entre mim e Lia.
Eu estava sempre cercada de admiradores... ao passo que Lia, vivia
sempre só.

--- Mas uma vez casada --- tornou Helena --- não será mais rival para
sua irmã.

--- Duvido! Lia é retraída, esquiva. Hoje em dia, é preciso ser esperta.

--- Você foi esperta?

--- Ora, se fui! Diverti-me a valer. Caso-me cedo, mas aproveitei
bastante. Fazia das minhas, e nunca me preocupei; porque os rapazes de
hoje não veem um palmo adiante de si. Tomei certa cautela, para
aparentar aos outros que era apenas divertida. Assim é a vida! Os
rapazes não são psicólogos. A psicologia está nos programas dos prés,
mas, até que eles deem por ela!...

Bárbara estava numa irritação crescente com a preleção de Ivete.
Continuava não admitindo que uma criança fútil como Ivete, pensasse em
formar um lar. Olhou para d.\textsuperscript{a} Alda e a sua irritação
cresceu ainda mais. Pensou em dizer alguma coisa que tivesse um eco
diferente naquela família, mas a voz de Ivete continuou:

--- Fiz dezessete anos no dia do meu noivado; as minhas ações serão
desculpa das criancices... e depois, não se esqueçam, eu era apenas
divertida.

Neste momento entrou Carlito. Vendo Bárbara, não escondeu a satisfação
de encontrá-la em sua casa e foi diretamente a ela:

--- Até que enfim a vejo novamente!

--- Não faz tantos dias assim que não nos vemos --- respondeu Bárbara
com amabilidade.

--- Estamos no verão; os dias são longos --- tornou o rapaz. E voltando
para Helena perguntou:

--- E você, Helena, que me conta?

--- Nada, Carlito. A vida recaiu na rotina.

--- Vamos tirá-la, então.

--- Como?

--- Indo a algum lugar. Que tal um Cassino?

--- O Atlântico? --- indagou Helena.

--- Concorda, Bárbara? --- acentuou Carlito, olhando para ela.

--- Naturalmente, Carlito --- volveu Bárbara, pensando que os convites
do rapaz geralmente eram para um cassino.

--- Em boa ocasião nos encontramos. Vamos todos ao Atlântico! ---
exclamou entusiasmado o rapaz.

--- Atlântico?! Oh! como eu gostaria de ir! E penso que não há nada...
Dançarei com maior seriedade --- disse Ivete, pensativa.

--- E Roberto, com quem dançará? --- retorquiu Carlito em atitude
provocadora.

--- Ah, eu não posso mesmo --- corrigiu Ivete imediatamente. --- Hoje
vem aqui a bordadeira; preciso decidir muita coisa. Um enxoval dá tanto
trabalho!...

--- Se soubesse que era assim, não se casaria, não é? --- continuou o
irmão.

--- Nem tanto, Carlito. Há trabalho, mas haverá compensações.

--- Faço votos que as descubra logo.

--- Pois você, que é inteligente, procure-as por mim. Dizendo isto,
retirou-se contrariada, sem pedir licença.

--- Não faça isto, meu filho. Ivete anda nervosa com o casamento, disse
com doçura d.\textsuperscript{a} Alda.

--- Mamãe, Ivete se casa mesmo?

\chapter{Capítulo 34}

O tempo mudou de repente e Bárbara chegou em casa sob um temporal
tremendo. Ainda antes de sair do carro, viu luz na sala. Abriu a porta
de entrada que, contra seus hábitos, não estava trancada a chave. Na
sala, em pé e de costas para a entrada, Álvaro lia uma revista. Não lhe
foi difícil distinguir a da Academia Brasileira de Letras; pois, além do
seu aspecto característico, ela a estivera lendo ali. Bárbara
aproximou-se e dirigiu-lhe a palavra.

--- Entrei assim tão de leve, ou será alguma coisa extraordinariamente
interessante?

--- Ó Bárbara --- disse ele deixando a revista e vindo ao seu encontro.

--- Vê que espetáculo? Uma chuva destas e eu toda de branco --- disse
ela sorrindo.

Ele a admirou naturalmente:

--- Bonito vestido!

--- Bem --- concluiu a moça --- com licença um momentinho.

--- Seus cabelos não estão molhados? Seria conveniente enxugá-los ---
recomendou o rapaz atencioso.

--- Serei cautelosa como a serpente --- gracejou Bárbara.

--- E simples como a pomba?

--- Farei o possível.

Bárbara afastou-se; foi guardar o chapéu, luvas etc. e passar um pente
nos cabelos. Ele ficou na sala e, aproximando-se da vidraça, pôs-se a
olhar a chuva, que caía incessantemente. Um relâmpago iluminou a rua e
um trovão ecoou a seguir pela casa.

--- Será que o mundo veio abaixo? --- falou Bárbara entrando na sala.

--- Talvez --- disse ele voltando-se.

--- Nem percebi como passamos da tarde à noite. Escureceu tão depressa.

--- E como foi de visita?

--- Como sabe que eu saí para uma visita?

--- A sua governanta me falou quando cheguei; há uns vinte minutos.

Bárbara fez uma expressão de quem compreendera e Álvaro perguntou ainda:

--- Gostou?

--- Não desgostei. E por falar nisso, vou telefonar, desfazendo um
encontro.

--- Por minha causa, não. Vou sair logo --- interrompeu.

--- Não é por sua causa e nem pode sair com este tempo.

--- Insisto na minha saída, Bárbara.

--- Convido-o para jantar comigo.

--- Num outro dia --- respondeu meio desapontado.

--- Mas hoje está aqui. E eu já mandei servir --- continuou Bárbara
compreendendo a sua atitude.

--- Receio estar sendo importuno --- disse em tom sério.

--- Se o fosse, tratá-lo-ia como tal --- respondeu a moça com
naturalidade.

Álvaro ficou ali parado, sem saber que direção tomar; até que Bárbara o
interpelou.

--- Vamos jantar; pois já é hora.

Ele a seguiu constrangido; sentia-se um intruso naquele ambiente de
conforto. Olhou para Bárbara, quando já sentados à mesa; a serenidade do
seu semblante deixou-o mais à vontade. E a intuição lhe dizia que eles
eram dessas pessoas que se entendiam desde as primeiras vezes.

* * *

Terminado o jantar, ambos se dirigiram para o terraço. A chuva cessara,
mas a noite continuava escura. Como Álvaro se mostrasse mais quieto,
Bárbara iniciou a conversa.

--- No passado, eu costumava comer a sobremesa, caminhando --- disse
ela.

--- Teve também um passado? --- perguntou Álvaro.

--- Quem não o teve? Afinal, a vida não começa aos quarenta.

--- Dou, a mão à palmatória. Vejo em você somente o presente --- afirmou
ele convicto.

--- Vivo o presente, sim. Mas que é a experiência senão a herança do
passado? É como se lêssemos um livro; o fato de estarmos no fim, não
impede a lembrança do começo.

--- Afinal, estamos falando como se já tivéssemos os quarenta ---
comentou a rir.

--- Invertamos então. Falemos dos vinte --- voltou Bárbara naturalmente.

--- Não tenho nada de interessante nos dias que já vivi.

--- O fato de viver as coisas pela metade, como disse várias vezes, já é
algo curioso e interessante. Se fôssemos esperar acontecimentos
sensacionais e completos, passaríamos a vida catalogando fatos,
julgando-os, para ver se podem constituir material de recordação ou não.
Perderíamos o presente na pesquisa do passado, quando o verdadeiro
prazer da vida, nós o encontramos, quase sempre, nas coisas de todos os
dias --- continuou a moça com a mesma naturalidade.

--- Eu perdi esse prazer, e com isso me tornei indiferente.

--- Tem certeza de que o perdeu?

--- Não o sinto em mim. E por isto mesmo caminho sem direção.

--- Muitos caminham sem direção --- ponderou Bárbara admirada pela
franqueza do rapaz --- poucos, porém, o reconhecem. E este
reconhecimento, Álvaro, não mostra o outro lado do caminho?

--- Eu já abri falência na vida.

--- Mas, eu sei que você luta --- insistiu a moça --- percebo isto
claramente.

Ele não procurou esconder a sua guerra íntima. E aquela voz amiga,
ouvida em momento difícil, deixou-o trair os seus sentimentos:

--- Luto, é verdade; porém, não o bastante. E na luta, só há duas
alternativas: vitória ou derrota. Não se tomando tudo do inimigo, o
inimigo toma tudo da gente. Neste ponto, Bárbara, admiro as mulheres;
têm mais força para encarar as coisas.

--- Acostumam-se a maiores sacrifícios; a sociedade exercita a mulher
nas suas forças.

--- Elas são o sexo fraco... --- ponderou com ironia.

--- E por isso mesmo, exigem dela todas as forças --- tornou a moça.

--- Enquanto que ao sexo forte permitem todas as fraquezas --- concluiu
o rapaz. --- A vida é isto mesmo, toda feita de injustiças. E ainda há
quem proclame a existência de um ser eternamente justo.

--- Deus?

--- Deus ou outro princípio qualquer; depende da imaginação de cada um.

Bárbara não respondeu. Falar àquele homem numa justiça que ela
acreditava existir, seria lembrar a água a quem morresse de sede no
deserto. Como não recebesse resposta, o rapaz nada mais disse. Ficaram
ali no terraço, encostados ao peitoril de cimento. Álvaro, um pouco mais
para trás, olhou para Bárbara e começou a pensar no modo original por
que ela entrara na sua vida. Tornou a notar o seu belo rosto, as linhas
acentuadas da sua plástica; aliás, essa particularidade não escapara a
ele, rapaz experimentado na vida livre. Havia muito que ele encarava uma
mulher pelo prazer que ela podia proporcionar-lhe; mas, ao lado de
Bárbara, parecia-lhe intolerável um pensamento destes. Seria ela que se
impunha vigorosamente, ou alguma chama de sentimentos bons do passado
que reviviam nele?

Despir aquela mulher, mesmo no pensamento, do encanto natural com que
lhe aparecera, seria um crime hediondo; um abominável assassínio! Tudo
que restava de nobre naquele homem vencido da vida, reergueu-se numa
defesa espontânea; num instinto de preservação de quando se encontram as
coisas boas da própria vida.

Um casal de namorados parou à vista deles, e, entregues a si mesmos,
ignorando que alguém os visse, trocaram um beijo apaixonado. Álvaro
contemplou-os.

--- São ainda jovens --- disse.

--- Parecem ter dezoito anos. Nesta idade, nem se sabe ainda o que é o
amor --- comentou a moça.

--- Nesta idade, também, um beijo decidiu a minha vida. Faço votos que
para eles não aconteça o mesmo.

--- Amou? --- perguntou Bárbara com naturalidade.

--- Não --- respondeu Álvaro --- casei-me.

\chapter{Capítulo 35}

Houve um silêncio prolongado. Após sua inesperada revelação, Álvaro
voltou-se para Bárbara sem dizer palavra. Ela percebeu que o rapaz
desejava falar; assim abriu-lhe o caminho, fazendo uma pergunta:

--- Você não foi feliz, Álvaro?

--- Não teria sido possível. As condições do meu casamento só poderiam
trazer infelicidades.

--- Não teve com quem aconselhar-se antes?

--- Nestas coisas os homens nunca pelem conselhos; consideram-se mais
livres que as mulheres.

--- E seu pai, Álvaro, não lhe disse nada?

--- Morreu logo após meu casamento, sem saber que eu cometera tal
loucura. Talvez tenha sido melhor assim, pois, se soubesse, poderia ter
sofrido por isso. E depois, Bárbara, entre pais e filhos não existe
intimidade para tais coisas.

--- Se não existe entre eles, entre quem deveria existir? Com os pais,
vivemos os primeiros anos de vida. E nestes primeiros anos, não
conhecemos as manhas do mundo; entregamo-nos confiantes às pessoas que
convivem conosco e que nos dedicam o seu carinho e afeição. Deste
convívio nasce a intimidade.

--- A minha vida não foi assim. Aliás, isto me pareceu tão natural, que
me causam espanto os casos que diferem do meu.

Bárbara lembrou-se de seu pai. Da sua dedicação constante, do seu
carinho, da sua compreensão. Lembrava-se de como contara com ele nas
suas alegrias e nas tristezas, embora, fosse ainda criança. Era algo
diferente dos pais que vira até agora. E porque fora assim, Bárbara o
perdera! Seria isto possível?! Permaneceria no mundo somente o que fosse
imperfeito? Sentiu que este pensamento a perturbava, e desta perturbação
nasceu o gesto de profunda simpatia pela sorte daquele rapaz. Voltou-se
para ele com uma expressão acolhedora, e Álvaro, como se esperasse
unicamente por isso, continuou a falar:

--- Eu era ainda muito criança quando fui interno num colégio do
interior. Meu pai tinha uma indústria de vidros nesta cidade, e achando
que a minha educação seria imperfeita ali, mandou-me para a Capital. Mal
saí do colégio, já ficou resolvido que iria para São Paulo. Com treze
anos, inexperiente ainda, iniciei os estudos num estabelecimento
recomendado por amigos de meu pai. Mas desta vez fui só e, querendo ser
mais livre, entrei numa república de rapazes. Ali começaram as minhas
quedas. Iniciei-me no jogo e na bebida. Tinha dinheiro demais e não só o
entregava aos meus companheiros pelas cartas, como o emprestava sem
devolução. Papai vivia ocupado, para examinar os meus gastos. Não tendo
tempo para mim, externava o seu carinho por cheques bancários que eu
recebia satisfeito. Mais tarde, um ano antes de concluir o ginásio,
conheci um rapaz que se tornou meu maior amigo. Era simples, pobre e
sincero. Orientou-me com a experiência de quem luta pela vida; e eu,
contente por encontrá-lo, fui morar com ele num quartinho escuro de
pensão barata.

Tirou outro cigarro do bolso, acendeu-o, deu algumas baforadas,
certificou-se da atenção de Bárbara e continuou:

--- Foi quando eu vivi. Faltava um ano e meio para concluir o ginásio.
Estudei metodicamente, e, guiado por ele, prestei exames na faculdade
com dezesseis anos. Cursei o primeiro e o segundo ano... aí cessaram as
coisas boas para começarem as coisas más.

Bateu o cigarro, cuja cinza se avolumava na extremidade e, após curto
intervalo, prosseguiu num tom de angústia e de desprezo:

--- Naquela temporada, encontrei Dalila em São Paulo. Fôramos
companheiros de infância e chegamos a ter grande intimidade. Vimo-nos
muitas vezes então; e a intimidade de criança abriu novos caminhos. A
mãe dela entrou em cena, ajeitou tudo de tal maneira que eu vim a ceder
por completo: casei-me com ela. Mal fizera isso, percebi o passo em
falso que dera; mas era tarde. E uma semana depois, antes de pensar em
qualquer solução, perdi meu pai.

O casal de namorados passou. Álvaro silenciou até que eles se
afastassem.

--- Quando parti para casa, já avisado da moléstia grave que o
acometera, esperava encontrá-lo capaz de levantar ainda. Não. Ele
faleceu dois dias depois e eu percebi que estava inquieto com a minha
sorte.

Olhou para Bárbara, e ela sorriu-lhe tristemente; testemunhava-lhe o
interesse com que o acompanhava na sua história.

--- Voltei para São Paulo, abatido com o que me acontecera. Percebi
então que a morte de meu pai deixava um vácuo na minha vida. Conquanto
não o procurasse, contara com ele sempre. Quando me faltou o seu apoio,
fui logo encostar-me ao primeiro poste da esquina. E você acredita
Bárbara que, mesmo sem a convivência habitual, eu gostasse de meu pai?

--- Acredito. A posição de pai já é alvo de todo afeto.

--- Afinal --- disse ele tristemente --- não sei por que a enfado com
estas coisas...

--- Não me causam enfado --- insistiu Bárbara.

--- Estou desabafando a minha dor; não tenho direito a fazer isso.

--- Continue --- tornou Bárbara com insistência.

Álvaro acendeu o cigarro apagado, olhou para a brasa que se formara e
voltou-se para ela:

--- Uma vez em São Paulo, continuei o meu curso; não dependia do
trabalho para viver. E quando não se trabalha para ganhar, não se sabe
avaliar o dinheiro. Talvez fosse melhor para os filhos que os pais não
lhes deixassem haveres; que se iniciassem na escola do trabalho,
obrigados a explorar a vida por si mesmos. Formar-se-iam homens mais
capazes. Você pode julgar-me um inepto, mas fui fruto da educação que
recebi. Deveria reagir, é verdade; mas o homem acostuma-se depressa com
o que lhe é mais fácil.

--- E terminou o curso?

--- Não. Uma série de complicações transformou o rumo das coisas, sendo
a financeira a mais imperiosa.

--- A financeira?!

--- Parece incrível; o dinheiro, custa ganhar, mas perde-se num
instante.

Levou o cigarro à boca novamente; e olhando em linha reta, como se
falasse consigo mesmo, prosseguiu:

--- E nesse tempo que vivêramos juntos, conheci o inferno descrito por
Dante. Para esquecer este inferno, atirei-me ao jogo, à bebida, aos
divertimentos mundanos. Diz o ditado: ``o dinheiro não aguenta
desaforos''... é bem verdade. Esquecido disto, contraí dívidas, fiz
loucuras. Quando terminava o quarto ano, saíram a protesto as minhas
primeiras letras e com isso perdi todo o crédito. Felizmente, em tudo há
o lado menos mau; estávamos já em férias e eu pude desaparecer da
faculdade sem grandes comentários. Em dois anos, joguei fora o que meu
pai acumulara em trinta. Envergonhado, não compareci para terminar o
curso; e, com o tempo, deixei de pensar na carreira.

--- Pena que não a concluísse. Teria algo de mais objetivo para a sua
vida.

--- Teria \textsubscript{que} ser atingida também; uma desgraça nunca
vem só. E depois, como se apresentaria um advogado falido; cheio de
títulos protestados? Só me restaria ser professor no ensino secundário;
a profissão aqui no Brasil que acolhe os fracassados nas demais.

--- Por quê?

--- Simplesmente porque não há a concorrência dos que realmente se
orientam neste sentido. Aos concursos comparecem poucos; e colégios há
muitos, pois é um dos bons negócios hoje em dia.

--- Poderia começar a vida em outro lugar.

--- Foi o que fiz, vindo para o Rio. Um lugar novo, entretanto, não nos
livra de pensar no passado.

--- E essa mulher vive ainda?

--- Vive; ainda por aí. Há seis anos que nos separamos; sem dinheiro, eu
não lhe interessava.

--- E o seu interesse por ela, já considerou isso?

--- Tenho consciência de que não a amei. Houve tempos de ter-lhe até
rancor; mas, pelo hábito de viver junto, adiei inconscientemente a
solução definitiva. E quando nos separamos, sabe o que ela me disse?

---?

--- Espero que se vá sem ressentimento. E finalmente... você não vai
zangar-se por eu ter guardado algum dinheiro; você estava gastando
tanto! Fui apenas previdente.

--- E o seu amigo, Álvaro?

--- Fez o que pôde por mim. Eu, porém, não quis pesar-lhe com os meus
problemas.

--- E são amigos até hoje?

--- Até hoje. Embora estejamos separados, mantemos correspondência; e
ele algumas vezes aparece. Teve a má sorte de apaixonar-se por uma moça
da classe alta, socialmente falando. Sendo de família simples e
desconhecida, foi afastado pelos pais dela, que pertenciam aos grã-finos
da cidade. Retirou-se então para o norte, onde foi construir estradas e
afogar no álcool as suas desditas. O mundo é isto mesmo. Que tem ele
para ser apreciado? Penso como Schopenhauer, ``a alegria é a cessação da
dor''. E ainda digo mais, é um pequenino intervalo na própria dor.

\chapter{Capítulo 36}

Álvaro retirou-se tarde. Tendo fechado a porta, Bárbara moveu-se pela
casa silenciosa; uma emoção esquisita deprimia a sua alma. Olhou para
aqueles móveis tão familiares e eles lhes pareceram fora de lugar. Teve,
então, plena consciência da sua mudança para o Rio. E num gesto natural
e muito humano, deixou-se transportar para o passado. Para os dias mais
alegres, cuja recordação se traz permanentemente. A casa de São Paulo
reconstruiu-se na sua imaginação--- os arranjos, as velhas amizades que
nela recebera. Mrs.~Patrice andando pelos cômodos, o jardim com as suas
flores prediletas, os Heliotrópios que não trouxera, o crepúsculo da
cidade industrial... tão lindo, contemplado da sua casa, lá no bairro de
Santa Cecília. Era certo que, agora, estava em lugar diferente; e aquela
noite pareceu-lhe a primeira na casa do Rio.

Os pensamentos acorriam-lhe em sequência e Bárbara não procurou fugir a
eles. Foi para o quarto e, enquanto mudava os trajes do dia pelos da
noite, Álvaro veio-lhe à mente repetidas vezes.

Recordou as suas atividades no Rio; ainda não se adaptara a ponto de ter
a vida organizada, mas, seus hábitos impuseram-se desde os primeiros
dias e Bárbara os seguira normalmente.

Deitou-se. Na penumbra do quarto semi-iluminado, pensou na vida. Suas
atividades sociais vieram-lhe ao espírito: chás, visitas, o jogo na casa
de Helena, o cassino da Urca, a praia, as pessoas que conhecera,
Carlito... a fisionomia alegre e despreocupada do rapaz, para quem as
lutas da vida eram vistas de relance. Passava por elas como se estivesse
na sua linda barata... em tudo via o bafejo do vento agradável,
acariciando o para-brisa do carro. Para Carlito a vida era um conjunto
de coisas fáceis; era um riso permanente.

Por essa lei que rege os contrastes, Álvaro apareceu de outro lado. Para
ele, o vento não tinha aparador; eram as suas faces o para-brisa humano
em que o vento vinha bater. Andava a pé no terreno barrento,
escorregadio; parando, algumas, vezes, para contemplar o barro amassado.
E tivera no passado as mesmas coisas que Carlito ainda tinha no
presente. Onde colocar, agora, esse rapaz que vivera duas vidas? Tornar
ao meio antigo, seria impossível; o Álvaro de hoje era outro--- não
caberia nos conhecimentos do passado onde figurara como filho de papai
rico. Novos conhecimentos? A sua provação fora tremenda e ele não estava
preparado para recebê-la. Dessa provação adveio um desprezo incontido
pelas coisas da vida. Desprezava os preconceitos e ria abertamente dos
tesouros valiosos, objeto de luta entre os homens. Álvaro desajustara-se
na sociedade e, sentindo o seu desajustamento, isolara-se por completo.

Bárbara apagou a lâmpada fraca do quebra-luz e seus olhos, aos poucos,
fecharam-se para o sono. Na semi-inconsciência de quem adormece, imagens
confusas passaram diante dela; o sr. do ``packard'' preto, o condutor do
caminhão, Álvaro, d.\textsuperscript{a} Alda, Ivete, Lia, Carlito,
Álvaro outra vez, e, de repente, uma pessoa já esquecida--- a moça loira
que jogara no cassino. Depois, veio Paulo Cunha, a fisionomia ainda
incerta do Dr.~Renato que ela vira de passagem na casa de Helena,
novamente Álvaro, e por fim, Helena. Helena parecia estar no mesmo
caminho de Álvaro, eram dois gritos iguais que Bárbara ouvia. Helena...
e Bárbara adormeceu.

\chapter{Capítulo 37}

Bárbara combinara com Álvaro o seu primeiro passeio. Ia tentar
infundir-lhe coragem para a luta. Sabia que nem tudo estava extinto nele
e que alguma brasa poderia reavivar-se, ateando os ideais perdidos.
Escolheram Paquetá, ``a ilha enamorada dormindo ao luar''.

Começava, o dia. Tomaram a primeira barca e partiram... escritor e
desenhista.

Na sua marcha lenta, imutável, a barca deixou o Rio e ancorou em
Paquetá. Deslizou preguiçosamente pelas águas densas, e, balançando-se
indecisa, deixou os passageiros.

Bárbara e Álvaro seguiram a marcha. Andaram pelas ruas, levaram frutas
de uma quitanda na passagem, e, de bolsa a tiracolo, prosseguiram o
passeio de bom humor.

--- Só você mesmo me faria levantar assim tão cedo! --- disse o moço.

--- E vai continuar? --- perguntou como a esperar uma só resposta.

--- Mas há vantagem nisso? --- indagou o rapaz meio descrente.

--- Há; a de começar o dia lutando.

--- É mesmo uma luta, deixar a cama de manhã --- concluiu.

--- E, depois, olhe a manhã; tanta luz, tanta vida! Foi feita para a
atividade --- disse Bárbara convencida.

--- É... eu pensava de outra forma.

Ela sorriu, e continuaram o caminho.

--- Para onde vamos?

--- Que pergunta! Em Paquetá, há de ser a Pedra da Moreninha ---
respondeu Bárbara.

--- Então vamos recordar Macedo.

A paisagem apresentava uma frescura matutina; o verde das plantas tinha
um brilho viçoso, aprazível. Passaram por pequenino bosque e pararam ali
para ver os raios do sol, que penetravam pelas gretas da mata. Listas de
luz e de sombra pelos intervalos dos arvoredos, formavam a bandeira
viva, verdadeira insígnia da natureza.

Na sua marcha longa, rumo à Moreninha, os excursionistas daquela manhã
rendiam culto à natureza, pois, diante das paisagens de Paquetá viveu
neles apenas a emoção. Não os impressionou as formas dos contornos ou a
combinação das cores suplementares. Era a criação de um artista
sobre-humano, cuja harmonia dos elementos sugeria ao expectador a
impressão de uma arte infinita.

Chegados à Pedra, instalaram-se lá no alto e Bárbara foi a primeira a
iniciar o seu trabalho. Bem depois, Álvaro seguiu o seu exemplo.

Permaneceram calados e à distância um do outro. Bárbara não lia os
escritos do rapaz e ele, por sua vez, não via os desenhos que ela
tentava fazer. O tempo avançava sem que o percebessem. Outros
excursionistas passavam ao longe, e eles não os viam. Álvaro tomara
impulso e escrevia sem cessar. Se parava com o lápis, continuava no
cérebro e, neste afã, o trabalho sucedia-se indefinidamente. Passaram-se
as horas. O sol ficou a prumo; queimou-os mais, e iniciou a queda para o
poente.

--- Que horas são? --- perguntou Álvaro, jogando papel e lápis.

--- Três horas, Álvaro! --- exclamou a moça admirada.

--- E o nosso almoço? Ainda não me acostumei a viver de brisa.

Tinham trazido tudo. Bárbara abriu os embrulhos e só então lembraram de
que o estômago também tem exigências.

--- Estou com sede --- disse ela a percorrer os arredores com o olhar.

--- Vou buscar água --- volveu o rapaz levantando-se. --- Ou prefere
refresco?

--- Á água é o melhor refresco, para mim.

Álvaro desceu até ao restaurante e regressou, trazendo nas mãos garrafa
e copo.

--- Está gostando, Bárbara?

--- Está muito agradável.

--- Sairia outras vezes comigo?

--- Com grande prazer, Álvaro.

--- E desenhou alguma coisa?

--- Tentei esta vista. Mas, é bastante extensa e eu estou sem exercício.

--- Tentará novamente, então.

--- Percebi que é simples\emph{;} mas não posso abusar da sua
simplicidade.

--- Magoa-me, ao falar assim. Deixe-me agir como eu achar que me agrada.

--- É o que eu mais desejo...

--- Então, já está me agradando.

Declinava a tarde. O cenário era amplo e suave. O vulto esbelto e esguio
das palmeiras projetava-se em sombra nas areias ou tremulava no movediço
das águas. Do sol em decadência vinham os últimos raios da iluminação
abundante e viva.

Escurecia... Bárbara e Álvaro puseram-se a caminhar.

--- É tarde; vamos tomar a última barca de volta, não é Álvaro?

--- Vamos sim, Bárbara. E vamos também levar Paquetá conosco.

--- Vê Álvaro? Já estamos à luz lunar --- observou a moça.

--- É verdade, lá está o disco de prata. Que espetáculo magnífico! Por
isto, o poeta cantou o luar de Paquetá," a ilha enamorada dormindo ao
luar".

\chapter{Capítulo 38}

Álvaro chegara em casa, profundamente emocionado. Separara-se de Bárbara
no largo da Carioca e viu-a tomar o bonde de Santa Teresa. Permaneceu
ali até perdê-la de vista; depois, comprou cigarros e seguiu para o seu
quarto pobre de pensão de segunda. Sentou-se à escrivaninha e continuou
o trabalho iniciado. As ideias lhe ferviam no cérebro. Sentiu forte a
vertigem do escritor que inicia a sua carreira por uma necessidade
íntima, imperiosa. Com facilidade pensava, com facilidade transportava
para o papel seus pensamentos. Acendeu o usual cigarro e deu umas voltas
pelo quarto, disfarçando a sua agitação. Tudo se tornava diferente para
ele. Observava, aos poucos, coisas que lhe passaram despercebidas antes;
e, sem o notar, começou a sentir um leve gosto pelo viver. Prestava já
mais atenção aos seus trajes, andava sempre barbeado, procurava manter a
linha que há tantos anos desleixara. Parecia, que ao abrir-se de uma
cortina, um mundo novo amanhecia para ele. Seria escritor? Faria alguma
coisa? De repente, uma outra ideia--- a sua carreira. Valeria a pena
continuá-la?

Álvaro riu de si mesmo. Continuar a carreira! Como? Estudava-se por
acaso, no Brasil, sem dinheiro? Que era um curso superior na sua
generalidade? --- A representação de um capital, o que a maioria
pensava.

Nisto, outra ideia--- e o ensino secundário? Não poderia lançar mão
dele, nas suas dificuldades? Quem sabe, à custa do secundário, cursaria
o superior. O ensino secundário não abrigara sempre tantos fracassados?
Não teria, então, lugar para mais um? Ninguém iria saber a sua falta de
especialização em determinada matéria; quando estivesse em dificuldade,
lançaria mão dos seus conhecimentos gerais. Falando em assuntos
diversos, os próprios alunos respeitariam os seus conhecimentos, e ele
seria capaz de tornar-se uma celebridade. Não ouvira tantos elogios a
meros repetidores de fórmulas livrescas? E livros para o ensino
secundário existiam muitos; a ponto de os professores mudarem a escolha
de ano para ano, impedindo assim que os estudantes adquirissem livros de
colegas mais adiantados. E não andava agora uma campanha contra os
formados pela faculdade de filosofia? Entraria nesse meio e seria um
professor do ensino secundário. Dizem que no Brasil os professores
nascem feitos... Quem poderia negar que também ele não tivesse nascido
professor?

Não bastariam as dificuldades para definir uma vocação? Os bons
professores de faculdade, segundo os comentários mais respeitáveis, eram
os da faculdade da Praça em São Paulo, mas, estes eram poucos, e os
concorrentes de faculdade livres não fizeram, ainda, sentir a sua força.
Qual, o caminho estava aberto; tentaria ser professor no ensino
secundário e faria tudo para ser digno do lugar.

Restava, agora, escolher a disciplina: inglês? Não; todos estudavam
inglês no Brasil. Poderia encontrar um aluno que soubesse mais do que
ele. Francês? Que coisa horrível! Os alunos não queriam saber do
francês; já há algum tempo que esta língua passara ao domínio da classe
culta do país. Português? Os estudantes não se interessavam pela língua
nacional; ele não tinha prática de lecionar; tomar ainda, matéria sem
interesse na aprendizagem? Ademais, tantas eram as línguas que se
ensinavam no Brasil que o idioma pátrio perdera, aos poucos, a sua
significação própria. Praticamente, o povo brasileiro esquecera-se de
que a língua é o primeiro elemento nacional com que o indivíduo toma
contato.

Como era difícil escolher! Mas era preciso que escolhesse; não poderia
ficar sem trabalho. Lembrou-se de um antigo professor, no ginásio, que
lecionava seis matérias; português, latim, ciências físicas, física,
matemática e literatura. Como fora ele escolher tantas e Álvaro ali, em
dificuldades para escolher uma, apenas.

Atravessou-lhe o espírito uma ideia luminosa: a geografia não se estava
ligando aos estudos sociológicos? Não poderia juntar à descrição dos
países, questões sociais de povos que os habitavam? Conhecia a
sociologia, estudara-a por um gosto próprio. Além disso, tivera
oportunidades para discutir com o amigo, firmando assim os seus pontos
de vista.

Álvaro estava inclinado a lançar mão daquele meio de vida, quando a
consciência gritou mais alto. Saberia dosar pedagogicamente as aulas de
geografia? Ele que tivera dificuldades para aprender porque lhe faltaram
bons mestres, iria fazer a mesma coisa? A sociologia que ele cultivava
não estava ligada à filosofia, à literatura, à história, aos sistemas
políticos? Como poderia, honestamente, enquadrar esses assuntos numa
forma didática e expressiva? Formava a sua cultura própria por um gosto
pessoal; pensava sobre os problemas da humanidade, observando-os e
discutindo-os. Nunca se lembrara de, um dia, transmitir seus
conhecimentos em forma organizada e sistemática. Pensava, pelo gosto de
pensar; e o que pensava, agora, era o fruto de vinte e nove anos
decorridos numa tremenda oscilação de acontecimentos. Não seria justo
tratar, da mesma forma, os estudantes que se iniciavam na escola da
vida.

Uma coisa era verdade--- não nascera professor. Poderia explorar a
classe como tantos outros antes dele; mas, as dificuldades que surgiram,
pesaram consideravelmente, porque as encarara com honestidade. Era,
porém, preciso lançar mão de alguma coisa; a necessidade do trabalho
parecia-lhe mais justa e começava a ser para ele uma causa verdadeira.
Fora rico; não se habituara ao trabalho; uma coisa, porém era certa---
tanto os ricos como os pobres deveriam trabalhar. E esta será a escola
do futuro--- o trabalho para todos.

Uma série de problemas surgia diante dele; um ponto de interrogação
aparecia ante os seus olhos, convidando-o a agir. Era preciso pensar na
vida.

Veio-lhe novamente a ideia de um jornal--- e Álvaro preparou-se para
iniciar a caminhada.

\chapter{Capítulo 39}

--- Helena! --- exclamou Bárbara satisfeita.

As duas amigas abraçaram-se demorada mente.

--- Vim ver o que faz aqui. Você desapareceu --- disse Helena em tom de
censura.

--- Eu?!

--- Sim, Bárbara; você mesma. E ainda vem receber-me com exclamações,
como se a fugitiva fosse eu.

Bárbara riu e, passando o braço em volta da cintura de Helena,
convidou-a para ver os novos arranjos da casa. Alguns móveis estavam em
lugares diferentes, e Helena admirou a simplicidade com que Bárbara
dispunha suas coisas. Na sala de música, em lugar bem visível, o busto
de Beethoven; numa tela de Baldochi, uma jarra de rosas vermelhas
sobressaía na cor neutra da parede. Somente na outra sala, telas de
pintores estrangeiros e duas miniaturas antigas, além de algumas
porcelanas raras que se destacavam sobre os móveis. Por toda a casa
havia flores naturais e delas se desprendia suave e delicado perfume.
Helena sentiu a tranquilidade do ambiente e voltando-se para a amiga,
disse-lhe com certa graça.

--- Está linda a sua casa! Agora só falta...

--- Helena, Helena... --- interrompeu Bárbara.

--- Bárbara --- insistiu a outra --- você não sente falta do amor?
Afinal, todos têm o seu romance; será você uma exceção?

--- Não sei --- disse Bárbara. --- Acho que tenho um coração mais
difícil.

--- Os que a conhecem, enchem-me de perguntas a seu respeito.

Bárbara riu e comentou:

--- Saber da vida alheia é um hábito tão antigo que os homens não
conseguem livrar-se dele. Mas afinal, em que poderia satisfazer a
curiosidade desta boa gente?

--- Não sei --- volveu Helena. --- Impressiona, porém, o fato de ser tão
bonita e não se casar.

--- Como a ideia de casamento está ligada à beleza! --- comentou
Bárbara. --- No entanto, preste atenção, as mulheres bonitas são as mais
infelizes, na maioria.

--- É fácil de explicar --- ponderou Helena --- pensam que tudo no lar
deve girar em torno dos seus encantos pessoais. E os encantos pessoais
passam ou cansam; estão muito ligados ao sexo e o sexo permanece
independente da beleza física.

--- Eu tenho impressão --- disse Bárbara --- que esta visão falta às
moças brasileiras; elas têm receio de falar no sexo com a frieza
necessária para compreender o problema. Isto, porém, não impede que se
divirtam à custa de anedotas licenciosas e leiam livros tremendos que
entram pelos lares a dentro com o nome de literatura. O artificialismo
da educação brasileira, por sua inadaptação, leva muitas moças a um
fracasso completo na vida.

--- E se você disser alguma coisa a esse respeito, ficará mal vista
pelos demais. Não é preciso ir longe; se em casa souberem que nós
conversamos tais assuntos, achariam logo que o cultivo desta amizade
seria impróprio para mim.

--- E impediriam mesmo. Eles se acham possuidores de tal direito. Tirar,
é sempre mais fácil. Não considerariam, depois, que o que foi tirado
exigiria algum outro cuidado em troca. E isto se dá mais com as
mulheres, porque elas não se libertaram do seu espírito servil da
antiguidade. Hoje, pensam que são livres porque trabalham para sustento
próprio, fumam e jogam como os homens. Embriagadas por esta dose de
falsa liberdade, não percebem que a sua escravidão é maior. Pensam que
são livres porque imitam a liberdade. Falta-lhes a condição essencial
para serem livres: saberem o que é liberdade. Aliás, mesmo entre os
homens, poucos o sabem.

Bárbara parou um instante como a refletir em suas próprias palavras e,
depois, se voltou para a amiga:

--- Helena --- falou com seriedade --- quantos casais não estavam em sua
casa, naquela noite de jogo! No entanto, qual deles lhe deu a impressão
de felicidade? Observei-os todos; e não seria necessário grande alcance
para perceber o que se passa no espírito daquela gente. Nenhum deles é,
pelo menos, conscientemente infeliz. Isto os deixa alheios à realidade.
Vivem num disfarce contínuo, enganando a si próprios. E as mulheres?
Riam, falavam de coisas fúteis; cada uma delas querendo mostrar-se a
mais feliz. Mas, minha amiga, que haveria por trás daqueles risos? Uma
preocupação constante que as afastava do riso verdadeiro. Percebia-se
que aquelas mulheres se casaram pelo hábito social do casamento. Não as
levou a isso o afeto consciente, mas o costume de procurar um marido.
Depois, vem a preocupação feminina de reter o marido junto ao lar; a
substituição da esposa seria a maior humilhação social que a mulher
poderia suportar.

--- Mais que esta substituição --- interrompeu Helena --- é o comentário
social.

--- Na maioria dos casos, é mesmo verdade --- concordou Bárbara. --- Uma
ou outra mulher, de personalidade mais forte, enfrenta o comentário
social. Você não se lembra das palavras de Elvira Cunha, neste último
chá em que estivemos juntas?

--- Ah... --- riu Helena --- lembro-me perfeitamente. Disse, com grande
ênfase, que antes do nome do marido, tem o seu próprio para zelar.

--- É interessante; não se mantêm fiel por um princípio e nem pelo afeto
ao marido; simplesmente, pelo seu próprio nome. Como Elvira, é a
maioria; poucas mulheres lhe podem fazer crítica. E as que permanecem
solteiras, geralmente, não o fazem por deliberação pessoal, mas
compelidas pelas circunstâncias. Pelo fato de não acharem um marido, as
solteiras constituem objeto de caçoada entre os demais; no entanto, por
que não estão no mesmo caso os solteirões? Somente os homens podem
rejeitar um casamento? Somente eles têm o espírito livre para determinar
o seu estado civil? Que é afinal, a mulher diante da sociedade?

Helena repetiu a pergunta:

--- Que é, afinal, a mulher diante da sociedade?

--- Veja, minha amiga, embora com esta decantada civilização, a mulher
não está muito longe do tempo em que se reuniam concílios para discutir
a existência da alma feminina. Tudo isto, por quê? Porque as mulheres
não são realmente livres para escolher a sua situação. Uma ou outra
mulher, mais forte na sua personalidade ou mais firme nos seus
princípios, enfrenta a sociedade permanecendo solteira por deliberação
própria, ou desfazendo-se, conscientemente, de um casamento desastroso.
Seria preciso educar a mulher, lhe cultivando a liberdade de espírito;
ajudá-la no sentido de ela encontrar-se a si mesma, de encontrar a sua
própria individualidade! Seriam livres para escolher a sua situação, e
desapareceria da sua mente, como imprescindível, a dependência de um
casamento. Somente assim, não iriam elas encostar-se ao primeiro poste
da esquina...

\chapter{Capítulo 40}

Primeiro poste da esquina... Bárbara lembrou se de Álvaro. Com que
desânimo, ele usara esta mesma expressão. E ao repeti-la, veio-lhe ao
pensamento toda a tragédia do rapaz. Que fora aquele casamento, senão
uma conquista? Inexperiente, não fora ele apanhado de surpresa, aos
dezoito anos de idade? E por quê? Porque \emph{o} seu dinheiro atraíra a
atenção de uma mulher afoita. Outros que ouvissem a sua história,
acusariam sem restrições a mulher que o enganara; mas, esqueceriam que
todas as mulheres fazem mais ou menos as mesmas coisas. Os casos
contrários, tornaram-se tão raros, que os demais passaram a fazer parte,
das regras sociais. Instituiu-se o chamado ``bom partido'' e, por este,
se orientam as moças na procura de um companheiro para toda a vida; por
isso, um casal conscientemente feliz passou a constituir um ponto de
exclamação entre os homens.

--- Bárbara --- chamou Helena --- por que deixou de conversar e ficou
assim tão pensativa?

--- Pensava --- respondeu a outra --- que um casal conscientemente
feliz, passou a constituir um ponto de exclamação entre os homens.

Bárbara silenciou um instante; depois, voltando-se para a amiga,
concluiu:

--- É isto, mesmo, Helena, a felicidade espanta a toda a gente.

--- Bárbara --- indagou Helena --- você tomou já a deliberação de não
casar nunca?

\emph{---} Não, Helena --- respondeu com firmeza. --- Não conheço o dia
de amanhã, assim como não estou livre de prender-me por um homem. Você
se esquece de que sou mulher? Não sou contra o casamento; sou contra a
educação que coloca a mulher na dependência de um casamento. Havendo
esta dependência, as mulheres lutarão por um homem como algo
imprescindível à sua vida. E não é propriamente ao homem que elas
procuram; é à subsistência econômica que o casamento lhes trará; é a
conquista de alguém que lhes venha cultuar os encantos, mesmo
temporariamente. Quando a mulher prefere um médico a um motorista ---
continuou com seriedade --- não é a cultura do médico que a impressiona,
mas, a remuneração distinta entre os dois, como causa; e como efeito, a
posição social destacada que o médico pode proporcionar-lhe. E por
último, minha amiga, seriam considerados os candidatos. Mas o homem é
vaidoso, e a sua vaidade o impede de considerar tais coisas; julga-se o
escolhido, quando na realidade é o apanhado. O homem, entretanto, pouco
se impressiona com isso; uma vez fracassado no seu casamento, ou
subordinar-se às exigências da mulher numa condição ridícula e
desprezível... com o que deixa de ser um homem, ou, entrega-se aos
prazeres mundanos, fora do lar. Em nada, o detém a sociedade; a ele, o
sexo forte, são permitidas todas as fraquezas. Agora, Helena, volte-se
novamente para os que estiveram em sua casa, naquela reunião. Qual deles
você me diz ser diferente?

--- Não sei, Bárbara; todos eles me pareceram tão iguais.

--- E essa é a vida normal, Helena. Por isso, minha amiga, se você
pretende não se casar, procure fortalecer-se para enfrentar a situação.

Helena baixou os olhos, pensativa; depois, disse à Bárbara com amargura:

--- Sinto-me tão descrente da vida, para saber o que pretendo. E, no
entanto, como eu gostaria de crer, para ter esperança em alguma coisa...

\chapter{Capítulo 41}

Com o andar já meio cansado, Álvaro percorria as ruas da cidade sob o
sol quente do meio dia. Visitara já três jornais; em todos eles, deixara
o nome e endereço para ser chamado na primeira vaga. Andando lentamente,
Álvaro, indeciso, pensava em voltar para casa. O movimento dos pedestres
intensificava-se; era preciso, resolver de uma vez. Bem, concluiu o
rapaz, vamos a outra tentativa. Atravessou a rua e seguiu a passos
firmes; no seu íntimo, porém, desfalecia a esperança que tanto o animara
ainda pela manhã. Todavia, era necessário agir e ele estava disposto a
tudo. Parou em frente a um prédio velho, de pintura gasta, e, tirando um
papel do bolso, conferiu o número. Era aquele; Álvaro entrou e subiu. No
primeiro andar, bem no final da escada, estava a portaria. Foi direito
ao encarregado e perguntou pelo gerente. O homem olhou para Álvaro com
indiferença e indagou as suas pretensões. Álvaro insistiu em falar com o
gerente. O porteiro afastou-se lendo o jornal esportivo que tinha na
mão. Passou por uma porta de vidro no corredor da esquerda, fechando-a
atrás de si. Álvaro procurou uma cadeira para sentar-se; no início do
corredor, à direita, viu um banco de madeira. Mal pensou em caminhar
para ali, o porteiro apareceu novamente. Aproximou-se e disse que o
gerente não poderia atendê-lo; não obstante, lhe daria uma ficha que
Álvaro deveria preencher com suas pretensões. Um desânimo doentio
apoderou-se do rapaz; não teve força nem para revoltar-se.

O porteiro foi sentar-se no seu lugar e Álvaro permaneceu ali, em pé,
meio ciente da sua situação. Havia nele um sentimento novo que o impelia
a agir; e a sua ação, bem o sabia, deveria iniciar-se pelo trabalho
sistemático. O trabalho especializado, porém, não está à espera de
pessoas determinadas; se escolhera o jornalismo, era preciso que no
jornalismo houvesse lugar para ele. O que o desanimava, entretanto, era
que outros, menos capazes, usavam esses lugares como se lhes
pertencessem por herança. Bem, pensou o rapaz com amargura, este seria o
ramo em que eu poderia produzir honestamente; uma vez impossível, resta
ver o que me sobra. É triste sentir-se a vocação e tomar-se um caminho
oposto. Parou de pensar, olhou desanimado para a escada e dispunha-se a
descê-la quando a porta de vidro se abriu e um moço magro, de óculos,
passou. Tinha um cigarro na mão e vendo Álvaro, dirigiu-se a ele:

--- O senhor tem fósforo? --- perguntou.

Álvaro serviu-o maquinalmente.

O moço acendeu o cigarro; e, olhando para a fisionomia de Álvaro, disse
com interesse:

--- Tenho impressão de que o conheço. Qual é o seu nome?

Álvaro sorriu tristemente e respondeu:

--- É a primeira vez que me perguntam o nome, pessoalmente, num jornal.
O uso, segundo me parece, é deixá-lo escrito numa ficha que tanto pode
ser vista como inutilizada,

O moço deu uma gargalhada e perguntou outra vez;

--- Mas afinal, qual é o seu nome?

--- Álvaro Prado --- respondeu com desânimo.

--- Ah... eu não disse que o conhecia? É o tal da flor, perfume e
mulher.

--- É verdade --- disse Álvaro com ironia --- sou o tal da flor, perfume
e mulher; só não sou o tal do dinheiro.

--- E estas coisas custam caro --- frisou o outro com malícia.

Álvaro não se sentia disposto à brincadeira, mas o bom humor do rapaz
impediu que se retirasse.

--- E você tem escrito muito? Onde publica os seus trabalhos? ---
continuou o outro.

--- Em lugar algum --- disse Álvaro --- não sei onde arranjar impressão
para os meus artigos.

--- Por quê? --- perguntou o outro admirado.

--- Por quê? --- inquiriu Álvaro secamente.

--- Porque, sim --- afirmou o outro. --- Você quer desertar? Não faça
isso. Li o seu artigo e achei notável; era uma esplêndida combinação de
coisas fúteis e profundas. Você é desses escritores que prometem,
dependendo apenas de alguém o encontrar.

De alguém me encontrar, pensou Álvaro. Esta é boa; ser escritor não é o
suficiente, é preciso que alguém me descubra, me ache. Ora esta,
concluiu nas suas cogitações, eu estou perdido! Depois, olhou para
aquele moço, cujo nome não sabia ainda; pareceu-lhe familiar, embora
nunca o tivesse visto. Chamou-o por você desde o início e não se
desculpou; parecia à vontade no que fazia e dizia. Álvaro fitou-o com
atenção e perguntou-lhe o nome:

--- Caio Neiva, respondeu.

Era um nome estranho para Álvaro; nunca ouvira falar nele.

--- Trabalha no jornal? --- indagou ainda.

--- Trabalho; sou revisor. É um serviço aborrecido como não calcula:
leio, obrigatoriamente, artigos e anúncios. E vem cada um! --- disse ele
piscando.

--- Rever anúncio? Que coisa desagradável --- comentou Álvaro.

--- Assim mesmo --- volveu o outro --- chego a preferir os anúncios a
muitos artigos.

Álvaro riu e o rapaz voltou a perguntar.

--- Mas em que lugar publica os seus trabalhos?

--- Estou à procura desse lugar --- disse Álvaro não muito à vontade.

--- E já falou aqui?

--- Nem sequer fui recebido.

--- Oh, que coisa! --- exclamou Caio Neiva pensativo. --- Olhe, vou ver
o que é possível. Parece que o crítico de música vai deixar o jornal;
você não gostaria do trabalho?

--- Muito --- respondeu Álvaro.

--- Os críticos de música são amadores na profissão; não é coisa bem
remunerada, mas comecemos por aí.

--- Muito obrigado --- disse Álvaro comovido.

--- Não me agradeça --- volveu o outro --- no fundo, estou sendo até
egoísta.

--- Como isto? --- perguntou Álvaro admirado.

--- Gosto dos seus artigos. Você no jornal, seria ter algo agradável
para ler de vez em quando.

Álvaro riu da atitude do companheiro e tomou a dizer:

--- Mesmo assim, não posso deixar de agradecer. De outra forma, como
faria? Nem sequer fui recebido.

--- É assim mesmo a coisa, meu caro; tudo está na mão do mais forte. Os
homens buscam o poder para abusar, com mais segurança, da fraqueza
alheia. Mas, que se mantenham firmes, pois, no caso de uma revanche, os
oprimidos sempre se colocam no outro extremo.

Caio Neiva puxou a fumaça do cigarro, repetidas vezes, até que a brasa
se reavivasse; depois, voltando-se para Álvaro, disse em tom de gracejo:

--- Se esta gente soubesse que a lealdade é o meio mais acertado de
lidar com os homens, talvez, se tornassem leais até mesmo por
velhacaria. Mas qual, a primeira patifaria gera a segunda\ldots{} e
assim vai numa ordem sucessiva até perder-se o número de vista.

Álvaro ficou admirado de ouvir aquelas palavras de um rapaz que só
parecia rir na vida. O bom humor não impedia a visão dos problemas
graves do momento. E dando-lhe o seu endereço, deixou o jornal com uma
esperança mais viva no seu íntimo. Saía com uma recordação agradável de
Caio Neiva. Tinha a impressão de que ele conseguira o equilíbrio entre
as coisas boas e as coisas más da vida. Lera o artigo de Álvaro e
compreendera as suas ideias. Era dessas pessoas cuja apreciação, Álvaro
sempre estimaria.

\chapter{Capítulo 42}

Quando Álvaro voltou à casa de Bárbara, no dia seguinte, estava já
contratado pelo jornal. Caio Neiva o chamara na mesma noite para
substituir o crítico musical. Era certo que o crítico, tendo achado
ocupação bem remunerada, deixava de ouvir os concertos a que o jornal o
obrigava. Ambos ficaram satisfeitos--- o crítico por escapar à obrigação
de ouvir música e Álvaro por encontrar um trabalho do seu agrado. De
manhã, quando tocou a campainha da casa branca de esquina, Álvaro estava
alegre. Bárbara veio recebê-lo e ele, entusiasmado, contou-lhe do novo
emprego. Falou ainda que no outro dia, iniciaria a sua tarefa, pois já
estava marcado no teatro um concerto sinfônico. Entre outras coisas,
contou a Bárbara o seu ordenado--- trezentos mil reis.

--- Quanto! --- indagou Bárbara, duvidando do que ouvira.

--- Trezentos mil-réis --- repetiu meio desapontado. --- Pagam mal; mas
o trabalho é agradável e toma pouco tempo.

Trezentos mil-réis! Era quase o ordenado da sua cozinheira. Bárbara
sentiu-se perturbada com a revelação; era preciso, porém, não desanimar
o rapaz, uma vez que se conformara com a exigência do momento.
Convidou-o a entrar na sala, e ele, em caminho, perguntou a Bárbara:

--- Causar-lhe-ia transtorno consultar alguns dos seus livros?

--- A oportunidade de ser útil é muito agradável, Álvaro. Os meus livros
estão ao seu dispor.

--- Obrigado.

Quis dizer mais coisas; as palavras, porém, lhe faltaram e o assunto
pareceu morrer naquele obrigado.

Entraram na sala de leitura e Bárbara ofereceu-lhe a escrivaninha.
Álvaro aceitou. Tomando alguns dicionários, iniciou suas consultas.

Bárbara sentou-se em uma poltrona próxima e continuou a examinar uma
revista religiosa que o seu agente mandara de São Paulo. Fazia isto
quando Álvaro chegara à sua casa; era a primeira vez que deparava com
uma publicação de ideias religiosas aqui no Brasil. O que lhe chegara às
mãos foram sempre publicações de fatos ou acusações aos credos
contrários; a nova revista, porém, parecia desenvolver-se num outro
campo. A iniciativa, pois, era um passo apreciável no pensamento
brasileiro. Bárbara, que gostava do Brasil, acompanhava com interesse o
seu desenvolvimento. Pensou em comentar o fato com Álvaro. Olhou para
ele; viu-o entregue ao seu trabalho. Não quis interrompê-lo. E vendo
toda a revista, passou para um exemplar de Tagore. Já o tinha lido
várias vezes, mas costumava, reler trechos separadamente. Abriu o livro
e leu os pensamentos do poeta-filósofo. --- ``Poucos são os homens que,
no meio dos milhares da raça, têm suficiente discernimento para
desejarem chegar à perfeição. E destes poucos, tão raros são os que a
procuram com sucesso, que se acha apenas cá lá, alguém que a conhece
realmente.''.

A empregada entrou na sala trazendo o delicioso cafezinho. Ambos tomaram
com prazer a xicrinha costumeira. Álvaro acendeu um cigarro logo a
seguir, e Bárbara voltou ao poeta. Tomara contato com os pensadores da
índia longínqua. Sentindo nele, almas irmãs, cultivava com prazer a
leitura de suas obras. Assim como no passado, os romanos vencedores
trouxeram da Grécia os seus costumes e a influência da sua cultura, os
ingleses recebiam dos seus vencidos a filosofia poética e religiosa que
ainda vinha, influir em muitos indivíduos daquela raça tão diversa. Para
Bárbara, a índia não era o reservatório de imensas riquezas e nem a
senzala de um povo escravo, mas a terra em que a filosofia brotava unida
à poesia e à religião. Admirava a atitude de recolhimento do hindu: era
o povo que pensava metodicamente nos problemas espirituais. Conquanto
fizesse restrições às suas filosofias, cultivava-as pela sua parte
construtiva que era bastante compensadora. Ademais, Bárbara não aceitara
ainda um sistema organizado na sua totalidade. O homem ainda não chegara
à perfeição de criar uma filosofia, que não fosse passível de crítica. E
o único ser que legou à humanidade uma filosofia perfeita, deixou atrás
de si uma procissão de intérpretes que se contradizem mutuamente.

Bárbara pensou um instante sobre o texto lido e, sem o querer, chamou
baixinho por Álvaro.

--- Que é? --- disse ele voltando-se.

--- Empresta-me um lápis?

--- Aqui está a minha caneta --- respondeu, estendendo-a para ela.

--- Obrigada; não quero usar a sua caneta --- recusou a moça.

--- Por quê?

--- Todo o escritor tem ciúmes da sua caneta; a pena se habitua com a
maneira de escrever de quem a usa.

--- Faz lembrar um colega da Faculdade que dizia sempre: ``caneta é como
mulher, acostuma-se com uma'' --- comentou rindo.

--- Acostuma-se --- repetiu ela rindo também da ideia extravagante, mas
real do antigo colega.

--- Pois é, Bárbara, mas eu não tenho ciúmes da minha, quando se trata
de ser usada por você.

E insistiu para que ela a usasse. Bárbara tomando-a, grifou a frase que
lera. Restituiu-a ao rapaz:

--- Obrigada pela distinção.

--- Foi um prazer para mim.

--- E quando encontrar o antigo colega, diga-lhe que outros escreveram
com sua pena.

--- Oh, isto foi há tanto tempo! Foi ainda nos tempos de Faculdade;
agora, não o vejo mais.

--- Tem saudades da Faculdade? --- indagou a moça.

--- Nem sei dizer, Bárbara. Eu, porém, gostava da carreira.

--- Por que não volta a ela?

Voltar à carreira? Porque repetiria Bárbara a sua ideia? Pensara na
carreira tão de passagem como se fosse algo inatingível. Conformara-se
ao emprego insignificante, no qual tivera a ilusão de sentir-se
satisfeito; e já uma luz mais alta brilhava na sua imaginação. A ideia
da carreira tivera ele, mas, não chegara a deter-se na sua realização.
Ao lado de Bárbara, porém, tudo aparecia de maneira diferente.

Bárbara permaneceu calada; pois, percebeu a luta que se travou no íntimo
de Álvaro. Não conhecendo bem as possibilidades do rapaz, receou ter
precipitado os passos. Teria sido oportuna a sua lembrança? Ou deveria
calar-se à espera de outra oportunidade. Álvaro voltou-se para Bárbara;
sentia-se a inquietação em seus menores gestos. O que lhe parecera certo
há uns dias atrás, apresentava alternativas agora. Por fim, olhou
demoradamente para a moça e ponderou:

--- Não seria acalentar um sonho absurdo?

Uma vez que entrara no caminho, ela resolveu não parar ao meio.
Sustentou com firmeza o olhar do rapaz e respondeu:

--- Não o acalente, enfrente-o.

--- E por uma ironia, talvez, este é o tempo das inscrições --- comentou
ele, apertando os lábios.

E após um instante de indecisão:

--- Bárbara --- chamou --- que acha possível num caso como o meu?

--- A primeira coisa seria ir à escola para ver as suas possibilidades.
Depois, então, consultar o diretor com o fim de resolvê-las.

Álvaro riu:

--- Consultar o diretor? Você não conhece essa gente, Bárbara. Para que
eu pudesse falar-lhe seria necessário a proteção de alguém. Não é pelo
interesse do ensino, nem pelo dos estudantes, que se escolhem homens
para a direção de tais estabelecimentos.

--- Eu estudei aqui também, Álvaro; e estas mesmas coisas me
aconteceram. Conheço, portanto o problema. E falando nisso, recordo-me
da atitude do diretor do ginásio em que eu estudei. Foi sempre tão amigo
dos alunos e batalhava com ardor pelas causas justas. Certa vez, quase
perdi um ano por uma questão de matrícula; e não fossem os empenhos
desse diretor, ficaria impedida de frequentar as aulas dentro das
exigências legais.

--- Mas esse diretor continua em São Paulo, Bárbara --- gracejou o
rapaz.

Álvaro ficou pensativo algum tempo; depois perguntou a Bárbara:

--- E se for uma ilusão?

--- Pode-se vencer a ilusão, Álvaro quando se está em busca da
realidade.

\chapter{Capítulo 43}

Meia-noite. Da casa de d\textsuperscript{a}. Alda saíam alguns
visitantes. Quando o Dr. Renato, a esposa e as duas gêmeas se recolhiam,
o filho mais velho entrou cantando. Carlito, que tanto esperava por
Bárbara naqueles dias, resolvera aproveitar a noite de forma diferente.
Telefonara várias vezes; primeiramente, não a encontrara e, depois,
Bárbara recusara o seu convite, alegando outros compromissos. Ele
preferiria sair com Bárbara, era certo, mas, uma vez que ela se fazia
tão difícil, nada lhe restava senão divertir-se de outra forma. E foi o
que fez. Vendo entrar o irmão, Ivete suspirou com inveja:

--- Ah... que maravilha!

Carlito jogou-se na primeira poltrona e olhando para a irmã, indagou:

--- Ah... o quê?

--- Quem me dera ser homem! --- exclamou Ivete. --- Entrar em casa a
qualquer hora! Ter inteira liberdade!

Carlito deu uma risada alta e comentou:

--- Esta menina só diz bobagem.

--- Que menina? --- indagou d.\textsuperscript{a} Alda meio contrafeita.

--- Ivete, mamãe; quem mais havia de ser?

Ivete irritou-se com as palavras do irmão e, gaguejando, retrucou-lhe:

--- Se acha bobagem ser livre, por que não fica em casa?

--- Você não sabe que estas coisas não condizem com uma moça de família,
minha cara irmã? --- perguntou o rapaz, sorrindo maliciosamente.

--- Já não chega ser cautelosa lá fora, e ainda tenho de calar-me em
casa? --- tornou Ivete com rispidez. --- Ao menos aqui hei de dizer o
que penso.

--- E pensa assim?

--- Penso; há mal nisso?

--- Muito --- volveu o irmão com autoridade.

--- Muito?! --- e Ivete deu uma gargalhada nervosa. --- Os irmãos são
sempre assim; ditam normas familiares o dia inteiro, como se as irmãs
tivessem de segui-la incondicionalmente. Exigem das irmãs uns modelos de
virtude e distinção; e procuram para si, modelos de sapequismo e
leviandade. Por que é que todos os irmãos acham defeitos tremendos nas
irmãs e não os reconhecem nas namoradas?

--- Por que pensa que seja?

--- Por terem o gosto de destruir tudo. Olhe para o irmão de Helena;
martirizou a coitada enquanto foi solteiro e, no entanto, com quem se
casou? Com uma mulher sapeca que o põe em ridículo pelas suas
exigências. E ele o que faz? Submete-se porque está sob o domínio de uma
mulher esperta. Chegando-se para mais perto de Carlito, falou entre os
dentes: --- Não pense que vou segui-lo; também sou esperta.

--- Vejo-a, neste momento, como um símbolo da esperteza, moderna, minha
cara irmã.

A empregada trouxe o carro de chá e deixou-o perto de
d.\textsuperscript{a} Alda; depois, pediu licença e afastou-se.
D.\textsuperscript{a} Alda deixou o exemplar de modas que folheava
atentamente; serviu ao marido, aos filhos e finalmente a si. Ao lado, a
discussão prosseguiu:

--- E verá a que perfeição chegarei --- continuou Ivete em resposta às
palavras do irmão.

--- Ah, Ivete --- disse Lia pela primeira vez.

--- Vai contradizer-me, Lia? Você não faz outra coisa!

--- Não quero contradizê-la; mas, isto não será bom para você --- volveu
Lia no mesmo tom.

--- E será para você? O melhor, Lia, é cuidar cada um da sua própria
vida.

Embora ofendida pela rispidez de Ivete, Lia tentou explicar-se.

--- Não quero entrar na sua vida; é que não está vendo direito as
coisas, Ivete.

--- Talvez eu precise de óculos; sou sempre a defeituosa ...

Magoada pela atitude da irmã, Lia respondeu no mesmo tom:

--- Use-os antes no coração, Ivete.

D.\textsuperscript{a} Alda tomava o seu chá, indiferente ao debate dos
filhos; todavia, a esta altura, voltou-se para eles:

--- Por que hão de se portar assim? São apenas três irmãos e poderiam
ser tão unidos. Vocês não sabem do provérbio antigo ``a união faz a
força''?

--- Às vezes, desfaz --- tornou Lia com seriedade.

--- Como?! --- perguntou D.\textsuperscript{a} Alda admirada.

Era a primeira vez que Lia lhe falava naqueles termos. Dr.~Renato
abaixou o jornal que passava em revista e achou que era tempo de pôr fim
à discussão. E como Lia fosse a última a falar, foi por ela que o
Dr.~Renato começou:

--- Lia --- perguntou --- onde está a sua meiguice?

Lia não se intimidou; olhou para o pai e respondeu calmamente:

--- Só assim, papai, fiquei sabendo que o senhor notou, um dia, a minha
meiguice.

D.\textsuperscript{a} Alda quis repreendê-la pela altivez da resposta,
mas viu o marido ficar embaraçado pelas palavras da filha e desviar-se
do assunto. Ouviu-se novamente a voz áspera de Ivete:

--- Pois saiba Lia, que eu não desejo a sua intromissão na minha vida.
Somos gêmeas no sangue, mas, por um capricho da natureza, não o somos
nas ideias.

--- Ainda bem --- gracejou Carlito.

--- Não interrompa o chá; poderá engasgar --- replicou Ivete com ironia
para o irmão.

--- O que seria divertido para você... --- comentou o rapaz.

Os irmãos continuaram a divergir. Dr.~Renato sentiu-se incomodado pelo
palavrório e olhou para eles:

--- Parem com isso --- ordenou com secura.

Começaram a discutir mais baixo. Todavia, esquecendo-se da recomendação,
elevaram novamente as vozes.

Dr.~Renato bateu irritado no jornal.

--- Ficam quietos ou não?!

Ante aquela ordem imperiosa, calaram-se os três.

\chapter{Capítulo 44}

Aos poucos, Álvaro ia penetrando mais na intimidade de Bárbara. Vinha
muito à sua casa e, animado por ela, tomava os seus primeiros impulsos.
Consultava-lhe os livros, expunha-lhe ideias e, discutindo-as,
firmava-as em caminho mais seguro. Naquele dia, como nos demais, fora em
busca das suas opiniões e finalmente, ficara para ultimar alguns trechos
do seu trabalho. Tivera a boa sorte de cooperar no jornal, em alguns
trabalhos extras; embora os vencimentos não fossem animadores vinham-lhe
em momento difícil. Álvaro atirou-se à oportunidade, como se uma fortuna
lhe chegasse às mãos; na verdade, aqueles trabalhinhos extras, talvez
fossem a chave que abriria a porta à sua carreira.

Sentado ali, à escrivaninha de Bárbara, trabalhava febrilmente. Por mais
que desejasse estar calmo, não conseguia conter a agitação íntima.
Alguns artigos esparsos e umas horas na redação, trouxeram-lhe
inesperadamente uma quantia que talvez permitisse a sua entrada na
faculdade. Álvaro sabia que a inscrição, com as suas devidas taxas,
constituía a parte menos difícil da carreira; mas, sentia-se perturbado
para pensar no futuro. Atirara-se ao que aparecera e deixara o resto
para resolver mais tarde. Às três horas deveria sair para saber o
resultado do seu requerimento e, por uma felicidade incrível, fora
passar umas horas em companhia de Barbara, onde o tempo correria mais
depressa. Continuava o ensaio sobre ``Estados de Alma'', e dois
capítulos haviam sido publicados no mesmo jornal em que trabalhava. No
dia seguinte deveria sair mais um, no qual Álvaro analisava a emoção,
detendo-se com rara felicidade no estudo da atitude contemplativa.

Na mesma sala, um pouco mais distante, Bárbara, na sua poltrona
costumeira, comparava duas traduções portuguesas do ``If'' de Rudyard
Kipling. Pôs-se depois a desenhar, tentando, a lápis, o busto de
Beethoven que estava sobre a estante. Conquanto Álvaro desejasse ser
delicado, interrompia bruscamente as ocupações de Bárbara, para
falar-lhe com entusiasmo das suas próprias.

--- Bárbara --- exclamou o rapaz entusiasmado, penso que nestes quinze
dias, estarei com o livro terminado.

Ela não chegou a responder-lhe; Álvaro continuou no mesmo entusiasmo:

--- Imagine... quando Paulo vir isto! Ele que lutou tanto para que eu
não me deixasse vencer.

--- Quem é Paulo? --- indagou a moça, prosseguindo no desenho.

--- O amigo de que lhe falei. Coitado! Quis infundir-me coragem, e não
foi mais feliz do que eu.

--- Infeliz no casamento, também?

--- Não; o caso de Paulo foi o inverso do meu. Já lhe falei sobre isso,
Bárbara.

--- É verdade --- concordou a moça --- recordo-me agora.

--- Quando se despediu de mim, disse uma frase que de vez em quando
repete: ``o meu nome foi o machado que vibrou o golpe no meu destino''.

--- Como?!

--- Ficou admirada? --- perguntou Álvaro, relendo a última frase que
escrevera; --- o seu nome era Paulo Machado.

--- Paulo Machado! --- repetiu Bárbara com surpresa.

Álvaro estava muito entregue a si e a seu trabalho; não pôde observar o
efeito da revelação, e passou adiante.

--- Quando sair o livro, hei de mandar-lhe um exemplar. Com grande
pasmo, vai encontrar-se com assuntos já velhos nas nossas discussões.
Assim mesmo, Bárbara, há aqui coisas que eu nunca disse; e outras ainda,
que eu ignorava estarem em mim. Como o homem se desconhece!

--- Aderindo a Carrel? --- tornou Bárbara, ainda não refeita da
surpresa.

--- Carrel disse uma verdade --- tornou o rapaz, pensativo --- o homem
vive em tudo, menos em si mesmo.

--- Por que acha que seja? --- perguntou Bárbara com o olhar distante.

--- Por diversas razões --- volveu ele --- mas, a maior delas é a
comodidade moral. Você já observou, alguma vez, um período de transição
entre estados de alma?

A pergunta de Álvaro prendeu a atenção de Bárbara, e ela tentou pensar
na resposta. Uma emoção estranha, porém, lhe perturbava o espírito. Não
conseguiu dizer mais que um:

--- É...

Aquele é... pronunciado tão vagamente, não chegou à consciência de
Álvaro; ouviu-o como um som apenas... sem um sentido exato na realidade.
E alheio ao que lhe fosse exterior no momento, continuou:

--- O homem evita esse período de transição. Teme a insegurança, pois,
tem receio da sua fraqueza própria. Uns ignoram totalmente estes
choques, outros conhecem-nos e não os enfrentam.

Uma nuvem sombria pareceu lhe alterar as feições e, voltando-se para
Bárbara, disse com certa tristeza:

--- Eu estou incluído nestes últimos.

--- Preferiria pertencer aos primeiros?

--- Não, Bárbara; prefiro a fraqueza consciente.

--- Estou com você, Álvaro; o reconhecimento da fraqueza, disse ela, é o
primeiro passo em direção à força.

Álvaro acendeu o cigarro com vagar; mordeu o canto esquerdo do lábio
inferior e, de repente, voltou-se para a moça:

--- É verdade.

Bárbara, que acompanhara as atitudes de Álvaro, compreendeu todo o
significado daquelas duas palavras. Separadamente, pareciam um tanto
vagas, mas, para Bárbara eram compreensíveis e explicáveis. Estribado
nessa compreensão, ele, às vezes, dizia as coisas pela metade. O
cafezinho habitual foi trazido à sala e Álvaro, vendo-o entrar,
perguntou preocupado:

--- Que horas são?

--- Duas e meia --- respondeu Bárbara lembrando-se do compromisso.

O rapaz tomou apressado o café e dirigindo-se para a porta, gritou um
``até logo'' de despedida. Mal saiu da sala, parou repentinamente e,
voltando-se, viu Bárbara na sua poltrona costumeira. Olhou para ela e
com afeto e repetiu:

--- Até logo, Bárbara.

Ela sorriu e respondeu:

--- Até logo, Álvaro. Felicidades.

--- Tentarei --- disse o moço --- uma vez que tudo na vida não passa de
uma tentativa.

A nuvem sombria, de há pouco, desaparecera completamente. Álvaro estava
satisfeito; e movido por seu temperamento impulsivo cedeu à alegria do
momento.

\chapter{Capítulo 45}

Quando Álvaro se retirou, Bárbara, mais à vontade, entregou-se aos seus
pensamentos. Lembrou-se do que ele dissera; Paulo Machado, o amor
infeliz de Helena, era o grande amigo de Álvaro. Como é caprichoso o
destino no seu vai e vem interminável! Falaria à amiga? Seria
conveniente? Qual a medida mais sábia? Helena sofria, mas estava
subjugada pelos pais, não teria forças para impor-lhes a vontade. Seis
anos se haviam passado e Helena mantinha sempre a mesma atitude. Falava
em Paulo com saudade; rejeitava namoros firmes e pensava que se ele
voltasse, ela enfrentaria os pais. Mas Bárbara que a conhecia, conhecia
também a sua indecisão. Não que tivesse uma personalidade dúbia; Helena
tinha convicção dos seus sentimentos, mas, permanecera amarrada ao
hábito de obedecer sempre. No seu íntimo vivia ela mesma, nas suas ações
vivia o ponto de vista paterno. Bárbara achou prudente silenciar. As
circunstâncias indicariam o caminho a seguir. Era preciso estar atenta
para não deixar passar; pois, como tudo na vida, as oportunidades também
passam...

Helena, como se pressentisse os pensamentos de Bárbara, bateu à porta
sem ser esperada.

--- Vim surpreendê-la --- disse ao ver Bárbara,

--- E que boa surpresa --- exclamou Bárbara com ternura.

--- Tive vontade de vir. Estava na cidade, fazendo algumas compras com
mamãe e d.\textsuperscript{a} Alda, quando me lembrei de que poderia
deixá-las e vir até aqui.

--- Boa ideia; e, agora, fique para jantar comigo.

--- Está bem --- concordou Helena --- mamãe voltará tarde para casa.
Está auxiliando d.\textsuperscript{a} Alda nas compras do enxoval.

--- De Ivete? --- indagou Bárbara.

--- É. Estão para marcar o casamento.

--- Assim tão depressa?!

--- Assim, minha amiga. Os casamentos hoje, ou se realizam na hora, ou
se prolongam indefinidamente.

--- Mas Ivete está noiva há um mês apenas --- objetou Bárbara admirada.

--- E não chegará a dois esse noivado. D.\textsuperscript{a} Alda
aceitou a proposta de realização imediata do casamento; pois, as
famílias são conhecidas antigas, e não será necessário esperar mais.

Como Bárbara não dissesse coisa alguma em resposta às ideias de
d.\textsuperscript{a} Alda, Helena deu umas voltas pela sala e acabou
sentando-se na cadeira da escrivaninha.

--- Não sei por que, Bárbara --- disse ela --- eu me sinto tão bem aqui.
A atmosfera de sua casa é leve, descansa o espírito.

--- Pois se acha isso --- volveu Bárbara --- precisa vir mais vezes.

--- Preciso mesmo --- respondeu Helena, olhando para os papéis esparsos
sobre a escrivaninha. --- Mas... o que é isto aqui? --- indagou,
apontando um deles. --- Não é a sua letra, Bárbara.

--- Não é mesmo. São escritos de um amigo meu.

--- Amizade nova? Nunca me falou dele;

--- Conheci-o antes de mudar-me para aqui. Chama-se Álvaro Prado; sabe
quem é?

--- Álvaro Prado? --- repetiu a outra. --- Não me é estranho.

--- O autor do artigo ``A flor, o perfume e a mulher''.

--- Talvez seja daí o meu conhecimento; mas, este nome se associa a
qualquer coisa no meu espírito que eu não posso explicar.

--- Por quê?

--- Não sei. Parece-me tê-lo ouvido em alguma circunstância diferente.

--- Com o tempo talvez se lembre --- tornou Bárbara pensativa.

--- Mas --- tornou Helena --- por que não me falou nele, antes?

--- Falta de oportunidade; você me conhece e sabe como costumo narrar as
minhas coisas.

Helena silenciou; era bem verdade o que a amiga dizia. Os fatos da vida
de Bárbara, ela os contava normalmente; em geral, sem a preocupação de
ter uma coisa para narrar.

Ouviram barulho no portão e, quase a seguir, Álvaro entrou na sala:

--- Bárbara, parece que é possível --- disse emocionado. --- Nem ouso
afirmar, tanto receio tenho de iludir-me.

--- Álvaro --- respondeu com ligeiro tremor na voz --- poderá ser uma
esperança demorada e não uma ilusão desvanecida.

Álvaro olhou para o outro lado e vendo Helena, perturbou-se:

--- Não a tinha visto; queira desculpar-me.

--- Vocês não se conhecem ainda --- falou Bárbara --- quero
apresentá-los. Esta é Helena, Álvaro; foi a minha primeira amizade no
Brasil.

O rapaz estendeu-lhe a mão:

--- Álvaro Prado, às suas ordens. E insisto em desculpar-me, pois, fui
indelicado sem o querer.

Helena correspondeu à atitude delicada do rapaz, e acrescentou ainda:

--- Um amigo de Bárbara, é também meu amigo. Tenho imenso prazer em
conhecê-lo.

Álvaro sorriu e perguntou, atencioso:

--- Desde quando se conhecem? É a amizade mais velha de Bárbara, no
Brasil?

--- Desde antes do Brasil. Viemos juntas no mesmo vapor e não nos
desligamos mais.

Álvaro, percebendo que estava diante de pessoa íntima de Bárbara, contou
o que se passava na faculdade e a sua perspectiva de reingresso no
curso; mesmo sem o auxílio do diretor ou de qualquer outra figura
proeminente no ensino. Helena tratou-o com intimidade, e reinou logo
grande camaradagem entre os três. Pouco depois, jantavam juntos,
conversando alegremente sobre os assuntos mais diversos.

\chapter{Capítulo 46}

Dias depois, Bárbara examinava os heliotrópios que Álvaro semeara,
quando a empregada veio perguntar-lhe se estaria em casa à tarde; pois
Lia viria, então, trazer-lhe pessoalmente o convite para o casamento de
Ivete.

--- Estarei sim --- respondeu Bárbara --- ela está ao telefone?

--- Está esperando a resposta, sim senhora.

--- Vou falar com ela --- tornou Bárbara dirigindo-se para a casa.

Tomando o fone, ouviu a voz meiga e delicada de Lia.

--- Bárbara --- disse a menina --- queria levar-lhe o convite para o
casamento; estará em casa à tarde?

--- Com muito prazer, Lia.

--- Assim terá de prometer-me que não faltará ao casamento.

--- Farei todo o possível --- tornou Bárbara.

--- E eu quero que faça mesmo --- tornou Lia do outro lado. --- Até à
tarde.

--- Até à tarde --- respondeu, desligando o fone.

Ao lado estava o jornal do dia e Bárbara, abrindo-o, passou os olhos
sobre as notícias. Não se deteve em nenhuma delas, pois, Lia voltou-lhe
ao pensamento. Simpatizara-se com ela. Deixou o jornal para relembrar as
atitudes daquela menina de aparência tímida e sentimentos delicados.
Percebera nela, um desses caracteres invulgares, porém, pouco trabalhado
e ainda inseguro. Incompreendida no seu meio, não se integrava, nele;
tinha voos de espírito que a visão prática e normal da mãe e irmã,
procurava cortar pelo ridículo. Mas Bárbara esperava que Lia rompesse
estas barreiras, pois, percebeu desde os primeiros instantes que sob
aquela aparência tímida, estava um espírito forte e um caráter definido.
Era a única na família; e como sempre acontece com a minoria, ela é que
passou a ser o símbolo do errado. Sorrindo, Bárbara lembrou-se do homem
normal de Lombroso. O de ``bom apetite, trabalhador, ordenado, egoísta,
apegado aos seus costumes, misoneísta, paciente, respeitoso a toda
autoridade, animal doméstico''. Em todo caso já era mais lisonjeiro que
o bípede sem penas de Platão.

\chapter{Capítulo 47}

Sem que Bárbara esperasse, Helena apareceu em sua casa à hora do almoço.
Bárbara terminava a sua refeição; não obstante, ofereceu o que fosse
possível arranjar. Helena recusou, embora a outra percebesse que ela não
almoçara ainda. Vendo a amiga sentada só, à mesa, Helena disse de
repente:

--- Você está correndo o risco de ficar solteirona, Bárbara.

--- O quê?! --- perguntou a outra muito admirada.

--- Solteirona, Bárbara --- repetiu Helena meio impaciente. --- É isso
mesmo: o que me disseram hoje, serve também para você.

--- Quem foi assim delicada? --- indagou Bárbara.

--- Ninguém; não vale a pena.

--- Então tome um cafezinho --- volveu Bárbara, passando-lhe a xícara.

Helena tomou a xícara, maquinalmente; segurou-a por algum tempo,
esquecida de que o café era para ela, depois voltou-se para a amiga:

--- Você não sente vontade de ter um companheiro? Sempre morou sozinha.

--- Às vezes, sinto; o que me falta é o companheiro.

--- Mas, tantos a procuram --- ponderou Helena pensativa.

--- Não chegou ainda a minha vez. Você é que anda, ultimamente,
preocupada com o problema do casamento.

Helena calou-se. Bárbara observou-a em silêncio e perguntou-lhe com
afeto:

--- Que aconteceu, Helena?

Ela não pôde responder; e Bárbara, para não prolongar o silêncio,
convidou-a para a sala. Helena depôs sobre a mesa a xícara ainda cheia
de café e dirigiu-se lentamente para o quarto da amiga. Bárbara
acompanhou-a. Vendo-a deitar-se, sentou-se ao lado. Embora compreendesse
que o mal de Helena era puramente moral, ofereceu-lhe um chá. Deixou-a
só no quarto. Quando regressou, trazia já a bandeja com chá e algumas
bolachinhas. Preparou-lhe a xícara, e diante daquelas atenções
carinhosas, Helena pareceu animar-se. Tomou alguns goles e olhou para
Bárbara:

--- Estou convencida de uma coisa; as filhas quando não se casam,
tornam-se um peso para os pais. Eu gostaria imensamente de ter nascido
homem.

--- Por que diz isto, Helena?

--- Pela prática, unicamente. Sei que papai e mamãe desejam ver
terminada a sua responsabilidade, e eu, Bárbara... tenho o coração
fechado.

--- Que coisa difícil são os seus pais, Helena; impediram o casamento
com Paulo... e desejam vê-la casada!

Bárbara pronunciou um Paulo diferente; a emoção intensificou-se no seu
íntimo. E para reavivar os sentimentos de Helena, de cujos reflexos ela
queria deduzir sua atitude, Bárbara, insistiu no assunto:

--- E se Paulo regressasse, Helena, não mais impediriam o casamento?

Helena fitou a amiga; depois, voltou-se para o lado e respondeu com
desânimo:

--- Creio que sim.

--- E por quê?

Não houve resposta. Não era necessário, Bárbara sabia por quê.

--- Minha amiga --- tornou a perguntar Bárbara --- você não se interessa
por nenhum outro homem?

--- Nenhum, Bárbara. Eu não formei o meu caráter, pensando em
conquistas; e sim moldando-me a um namorado que me amava e que me sabia
guiar. O carinho que me faltou em casa, eu o recebi dele. Fui
compreendida e senti essa compreensão. Foi completo demais e foi o
primeiro amor. Paulo guiava-me como se guia uma criança; e eu me deixei
guiar. O que vivi antes, ficou com ele; o que vivo hoje é outra vida,
embora eu me sinta a mesma Helena.

--- Por que não procura saber dele?

--- Papai disse à mamãe, que ele morreu não sei onde. É tudo tão vago.

--- Será?...

--- Papai não chegaria a isso. Falar de uma morte suposta... Não; não,
seria impossível --- concluiu Helena convicta.

--- Que pensa fazer, então?

--- Nada. Isto é o que me aflige: eu nunca faço nada.

--- Antes isso que precipitar-se --- ponderou Bárbara.

--- Mas eu tenho o espírito precipitado. Não sei que rumo tomar.

--- Talvez, Helena, se seus pais não lhe proibissem, você mesma se
esqueceria dele --- lembrou a outra procurando analisar a atitude da
amiga.

--- Mamãe diz isso. Pensa que é teimosia da minha parte.

--- Uma coisa é verdade, Helena, o fruto proibido tem mais sabor.

--- Acha que estou sendo teimosa?

--- Não, Helena. Ninguém poderia afirmar tal coisa. Isto é um assunto
seu unicamente; quem poderia penetrar nas profundezas dos seus
sentimentos?

--- Lá em casa afirmam, e me fazem sofrer por isso.

--- Você se deixa influenciar pelo mundo exterior. Não se modifica sob
essa influência, mas vacila ao seu contato. É preciso ser mais
individual. Sei que isto é difícil, porque o seu pessoal procura esmagar
a sua personalidade; mas, não há outro meio, minha amiga. Se não
compreendem que o direito de interferir nos seus sentimentos não lhes
compete, é preciso que você compreenda e disso se certifique para não
cair em uma falsa armadilha.

--- Como sabe todas essas coisas, Bárbara? E tem assim certeza do meu
sofrimento?!

--- Está nos seus olhos, Helena; na sua fisionomia. É uma expressão de
amargura! Não sei como pode uma família permanecer tão indiferente num
caso como o seu!

--- E quando esta amargura cresce muito, eu venho à sua procura. Eu me
envergonho da minha fraqueza, Bárbara.

--- Não vem à minha procura em busca de salvação, vem para fugir ao seu
ambiente. É uma evasão, Helena.

--- O fato é que nas minhas dificuldades, quero estar sempre ao seu lado
--- respondeu Helena com tristeza.

--- Mas deveria estar em você mesma, e não à sua margem apenas ---
repetiu Bárbara com afeto, mas com energia.

Os olhos de Helena encheram-se de lágrimas; ela apoiou a cabeça no
travesseiro, permanecendo longo tempo em silêncio. Bárbara calou-se
também, respeitando a atitude de dor da sua maior amiga. Quando Helena
sentiu ânimo para prosseguir a conversa, voltou aos mesmos problemas.

--- Como foi terrível na minha vida o desmoronar das minhas ilusões. Se
os pais soubessem o que é desmanchar, bruscamente, um sonho de amor nos
primeiros anos da mocidade\ldots{}

--- Isto é que a persegue, Helena. Se pudesse vencer estas impressões...
--- disse Bárbara.

--- Não posso, Bárbara; lamento do fundo da alma, não ter casado com
Paulo. Ele seria o único para compreender-me; a seu lado, eu teria
certeza da felicidade.

--- Mas Helena, quando uma coisa não se realiza, nós nos prendemos à sua
irrealização, imaginando um mundo que talvez nunca existisse para nós.

--- Não compreendi bem --- volveu Helena indecisa.

--- Está se prendendo muito a este passado. Pensa que se este casamento
se realizasse, as coisas seriam todas como você e Paulo quisessem?

--- Não, Bárbara. Não estou imaginando o que seria, mas vendo o que é
agora. Com os meus sonhos dissipados de forma tão cruel, tornei-me
descrente. Acabaram-se as minhas ilusões de mulher; para mim tudo perdeu
o encanto. Pensa que não tenho desejado amar? --- perguntou com
desespero. --- De todo o coração. Às vezes, peço a Deus que me conceda,
um novo amor na vida; e cada vez que penso nisto, sinto a minha
incapacidade para gostar outra vez. Tudo o que estava em mim ficou na
adolescência, prendeu-se nesse sonho irrealizado com que se desmoronou a
minha vida.

--- E no caso, ainda há a considerar a incompreensão da sua família.
Paulo não foi somente objeto do seu amor, foi um ponto de apoio nas suas
dificuldades.

--- Como disse, conduzia-me como a uma criança; me compreendia e sabia
enxugar as minhas lágrimas. Depois, eu estava na idade de absorver,
assim me fui amoldando a ele. Paulo era um espírito forte, experimentado
na luta pela vida; e eu, percebendo isto, me deixei guiar por ele. E
ainda teci de sonhos, um castelo que ruiria por terra. Como arruinaram a
minha vida! --- repetiu exasperada.

--- Talvez julgassem fazer-lhe o bem... Nos problemas alheios, o homem é
mais cego, mais inexorável que nos seus próprios. Mas, procure reagir,
Helena; não se deixe levar assim. Lute contra o seu meio e crie uma vida
para você mesma.

--- É difícil, Bárbara --- tornou Helena no seu crescente desânimo. ---
O meu pessoal esqueceu-se de que eu também tenho uma vida! Eles têm
prazer nas perguntas que me magoam; exasperam-me... Tenho receio de vir
a odiá-los um dia.

--- De nada adiantaria, Helena; conheço-a bem, para afirmar a sua
afeição por eles. E isto é nobre em você, minha amiga; depois, não se
inicia uma luta, destruindo, mas sustentando o que é nobre --- insistiu
Bárbara com brandura, mas com firmeza.

--- É difícil... é difícil --- repetiu Helena.

--- É mesmo Helena, mas o que há na vida de compensador, que não seja
``difícil e raro''?

\chapter{Capítulo 48}

Como prometera a Bárbara, naquela mesma tarde de fins de março, Lia
chegava à sua casa. Bárbara deixou Helena no quarto e veio ao encontro
de Lia:

--- Sinto-me feliz pela sua distinção. E como está graciosa, Lia.

A menina sorriu. Sentiu-se à vontade na presença de Bárbara. A maneira
com que ela a tratava, fazia desaparecer toda a sua timidez

--- Você não foi ao chá e eu espero que a sua ausência não se repita.
Trazendo o convite, insisto para que não falte desta vez.

--- Desta vez --- repetiu Bárbara --- não faltarei. E sente-se aqui,
Lia, ficaremos mais perto uma da outra.

Olhando em redor, Lia exclamou com entusiasmo:

--- Que linda casa! Como é tudo tão claro!

--- Eu gosto da luz --- disse Bárbara --- por isso abro todas as
janelas. Além disso, o lugar é agradável e eu me sinto bem aqui.

--- Carlito disse isto mesmo; embora lamentasse você morar tão distante
de nós.

--- E como vai Carlito? --- indagou Bárbara.

--- Vai bem, obrigada. Talvez venha me buscar; entretanto, não deu
certeza.

--- E você --- perguntou Bárbara --- está entusiasmada com o casamento?
Ter um cunhado, vai ser uma experiência nova, não é verdade?

--- É... estou... estou sim --- respondeu gaguejando.

Bárbara percebeu que o assunto a embaraçava e desviou naturalmente a
conversa.

--- Já escolheu o seu vestido? --- indagou.

--- Não.

--- Não?! Já é tarde. Se não anda depressa...

--- Mamãe escolhe por mim. E eu não estou com muita vontade de fazer
vestido novo.

--- Na sua idade gosta-se de vestidos novos --- ponderou Bárbara.

--- Eu gosto também, mas...

--- Oh --- disse Bárbara desviando novamente a conversa --- aqui está o
café. Ou prefere chá?

--- Gosto mais de café --- tornou Lia satisfeita por não retornar ao
assunto.

--- Então, tome esta xícara --- disse-lhe servindo-a. --- E prove estes
bolinhos; eu mesma os fiz.

Lia aceitou os oferecimentos de Bárbara e ficou silenciosa. A
governanta, após ver que nada faltava, afastou-se sem ruído.

--- Há de pensar que sou boba, não é? --- insistiu a menina.

--- Absolutamente, Lia, não há razão para isso.

--- Há sim; eu bem sei. Mas cada um tem que representar o seu papel na
vida.

Bárbara, preocupada, começou a tomar o café. Helena continuava no
quarto, procurando refazer suas energias. E agora, Lia vinha trazer-lhe
o convite para uma festa como se fosse para um enterro. Como iria
concluir aquele dia? E mais que isso, por que atraía Bárbara a
confidência das pessoas em dificuldade? Fitou atenciosamente a menina; a
expressão meiga e encantadora da jovem lhe atraiu a simpatia. E ela lhe
perguntou com afeto:

--- Que se passa com você, Lia? Não se deixe magoar assim.

--- Faço o possível, mas eu não sei...

--- O que é que não sabe? --- indagou num tom maternal.

--- Muita coisa. Ser mais alegre, por exemplo.

--- Posso ajudá-la em algo?

Ela não respondeu. Começou a tomar o café; comeu um bolinho e disse à
Bárbara:

--- Vou comer agora, para não ter fome mais tarde.

--- Ficarei satisfeita se apreciar os meus préstimos. Mas por que não
comer agora e mais tarde?

--- Dizem que comer à noite dá pesadelos. Eu tenho tido vários.

--- Tomou alguma providência?

--- Primeiro, deixei o chá; como não passaram os pesadelos, deixei
também o jantar.

--- Mas a causa pode ser outra que não o alimento. Que vê nos seus
sonhos?

--- Uma coisa esquisita: quase sempre Ivete no topo de uma alta escada e
eu num pavor de vê-la cair. Grito para que ela se agarre ao corrimão;
ela, porém, não me ouve, por mais que eu grite. Parece que outros ruídos
abafam a minha voz. Outras vezes, vejo-a trajada de noiva, a tropeçar
perigosamente na cauda do vestido. Acordo sempre numa angústia terrível
e não sei o que fazer.

--- Sonha assim muitas vezes? --- perguntou Bárbara com grave seriedade.

--- Frequentemente. É mau?

--- Não. Mas posso assegurar-lhe que isso não é estômago pesado, minha
querida. Continue com o chá, jantar e tudo o mais.

--- Uma noite, quando acordei, fiquei pensando que Ivete poderá mesmo
cair na escada no dia do casamento. Quero lembrar-me disso para avisá-la
na hora. Já pensei até em falar à mamãe para que ela se apronte lá
embaixo.

Bárbara sorriu com afeto. E Lia, tendo tomado todo o café, pareceu
erguer-se na cadeira com uma decisão repentina:

--- Acho que vou para casa.

--- Não ia passar a tarde comigo?

--- Ia, mas acho melhor eu ir.

Levantou-se da poltrona, calçou uma das luvas e estendeu a mão a
Bárbara:

--- Até logo. Vê-la-ei no casamento, não é?

Barbara segurou a mão trêmula de Lia e reteve-a entre as suas num gesto
carinhoso e amigo. Com uma expressão maternal acrescentou:

--- E acerca do sonho, Lia, você sabe como poderá evitá-lo?

Ante o grande interesse da menina. Bárbara prosseguiu:

--- Uma escada alta é a noção pueril do perigo que você temia, quando
criança. Ivete está no topo; quer dizer, sobre o perigo. Se cair,
acontecerá o que você receia. Mas se você conseguir mudar seu receio, já
não haverá perigo na escada e você deixará de sonhar com ela. Porque,
Lia, você sonha aquilo que prepara quando acordada.

--- Poderei então afugentar esses sonhos, pensando em outras coisas?

--- Talvez. Ao menos poderá mudar em sua consciência a ideia apavorante
que tem da escada.

--- Sabe, Bárbara... --- balbuciou a menina lentamente --- tenho medo
deste casamento.

--- De Ivete?

--- De Ivete sim. Receio que ela não seja feliz.

--- Por quê?

--- Ivete casa-se com Roberto, porque este foi o primeiro que a pediu em
casamento. Está somente no presente, o futuro para ela não existe.

--- Que quer dizer com isto? --- disse Bárbara, olhando para aquela
criança que falava em futuro.

--- As minhas conclusões são pessimistas. Não sinto confiança, E quando
procuro falar com ela, não me ouve e irrita-se comigo. Pensa que eu lhe
desejo mal.

Tirou um lenço do bolso e enxugou as lágrimas que brotavam dos seus
olhos ternos.

--- Lia --- chamou afetuosamente Bárbara --- na sua idade tomam-se muito
os extremos: ou se espera por um belo sonho ou se conclui que tudo é
mau. As coisas não são assim... você vai ver como Ivete ainda pode ser
feliz.

--- E se não for?

--- Escute uma coisa, Lia; por que você vai ao dentista?

--- Por quê?! Para tratar dos dentes --- respondeu admirada.

--- Você, algum dia, pensou em extrair um dente porque poderia cariar
com os anos?

--- Não...

--- Você esperaria que o dente tivesse cárie para ir tratá-lo, não é
verdade?

---!...

--- Assim são certas coisas na vida. Você pensa que seus dentes poderão
cariar, mas, não os tira por isso. Espera que a cárie apareça em alguns
deles, para ir obturar.

Lia prestava atenção às palavras de Bárbara; que modo estranho ela lhe
falava! O interesse da menina crescia visivelmente e a moça prosseguia
com o mesmo carinho:

--- Se você insistisse em desfazer esse casamento, seria como a extração
de todos os dentes.

--- Por quê?!

--- Porque é do gosto de Ivete e de toda a família. Roberto é um marido
como Ivete quer; atraente, galanteador, rico e de família tradicional no
país. Ela acha que descobriu tudo nessa união; tentar evitá-la, seria
decepcioná-la cruelmente. Sofreria tanto por perder esta oportunidade,
que você ficaria em dúvida, pensando qual dos males seria o menor.

--- Mas eu não queria isso, Bárbara; queria que ela compreendesse o
quanto se arrisca...

--- É mesmo arriscado, Lia; mas Ivete, se não se arriscar aqui, irá se
arriscar mais adiante. Tem esse temperamento e isto você não pode mudar.

--- Acha, então, que devo deixar de ter medo? Que poderia fazer mais num
caso desses? --- perguntou pensativa.

Bárbara sorriu.

--- A vida não se guia pelos nossos desejos, Lia. É uma sabedoria saber
aceitar o que ela nos proporciona, seja mau ou seja bom. Não quero dizer
que você deve aceitar resignada coisas contrárias aos seus desejos; isto
seria viver sem direção. Há coisas que podemos evitar e nestas devemos
colocar o nosso empenho; outras, porém, independem de nós.

--- Este casamento não depende de Ivete? --- insistiu a menina.

--- De Ivete e outras circunstâncias mais. D\textsuperscript{a} Alda e
toda a família desejam este casamento; desmanchá-lo seria um
desapontamento geral, cujas consequências tornar-se-iam incalculáveis.
Se você tivesse certeza de que os supostos males do futuro seriam
maiores que os do presente...

--- Você vê as coisas de uma forma tão diferente!

--- Não é isso, Lia; é que você está vendo apenas um lado da questão.

--- Mas toda a gente é assim.

--- No geral, é mesmo; a felicidade na vida, porém, está no acerto entre
aquilo que desejamos e o que a vida nos proporciona. Não podemos
governar a vida e nem nas submetermos a ela.

Bárbara fitou a expressão meditativa daquela menina ingênua e disse com
seriedade:

--- Você vai aprender muito com o tempo. Você... é dessas pessoas que
caminham na vida.

\chapter{Capítulo 49}

Carlito parara a sua linda barata em frente à casa de Bárbara. Sua
presença fora uma nota alegre naquela sinfonia melancólica em torno da
moça. Cumprimentou-a cordialmente e, embora quisesse demonstrar certa
indiferença, dirigia-lhe de quando em quando uns olhares apaixonados.
Bárbara sentiu o contágio dessa alegria; a presença de Carlito foi-lhe
extremamente agradável. Envaidecido pela recepção, ele quis segurar a
oportunidade e pensou em insistir num convite para o cinema, uma vez que
ela recusara tantos para o cassino. A insistência, porém, foi
desnecessária. Bárbara acedeu logo ao convite.

--- Iremos todos --- disse ela --- você, Lia, Helena e eu. Vai ser
divertido!

Carlito preferiria não ir assim acompanhado; mas, a alegria com que
Bárbara pronunciara o ``vai ser divertido'' trouxe-lhe a perspectiva de
uma noite agradável. Lia, porém, recusou logo a ideia; a mãe esperava
por ela para decidir o vestido do casamento. Carlito prontificou-se a
levá-la para casa, antes da sessão combinada; mas, como havia tempo,
quis aproveitá-lo na companhia, de Bárbara.

--- Você toca em dois pianos? --- perguntou.

--- Em um, na realidade; em outro, na imaginação --- respondeu sorrindo.

--- E quem aparece, tocando... na imaginação?

--- O piano --- disse ela com a mesma malícia.

--- É auto piano?

--- Não; simplesmente um piano.

--- E toca sozinho?!

--- Na imaginação tudo é possível.

--- Ah... E de quem é aquele busto --- perguntou indicando-o.

--- É de Beethoven --- disse ela rindo.

--- Vou prestar atenção --- volveu o rapaz --- você gosta tanto dele!

Aproximou-se do busto e examinando-o atenciosamente, exclamou:

--- Que sujeito feio!

Bárbara riu. A simplicidade com que Carlito emitia suas opiniões,
desculpava-o perante a moça. Depois, passando à sala contígua, começou a
ler os títulos dos livros. Admirou o número de volumes sobre o Brasil;
tratados históricos sobre a política, economia, raça e até sobre a
educação do país. Alguns livros de filosofia, outros sobre arte em geral
e vários sobre música. Mais adiante, ele parou:

--- Que é isto, Bárbara? Nunca vi nome mais esquisito.

--- É a Escritura Sagrada dos Brâmanes, cuja tradução do sânscrito quer
dizer ``Sublime Caução''.

--- Você conhece o sânscrito!

--- Não, Carlito; mas no prefácio traz a explicação.

--- Finalmente, por que lê livros assim? Você é teosofista?

--- Também não --- volveu ela. --- Simplesmente por um gosto pessoal.

--- Mas que gosto atrapalhado, Bárbara!

O rapaz tirou o exemplar da estante e abriu-o. Não lhe passara pela
mente que um povo, lá do outro lado, pudesse ter um livro traduzido para
o português. Lembrou-se dos contos misteriosos da Índia e, curioso, leu
uma passagem:

\emph{"Cada ser age em conformidade com a sua natureza; também o sábio
procura o que se harmoniza com a sua própria natureza, de acordo com
aquilo que é o mais alto no seu caráter.} Meio desapontado, dirigiu-se a
Bárbara:

--- Mas, não há nada de mistério, aqui?

--- E por que esperou encontrar mistérios aí? Os hindus não podem ser
simples?

--- Não sei; as fitas sobre a Índia são mistérios; os contos, mais
ainda. Enfim, a Índia me parece a terra dos faquires.

--- E é --- concordou Bárbara. --- O fato, porém, é que você está vendo
um lado da Índia.

--- E através do cinema, para dizer a verdade --- concordou ele rindo.

Colocou o livro na estante e leu o título de outros: ``Novo,
Testamento'', ``Novo Testamento''...

--- Para que vários livros iguais?

--- São traduções diversas --- explicou a moça. --- E esta, é uma
tradução feita no Brasil diretamente do grego.

--- Ah... --- disse ele meio contrafeito.

--- A religião está muito ligada à filosofia. Quem faz um curso de
filosofia, toma contato com os pensadores religiosos.

--- E é preciso ler toda esta gente para se ter uma fé segura? Talvez
seja por isso que eu nem sei o que sou, por mim mesmo.

--- A fé não é um atributo intelectual, Carlito --- objetou Bárbara.

--- Que é então?

A moça olhou para Carlito e pensou como seria difícil responder àquela
pergunta. Se o rapaz estivesse realmente interessado, poderia pensar com
ele; mas, fazia aquilo apenas para conhecer o gosto de Bárbara. Então,
tornou com simplicidade:

--- Você está interessado em crer?

Ele sacudiu os ombros:

--- Será isto uma coisa muito necessária?

--- Ah... Carlito --- volveu ela graciosamente.

O rapaz deu uma gargalhada e disse à moça:

--- Sabe Bárbara, o que falta aqui são revistas do cinema. Na próxima
vez trar-lhe-ei algumas, como: ``Cinearte'', " Cena muda", e outras. São
revistas alegres e trazem a vida de muitos artistas. Tenha paciência,
mas isto é bem mais interessante que estas vidas de santos que você tem.

Em seguida olhou para o relógio e chamou pela irmã que se entretinha a
folhear um álbum de fotografias. Lia despediu-se de Bárbara e fê-la
prometer que iria ao casamento. Carlito combinou que viria buscá-la para
a sessão das oito. Separaram-se.

\chapter{Capítulo 50}

Álvaro, no seu quartinho humilde, pensava nos últimos acontecimentos.
Tudo saía a contento nas suas tentativas e já frequentava as primeiras
aulas na faculdade do Rio. Alguns jornais publicavam trechos de seu
livro inacabado e, lutando penosamente, conseguia algum dinheiro para
sustento próprio. Era ainda orgulhoso; e tinha a satisfação de viver por
si mesmo. Entregue aos seus pensamentos, foi despertado por alguém que
batia à sua porta.

--- Entre --- respondeu sem mover-se do lugar.

Não teve resposta. Esperou mais um pouco e ouviu baterem novamente.
Contrariado, foi verificar porque o importunavam assim. Abriu a porta
com rispidez e surpreendeu-se.

--- Paulo!

--- Paulo Machado em pessoa! --- exclamou o outro, entrando no quarto.

Os dois amigos abraçaram-se com espalhafato.

--- Então, algo de novo? --- perguntou Álvaro.

--- Não --- tornou o outro displicente --- a não ser um trecho de
estrada já terminado.

--- Por isso, veio dar um passeio; bravo!

--- Venho mudar de ares.

E Paulo contou ao amigo como se divertiram ao finalizar o trabalho.
Deram férias a todos, depois de uma comemoração em regra.

--- Assim foi que eu resolvi dar um pulo até ao Rio, para ver o que se
faz por aqui --- concluiu o visitante. E você, o que me conta?

--- Eu? --- exclamou Álvaro meio misterioso, como a explorar a
curiosidade do amigo --- imagine.

--- Não sei --- tornou o outro, percebendo no tom da voz de Álvaro que
grande mudança se operara nele.

--- Voltei para os bancos da faculdade --- contou Álvaro numa alegria
quase infantil.

Paulo segurou o braço do amigo:

--- Formidável! --- exclamou entusiasmado.

--- É incrível rapaz! Mas é verdade.

--- Sendo verdade, já não imposta ser incrível --- comentou o outro
mostrando grande satisfação.

E Paulo pôs-se a andar pelo quarto. Álvaro esperou um instante, como se
pensasse nas suas próprias palavras e depois dirigiu-se ao amigo.

--- Ei... pare um pouco: pois não há o que ver nesta pocilga.

--- Isto não faz mal. O público começa a ler os seus artigos --- disse
Paulo que acompanhava os trabalhos do amigo. --- Você já começa a
escalar o caminho da glória. Aí, então, moraremos num castelo!

--- Como a glória se relaciona com o dinheiro! --- tornou Álvaro com
zombaria.

--- É bem verdade --- concordou o outro --- o dinheiro compra tudo, até
os homens.

Aproximando-se da escrivaninha, Paulo perguntou:

--- E estes papéis?

--- Um novo trabalho. Pretendo publicar um livro --- explicou o outro
emocionado.

--- É, meu caro, parece que você pôs o pé na senda... mas...

--- Mas o quê? --- indagou Álvaro aflito, observando os gestos do amigo.

--- Conhece esta moça? --- balbuciou Paulo com a voz alterada. E tomando
o jornal que trazia a fotografia de Bárbara, examinou-o detidamente.

--- Conheço --- tornou Álvaro com a respiração em suspenso.

Num segundo veio-lhe à mente a surpresa de Bárbara ao pronunciar o nome
Paulo Machado. Tão entregue estava a si próprio, aos seus escritos, que
nem sequer se deteve nas palavras dela. A cena voltava-lhe como uma
ameaça.

--- Onde está Bárbara? --- indagou o amigo, no mesmo tom.

Aquela intimidade ao pronunciar o nome de Bárbara, causou violento
choque ao rapaz.

--- Bárbara?! Paulo, seria ela...

Álvaro não teve coragem para continuar. Ficou imóvel com os olhos fixos
no amigo.

Paulo sentou-se na cadeira da escrivaninha e voltando-se para Álvaro,
lembrou-se com amargura da sua história:

--- Bárbara...

E Paulo fez uma pausa angustiosa. Fechou as mãos com força, para depois
completar com voz cansada:

--- Bárbara foi a maior amiga de Helena.

--- Helena? --- repetiu Álvaro compassadamente. --- É então a Helena de
Bárbara?

--- Que sabe você de Helena? --- indagou o outro com grande interesse.

--- Conheci Helena. É ainda amiga de Bárbara.

--- Você a viu? --- interrompeu o amigo --- qual foi a sua impressão?

--- Não sei dizer ao certo, mas...

--- Ela estava triste ou alegre?

--- Não me apareceu alegre --- continuou Álvaro. --- Surpreendi-lhe
mesmo certas expressões duras; mas quando falava chegava até a timidez.
Julguei que algum mal a atormentasse; todavia, não quero afirmar isto,
pois foram impressões muito vagas. Eu estava um vulcão naquele dia.

--- Ela ainda, cerra os olhos quando ri? --- perguntou Paulo com o mais
vivo interesse.

--- Não cheguei a observar isso...

--- Não me diga, rapaz. Era uma particularidade atraente de Helena.

Paulo deixou a escrivaninha e jogando-se sobre a cama do amigo
continuou:

--- Eu gostaria de vê-la. Você não calcula, Álvaro, como eu fiquei preso
àquela menina. Tinha um riso espontâneo, atitudes francas... nunca lhe
surpreendi um gesto astucioso ou premeditado. Ela costumava estender-me
as duas mãos quando se encontrava comigo. Uma vez em que esperava por
mim, viu-me a certa distância e saiu correndo ao encontro. Eu detestava
mulher que corresse na rua, mas Helena era tão espontânea, tão graciosa,
que eu achei interessante a sua corrida --- contou Paulo rindo.

Terminou de falar e ficou pensativo. Aquela menina alegre, que planejara
caminhar a seu lado na vida, como estaria agora? Como se teria arranjado
sozinha, pois se até as suas leituras eram escolhidas e indicadas por
ele. Que duras lições lhe teria dado a vida! E ela era tão frágil para
recebê-las. Paulo, com a sua força, cultivara, aquela fragilidade; pois,
estava certo de ampará-la sempre. Fora egoísta, concluiu no íntimo o
rapaz; ao invés de prepará-la para a vida, preparava-a unicamente para
ele.

\chapter{Capítulo 51}

Era noite adiantada. O silêncio da cidade indicava a hora de repouso. No
quarto da pensão, tudo parecia em calma.; luzes apagadas, nenhum ruído,
nenhuma atividade. O relógio bateu horas repercutindo na quietude da
noite.

--- Álvaro! --- chamaram baixinho.

--- Que é, Paulo?

--- Não dormiu ainda?

--- Não.

--- Vou fumar um cigarro, quer?

--- Aceito --- disse estendendo-lhe a mão.

As camas estavam emparelhadas, muito próximas uma da outra; o problema
de espaço era levado em conta ali. Acenderam os cigarros; cada vez que
os levavam à boca, as brasas se avivavam, deixando transparecer as
fisionomias abatidas dos dois amigos.

--- Por que não dormia, Álvaro?

--- Pela mesma razão que você, amigo.

--- Está amando?

--- Tive a certeza disto, hoje.

--- Bárbara?

--- É.

Continuaram fumando em silêncio. Um móvel deu um estalido seco; ouviu-se
o barulho de um automóvel que passava na rua, e tudo recaiu no sossego
aparente da noite.

--- Paulo...

--- Que é?

--- Você conheceu bem Bárbara?

--- Conheci; mas isto foi há seis anos. Parece uma existência e parece
um dia, Álvaro.

--- Que achava dela?

--- Nós a admirávamos muito: além disso, Helena adorava Bárbara. Era um
temperamento diferente de Helena; mais reflexiva e mais consciente da
sua individualidade. Isto, porém, não impedia que auxiliasse os outros;
e os seus gestos não se padronizavam por esta virtude árida que se
apregoa por aí. Bárbara nasceu assim; era um carácter perfectível e como
tal se desenvolveu.

--- Teve algum namorado?

--- No meu tempo, não. Helena costumava dizer-lhe que os ingleses são
frios por natureza.

--- Acha que ela era fria?

--- Nunca achei. Era uma mulher diferente sim, mas afetuosa e capaz de
amar. Capaz, Álvaro, de compreender um amor.

--- Que fez quando você se separou de Helena?

Paulo pensou um instante:

--- Estranho! Barbara não fez coisa alguma. Foi a espectadora silenciosa
da minha desgraça --- repetiu ele.

--- Indiferença?

--- Não; tenho certeza. Deve haver alguma outra razão; é a primeira vez
que penso nisto.

--- Viu-a depois?

--- A última vez que a vi foi na minha despedida. Contei a ela o que se
passara e deixei-lhe as coisas de Helena, pedindo que as devolvesse por
mim.

Álvaro tirou uma baforada sem olhar para o amigo:

--- Que triste coincidência! Agora estamos ligados até pelo mesmo
infortúnio. Estou ficando com tendência para crer no destino.

Calaram-se os dois. O relógio do corredor bateu a meia hora. Ouviram-se
uns passos cautelosos que se aproximaram e novamente se perderam.

--- Você não dorme? --- indagou Álvaro.

--- Estou pensando no que você me disse.

--- Que foi que eu disse?

--- Que surpreendeu algumas expressões duras em Helena --- tornou Paulo
compassadamente. --- Será que a vida a mudou muito?

--- Foi tudo muito vago; não se guie pelas minhas palavras.

--- Era tão meiga... Chegava, às vezes, a ser um pouco fútil; mas era
simples de espírito, delicada e me atendia afetuosamente. Helena era
mulher em todos os sentimentos; lembro-me como amava com ternura e era
despida de todo egoísmo. Tinha um jeitinho de querer as coisas que era
impossível não lhe satisfazer os gostos. Que boa esposa teria sido e
como eu a teria adorado!

--- Eu já não posso pensar nisso --- volveu Álvaro com tristeza. ---
Estou amarrado à lei e com a vida estragada para sempre. E amarrado por
quê? Porque aos dezoito anos uma mulher quis o meu dinheiro. Estou atado
de mãos e pés; só não estou no coração. Como é injusta esta vida! ---
disse amargamente.

--- A vida foi feita para ser vivida em comum, meu caro --- ponderou
Paulo. --- Veja você, quanta, dificuldade surge, quanto o homem luta
para vencer. Tudo por quê? Para se viver em comum com uma mulher que se
ama. Uma mulher que vem aumentar as nossas dificuldades, que vem a ser
uma responsabilidade tremenda no lado financeiro, na nossa liberdade...
em tudo enfim.

--- Para ver meu caro --- tornou Álvaro, repetindo a expressão do amigo
--- com tudo isso, nós dois sofremos por não poder tomar sobre nós estas
dificuldades.

--- E tantos que não pensam nelas, entram pelo casamento afora, como se
o casamento fosse uma viagem de aventuras.

--- Casamento! Que coisa magnífica a liberdade de casar! Dar à mulher
amada o nosso nome... Viver dia a dia, hora a hora, minuto a minuto, ao
lado de alguém que nos possa encher a vida.

--- E dos que podem ter esta regalia, poucos a percebem. Penso que nem
um por cento chega a ter consciência destas coisas.

--- É mesmo injusta esta vida! Injusta, injusta, terrivelmente injusta
--- comentou Álvaro com rancor.

--- Não adianta, revoltar-se. Também fiquei assim muitas vezes; e cedi
ante o peso da sorte desgraçada.

Álvaro acendeu o mesmo cigarro que se havia apagado entre os dedos.
Tirando uma baforada, assoprou longe a fumaça e voltou-se para o amigo.

--- Que criatura estranha é a mulher! --- disse, tentando levar o
assunto para um caráter mais geral. --- Encontram-se, às vezes, homens
de grande elevação completamente subjugados por ela.

--- E vice-versa --- tornou o amigo. --- É curioso, entretanto, notar
que nem sempre é por amor. Há homens escravizados que sentem repulsa
pela mulher que os escraviza.

--- A mulher, como o homem, é, como diziam os sofistas, a medida de
todas as coisas. Nela está o mais sublime \emph{e} o mais perverso.

--- Assim meu caro, está-se sempre a tentar; ou se cai no sublime, ou...

--- Amigo --- tornou Álvaro compassadamente --- nós temos as nossas
vidas cortadas. Agora, é deixar correrem as coisas.

Calaram-se. Ambos procuravam entregar-se à inconsciência, adormecendo,
mas as imagens das eleitas ficaram ali acariciando e perturbando o sono
dos dois amigos.

\chapter{Capítulo 52}

Carlito parou o carro em frente à residência de Bárbara. A moça desceu,
agradecendo as horas agradáveis que passara em companhia do rapaz. Ele
permaneceu ali até que ela fechasse a porta que dava para o terraço;
depois, desceu velozmente as ruas em declive de Santa Tereza. Levara-a
ao cinema, e notara a sua disposição durante todo o tempo. Boa
disposição! Ora, deveria ser mais que isto. Ela estava contente por
passear com ele. Isto, porém ele notara: ela não era dessas moças que
dão certeza das coisas. Coçando a cabeça o rapaz concluiu--- é uma
mulher difícil. Mas, como era fascinante a conquista de uma mulher
difícil! Talvez fosse por isso que ela aparecesse aos olhos dos outros
como uma criatura atraente e encantadora. Ficaria, entretanto, a
fazer-se difícil por toda, a vida? Seria um procedimento esquisito para
uma mulher. Correria o risco de ficar solteirona, e com toda aquela
beleza! Que ideia descabida essa que lhe atravessara o espírito. ---
Hum... --- fez o rapaz num trejeito de lábios --- essa mulher é
perigosa...

* * *

Bárbara fechou a porta à chave e dirigiu-se para o quarto. Passara uma
noite agradável e sentia sono; ia dormir profundamente. Tirou o casaco,
despiu a roupa de passeio e vestiu a camisola rosa com que Helena a
presenteara no seu último aniversário. Sentou-se ao penteador e começou
a escovar os cabelos. Demorou-se nos seus preparos. Olhando-se ao
espelho, e, notando a camisola que vestia, lembrou-se da amiga. A cena
da tarde reconstituiu-se no seu pensamento. Bárbara lamentou a sorte da
amiga; quisera que o seu ânimo se levantasse e ela pudesse vencer as
dificuldades. Era certo que Bárbara a queria muito; mas, querer bem uma
pessoa não resolve os seus problemas. E os problemas de Helena eram
exclusivamente pessoais e familiares; ela não deixaria nunca a família,
e a família, por sua vez, não lhe prestaria auxílio. Insistir no
absurdo, seria um crime maior --- eram dificuldades sem solução no
presente.

Bárbara deixou a escova e tomou o pente que estava sobre o convite do
casamento de Ivete. Olhando para a letra grande e desenhada do envelope,
pensou--- a intuição revelou a Lia as improbabilidades de êxito desse
casamento. Lia, porém, é criança demais para ser ouvida num assunto
desta natureza; não saberia explicar seus pressentimentos. Feminina como
era, e ainda criança, Lia sabia, apenas, sentir. Os que a cercavam,
porém, pouco a atendiam; ela não poderia fazer mais que tentar ser
ouvida, sem resultado satisfatório. Era deixar correr; havia certas
coisas que faziam parte do entroncamento da vida. E Carlito? Notara já
que ele era afeiçoado a Lia; mas, que poderia fazer aquele rapaz
grã-fino, tão inconsciente das coisas quanto vaidoso dos seus atributos
pessoais? Carlito era bom de sentimento, Bárbara o percebera desde o
primeiro dia; porém, de alcance falho e estragado pela própria educação.
Tinha dinheiro e, empurrado pelos pais, tentava diplomar-se em
advocacia; contudo, estes fatos eram resolvidos e encarados, o mais
possível, dentro dos haveres da família. E assim, concluiu Bárbara nos
seus pensamentos, estragava-se um rapaz aproveitável que poderia
realizar alguma coisa na vida.

\chapter{Capítulo 53}

Na manhã seguinte, Bárbara foi atender pessoalmente à porta. Mal a
abriu, não pôde conter a surpresa.

--- Paulo!

Ele cumprimentou-a com amizade, e, segurando-lhe afetuosamente a mão,
perguntou:

--- Como está você, Bárbara? Bem?

--- Bem. E você, Paulo, que bons ventos o trazem?

--- Os mesmos que passam pelas estradas prontas lá do Norte.

--- Prontas?! Terminou tudo? --- perguntou Bárbara alegremente.

--- Não; alguns trechos apenas. A construção de estradas é trabalho
penoso e demorado.

--- E você, Álvaro --- disse Bárbara dirigindo-se a ele --- que rosto
amanhecido!

--- Ficamos conversando até tarde --- respondeu logo.

--- Com isso, vamos conversar na sala. Que poderíamos fazer de pé, aqui
na porta!

--- Nada, a não ser olhar a rua --- tornou Álvaro em brincadeira.

Paulo era o mesmo rapaz de seis anos antes, e estava muito mudado!
Bárbara encontrou nele elementos para afirmar e negar. O mesmo físico,
os mesmos gestos, traços, mas a expressão diferia. O lábio inferior um
tanto caído no canto esquerdo e um olhar profundamente penetrante
pareciam indicar, ao mesmo tempo, desprezo e revolta. Bárbara o
reconheceria em qualquer parte, pela lembrança do passado; mas descobria
nele as marcas pesadas da vida. Alto, excessivamente magro, cabelos e
bigodes castanhos claros, mostrava uma linda dentição ao sorrir. Helena
dizia no passado, que aqueles dentes eram as pérolas mais preciosas que
encontrara. Um olhar triste, algumas vezes duro, transparecia sob os
olhos longos. Perdera, aquele aspecto alegre, agradável; parecia alguém,
vergado pelo peso de uma carga permanente. Chegando à sala, sentaram-se
todos. Coube a Paulo uma poltrona voltada para uma mesinha de canto,
onde, em lugar bem visível, aparecia a fotografia de Helena. Ele
contemplou-a em silêncio; Álvaro e Bárbara calaram-se também. Paulo
levantou-se, tomou nas mãos a moldura que continha a preciosa imagem e,
contemplando-a, permaneceu de pé. Era somente a cabeça de Helena; tinha
os cabelos soltos e sorria. O seu sorriso, porém, tinha algo
inexplicável, parecia estar longe da objetiva fotográfica. Em baixo, a
um canto, com letra bem feminina e legível. --- \emph{Para você,
querida, minha afeição. Helena. Natal de 37.}

--- Como é recente! Três meses apenas.

Bárbara contemplou a figura máscula de Paulo; aquelas mãos grandes,
fortes, segurando o retrato de uma mulher. E para mudar a situação de
constrangimento que se seguiu, perguntou:

--- Nota alguma diferença?

--- Está mais mulher --- tornou ele sem levantar os olhos da fotografia.
­--- Deixei-a com dezessete anos, hoje, tem vinte e três.

Álvaro tirou um cigarro do bolso e Bárbara o deteve:

--- Espere; estou vendo o café daqui --- disse olhando para o interior
da casa.

--- Helena não pensa em casar-se? --- indagou Paulo sem olhar para
Bárbara.

--- Como sabe que ainda não se casou?

--- Por intuição apenas. E os seus sentimentos, Bárbara, são muito
diferentes?

A empregada entrou na sala, com o café. Bárbara tomou a bandeja e
dispensou-a. Pôs açúcar nas xícaras e serviu; deu a primeira xicrinha a
Álvaro e quando foi passar a de Paulo, o rapaz olhava firme para ela.
Com a xícara na mão. Bárbara hesitou ainda. Paulo não se moveu; o seu
olhar fixo era uma intimação e ela respondeu à sua pergunta:

--- Quanto a você, Paulo, Helena não mudou.

Paulo respirou com força; ficara visivelmente perturbado. Tomou o café
em um só gole, deu uns passos pela sala e voltou a sentar-se na mesma
poltrona.

\emph{---} Não se casou --- repetiu mais uma vez.

Colocou a fotografia sobre uma coluna próxima, bem ao alcance de sua
mão. Bárbara reparou-lhe os gestos e nada disse.

--- Fala em mim alguma vez? --- indagou o rapaz, adiantando-se nas suas
perguntas.

--- Fala. Recorda muito o passado.

--- E os pais?...

--- São os mesmos que você conheceu.

--- Essa gente custa mudar --- ponderou ele --- e você acha que Helena
sofre por isso?

Bárbara demorou para responder:

--- Sofre por isso e por outras coisas mais...

Paulo sorriu tristemente e comentou:

--- Como sou egoísta! Alegro-me em saber que ela sofre por minha causa;
no entanto, quem me dera, não deixá-la sofrer nunca.

Bárbara recordou a Paulo o temperamento dócil de Helena. A pessoa que
sofre porque sabe o que quer mas, acostumada a obedecer e presa a
preconceitos sociais, subjuga-se às exigências tradicionais da família.
Delicada de sentimentos e crescendo em um lar sem compreensão, Helena,
sentiu-se desnorteada para prosseguir na. vida.

--- Era isto mesmo --- concluiu Paulo --- mas era uma criatura adorável!
Bárbara... --- chamou o rapaz.

--- Será que?... Isto é... Não, não é isso --- falou bruscamente como a
combater uma lembrança.

--- Já sei, Paulo; pode levar a fotografia --- disse num tom sério.

--- Não ousei pedi-la. Agradeço muito.

--- Não concebo isto como ousadia --- explicou a moça.

--- Você sabe compreender, Bárbara. Confiando na sua compreensão, levo a
fotografia.

A um lado, Álvaro fumava sem dizer palavra. Espectador silencioso de uma
cena de dor, quando sentia já em si os sintomas do mesmo mal.

\chapter{Capítulo 54}

Paulo retirou-se. Bárbara e Álvaro, no terraço da casa, contemplaram-no
até que ele desaparecesse de vista.

--- Bárbara --- disse Álvaro ainda não refeito das emoções --- você
encontra motivos para que impeçam este casamento?

--- Encontro.

--- Encontra?! --- indagou admirado.

--- Sim: o egoísmo. O egoísmo, Álvaro, é algo tremendo quando em relação
aos outros.

--- Por que diz isto, Bárbara?

--- Porque os pais de Helena agem assim. Veem na filha, não um ser
humano, mas uma projeção de si mesmos; procuram, por intermédio dela,
elevar-se a si próprios. Satisfazem na filha os seus caprichos,
esquecendo-se de que nela também vive uma pessoa com seus gostos e
preferências. E tudo para quê? Para moldar a filha aos seus padrões.
Insistem na conservação da nobreza de família quando a nobreza da alma
está em jogo.

--- Por que não ajuda Helena a romper com tais tradições? --- perguntou
o rapaz.

--- Mas quererá ela romper? Se tentar e ficar pelo caminho, será ainda
pior para ela. Uma resolução destas, só pode ser pessoal e espontânea.

--- Paulo gosta muito de Helena --- afirmou Álvaro.

--- E quem poderia fazê-la mais feliz senão ele? --- confirmou Bárbara.
--- São temperamentos diferentes, próprios para viver juntos. Helena...
feminina, dócil, afetuosa; Paulo... homem forte, experimentado na luta
pela vida.

--- É verdade, Bárbara. Paulo é um homem para enfrentar a vida; tem
qualidades de caráter e aptidão para o trabalho. No entanto, o que faz?
Nada! Perdeu de vista o seu objetivo.

--- Paulo --- disse Bárbara --- tem o temperamento do artista. Poeta por
nascimento, viveu com intensidade as suas alegrias, mas, ninguém sofreu
mais a própria dor.

Bárbara olhou para a cidade que se estendia lá embaixo; pareceu-lhe, de
repente, que ela e Álvaro estavam num mundo à parte; sem voltar-se para
ele, disse com decisão:

--- Você também é assim, Álvaro.

Ele perturbou-se e respondeu, apenas:

--- Não sou poeta, Bárbara.

--- Não me refiro ao poeta de rimas, mas ao poeta de alma.

---\ldots{}

--- Você não se conhece, Álvaro?

O rapaz sacudiu os ombros:

--- Talvez não.

Bárbara fitou-o detidamente. Ele continuou no mesmo tom:

--- Afinal, quem sou eu? Qual, Bárbara, nem sei o que sou.

Como não pretendesse deixar o assunto, Bárbara insistiu:

--- E onde está o homem que sabe exatamente aquilo que é?

Ele não respondeu.

--- Então --- prosseguiu a moça --- resta examinar aquilo que já
sabemos.

Álvaro passou a mão pelos cabelos e, sem olhar para ela, disse com
desapego:

--- Sei já muitas coisas... sou um falido, interrompi a carreira,
estudei um pouco de música, tive algum dinheiro e finalmente, iniciei
várias tentativas para ser escritor.

--- Nos momentos de desânimo, Álvaro, as coisas ruins da vida têm
preponderância sobre o espírito. Você não se lembra agora que já
reiniciou a carreira; voltou a escrever; sente ainda a verdadeira
música, enfim, que começou a vida novamente?

Álvaro perturbou-se ante as palavras dela. --- Começou a vida novamente.
Seria possível começar a vida novamente, depois do que descobrira
naquela noite?!

--- Álvaro --- chamou a moça --- você não disse há pouco que iniciou
várias tentativas para escrever?

--- Disse.

--- Pois então, não é o que mais tem permanecido na sua vida? Depois ---
ponderou a moça --- você sabe e sente que tem a alma aberta para a
contemplação do belo. Isto ninguém pode tirar de você; é uma dádiva que
veio com a vida. Você tornou-se escritor --- repetiu olhando para ele
--- por uma necessidade íntima que o obrigou a desafogar esses
sentimentos; a partilhá-los com alguém --- asseverou convicta. --- E se
no meio onde você vivia, o consideram um fracasso, que importa? O homem
conta o fracasso alheio para disfarçar o seu próprio. Pois é mais fácil
ver o que está errado nos outros do que em si mesmo. Por isso, Álvaro,
para se ir em busca da perfeição, parte-se de si mesmo. O seu
desajustamento na sociedade não é por uma fraqueza sua, mas por uma
sensibilidade de espírito que o afasta dos demais. Ainda há a considerar
o caminho errado que seguiu por muitos anos; você bem sabe que andou por
aí, tomando encruzilhadas falsas. Agora que se encontrou... que se
achou... no meio de tantas coisas, só resta prosseguir no caminho certo.

--- Caminho certo... qual é o caminho certo? Aquele que nos conduz ao
lugar de toda a gente? Em que padrão você se baseia, para me falar no
caminho certo?

--- Não me baseio em um padrão preconcebido. Apenas procuro ver o que
está mais com o seu temperamento. Se quisesse moldá-lo a um padrão
instituído, teria citado o padrão social comum. Nesse caso, Lombroso
falaria por mim, e, ouvindo-o, talvez fizesse das palavras dele um
caminho a seguir.

--- Que disse Lombroso? --- insistiu Álvaro.

--- Ainda ontem pensei nisso --- volveu Bárbara. --- Ele apenas
respondeu a uma pergunta que lhe fez um grande jornal dos Estados
Unidos: ``Qual é o homem normal?''

---\ldots{}

--- Diz o narrador que a pergunta tipicamente americana, era acompanhada
de um cheque sugestivo; e que a resposta desconcertou os leitores. O
homem normal, disse Lombroso, traz os seguintes característicos: ``bom
apetite, trabalhador, ordenado, egoísta, apegado aos seus costumes,
misoneísta, paciente, respeitoso a toda autoridade--- animal
doméstico''. E se o homem normal é de tal natureza, está claro que o
artista ou o filósofo está fora da normalidade. Por isso, às vezes,
costuma-se dizer por aí, que o artista é um desiquilibrado. Eu não
aceito o desiquilíbrio de um homem, simplesmente por ser artista. O que
acontece, porém, é que o artista tem um interesse diferente na sociedade
e por isso se desajusta no seu meio. A sensibilidade de que é dotado,
afasta o seu interesse pelas coisas práticas. Sendo assim, como pode
integrar-se no meio social comum?

--- Mas, Bárbara --- advertiu Álvaro --- você fala como se eu fosse um
artista. Não se esqueça de que eu fracassei na arte; não continuei
música e ainda não sou escritor.

--- Álvaro, você me faz lembrar duas colegas que eu tive no meu estudo
de piano. Uma, quando alcançou o auge, foi se descuidando nos seus
estudos, simplesmente porque não possuía hábitos disciplinados. A outra
alcançou-a e passou adiante; chegou a ser boa pianista e o público
aplaudiu-a calorosamente. Mas o talento estava com a primeira; o
sentimento artístico era um elemento integrante da sua natureza; ao
passo que na outra, um elemento superficial, conquistado pelo muito
esforço. Se a segunda tivesse deixado o piano, seria um desastre
irremediável, porque a sua conquista foi pela disciplina; mas, a
primeira poderia recomeçar após o seu descuido, porque o sentimento
artístico a faria tomar impulso novamente. Você, Álvaro, não teve força
de vontade e vagueou na sua adolescência como o judeu errante. Agora,
sabe o que tem e o que pode fazer, só resta uma coisa: agir.

--- Agir... --- disse ele.

--- Sim, agir. A sociedade não o aceitará na metade do caminho. E convém
lembrar que os artistas, mesmo depois de feitos, o povo não os sabe
aceitar. Às vezes um Brailowsky derruba um teatro com um programa sem
valor; e um outro que lhe iguale ou mesmo o ultrapasse, recebe uns
aplausos frios, próprios para quem não tem nome ainda. Por isso, às
entradas dos teatros costuma-se dar uma ligeira biografia do artista da
noite, enumerando os sucessos obtidos. O povo se sugestiona e gosta. Eu
frequento muito, Álvaro e já estou bem familiarizada com esses fatos.
Antigamente, ainda se esperava a compreensão do pessoal da galeria; mas,
hoje, os batedores de palmas também se agrupam nas galerias e, a bem
dizer, anulam a outra parte.

--- E como se poderia solucionar um problema dessa natureza? ---
perguntou interessado.

--- Cultivando o temperamento do povo. Fazendo com que ele ouça\ldots{}
ouça, sem comentário prévio. Que forme a sua opinião, que se desperte.
Música nas estações de rádio, mas muita música; e depois de despertar o
gosto, então o aprimoramento. Caso contrário, seria querer ensinar latim
ao estudante porque ele ainda não sabe o português. E principalmente,
nada de contar ao povo que tal ou tal artista já se apresentou na
América do Norte ou nas Filipinas. O povo precisa conhecer por si mesmo
e aprender a descobrir os artistas que ainda não têm nome ou que ainda
não saíram do Brasil. E por isso, Álvaro, você precisa se fazer para se
apresentar. Deixe de lado a ideia do fracasso; o fracasso é relativo, os
homens é que o querem tornar absoluto.

Bárbara falara num tom amigo; compreendia o temperamento do rapaz e
sentia o seu desajustamento. Das suas palavras nenhuma foi pronunciada
em tom catedrático; não quisera ser a doutrinadora, mas a amiga igual
que se lhe pusesse ao lado. Álvaro ouviu-a com interesse, e, quando ela
silenciou, impressionado pelas suas palavras, não encontrou o que dizer.
Olhou várias vezes para a moça, ensaiou alguma coisa... e acabou por não
dizer coisa alguma. O desejo de vencer tomava impulso no seu íntimo, e
quando ele contemplou novamente a cidade que se estendia lá embaixo,
tinha no olhar uma expressão de firmeza, de senhor da casa-grande, como
se tudo aquilo lhe pertencesse.

--- Álvaro --- chamou Bárbara com brandura.

O rapaz voltou-se para ela.

--- Você vai lutar, não é? --- indagou no mesmo tom.

A atitude da moça mostrava esperar uma só resposta.

--- Desejo ardentemente --- tornou o rapaz com a voz alterada. Receio,
porém, não encontrar a direção. Se você pudesse seguir comigo,
indicar-me-ia o caminho... mas, isto seria pedir muito.

--- Farei o que puder, até que esteja firme nele.

--- Seria uma Beatriz para mim.

--- Mas, não iria lhe mostrar um céu, Álvaro.

--- Nós estamos na terra, Bárbara. E eu faria aqui mesmo esse céu, se
pudesse. Mas quem sou eu no rol das coisas?...

\chapter{Capítulo 55}

Um concerto estava marcado para aquela noite. Um artista de renome
internacional deveria exibir-se ao público carioca. Em companhia de
Álvaro, Bárbara subiu as escadarias do teatro; esperava uma noite
agradável, ouvindo boa música e um intérprete consagrado. Ao atravessar
o saguão, o rapaz notara contrafeito que a beleza de Bárbara atraíra a
atenção dos homens que fumavam ou conversavam por ali. À porta principal
da plateia, alguns estudantes contavam fatos engraçados; um deles falava
e os demais ouviam atenciosamente. Seguiu-se depois uma sonora
gargalhada de toda a rapaziada. Quando Bárbara e Álvaro passaram por
eles, o riso estancou de repente; voltaram-se todos, dali partiu um
assobio longo e agudo. Álvaro reconheceu alguns colegas de faculdade;
irritado, passou por eles sem mesmo cumprimentá-los. Notando como
olhavam para Bárbara, teve ímpetos de segurar-lhe o braço; chegou a
aproximar-se muito e, finalmente, cerrou os punhos com força, como a
castigar-se pelo seu desejo. O encarregado mostrou-lhes suas poltronas;
acomodaram-se. Álvaro desejou no seu íntimo que Bárbara não mais saísse
dali. Estava quase na hora e os espectadores começaram a entrar,
apressados. Álvaro quis iniciar uma conversa, falar de alguma coisa; as
ideias, porém, não lhe vinham; acabou por fazer uma pergunta rotineira:

--- Será que vai ver bem daí?

--- Penso que sim --- respondeu a moça com delicadeza.

Tudo parecia normal para ela; o rapaz observou-a furtivamente; mas, em
vão pesquisou na atitude de Bárbara qualquer expressão diferente.

O pianista entrou no palco. Já com as luzes apagadas, Álvaro esforçou-se
por ler o programa. Era, porém, tarde demais. Ouviram-se as primeiras
notas... reconheceu logo o estilo clássico de Bach. Não precisou a peça,
mas, que importava isso? A música divina, de um ser humano,
transportara-o já às regiões em que tanto desejara estar. Desaparecera o
crítico; ficara somente o artista... que sentia Bach, sentia o
intérprete, sentia a música! Quando o pianista terminou e o público
aplaudiu com loucura, ele permaneceu imóvel, como se a música
continuasse ainda. Veio a segunda peça, e ele se esqueceu de olhar o
nome da primeira; a mesma pergunta, porém, se repetiu no seu espírito:
que importa isso? Era Bach--- e Bach é a música. A segunda peça,
reconheceu-a logo: ``Jesus, alegria dos homens.'' O espírito religioso
de Bach, elevava a Deus o seu hino; era ainda o artista, cujo canto
divino-humano sugeria uma ideia da eternidade. Álvaro não era religioso;
dizia-se mesmo ateu e cético. Ao ouvir, porém, aquela música, sentia-se
transportado, como se conhecesse essa mesma eternidade que a alma crente
de Bach soubera cantar. Uniam-se na harmonia daquelas notas o espírito
do ateu e o do crente: havia naquelas duas almas um elemento comum que
as aproximava. Carrel afirmou um dia que os grandes santos e os grandes
criminosos eram feitos da mesma massa. E nessa proporção, por que não
dizer o mesmo do ateu e do crente? A igualdade extremada de Carrel
estendera-se àqueles dois espíritos, colocando numa equação simbólica os
temperamentos de Álvaro e de Bach--- o ateu e o crente. Seguiu-se mais
um coral ``Jesus, filho de Deus'' e encerrou-se a primeira parte. Palmas
estrondosas ecoaram por todos os lados; Álvaro e Bárbara levantaram-se
na plateia, para aplaudir o intérprete magnífico que lhes proporcionava
a boa música.

Ao lado, uma senhora comentou em voz alta:

--- Será que a minha filha está aproveitando bem o concerto? Como foi
desastroso ficarmos assim separadas! De outra vez, procurarei os
bilhetes com bastante antecedência.

O senhor, que a acompanhava, levantou-se e, nas pontas dos pés, pôs-se a
olhar por cima das cabeças dos espectadores; a certa hora, exclamou
satisfeito

--- Lá está a Lininha aplaudindo horrivelmente.

A senhora levantou-se e perguntou várias vezes:

--- Onde? Onde está a Lininha?

Outras pessoas passaram por ali e cumprimentaram o casal; houve trocas
de palavras. Álvaro e Bárbara perceberam logo que os seus vizinhos de
espetáculo eram os pais de uma menina prodígio. Mais tarde, veio a tal
Lininha; era já moça formada, porém, com um penteado de cachinhos e uma
fita no cabelo, causava a ideia de menina, à primeira vista.
Aproximou-se dos pais contando que o pianista deixara passar uma pausa
importante na repetição do tema da primeira peça. Uma vez que a mesma
frase se repetia no grave, era necessário realçá-la mais para que todos
pudessem compreendê-la; da maneira como o pianista a tocara, somente os
entendidos perceberam que a mão esquerda repetia no grave a mesma frase
da mão direita, no médio. E quando a mão esquerda executava, a frase,
num ligado absoluto, a direita deveria manter-se num ``staccato'' vivo e
pronunciado. A expressão fora boa, concluiu a moça, deixando a desejar a
parte de pedal que foi demasiado para Bach. Talvez o artista ignorasse
que as peças de Bach foram escritas para o cravo. Álvaro lembrou-se
então de verificar o nome da primeira peça e viu que era um dos muitos
``Prelúdios'' que o compositor organista escrevera.

Uma outra senhora, ao lado de Bárbara, desmanchava-se em exclamações:

--- Oh, que charme!!!

E contou à conhecida da fileira da frente que o pianista fizera um
grande progresso desde a última vez que o assistira em Paris. Álvaro,
estreante na carreira de crítico, procurava, ainda, ouvir as opiniões
esparsas do público. Nos dois ou três concertos a que comparecera,
começou a compreender que o público devia estar à parte do crítico. O
crítico ajudaria o público a compreender a peça, a, técnica da execução,
o temperamento artístico do intérprete; mas, uma coisa era certa, não
podia prender-se às suas opiniões. Era um público heterogêneo que se
reunia num mesmo teatro --- pessoas que nada entendiam de música, mas
que a cultivavam por um dever de boa educação; outras que estudavam
música, mas apenas se mantinham nesse plano de estudo, como se a música
se representasse pelos sinais que a tornaram uma escrita universal; e
por último, os que realmente sentiam a música. Estes eram tão poucos,
que ouvir música parecia mais uma profissão que um gosto. E o mal de
tudo isso era que a falsa apreciação perturbava, muitas vezes, a
carreira de talentos nacionais que lutavam por estabelecer-se.

Álvaro sentira-se desnorteado na sua primeira crítica, mas, já se
iniciava na carreira com firmeza. Não se colocava apenas nos defeitos do
executante; pesava, nos seus comentários, a visão do conjunto. A emoção,
cultivada pelos bons intérpretes, tornou-se um guia seguro na apreciação
da arte.

A segunda parte, dedicou-a o pianista, aos apreciadores do Romantismo---
Schubert, Schumann e Chopin.

Ao finalizar o espetáculo, voltara a Lininha, como um arauto, a apregoar
que o pianista esbarrara uma nota na valsa de Chopin; que o noturno,
cuja revisão fora a de Brugnoli, podia ter uma execução mais elegante
nas notas agudas do final. Ele tocara as quatro notas com a mão direita
e era aconselhável, por um requinte de elegância artística, que o dó
bemol e o si bemol passassem para a esquerda, enquanto que a direita
daria apenas o dó e o lá. Álvaro sorriu com desprezo e voltando-se para
Bárbara comentou:

--- Nada mais tremendo que uma pessoa dessas a deitar erudição
artística.

--- Coitada! --- volveu Bárbara --- não sabe sentir o belo da arte. É
dessas pessoas que só sabem enxergar o defeito das coisas.

Quando se calaram, a mesma voz continuou na fileira de trás:

--- É preciso ser grande para me satisfazer. Um artista qualquer não me
faz sair contente do teatro.

--- Felizmente --- volveu Álvaro, não estou ouvindo música ao lado de
tal criatura --- como me arruinaria o espetáculo essa erudita de
fórmulas.

Após a terceira parte, onde figurava um autor brasileiro, Álvaro
convidou Bárbara para sair:

--- Poderíamos nos afastar antes que a tal Lininha --- murmurou
gracejando --- venha dar suas dispensáveis impressões.

--- Você logo se acostumará com essa gente --- volveu Bárbara. --- Não
tardará a ouvi-las com tolerância porque, na realidade, são crianças em
desenvolvimento.

--- Talvez você as suporte, Bárbara; eu não as tolero.

Deixaram a plateia e dirigiram-se para a caixa do teatro; Álvaro ia
fazer perguntas ao pianista e colher algumas das suas impressões. Ao se
aproximarem, Álvaro parou admirado pelo que presenciava; Bárbara sorriu
complacente, acostumada a cenas semelhantes. Lininha pedia o autógrafo
do artista:

--- Oh! --- exclamava a prodígio --- que vous êtes merveilleux! Quelle
perfection d'exécution! Votre école, l'arte que vous avez\ldots{} Oh,
mon Dieu, mon Dieu! Je ne comprends pas une telle elevation d'esprit...
Vous êtes un priviligié parmi les autres. La vie vous a donné tout ce
qu'il faut pour être le premier parmi les premiers.

Bárbara continuou no seu sorriso complacente e disse a Álvaro:

--- A arte serve para muito, até para revelar os espíritos\ldots{}

\chapter{Capítulo 56}

Lia ainda quis dizer qualquer coisa a Ivete. Criança demais para ter
experiência, sentia um medo vago daquele casamento. Desejava saber os
sentimentos de Ivete, queria ter certeza de que ela amava o noivo; mas,
a oportunidade não se apresentava. Notara que Ivete, mesmo nas vésperas
do casamento, olhava com disfarce para outros rapazes. Vira também
Roberto ser excessivamente amável com a vendedora da perfumaria. Notara
ainda que Roberto mudava de atitude com as moças quando na ausência de
Ivete. E em uma noite, entrando distraidamente na sala de palestra, viu
que Roberto e Ivete se beijavam. Não achou estranho que como noivos se
beijassem; mas a atitude de paixão incontida, grosseira, que os
dominava, chocou-a profundamente. Não pôde pensar mais no caso, sem
sentir aversão; todavia, insegura no terreno, tinha apreensões que os
outros denominavam vagas; pois, Lia não sabia como explicá-las. Além
disso, não a auxiliavam nas suas ideias. Viu no dedo de Ivete o
brilhante enorme, dado por Roberto, que ela trazia com satisfação.
Notava com certeza que aquele casamento era do gosto de sua mãe.
D.\textsuperscript{a} Alda falava nele como em um conto de fadas;
admirava o lindo brilhante no dedo da filha, e fazia-a mostrá-lo a toda
gente. Louvava, agora mais que antes, os encantos de Ivete; indicava-os
com uma admiração exagerada. Certa vez, em presença de outras moças mais
gordas e de Roberto, d\textsuperscript{a}. Alda, alegando pretextos de
enxoval, medira a cintura da filha, exaltando suas formas acentuadas.
Elogiava-lhe a graça, a vivacidade, enfim, tudo o que a filha possuía de
encantador. Ivete, por sua vez, admirava o porte elegante da mãe, os
cabelos bem tratados, os vestidos de fino gosto. Só as qualidades morais
permaneciam esquecidas. Não que elas não as tivessem de uma vez; apenas,
perderam o hábito de notá-las. A síntese da perfeição estava nas
exterioridades; um predicado passava a ser virtude se se tornasse parte
integrante da apresentação pessoal. Lia não chegava a apreender
totalmente as coisas, mas sabia que o assunto era complexo e difícil.
Quanto a Ivete, esta a dominara sempre; ouvia-a com dificuldade, e a
tudo o que via, lançava os seus vereditos extremados, ``Formidável,
muito bom''; ou ``não presta, horrível, tremendo'' etc. Bastante curta
para perceber coisas quando evidentes, quanto mais descobri-las em meios
confusos. Levava ao ridículo as atitudes de Lia, pedindo exemplos
práticos das suas observações. Assim viveram as duas irmãs até o dia em
que Ivete ia deixar a casa paterna.

A modista veio experimentar o traje nupcial e Lia, descendo a escada,
viu na sala de visitas, a irmã vestida de noiva.

Apoiou-se ao corrimão e ficou silenciosa, contemplando Ivete de longe.
Como era lindo o vestido e que encantos diferentes parecia ter Ivete
naqueles trajes. Esqueceu-se das suas apreensões e quando Ivete,
vendo-a, gritou da sala:

--- Que tal?

Lia exclamou entusiasmada:

--- Maravilhoso!

Ivete afastou-se e quando Lia a viu novamente, já estava sem o vestido
de noiva. A irmã passou por ela rindo e comentou:

--- Em vinte dias serei uma senhora; terei já uma posição definida na
sociedade. Madame Roberto Bastos --- pronunciou enfaticamente --- a
esposa de um rapaz elegante e rico! Oh, que sorte eu tive na vida!

A visão agradável de Lia desaparecera. Tornaram a voltar as apreensões;
esqueceu-se do lindo vestido branco que a deixara extasiada ainda há
pouco. Que estranhas eram aquelas apreensões e como a perseguiram
dolorosamente! Sentiu um aperto na garganta e teve um desejo imenso de
chorar. A porta da saleta de entrada abriu-se de repente; sem verificar
quem estivesse chegando, Lia voltou-se na escada e começou a subir os
degraus devagar.

* * *

--- Posso entrar?

--- Pode, Carlito --- respondeu Lia. --- Quer alguma coisa?

--- Onde está você? --- insistiu o rapaz no quarto de vestir.

--- Aqui; vim descansar um pouco.

--- Deitada? Que instante! Agora, mesmo eu a vi subir.

Lia não respondeu e Carlito entrando no quarto de dormir, perguntou à
irmã:

--- Quer alguma coisa?

Na atitude do rapaz percebia-se o desejo de agradá-la e Lia pediu então
que lhe trouxesse um copo de água.

Muito gentil para com a irmã, o rapaz não se demorou, e voltou com o
copo na mão.

--- Pus um pouco de açúcar; é um calmante caseiro... enfim, o único que
conheço.

Lia sorriu:

--- Sempre, já conhece um.

Ela não gostava do remédio; em tudo o que tomava, fazia pouco uso do
açúcar, mas era delicada e não diria isso após aquela atenção de
Carlito.

O rapaz tomou o copo vazio e olhando para a irmã, falou meio misterioso:

--- Lia, eu estive pensando e resolvi uma coisa.

--- Importante?

--- Importante, sim. Estou querendo ir passar dez dias na fazenda de um
amigo.

--- Agora?

--- Agora --- repetiu ele.

--- Nas vésperas do casamento?

--- Que tem isso? Chegarei ainda com dez dias de prazo.

Lia sacudiu a cabeça.

--- E não é só --- tornou o rapaz --- pensei em levá-la comigo.

--- Eu?! --- exclamou a menina duvidosa.

--- Você, sim; a quem mais estou me referindo agora?

--- Mamãe não me deixaria sair agora. Nem quero pensar...

--- Deixará, sim. Usarei a influência de filho mais velho.

--- Ela achará um absurdo, Carlito!

--- Eu sei disso; mas insistirei e mostrarei que a nossa ausência
facilitará os preparativos. No princípio, custará um pouco... mas
acabará cedendo. Mamãe não é dessas que sustentam uma ideia por muito
tempo, desde que alguém a atormente por causa disso.

Carlito sabia pedir. D\textsuperscript{a}. Alda já cedera ao filho
muitas vezes. Além disso, era tão mais cômodo ceder.

--- Escute --- voltou Lia. --- Mesmo que mamãe não deixe, fico-lhe grata
pela lembrança.

--- Pode estar certa de que eu conseguirei. Insistirei na ideia; e mamãe
já anda cansada para resistir. Essa gente casada, cansa depressa...

\chapter{Capítulo 57}

A lagoa Rodrigo de Freitas foi a seguir o local escolhido. Álvaro e
Bárbara seguiram para lá. Procuraram um lugar cômodo e poético.
Instalando-se o melhor possível, dedicaram-se às suas tarefas
diferentes. Álvaro escrevia sem cessar--- finalizara o livro. O
turbilhão de ideias que lhe deu impulso na carreira, não se extinguiu
mais. Bárbara, sentindo-se menos disposta naquele dia, deixou o desenho
e olhou para o companheiro. Viu-o entretido no seu trabalho. Levantou-se
sem fazer ruído e foi visitar o lugar. Álvaro continuava a escrever;
contemplava a natureza e, num ensaio tendendo a ponderações filosóficas,
comparava-a à própria natureza humana. Procurava atingir o âmago das
coisas; usava a sonda e o escalpelo, numa tentativa de aprofundar-se
cada vez mais. Quando descansou a pena, o seu primeiro gesto foi olhar
para o lado; não encontrando Bárbara, sentiu-se inquieto. Desceu a
prancheta e saiu-lhe à procura. Viu-a logo adiante, à margem da lagoa e
debruçada para as águas.

Achegou-se a ela e a sua imagem refletiu-se ao lado da de Bárbara, na
superfície daquelas águas densas.

--- Por que me deixou? --- perguntou sem a fitar.

Bárbara sorriu:

--- Vim dar uma volta; não consigo fazer nada hoje.

Álvaro sentiu-se desapontado pela sua insistência e propôs um passeio em
torno da lagoa.

--- Mas --- replicou a moça --- e o livro que está por terminar?

--- Eu gostaria de andar um pouco com você; vamos, Bárbara --- convidou
novamente o rapaz.

Começaram a contornar as margens da lagoa. Formaram o hábito de caminhar
juntos, longas distâncias conversando sobre vários assuntos. Álvaro não
tinha automóvel e Bárbara não usava o seu quando em companhia do rapaz.
Temia melindrá-lo, pois, ele tivera as mesmas comodidades no passado.
Seria lembrar uma situação que se procurava sempre esquecer. E nessas
voltas agradáveis, observavam tudo; e em torno desse tudo passavam horas
a falar. Álvaro entregava-se às dissertações, pondo nelas toda a sua
razão; pensava encontrar assim, um meio para fugir aos seus sentimentos.
Evitava pensar em si mesmo desde aquela noite em que ao lado de Paulo,
descobrira a triste verdade. No íntimo, sentia a impossibilidade de uma
aproximação. Não compreendia bem Bárbara e a atitude não lhe permitia
maior intimidade. Tomaram um outro lado e viram no alto o Cristo do
Corcovado. Álvaro, contemplando-o, comentou:

--- Que altura!

--- Que preferiria você, Álvaro --- indagou ela --- olhar lá do alto
para cá, ou daqui para as alturas?

--- Se eu estivesse lá, ao certo não estaria olhando para baixo.

--- É verdade; já notei a sua atração pelas alturas --- tornou a moça.
--- Você não pensou ainda em ser aviador? É uma carreira da moda, no
momento.

--- Pensei quando mais jovem; hoje, não. Mas, a altura que eu tanto
desejava alcançar não está na estratosfera. O fato de olhar para o alto,
talvez, seja uma fuga --- interveio ele.

Bárbara sorriu:

--- Não me venha falar mal de você, Álvaro; eu não gostaria.

Ele continuou sério e, sem voltar-se, prosseguiu:

--- A pior coisa para um homem é sentir o germe de um ideal e não lhe
cultivar a vida. Permanecer entre as metades e conhecer o impulso das
alturas. Às vezes, chego a preferir o mau ao meio justo; o derrotado ao
meio vencedor; e sabe de uma coisa, Bárbara? É preferível ser miserável
a apenas vislumbrar a felicidade.

* * *

Após ligeira refeição pelo caminho, dirigiram-se à casa de Bárbara.
Quando chegaram, era já noite escura. Bárbara, desejando que Álvaro
prosseguisse no seu trabalho, foi logo providenciar os apetrechos que
ele usava para escrever. Trouxe ao terraço a prancheta portátil e ligou
uma lâmpada de leitura que fixou na própria prancheta.

A luz do terraço tendia à cor verde e era pálida. Bárbara e Álvaro
sentaram-se ali. Antes que reiniciassem o trabalho, conversaram um
pouco. Relembraram as ideias ventiladas no passeio, e Bárbara falou
delas novamente. O que ele lhe dizia ficava armazenado e sempre voltava
no momento oportuno. Tal como o escravo de Sócrates a pensar na
geometria, Bárbara devolvia-lhe as ideias; com a diferença, porém, de
que Bárbara tivera consciência delas. Sabia ajudar Álvaro e, por seu
intermédio, ele percebia aos poucos a sua participação na sociedade.
Deixara já em livro as ideias que tantas vezes pensara, ao acaso.

Quando o rapaz começou a escrever, Bárbara ligou o rádio em surdina, e
pegou o melhor programa da hora. Ouviu uma melodia conhecida, logo
descobriu no violino de Heifetz, o ``Romance'' de Beethoven. Deixou bem
baixinho para não perturbar; de ouvido quase colado ao rádio, sentiu o
contágio daquela música que tanto lhe falava à alma. Sentia a música de
Bach, de Mozart, de Haydn, mas Beethoven... este parecia ter escrito
para ela. Qualquer coisa de mágico na sua música, fascinava-a, trazia-a
presa de uma emoção diferente. Era uma força indescritível que emanava
do espírito \emph{beethoveniano} para os que lhe ouvissem o apelo;
embora, muitas vezes, aquela força a prostrasse, uma atração
irresistível impelia-a, lavava-a... para penetrar naquela floresta
mágica de sons. Terminada a música, Bárbara olhou para Álvaro e viu-o de
forma diferente. Estava de perfil e entregue ao seu trabalho; suas mãos,
iluminadas pela lâmpada portátil, pareciam desligadas do corpo.
Trabalhava febrilmente; corriam pelo papel branco, deixando à sua
passagem, uma letra quase ilegível, porém, firme, em tinta, azul-escuro.
O seu rosto tomara a atitude enérgica do homem que trabalha com vontade;
embora menos iluminado e curvado sobre o papel, percebiam-se suas
ligeiras contrações nervosas. Bárbara olhou demoradamente aquele quadro
e lamentou não ser grande pintora para reproduzir fielmente aquilo que
via. A voz de Álvaro despertou-a:

--- Bárbara, não sei se estou sonhando...

--- Por que? --- indagou com interesse.

--- Por quê?! Terminei o livro --- volveu agitado.

Como Bárbara silenciasse ante a brusca revelação, repetiu a esmo:

--- Terminei o livro. Terminei sim, Bárbara.

Ela ficou ali, a olhar para Álvaro, quase não acreditando no que ouvira.
Não sabia o que dizer; animara-o tanto e agora que o livro estava
concluído, recebia a notícia, paralisada pela emoção. Álvaro chamou-a
novamente:

--- Ouça, Bárbara; esta é a última frase: \emph{``E para terminar, deixo
aos meus leitores esta pergunta--- Afinal, que é o Homem ?''.}

--- O desconhecido de Carrel? --- volveu Bárbara.

--- O procurado de Diógenes ou o bípede de Platão; até hoje, quem o
definiu?

Álvaro colocou em ordem as últimas folhas escritas; juntou as demais e,
tirando um grampo do bolso, prendeu-as todas. Mas a primeira ficou em
branco; bem no centro ele escreveu:

À BÁRBARA,

\begin{quote}
ÁLVARO.
\end{quote}

Passou o livro à moça que o segurou perturbada. Deixando cair nele os
olhos, viu a dedicatória do rapaz. As emoções sucediam-se rapidamente---
primeiro a música, depois o quadro de Álvaro no trabalho e agora, o
livro; muito contra o seu temperamento, Bárbara não soube o que dizer.
Uma, emoção estranha apoderou-se dela; teve vontade de chorar. Segurou o
livro com mais força e, fitando o rapaz, continuou silenciosa. Álvaro
também não falou. Um ar diferente pairava no terraço. O silêncio
prolongou-se indefinidamente\ldots{}

\chapter{Capítulo 58}

Dezenove de Abril. Dia do casamento.

Ivete saía da igreja e dirigia-se para a casa dos pais. O mundo social
carioca e muitos paulistas estavam presentes à solenidade. Chegando à
casa paterna, ficou ao lado de Roberto, na sala de visitas, recebendo
cumprimentos dos convidados. Estava ricamente trajada; seu vestido
juvenil de tule e aplicações de renda verdadeira, tinha a saia farta e a
cauda ampla, em recortes. Um veuzinho curto caía-lhe ao rosto, todo
contornado pela mesma renda do vestido. Era uma noiva de dezessete anos
apenas, usando traje próprio para a sua idade. Não trazia joia alguma.
D\textsuperscript{a}. Alda, supersticiosa desde os primeiros anos, teve
receio de acompanhar a moda deixando que a filha usasse um colar de
pérolas. Como estas eram as únicas joias permitidas para um traje
nupcial, preferiu aboli-las completamente; pois, havia muito se dizia,
que as pérolas eram o emblema da infelicidade. De personalidade forte,
mas não o bastante para enfrentar crendices, d\textsuperscript{a}. Alda
se acovardava ante a possibilidade de maus presságios. No dedo dos
noivos, a aliança brilhava, solitária. Aquele fio de ouro, trazendo
impresso um nome, simbolizava a algema, ligando duas vidas para sempre.
Ivete, porém, estava muito atarefada para lembrar-se disso. Tudo fora
tão bem e Roberto dizia ao seu ouvido que ela estava linda, linda como
nunca existira, outra igual.

Roberto era de estatura média, cheio de corpo. Cabelos e bigodes louros;
sua pele tornara-se bronzeada pelo sol, o que lhe formava um contraste,
dando-lhe um tipo original. Um sorriso malicioso estava sempre nos seus
lábios; e seus olhos azuis tinham a expressão lânguida do eterno
apaixonado. Não tinha beleza e era vulgar na sua aparência. Tipo
aperfeiçoado do rapaz rico que consegue um diploma superior para nunca
ser usado, a não ser pelo título de doutor por que os outros o chamariam
depois. Sabia ser amável e tinha aquele ar despreocupado, jovial,
próprio dos que nunca lutaram pela vida. Extremamente elegante no seu
traje de solenidade, tinha sempre um gesto amável para os que se
aproximassem dele.

Assim, os parentes, amigos e conhecidos chegavam até aos noivos,
apresentando-lhes as clássicas felicitações. Entre eles estava Lia, num
desejo imenso de abraçar a irmã. Despedia-se de Ivete e queria que ela
fosse feliz; ainda na reza daquela manhã, pedira fervorosamente a Sto.
Antonio que protegesse Ivete contra todo o mal. Abraçaram-se; e Ivete,
comovida, teve um carinho repentino pela irmã. Lia não pôde conter as
lágrimas e Ivete chorou também; parecia sentir naquele momento toda a
bondade da irmã que ela desprezara tanto. Os que esperavam a sua vez
para abraçar a noiva, não foram indiferentes àquela cena; mostraram-se
comovidos e ouviram-se então as frases adequadas para ocasiões
semelhantes.

--- Oh! Separam-se, coitadinhas --- disse uma senhora ao lado.

--- Como se querem! --- comentou outra.

--- Também, não é para menos --- arriscou uma terceira --- são gêmeas.

--- Que beleza de família --- exclamou uma moça loira que as
contemplava.

--- Oh, elas se adoram! --- tornou d\textsuperscript{a}. Alda em voz
alta como a responder a todas as exclamações --- são gêmeas no sangue e
nos sentimentos.

Helena, Carlito e Bárbara, próximos dali, ouviram aquelas palavras.
Bárbara percebeu que Carlito chocara-se com a atitude insincera da mãe.
Mas, tudo pareceu ficar ignorado. Durante os cumprimentos a orquestra
tocava em surdina algumas peças românticas. Terminados os abraços, deu
início a uma valsa lenta, e os noivos, dançando-a, abriram o baile.

Estava ali também um conhecido do Dr.~Renato. Um homem de seus trinta e
oito anos, de feições vulgares e um rosto alongado; o tipo comum do
homem que não atrai e nem repele. Contava-se, porém, nos círculos
sociais que este era possuidor de uma enorme fortuna. Havia mesmo quem
afirmasse ser ele o herdeiro da maior fortuna do Brasil; o que lhe vinha
aumentar os predicados em todas as dimensões. D\textsuperscript{a}.
Alda, ao lado do milionário no momento, chamou a filha solteira que
conversava em um pequeno grupo de moças e rapazes. Lia foi atendê-la
imediatamente.

--- Anda não conhece o Dr.~Rocha? Quero apresentá-lo. É minha filha,
doutor --- disse, dirigindo-se ao moço.

--- Muito prazer. Luiz Porto da Rocha --- respondeu este, cumprimentando
a menina.

--- Lia Macedo --- respondeu ela naturalmente. --- O senhor não quer
aproximar-se daquele grupo? Estamos ali em uma prosa animada.

--- Gostaria muito, mas prefiro dançar esta valsa primeiro.

--- Prometo-lhe uma valsa, Dr.~Rocha; esta, porém, já a tenho
comprometida.

--- Avise o seu par, minha filha e conceda esta ao Dr.~Rocha --- tornou
d\textsuperscript{a}. Alda num desejo de acertar as coisas.

--- Seria indelicado, mamãe, eu prometi.

--- Tem toda a razão --- esclareceu o Dr.~Rocha --- esperarei por outra.

--- Fico-lhe grata --- tornou ainda Lia, sorrindo graciosamente.

\chapter{Capítulo 59}

Dr.~Rocha acompanhou a pequena com o olhar. Viu-a regressar ao grupo e
correu os olhos pelas pessoas que o formavam; demorou-os em Helena. O
seu gesto não passou despercebido a d\textsuperscript{a}. Alda; esta,
porém, deu mostras de não ter visto nada.

Lia aproximou-se de um rapaz que parecia esperar por ela; sorriu-lhe com
ternura e saíram ambos para dançar. Carlito, já ao lado de Bárbara,
disse-lhe baixinho:

--- Vamos dançar Bárbara; quero a minha coroa antes que algum
aventureiro lance mão dela.

O rapaz mostrava um excelente ânimo. Quando longe de Bárbara, chegava a
esquecer de pensar nela; mas, ao seu lado, não podia esquivar-se aos
seus encantos. Onde estivessem juntos, seria sempre a primeira. E
naquele ambiente de casamento, onde trocavam promessas de amor, chegou a
crer-se apaixonado por Bárbara. Saiu pelo salão com a moça nos braços e
ela deixou-se levar sob o compasso da música. Gostava de dançar. De
repente ouviram uma voz feminina que lhes pareceu familiar: --- Que é
meu amor? --- E antes que Carlito se voltasse, um par bateu-lhes de
encontro.

--- Desculpe --- disse o rapaz sorrindo --- não os tinha visto, Carlito.

Lia dançava com o rapaz que se dirigia a Carlito; este percebeu logo o
que se passava entre eles. Meio contrafeito, disse a Bárbara:

--- Ali há namoro. Que estupidez a minha em não perceber semelhante
coisa.

--- E parece que não procuram esconder --- volveu Bárbara
contemplando-os.

--- É mesmo --- respondeu o moço, olhando para o amigo.

--- Por que se mostra contrariado com o namoro?

--- O namoro das irmãs é sempre uma preocupação para a gente. Não quero
cometer a estupidez de atrapalhar; tem direito a isso. Mas os rapazes de
hoje...

--- E quem é este? --- interrompeu Bárbara.

--- É o Sérgio Augusto de Toledo; você o conhece.

--- Eu? Onde foi que o conheci?

--- Na praia. Foi o vencedor da bicicleta a quem você prometeu uma
contradança como prêmio, lembra-se?

--- Mas eu não tenho ideia de ter dançado com ele --- volveu a moça
intrigada.

--- Pois você e Helena saíram naquela tarde e acabaram por esquecer do
cassino. Por essa razão, não houve a tal contradança.

--- Como está informado, Carlito --- gracejou Bárbara.

--- Sempre sei aquilo que me interessa --- tornou o rapaz numa entonação
maliciosa.

Carlito quis aproximar-se de Bárbara; sabia, porém, que não seria fácil
tal aproximação. Formulara planos quando ainda na igreja, para
confessar-lhe todo o seu amor e estabelecer namoro firme entre eles;
mas, o rapaz que passava sorrindo por todas as conquistas amorosas,
intimidou-se diante de Bárbara. Como não encontrasse ensejo para as suas
tentativas, continuou a dançar e falar de assuntos sem importância.
Bárbara mostrava-se alegre e despreocupada; todavia, no seu íntimo, a
lembrança de Álvaro se avivara por uma frase de Carlito: ``sempre sei
aquilo que me interessa''. Era curioso que os dois rapazes, tão
diferentes em espírito, tivessem expressões em comum. E a certa hora,
Lia passou por eles, distraindo-os de suas cogitações. Vendo-a. Carlito
disse a Bárbara:

--- Não sei por que, mas esse namoro de Lia...

--- Sem namoro não há casamento --- redarguiu ela.

--- Nós estivemos juntos na fazenda; e eu, confiante no desinteresse de
Lia, deixei-os sós muitas vezes.

A cena da praia voltou ao pensamento de Bárbara. Lembrou-se então que
Sérgio Augusto a cortejara e, que ao mesmo tempo, olhara interessado
para Ivete; só não tomando conhecimento de Lia que se manteve retraída e
esquiva. Como as coisas poderiam mudar assim! --- pensou Bárbara. Lia,
contudo, parecia alheia às preocupações do irmão; tinha uma expressão
feliz como Bárbara não lhe vira ainda.

--- Bárbara --- chamou o rapaz quebrando o silêncio que se fizera entre
eles.

--- Que é?

--- Você me promete uma coisa?

--- Que coisa?

--- Di-la-ei depois.

--- Oh, isto é perigoso --- respondeu com malícia.

--- Que prudência! --- comentou ele.

--- Nem tanto assim.

--- Promete-me todas as contradanças?

A música parou e um outro rapaz aproximou-se:

--- Desculpe, mas esta é a minha vez. Se me permite, Carlito...

--- Veja --- atalhou Bárbara sorrindo --- eu prometi e não poderia
faltar à promessa.

Carlito afastou-se contrariado e Bárbara foi dançar com o outro.

* * *

Lia e Sérgio conversavam ao pé da escada quando o Dr.~Rocha se
aproximou.

--- Não vim reclamar a minha valsa --- foi logo dizendo para não ser
importuno aos namorados. --- Vim pedir um favor.

Lia apresentou o Dr.~Rocha a Sérgio Augusto e mostrou-se afável em
atendê-lo.

--- Gostaria --- continuou o Dr.~Rocha --- de conhecer aquela moça loira
com quem conversavam antes.

--- Helena --- inqueriu Lia.

--- Não sei --- tornou o milionário, mas distingui-la-ei em qualquer
lugar.

O jovem par mostrou-se pronto para sair em busca de Helena e Lia, tendo
de um lado Sérgio Augusto e de outro o Dr.~Rocha, iniciou a procura
pelas dependências da casa. Passaram por d\textsuperscript{a}. Alda;
esta sorriu satisfeita vendo a filha entre dois admiradores, ainda mais
sendo um deles o milionário. Logo adiante viram Helena que conversava em
um pequeno grupo; Lia chamou-a de lado, ao que ela atendeu prontamente.

--- Está linda a festa; tudo com muito gosto! --- exclamou Helena, vindo
ao seu encontro.

--- É trabalho e gosto de mamãe --- tornou a menina satisfeita. E
escute, Helena, você ainda não conhece o Dr.~Rocha?

O milionário aproximou-se, apertou a mão de Helena e tirou-a para
dançar. Parecia satisfeito por conhecê-la e não escondeu o seu interesse
pela moça. D\textsuperscript{a}. Alda, contrariada, presenciou a cena.
Intimamente não deixou de perguntar a si mesma, se Ivete seria tola de
apresentar um rapaz à outra; e justo o melhor.

Carlito passou na ocasião e, pedindo licença a Sérgio Augusto, convidou
a irmã para dançar. Mostrava-se um tanto irritado e Lia, percebendo, não
lhe fez perguntas. Sérgio foi então reclamar a prometida contradança de
Bárbara e, logo depois, rodava com ela pelo salão. Não tardou que
Bárbara percebesse pelas atitudes do rapaz que ele estava enamorado de
Lia. Fez algumas perguntas ao acaso, referindo-se à menina
propositadamente e não teve dúvidas quanto ao interesse sincero do
rapaz. Sérgio Augusto estava apaixonado.

\chapter{Capítulo 60}

``COMPARAM-SE OS HOMENS...'' O primeiro livro de Álvaro vinha à luz.
Estribado no sucesso de seus artigos anteriores, Álvaro colocou sem
empecilhos consideráveis o seu trabalho recente. Em todas as livrarias,
um cartaz contava ao público que o autor de ``COMPARAM-SE OS HOMENS
...'', era o mesmo de ``A flor, o perfume e a mulher''. Sem saber por
que, Álvaro, que tivera dificuldades para ingressar-se no jornalismo,
encontrara facilidade para editar o seu trabalho completo. E os
livreiros, ávidos de propaganda, levaram ao conhecimento de todos, o
artigo que os elegantes da cidade tinham lido.

Vendeu-se regularmente o livro; pois o gosto da leitura difundia-se
entre o povo. Na capital paulista, o trabalho de Álvaro fora bem aceito.
O editor escrevera contando que uma livraria de São Paulo chegava a
vender noventa exemplares numa semana. Para um livro nacional, era
muito.

Animado com os resultados, Álvaro entregava-se ao estudo com maior
satisfação. Vivia do trabalho intelectual; pôs nele, então, todo o seu
ardor. O seu estilo simples, claro, dava-lhe o dom de ser profundo com
naturalidade.

No seu trabalho, analisou o homem individualmente, antes de colocá-lo na
sociedade. Ressaltou os direitos humanos desde os ricos até os pobres;
dos mais salientes aos mais obscuros; dos mais sábios aos mais
ignorantes; dos religiosos até os ateus; dos fortes aos mais fracos; em
todas as classes, os direitos humanos respeitados por convicção íntima
\emph{em cada indivíduo.} ``O direito de outrem é dever para nós e o
nosso direito é dever para outrem''. Cada indivíduo exercendo o direito
e sustentando-o. Exercendo-o, quando chamado pela razão à verdade de uma
causa; sustentando-o, quando em perigo de ele ser negado. E sobretudo,
que o direito de um ser humano não implica \emph{na desistência, ou no
sacrifício de um outro ser semelhante.} Não existe o sol de verão para
uns e o sol de inverno para outros. Perante a sua consciência, todos os
homens são livres. Mesmo que essa liberdade se tome no sentido relativo,
como todos os atributos humanos, a sua relatividade não exclui a sua
existência.

E quanto aos seus sentimentos, Álvaro estava em luta. Evitava pensar em
Bárbara como mulher; procurava ser, com lealdade, o amigo a que ela se
referia. Nas suas agitações, pensara alguma vez em deixá-la; a sua
força, porém, não chegara a tanto. Álvaro conhecia a situação. Não
obstante, tratava-a com camaradagem e reprimia-se cuidadosamente nas
suas tendências amorosas. E saberia Bárbara que ele a amava? Pensaria
alguma coisa a seu respeito? Umas vezes revoltado, outras mais submisso,
Álvaro procurava orientar-se. Quem sabe se ainda, brevemente, teria
forças para não mais perturbá-la? Recordava-se muitas vezes, com certo
rancor, da atitude firme e impassível de Bárbara. Contudo, era-lhe
atenciosa, solícita. Não sentiria nada por ele? Fora assim tão forte a
ponto de dominar uma paixão, ou apenas lhe inspirara um sentimento de
proteção, quase maternal? Feriam sua vaidade tais pensamentos, mas a
atitude de Bárbara deixava-o inseguro nas suas deduções. Dedicando-lhe
atenção, ela continuara a frequentar outras rodas; ia a clubes,
cassinos, e ouvira de Helena que Bárbara era muito admirada. Quando
longe, todas estas coisas vinham-lhe à mente para torturá-lo; ao lado de
Bárbara, porém, tudo ficava esquecido.

No seu quartinho humilde, Álvaro pensava. Embora aqueles pensamentos o
fizessem sofrer, não conseguia furtar-se a eles; quando tomava o cálice,
tinha que ir até o fim.

O relógio do corredor bateu quatro horas; Álvaro despertou das suas
cogitações. Prometera levar Bárbara ao Pão de Açúcar, para ver de lá o
cair da tarde. Ia mostrar-lhe o acender das luzes da baía, espetáculo
que atraíra a atenção de inúmeros estrangeiros, e cuja narrativa levara
a outros países o nome da capital do Brasil. Álvaro fora já várias vezes
ao Pão de Açúcar; a beleza do espetáculo não o cansava. Estava ansioso
por levar Bárbara àquelas alturas mágicas; ela iria ver, a seu lado, o
acender das luzes da baía, a cidade entrando em sua vida noturna.

\chapter{Capítulo 61}

Álvaro e Bárbara, nos últimos tempos, quase não passeavam silenciosos. A
prosa de Bárbara era atraente e prendia a atenção do rapaz; assim,
deixavam-se conduzir pelo agradável de pensar em comum. Mas a verdade,
porém, era que Álvaro, embora gostasse de conversar, encontrava em tais
discussões uma fuga para os seus sentimentos. Era uma atitude de defesa,
instintiva, que, fazendo trabalhar a razão, desviava o seu forte
interesse pela mulher. Por isso, quase sempre, era ele o primeiro a
iniciar a conversa. Assim, voltando-se para Bárbara, indagou:

--- Por que está tão quieta? Ainda não disse uma palavra desde que
subimos.

--- Estou abismada com tanta beleza, Álvaro --- respondeu a moça olhando
ao longe.

--- Você não tem a impressão de que está num mundo à parte? ---
continuou ele.

--- Essa beleza, Álvaro, é tão estranha, tão diferente do que tenho
visto, que parece alguma coisa irreal. Francamente, nunca vi lugar mais
lindo em minha vida.

Álvaro mostrou-se alegre; parecia ter dado alguma coisa à moça, por
tê-la trazido ali. Bárbara continuou na sua atitude contemplativa e a
alegria de Álvaro foi desaparecendo aos poucos. Sentiu uma opressão no
íntimo, que talvez lhe estivesse indicando o perigo. Não queria
interromper Bárbara, mas era preciso fazer alguma coisa. Aquela tarde
quente, aqueles restos de luz viva, aquela nuvem colorida que se
alongava preguiçosamente... e o ar violáceo pairando sobre tudo e
fazendo ver as coisas como por uma lente colorida... largada a esmo no
espaço. Álvaro sentiu-se inquieto; teve medo de viver dentro de si
mesmo. Num esforço, tentou precipitar-se.

--- Bárbara --- disse interrompendo-a --- você, às vezes, não se sente
fora das coisas?

Ela voltou-se para o rapaz e esperou que se explicasse. Embora um tanto
agitado ele continuou.

--- Você não sente que dentro de você mesma, vive um outro mundo?

--- Sinto --- respondeu --- e algumas vezes esse mundo assume tais
proporções que eu chego a esquecer do mundo de toda gente.

--- E como vive então? --- indagou com a voz alterada, receando que ela
descobrisse o que vivia nele agora.

--- Intensamente. Pois é o momento que vivo mais de mim mesma.

--- E se sobrevém uma ocupação material, você não perde o controle? ---
prosseguiu o rapaz já com certo interesse no assunto.

--- Às vezes. O despertar de um sonho belo é penoso --- tornou a moça
com naturalidade.

Álvaro ouviu as palavras de Bárbara e no íntimo, como a reforçar-se nos
seus desígnios, comentou: --- Estou tentando fugir a um sonho belo;
preciso manter-me sempre desperto. --- E acendendo um cigarro, voltou ao
assunto primeiro:

--- É pena que exista o outro lado da vida; como seria formidável
viver-se das coisas belas!

--- Eu diria que as coisas belas são o outro lado da vida, Álvaro.

Ele sorriu:

--- É verdade, Bárbara; este será o outro lado da vida. A mediocridade e
a dor antecedem sempre à beleza. Há de haver um caminho antes deste
outro caminho que nós poderíamos chamar de o outro lado da vida.

--- Não disse o poeta que a criança ao nascer começa a chorar? E a
sabedoria, Álvaro, não está em revoltar-se contra estas primeiras
coisas; elas são o contraste, a causa eficiente, digamos assim, do
relevo dos coisas boas da vida. Isto, apenas por umas observações de
ordem prática\emph{;} não que eu deseje estabelecer a ordem definitiva
do que é para nós o bom e o mau.

--- Quer dizer que devemos aceitar as boas e as más coisas da vida.
Intelectualmente, eu aceito esta sabedoria a que você se refere; mas eu
não tenho o coração sábio do salmista, Bárbara. Essa desproporção entre
as boas e as más coisas da vida, me desconcerta, me desorienta dentro
desta própria vida --- respondeu sem esconder a sua revolta.

--- Mas essa desproporção --- objetou Bárbara --- se nós a
conhecêssemos, chamá-la-íamos mesmo de desproporção?

--- Você crê que essa desproporção, aos olhos humanos, seja, na
realidade, um equilíbrio? O raciocínio metafísico poderá nos levar a
isso, mas a vida... E escute, Bárbara, na verdade eu não sei como seria,
mas o fato é que pagamos um preço excessivo em dor por um minuto feliz.
Eu estou sempre de prevenção contra a vida.

Bárbara compreendia aquela revolta íntima; via-a através do sofrimento
do rapaz. E Álvaro vendo-a silenciosa, receou tê-la magoado: olhou para
ela com ternura e voltou ao assunto em discussão:

--- Você não pensa como eu, não é, Bárbara?

--- Em alguma coisa, Álvaro --- respondeu com delicadeza.

--- Que pensa dessa desproporção?

--- Pode ser uma criação da mente humana. Olhe a natureza e repare no
seu equilíbrio; por que haveria desproporção somente para com os homens?
Há nisso alguma coisa que ultrapassa às nossas faculdades.

--- Mesmo na natureza, Bárbara, me parece que as coisas boas e as coisas
más não estão em equilíbrio --- reafirmou o rapaz.

--- Por quê?

Álvaro desconcertou-se ante a simplicidade da pergunta.

--- Por quê? --- repetiu. --- Olhe um pouco para tudo.

--- E que há nesse tudo? --- perguntou a moça na mesma atitude.

--- Vivendo, vê-se em tudo sempre a mesma desproporção. Pense num
jardim. O mato não cresce mais do que as flores? Nas plantações, as
formigas não devassam e aniquilam, muitas vezes, o trabalho de dezenas
de braços? Você não se vê cercada por inúmeras coisas contrárias, quando
as do seu gosto podem ser contadas nos dedos da mão? E depois, Bárbara,
você não se diz cristã? Pois o próprio Cristo disse ``Basta a cada dia o
seu mal''.

--- Álvaro --- volveu ela com calma --- que são as coisas boas e que são
as coisas más?...

---\ldots{}

--- Quando o homem prova uma fruta e acha que é azeda, joga fora e diz
logo: não presta. Em tudo se põe como o centro, porque crê que o
universo foi criado para ele. Esse egocentrismo desloca o homem dentro
deste universo. Quando o homem se irrita contra o mato do seu jardim,
está esquecido de que o mato nascera ali espontaneamente; e ele o
substituiu por uma flor, contrariando a espontaneidade da natureza, mas
revolta-se quando a natureza desvia os seus desígnios. E você sabe
Álvaro, que para contrariar o que é espontâneo é necessário força. Não
tenho a menor ideia de aceitar a inação do homem; pelo contrário, acho
que deve lutar. Arrancar o mato do jardim, matar a formiga da plantação,
ir em busca da fruta doce; tudo isto ele deve fazer. Mas a uma coisa não
tem direito: queixar-se de uma luta que orienta para seu próprio
proveito.

Álvaro pensou um instante na verdade das palavras de Bárbara; não
poderia respondê-las. Era, porém, contra alguma coisa mais particular
que se revoltava; e estendia a tudo a particularidade das suas
circunstâncias pessoais.

--- Você faz considerações, confiando sempre, Bárbara; e eu descreio de
tudo --- tornou explicando-se. --- Afinal, quem me fez assim? Não fui eu
mesmo --- continuou Álvaro justificando a sua atitude.

Muitas vezes chegara àquelas mesmas conclusões de Bárbara; mas, não
passavam de conclusões ideais, pois, ao vivê-las, sentia-as
destruírem-se irremediavelmente. Álvaro vivia em eterno paradoxo. Sua
alma, cheia de coisas belas; sua vida, um edifício em ruínas. Após curto
intervalo, falou:

--- Não posso lutar contra o destino. Sorrindo, acrescentou: --- Seria
ir contra o próprio Deus.

--- Seria isto lutar contra o destino? Ou apenas orientar a sua luta?
--- indagou Bárbara com a mesma firmeza.

--- Já não me importa a orientação, uma vez que tudo \emph{é} luta ---
completou com uma revolta zombeteira. Depois sorriu outra vez e disse:
--- Pode ser que no céu não haja luta; eu, porém, nada conheço deste céu
e muito menos do equilíbrio que se diz existir nele.

Bárbara correspondeu-lhe ao sorriso e perguntou:

--- Você não conhece a história de Cauchy?

--- O matemático?

--- Sim. Ele e um sacerdote entraram em conjecturas sobre o céu. E o
sacerdote disse a Cauchy que o céu talvez lhe trouxesse todos os
problemas da matemática, resolvidos. Se for assim, respondeu Cauchy, eu
não quero ir para o céu, pois, o que há de belo na matemática é a luta
pelo alcance da verdade. Não é isto um exemplo de que a luta, quando bem
orientada, traz compensações?

--- A ponto de Cauchy não querer ir para o céu --- volveu ele com certa
ironia. --- E recordando as palavras de Cristo ``basta a cada dia o seu
mal'', que é que você responderia?

--- Não tenho a pretensão de achar respostas às suas perguntas; mas uma
ideia interessante me veio ao pensamento. O texto evangélico diz antes:
``Não vos inquieteis, pois, pelo dia de amanhã, porque o dia de amanhã
cuidará de si mesmo. Basta a cada dia o seu mal''. É psicológica essa
recomendação, Álvaro; e teria de ser feita por alguém que conhecesse
profundamente os homens. Você já observou que o homem pensa mais no mal
a evitar que no bem a fazer?

Álvaro acompanhou-a nos pensamentos e reconheceu intelectualmente a
verdade no que ela dizia. Mas a compreensão intelectual não lhe bastava;
Álvaro era sentimental e não conseguia ajustar a vida aos raciocínios
exatos. Aquelas demonstrações que lhe pareciam axiomáticas quando numa
filosofia, levavam-no a uma revolta íntima, da qual ele saía quase
sempre vencido. Álvaro sentia-se desnorteado num mundo onde a política
da sociedade se esforçava para que homens não pensassem. Aí estava o seu
desajustamento; pensava com ideias próprias e não tinha forças para
sobrepor-se àqueles que pensavam com ideias alheias. Sentindo que
Bárbara olhava para ele, concluiu com desapego:

--- Pode argumentar como quiser; as injustiças da vida percebem-se
quando são vividas. Têm início quando se abre os olhos para a luz. Veja
só isto: uns nascem em berços de ouro, outros nascem nas sarjetas.

--- Esta injustiça. Álvaro, chamam-na por aí, a justiça do próprio
homem. Olhe lá para baixo e contemple esta cidade. Que maravilhosos
arranha-céus, que ideia de riqueza para quem os vê. Mas, ao lado destas
riquezas há uma miséria imensa. Desça até a cidade e verá quantos lhe
estenderão a mão, implorando auxílio. É assim que se formam as cidades;
ao lado da magnificência, a fome. E não é Deus que tem de resolver estes
problemas; seria a humanidade entregando os pontos.

Álvaro quedou-se pensativo: Bárbara atraía-o, empolgava-o. Naquele
momento, ela lhe pareceu mais do que um ser humano. Era mais que uma
mulher, era uma ideia que tinha vida. E Paulo lhe dissera, um dia, que
Bárbara seria capaz de amar, capaz de compreender um amor. Capaz de
amar? Que tristeza o invadira ao fazer-se tal pergunta. Capaz de
amar?... Ele não ousava uma resposta. Capaz de compreender um amor, sim;
mas, amar...

\chapter{Capítulo 62}

Ao sair das meditações a que se entregaram, viram que era já noite
cerrada. E o espetáculo que os levara ao Pão de Açúcar, fora esquecido
naquele turbilhão de ideias; nem Álvaro nem Bárbara se lembraram do
acender das luzes.

Ainda não haviam jantado; Álvaro convidou a moça para fazerem juntos a
refeição. Desceram até ao restaurante da Urca e lá, ocuparam uma mesa do
terraço. A situação de Álvaro permitia já algumas extravagâncias sem lhe
pesar no orçamento. Um rádio trazia música àquele pequenino mundo; e ao
som dessa música, alguns pares dançavam. A refeição decorria com
simplicidade e Bárbara notou que Álvaro apreciava os pratos que ela
escolhia para ambos. Assim as horas passavam sem que as percebessem.
Álvaro, pelo trabalho irregular, perdera o hábito de controlá-las:
Bárbara, já há algum tempo, se desacostumara dos horários. Sua vida
tranquila, sem obrigações diárias que lhe pesassem no seu tempo,
tiraram-lhe o hábito de contar os minutos. Após a refeição prolongada
pela conversa e pelo serviço da casa, Álvaro convidou Bárbara para rever
a baía; ela, esquecida de que o tempo não para, acedeu ao convite. Ao
deixarem o recinto iluminado, a mudança brusca lhes ofuscou a vista; não
tardou, porém, que se acostumassem à penumbra e logo andassem com
desembaraço por aqueles jardins suspensos. Passaram pelo jacaré, e a
lembrança, de que o animal dormia, fê-los continuar em sossego. De
repente, pararam extasiados: era belo demais! Lá em baixo, a baía se
estendia tranquila nas suas linhas circulares; tinha formas qual um
corpo de mulher. A seus pés vinham espreguiçar-se as ondas do mar.

--- Bárbara! --- disse o rapaz emocionado.

--- Que é? --- atendeu a moça inadvertidamente, não chegando a crer que
se pudesse contemplar tanta beleza.

Edifícios iluminados formavam toda a espécie de linhas. Era a geometria
que se desenhava nas luzes da cidade, diria o matemático; era a vitrina
do ourives na ousadia de joias inalcançáveis, diria a mulher. E não
faltariam à exposição nem mesmo as pedras coloridas que começavam a
impor-se nas joias modernas.

--- Este é o nosso Parnaso! --- exclamou Álvaro num arroubo de
entusiasmo. --- Daqui se contemplam todas as belezas; e as que não se
contemplam, imaginam-se.

--- A imaginação, Álvaro, é também um recurso para perpetuar as coisas
belas --- respondeu a moça.

--- Deve ser por isso, Bárbara, que você vive sempre na minha ---
avançou Álvaro sem poder conter-se por mais tempo. --- Procuro
integrar-me nas coisas e só o consigo naquilo que vem de você.

A voz de Álvaro estava alterada; Bárbara não lhe podia ver a emoção na
obscuridade do ambiente, mas, sentia-a. Perturbou-se e tentou desviar o
sentido da conversa. Mais uma vez a ideia triunfara.

--- Álvaro --- tornou a moça compassadamente --- a primeira condição da
vida, seria adaptar-se a si mesmo.

--- De todos os que dizem isto, só conheço você que realiza
verdadeiramente.

--- Não digo que seja verdadeiramente, mas é o que eu posso.

--- Você também tem medidas? --- perguntou com uma expressão estranha,
magoado pela atitude de Bárbara.

--- Tenho circunstâncias...

--- E as circunstâncias lhe são auxílio ou entrave? --- volveu a
perguntar o rapaz.

--- Alguma vezes auxílio; outras vezes, entrave.

--- Quais são as de auxílio?

Álvaro prosseguia nas suas perguntas; descobrir aos seus olhos aquela
mulher tornou-se, de repente, uma necessidade imperiosa do seu espírito.
Conhecia o seu sentimento maternal; mas era necessário que se
aprofundasse no outro lado daquela alma feminina, mesmo que fosse para
perder-se nas suas profundezas. E ele ouviu a voz de Bárbara, enumerando
as circunstancias de auxílio:

--- Nascimento, vida...

E como parasse por um instante, o rapaz antecipou:

--- Nega a sua contribuição pessoal?

--- Não chego a este ponto. Mas eu também não tenho queixa da vida; ela
não me foi adversa.

--- Mesmo sem o carinho de um lar? --- acentuou o rapaz.

--- A impressão do meu lar não se desvaneceu em mim. O caráter de meu
pai, tal como eu o conheci, tem sido um sustentáculo na minha vida. Eu
podia ser eternamente revoltada por tê-lo perdido; mas a vida também me
deu muito e eu aprendi dais minhas experiências, a não desejar o
inaccessível. Se eu tivesse um lar, talvez deixasse nele toda a força
que hoje tenho.

--- Que forma de interpretação! Às vezes, chega a não me parecer humana
--- retorquiu admirado daquela firmeza.

E pensando na força de Bárbara, lembrou-se abatido da fraqueza própria.

--- Bárbara --- chamou com voz estranha --- você me julga um covarde,
não é?

--- Poderia julgá-lo se as nossas lutas fossem iguais --- contestou a
moça. --- Quem sabe o que eu teria feito se estivesse no seu lugar?

E Bárbara calou-se inquieta.

--- Em que está pensando? --- indagou o rapaz observando-a.

--- No silêncio, Álvaro. Por que estará tudo assim tão quieto?

\chapter{Capítulo 63}

Olharam em redor. A calma era quase absoluta. Voltaram ao restaurante;
estava fechado e escuro. Só a luz lunar os guiava. Lembraram-se então de
olhar para o relógio; passava de meia noite. Lá se fora o último bonde;
nada restava a fazer, senão esperar pela manhã. Bárbara sentiu-se
aflita; compreendeu a gravidade da situação, todavia, não se lamentou. O
mal estava feito e não havia com que remediá-lo no momento.

--- Que faremos, Bárbara? Creio que não é preciso dizer o quanto me
sinto embaraçado --- desculpou-se Álvaro com visível perturbação.

--- Ambos temos culpa. O Pão de Açúcar não é lugar para se filosofar,
quando se está em perigo de perder a condução --- tornou Bárbara
conformada.

--- E além de tudo, não há um abrigo. Tem que dormir ao relento. Oh,
como fui desastrado!

--- Não tem importância, Álvaro; faremos o que for possível.

--- Ainda bem que o tempo está firme --- disse observando as nuvens.

Ficaram ali parados, contemplando a entrada dos bondes, até que Bárbara
sentiu-se cansada.

--- Como poderemos descansar um pouco? --- indagou.

--- Quer ficar ali? --- perguntou ele, indicando lhe um lugar que lhe
parecia mais cômodo.

--- Vamos tentar --- tornou a moça. --- Essa monotonia... tudo parado
assim, não deixa correr o tempo.

Sob um alto arvoredo, procuraram acomodar-se. Álvaro tirou o paletó e
estendeu-o no chão sobre a folhagem. Ali, Bárbara procurou repousar,
descansando a cabeça num tronco mais ajeitado. Vendo que Bárbara estava
bem, Álvaro recostou-se ao lado, respeitosamente, apoiando-se em uma
grossa raiz saliente do solo.

--- Você está bem, Bárbara? --- perguntou uma vez mais.

--- Estou, Álvaro, incomoda-me vê-lo deitado assim, em mangas de camisa,
na relva úmida.

--- Se você estiver bem, o mais não tem importância --- tornou ele.

Um arbusto próximo tinha os galhos quase ao chão, e Bárbara movendo-se,
sentiu que a ramagem lhe segurou os cabelos. Álvaro apressou-se a
ajudá-la e aproximando-se, tentou afastar o galho atrevido. A associação
de ideias fê-los lembrar \emph{o} primeiro encontro; nenhum deles,
entretanto, exprimiu em palavras o acidente do arrabalde silencioso. O
rosto de Álvaro estava sobre o de Bárbara; para ajudá-la, aproximou-se
tanto que chegavam a sentir a respiração um do outro. Álvaro olhou para
ela; num instante, perdeu a noção das coisas. Havia um ímã que os
atraía. Álvaro roçou os lábios pelos cabelos de Bárbara, beijou-os de
leve; encostou-os às faces da moça, e foi beijando devagar todo o seu
rosto, até que seus lábios quentes encontraram os dela. Álvaro segurou-a
com força; uma vertigem íntima consumia-o. Bárbara passou-lhe as mãos
pelos cabelos em desordem e correspondeu ao beijo apaixonado do rapaz.

\chapter{Capítulo 64}

Quando Álvaro adormeceu era quase dia. E a noite correu vagarosa,
tranquila\ldots{} não se apressou, nem se ocupou deles; seguiu a marcha
do tempo. Quando Álvaro abriu novamente os olhos, era já dia claro.
Voltou-se para o lado e não viu Bárbara. Sentiu-se inquieto; num impulso
instintivo, procurou-a por toda a parte. Inútil o seu esforço, Álvaro
não a encontrou. Compreendeu então o que significava tudo aquilo. Vestiu
o paletó e esperou a primeira condução. Ia já afastar-se quando viu ali
o estojo de pintura de Bárbara; abriu-o. O batom escorregou-lhe por
entre os dedos e foi esconder-se nas ramagens. Álvaro apanhou-o e,
contemplou-o, lembrando-se dos lábios quentes de Bárbara. Invadiu-o um
sentimento confuso que não sabia explicar; sentia uma espécie de
vergonha e, ao mesmo tempo, uma sensação agradável em recordar todas as
minúcias da ocorrência. Compreendia a extensão do seu procedimento;
entregara-se à violência daquele amor. Seus sentimentos, há tanto
recalcados, explodiram numa força incalculável, cujo domínio ele perdera
por completo.

Tomou o bonde para regressar. No curto trajeto, seu pensamento
trabalhou--- Beijei-a, sim, disse de si para si, mas como foi diferente!
Pus naquele beijo toda a minha paixão, é verdade; mas foi como um
namorado cheio de esperanças! Estranha sensação a do amor! Como está
diferente o Álvaro de hoje! Só agora, compreendo o que é o amor. Eu que
sempre me entreguei a uma mulher com o sangue fervendo nas veias; que
não a considerava senão pelo prazer que pudesse dar-me! Que estúpido
fui, até agora! Pensava estar tirando proveito, quando na realidade
estava sendo aproveitado.

Álvaro prosseguiu nas suas cogitações. Embora se comparasse ao Álvaro do
passado, lembrou-se de que desejara a moça; se não o tentara, tinha a
certeza de que não o fora por ele. A lembrança, porém, não o
envergonhou. Compreendeu então que o sexo é parte integrante do
verdadeiro amor. O rapaz deteve-se um instante, para depois reafirmar
com certeza: é parte, sim; os homens é que os separam, impelidos por
essa marcha destruidora que os leva ia saciarem-se de prazer. ---
Bárbara, disse o rapaz cheio de tristeza, como fui encontrá-la assim tão
tarde? Uma nova pergunta surgiu-lhe--- E se a encontrasse, antes,
estaria preparado para percebê-la? Desorientado, dirigiu-se para a
cidade a passos vagarosos. Sofria horrivelmente e uma força estranha o
impelia a alimentar aquele sofrimento.

* * *

Na casa de Santa Tereza, Bárbara, sentada à escrivaninha, tentava
responder algumas cartas. Escrevia várias frases, mas, terminava sempre
por amarrotar o papel e jogá-lo ao cesto. Era inútil a tentativa; não
conseguia sobrepor-se aos acontecimentos daquela noite. Passou a mão
pela fronte, como a arrancar dali um sonho mau. Bárbara conhecia a
realidade--- amava Álvaro com toda a força dos seus vinte e quatro anos.
Seu coração virgem entregava-se pela primeira vez, e Bárbara, já mulher,
sentia em si os impulsos do primeiro amor. Não tinha os sonhos da
adolescência, pois além dos seus vinte e quatro anos, Bárbara amava a um
homem. Sentiu-se profundamente abatida, e não chegou a tomar uma
resolução quando a governanta entrou na sala. Entregou um embrulhinho
que lhe deram à porta. Abrindo-o, viu o seu estojo de pintura, esquecido
lá no Pão de Açúcar. Procurou um bilhete, uma linha que fosse; mas, ele
não mandara nada." Deixou o estojo sobre um livro e seus olhos aflitos
caíram sobre a fotografia do pai. Bárbara falou em voz alta, como se o
pai pudesse ouvi-la naquele momento difícil. --- Que diria, se soubesse?
Compreenderia?

Lembrou-se de que o pai lhe dissera, certa vez, que mulher alguma
ocuparia na sua vida o lugar que a esposa deixara. Bárbara sabia que seu
pai amara apaixonadamente a companheira; que a morte de sua mãe abrira
nele um sulco profundo, incapaz de ser preenchido outra vez. Numa
associação de ideias, lembrou-se dos casamentos que recusara; dos homens
que passaram pela sua vida, cortejando-a, admirando-a, disputando-lhe as
atenções. Nenhum deles lhe ficara na memória; tudo se passara como um
sonho, do qual Bárbara, despertada, não mais se lembrou. Fora insensível
a tudo e a todos! Mas o seu dia chegara também... Mulher bonita que era,
incrivelmente dotada, Bárbara parecia ter nascido para se deliciar com a
vida. A lembrança do passado, porém, não a fez arrepender-se de ter
permanecido solteira. A recordação do lar em que nascera, as lembranças
que lhe ficaram através dos exemplos vividos do pai, fizeram-na
compreender em que bases se deveria firmar a união de duas vidas.
Desprezar este princípio, era ser contra a sua própria natureza. Era o
rebento de duas criaturas que se amavam; sentiu isso pelos seus dias
todos. Percebeu com certeza que agora, firmada a sua personalidade,
jamais se entregaria a um homem que não estivesse antes na sua alma. E
este homem chegara até ela--- mas em que circunstâncias! --- não lhe
podendo dar o seu nome nem lhe dispensar proteção perante as leis.

Veio-lhe ao pensamento a velha tradição baseada no texto sagrado: ``E da
costela, que Deus tomou ao homem, formou uma mulher''. Tirou a mulher do
próprio homem para que estivesse unida a ele; não buscou uma parte do
cérebro para que não lhe fosse superior; não buscou dos pés para que não
se tornasse inferior; mas buscou do meio para que a tivesse como sua
igual, perto do coração para que a amasse e dos braços para que a
amparasse. --- ``Assim Deus deu ao homem a mulher''.

Sentiu uma angustia imensa. E aquela mulher altiva, de atitudes
inconfundíveis, apanhou o fone e discou para a pensão de Álvaro.
Informaram-na de que ele não estava.

Bárbara procurou em que ocupar-se para não sentir o vagar das horas. Mas
quando voltou a telefonar, a senhoria contou-lhe que Álvaro partira, não
sabia para onde. Um empregado dissera ser para o norte, pois, vira-o
negociar uma passagem já adquirida. Era tudo o que lhe podia informar.

Para o Norte, pensou. Vai até Paulo Machado. Qualquer coisa ligou o meu
destino ao de Helena. E Bárbara, deixando o fone, recordou a cena da
noite. Aquela mulher que não vivia do passado, reconstruiu no seu
pensamento, passo a passo, o último encontro. Reconstruindo-o, sentiu
que o vivia novamente. Os lábios quentes de Álvaro achegaram-se aos seus
com vagar; e Bárbara, na força da sua imaginação, sentiu-se beijar outra
vez...

A imaginação, Álvaro, é também um recurso para perpetuar as coisas
belas.

\chapter{Capítulo 65}

Absorvida nos seus pensamentos, Bárbara deixou passar as horas. Só
voltou a si, quando a voz de Helena se fez ouvir na porta de entrada.
Bárbara foi até a amiga e, num arroubo de ternura, abraçou-a
demoradamente. Uma amizade mais sensível estreitou os laços existentes
entre elas. Quando caminharam abraçadas para a sala de música, havia
algo diferente na alma das duas amigas: queriam-se mais. O destino
esquisito e incompreensível preparara-lhes sorte semelhante.

Helena parecia mais aflita que das outras vezes; sua voz tinha timbre
nervoso acentuado. Sentaram-se juntas no divã de leitura e Helena falou
de várias coisas sem importância até que, dando pela falta da sua
fotografia, interpelou a outra:

--- Onde está a minha cabeça?

--- Olhe no espelho e verá.

--- Refiro-me à da moldura; você bem sabe.

Bárbara tentou sorrir; seus lábios, porém, mal se moveram. Helena,
preocupada com outros assuntos, esqueceu-se da fotografia. Correu o
olhar pela sala, dando tempo a que seus nervos se refizessem. Bárbara
compreendeu que estava para vir à tona alguma revelação séria.

--- Sabe? --- disse Helena abruptamente --- vou me casar.

---!!!

--- De que se surpreende? Não é o caminho de toda a gente?

--- Vem noticiar-me um casamento? Agora?

Ela não podia crer no que ouvira.

--- Venho, sim --- ajuntou Helena com tristeza.

--- Nesta atitude?! --- inquiriu Bárbara, aterrada pela revelação.

--- Nesta atitude --- confirmou Helena; como se a confirmação fosse a da
sentença de um réu.

--- Que a levou a isto? --- interrogou Bárbara.

--- Não sei... o que leva toda gente a casar.

--- Não sabe? O casamento não é para você, Helena?

--- Também não sei --- exprimiu-se com uma tristeza mesclada de ironia.

--- Que pensa fazer então?

--- Por que faz perguntas assim? Sabe que eu não sei responder ---
tornou Helena irritada.

Bárbara silenciou.

--- Não me pergunta com quem? --- voltou a outra, submissa como a
reparar a sua irritação.

--- Com quem, Helena? --- perguntou afetuosamente a amiga.

--- Luiz Porto da Rocha, o milionário.

--- Ah!... o milionário.

Helena desandou a chorar e Bárbara deixou que chorasse à vontade, para
aliviar o seu espírito. Contemplando-a, uma pergunta, se ergueu no seu
íntimo: estará Helena chorando pelo que foi ou pelo que ainda há de vir?
Helena expandiu as suas lágrimas e, quando mais calma, Bárbara
ofereceu-lhe um refresco. Mais tarde, perguntou-lhe com afeto:

--- Quer conversar, agora?

--- Quero sim, preciso dizer-lhe muita coisa.

--- Então fale, Helena; eu a ouvirei com o ouvido e o coração.

--- Não sei se eu já lhe contei alguma coisa...

Bárbara respondeu negativamente com a cabeça.

--- Tudo começou no casamento de Ivete --- iniciou Helena pausadamente.
--- Conheci Luiz naquela festa e desde então, ele não mais deixou de
procura-me. Eu tive certa culpa, pois, procurei gostar dele.

--- Procurou, Helena? --- insistiu a amiga.

--- Procurei, como de muitos outros. Penso que por isso, ele avançou
mais. Passou a ir diariamente à minha casa; e não sei de que modo
interpretou a minha tentativa. Depois, você sabe perfeitamente, estes
homens de dinheiro não concebem uma recusa. São sempre procurados;
acostumam-se a ser preciosidades. Assim, pediu-me a papai, julgando-me
fazer uma surpresa agradável. E percebo pela sua atitude que a ideia de
uma recusa não lhe atravessou o espirito.

Helena olhou tristemente para amiga, suspirou com desalento, e
prosseguiu a narrativa:

--- Papai falou comigo, e mamãe estava presente. Fiquei surpreendida,
porém, não dei resposta. Papai instou, queria saber o que eu pensava.
Mostrei-me indiferente à honrosa distinção e respondi que a ideia de
casamento não existia para mim. Mas o casamento, Bárbara, é do gosto
deles; não sei se pela posição de Luiz ou para se verem livres de mim.
Procuraram convencer-me de todos os modos. Mamãe perguntou-me, irritada,
se eu espero por algum príncipe encantado. Papai diz que eu sou livre, e
me caso com quem quiser; mas impedia Paulo e empurra-me para Luiz por
vias indiretas. Toma uma atitude irritante comigo, como você não
calcula. Zanga-se por qualquer coisa, grita sem razão. Enfim, eu estou
num inferno --- contou Helena com desespero.

Bárbara apiedou-se dela e considerou a triste dependência da amiga. Com
que intenção a criaram os pais, submetendo-a, daquela forma, a juízos
exteriores à sua personalidade? Como se adivinhasse os pensamentos da
amiga, Helena prosseguiu:

--- Afinal, depois de uma luta exaustiva, disse ontem que aceitava.

--- E seu pai o que disse?

--- Quase nada. Se você visse, Bárbara, como eu chora ao lhe dizer que
sim. Eu tenho a impressão de que este casamento é uma responsabilidade a
que eu não posso fugir; mas, de uma coisa você esteja certa; não fui eu
que escolhi.

--- Ele não lhe deu um conselho? Não disse nada?!...

--- A princípio tomou uma atitude que não me encorajou a voltar atrás.
Depois, falou que eu, agora, não compreenderia o valor deste casamento;
mas, mais tarde... Luiz é um marido seguro e de recursos. Mamãe e ele
estariam menos inquietos com a minha sorte. E como conselho, um único;
que procurasse estar sempre ao lado de Luiz; pois um marido mais velho
chega a ter um ciúme ridículo. Ah, Bárbara, como eu me revolto com tudo!
Se eu fosse mais livre, se pudesse ganhar a vida por mim\ldots{} Mas
você sabe como são as coisas lá em casa; mulher não tem voz ativa. Meu
irmão nem se lembra de que eu também sou uma criatura humana. Trata-me
como se eu fosse o gato ou cachorrinho da casa; no entanto, é um escravo
da própria esposa. Você se lembra daquela vez em que ele me tirou de um
baile, quando eu dançava com Paulo? Hoje, Bárbara, quem o tira do baile
é a mulher; e ele segue como um cãozinho domesticado que ela puxa pela
corrente. Com todos os direitos da esposa, não pensou ainda nos direitos
da irmã! Bárbara, às vezes, eu creio não pertencer ao lar, em cujo meio
eu vi a luz. Oh, vida! Que é que se pode fazer dela? Ser mulher... é uma
desgraça irremediável! Eu posso dizer isto convicta, porque sei o que é
ser mulher.

--- Helena --- tornou Bárbara com energia --- lute contra este
casamento. É uma loucura, cuja extensão você não pode avaliar agora.

--- Às vezes, eu fico pensando nos casamentos antigos; não eram por amor
e davam certo. Quem sabe se eu me casando, me tornarei mais livre, mais
independente da família e serei mais eu mesma. Quem sabe ainda virei a
gostar de Luiz. Quem sabe...

--- Mas Helena, a vida, não se constrói em quem sabes\ldots{} Calcule se
o engenheiro edificasse um prédio sem a certeza do resultado. Quem sabe
o prédio ficaria de pé e daria tudo bem; mas, quem sabe se ruiria por
terra, e esmagaria ali muita gente. Você viu tanta infelicidade na sua
casa e agora vai seguir o mesmo caminho!

--- Que mais eu poderia fazer?

--- Você bem sabe, Helena, que eu faria tudo por você; mas há casos em
que a resolução inicial e a luta consequente precisam ser pessoais. A
minha casa está às suas ordens; dinheiro, tenho para nós duas. Você sabe
que eu não vivo com grandes gastos e nem penso em fazer fortuna. Você
precisa afastar-se para ter o pensamento mais claro; para aconselhar-se
consigo mesma. E agora Helena, quero fazer-lhe uma pergunta: você ainda
gosta de Paulo?

--- Paulo?... É tudo tão distante, tão longínquo --- tornou a outra
pensativa,

--- Não é, Helena. Paulo ainda não se esqueceu de você.

Um choque elétrico não a teria perturbado mais. Helena fixou os olhos em
Bárbara, e, duvidosa, perguntou:

--- Como chega a dizer semelhante coisa? Recordar um passado extinto é
até crueldade, Bárbara! Paulo não existe mais; mamãe me afirmou que ele
morreu embriagado.

--- Só se for algum outro Paulo, porque o Machado existe, e eu estive
com ele.

O coração de Helena descontrolou-se; as batidas pareciam saltos, e
algumas manchas vermelhas formaram-se rapidamente nas suas faces.

--- Olhe para aquela moldura vazia --- falou indicando-a --- a sua
fotografia está com ele.

Helena, com a respiração ofegante, não conseguia articular uma, palavra.
Olhou várias vezes para Bárbara, agarrou-lhes as mãos com violência e só
há muito custo, conseguiu perguntar:

--- Quando? Como foi isso?

Bárbara contou-lhe, então, como Paulo batera à sua porta numa manhã. A
sua surpresa ao vê-lo, e como era ainda o mesmo Paulo que Helena amara.
Às perguntas que fizera sobre a antiga namorada, como comentara o
passado num enlevo ingênuo e romântico. A fotografia de Helena entre as
mãos grandes e fortes do rapaz e a paixão com que ele a contemplara.
Contou tudo; descreveu pormenores que ela mesma não tinha consciência de
os ter notado; e, à insistência de Helena, repetiu várias vezes as
mesmas coisas. Quando terminou, Helena perguntou numa ansiedade
tremenda:

--- E você sabe como encontrá-lo, Bárbara?

Encontrá-lo? Como encontrar, agora, Paulo Machado? Estava, no Norte, mas
o Norte era uma região bastante extensa para nele se localizar uma
pessoa. E agora que Álvaro partira também que lhe restava fazer? Como é
esquisita a vida! Se Helena a procurasse na véspera, tudo estaria
solucionado. Helena poderia comunicar-se com Paulo e ela saberia para
onde teria ido Álvaro. Mas, teriam acontecido estas coisas para que não
os encontrassem? Embora, confiado sempre, como Álvaro costumava dizer,
Bárbara sentiu que o jogo da vida começava a embaraçá-la.

\chapter{Capítulo 66}

Dias depois, Bárbara dirigiu-se ao Pão de Açúcar. Ao sair do bonde, o
coração bateu-lhe precipitadamente. Começou a andar pelos mesmos lugares
que viera conhecer com Álvaro. As emoções voltavam-lhe pouco a pouco;
ela sentiu que ``recordar é'' quase ``viver outra vez''. Chegou-se para
contemplar a baía.

Naquele cair de tarde, em que o sol desaparecia no poente, os últimos
raios refletiam ainda em uma pequena extensão das águas verdes. O ouro
do sol e o verde do mar, as cores da bandeira do Brasil, terra em que
vivera, crescera e amara. Encostou-se a um arvoredo, com as mãos no
bolso do casaco branco, comprido e desabotoado. Uma brisa leve batia-lhe
no rosto, levando para trás os seus cabelos; seu olhar distante
perdia-se no espaço. Álvaro parecia estar a seu lado e, cerrando os
olhos, pôde ouvir-lhe as palavras: ``Você está bem, Bárbara?'' Queria
dizer-lhe que não, mas permaneceu imóvel e continuou a sonhar. --- Se as
coisas belas vivem na imaginação, deve ser por isto que você está sempre
comigo. --- Bárbara abriu os olhos e contemplando as belezas naturais do
lugar, lembrou-se ainda de Álvaro. --- Daqui contemplam-se todas as
belezas e as que não se contemplam, imaginam-se.

Tal como Beethoven, completamente surdo, sentindo no seu mundo interior
a combinação harmoniosa dos sons, sem ouvi-los, Bárbara sentia em si
mesma toda a irradiação de Álvaro, sem vê-lo. Álvaro estava a seu lado
pelo pensamento; sabia que ela entrara na sua vida como ele entrara na
dela. Não se entenderam depois; nada disseram, nem se despediram.
Separaram-se bruscamente. Quando ele estava, ela saiu; quando ele saiu,
ela o procurou. Nem um adeus, nem uma palavra.

Contornou o restaurante, andando vagarosamente; demorava o olhar em cada
pedacinho de terreno em que estiveram juntos. Depois, regressou ao
ponto; um bonde ia sair. Entrou no meio dos passageiros e veio novamente
à cidade. Tomou o carro que deixara ali e saiu daquele recanto cheio de
saudade, onde cada objeto falava alguma coisa. Quando passou pelo Teatro
Municipal, viu um grande cartaz:

HOJE

BEETHOVEN--- NONA SINFONIA

Encostou o carro próximo dali e tentou chegar à bilheteria. Nem um
lugar. Tornou a olhar o programa; faltava meia hora para começar o
concerto. Na primeira parte--- Romance--- na interpretação de um
violinista que Bárbara não conhecia. Na segunda, a Coral. Pediu ao
bilheteiro que lhe cedesse alguma entrada, mesmo em pé; ou que lhe
fornecesse alguma, devolvida antes do espetáculo; ela esperaria. Sem
chapéu, casaco branco, esporte, e completamente só, Bárbara ficou ali, à
espera de que o acaso a favorecesse. Estava distraída e pensativa.
Alguém a chamou baixinho. Voltando-se, viu Carlito que a fitava
admirado.

--- Por que está aqui, Bárbara?

--- Hoje é um recital de Beethoven. E estou nos dias em que preciso
ouvir Beethoven --- explicou convicta.

--- Qual é o seu lugar? --- indagou o moço.

--- Não consegui entrada; estou aqui à espera; quem sabe...

--- Leve a minha, Bárbara --- e estendeu-lhe a entrada, precipitado.

--- Não, Carlito; eu não teria coragem --- retorquiu com firmeza.

--- Não a procurei, me deram por acaso --- insistiu o rapaz.

--- Mas ia ao concerto e eu não lhe tiraria este privilégio.

--- Por favor, Bárbara, aceite-a. Digo-lhe que eu gostaria de ir mais
para estar ao seu lado.

Bárbara não lhe deu resposta, e Carlito percebeu que as suas palavras se
perderam ao serem pronunciadas.

--- Eu não entendo de música --- tornou ele --- quando muito chego a
apreciar um romântico. Beethoven está além da minha alçada e embora o
tenha ouvido algumas vezes, parece que só o conheço de nome. E eu,
Bárbara, me sentiria feliz por ceder-lhe a entrada.

Carlito mostrava-se terno e comovido; a atitude de Bárbara provocou-lhe
um sentimento diferente. Viu-a ali, à porta do teatro, esperando
humildemente que lhe cedessem uma entrada, e sentiu desejo de falar-lhe
em outro tom. Aquela mulher altiva, que ele vira dominando sempre, como
poderia postar-se ali sozinha, à espera da boa vontade do bilheteiro?
Não seria o caso de desistir logo, mandando toda aquela gente para os
diabos? --- Carlito não compreendia que no portal da música os espíritos
mais altivos se postariam.

Bárbara não fez um movimento para receber a entrada do rapaz. Carlito
colocou-a no bolso do casaco branco; sentindo nele a mão quente de
Bárbara, sem poder conter-se apertou-a... e retirou-se depressa. Bárbara
ficou ainda ali, por mais alguns minutos; sentiu o contato com o papel
áspero da entrada que viera de forma tão singular. O sinal do teatro
avisou pela primeira vez. Deu os primeiros passos e resolveu. Entrou no
teatro, procurou a poltrona que a sorte lhe trouxera e esperou a
mensagem que Beethoven deixara aos que lhe pudessem acompanhar na
jornada.

\chapter{Capítulo 67}

Beethoven. --- Romance.

Bárbara ouviu as primeiras notas; a beleza daquela música transportou-a
a outras regiões. Dali por diante, o violinista não usava as cordas do
instrumento; o seu arco feria as cordas humanas do coração de Bárbara. O
seu romance estava escrito naquelas notas. Era triste, sem ser mórbido;
a tristeza de quem sente em si o peso da vida, mas que caminha a passos
firmes porque tem a \emph{alma livre.}

Desprezando o espírito servil, Beethoven não cedeu à própria vida.
Quando ela lhe tirou o seu mais precioso tesouro, ele lhe devolveu os
sons combinados e harmonizados, numa afronta suprema, como nenhum mortal
o fizera antes dele. O seu exemplo era único na história. No desfilar
das grandes personalidades que passaram pelo mundo, Beethoven ia à
frente como a baliza de todos os tempos--- símbolo de uma força que
desafiou a força da própria vida. Por isso ele dissera: \emph{a força é
a moral dos homens que se destacam dos demais; e é também a minha.}

Bárbara sentia plenamente a sua mensagem; nunca ela lhe falara como
naquela noite. Ouvia sempre Beethoven, e, cada vez, a sua música lhe
trazia emoções novas; respondia sempre ao apelo do seu espírito.

Começava \emph{o} ADAGIO da nona sinfonia; página de recolhimento,
elevando a alma ao êxtase supremo. Bárbara lembrou-se da frase do mestre
e ouviu-a através da música. \emph{Desejo a redenção de toda a miséria
para quem penetre o sentido dia minha música.} O coração de Bárbara
batia com as notas; no seu mundo interior ecoava ainda a frase musical,
captada pelas antenas da alma. De grande poder emotivo, aquele movimento
não a prostrou. De olhos enxutos, acompanhou a meditação do Mestre
sentindo, cada vez mais forte, a centelha do fogo sagrado que se
acendera no seu espírito. Deste mesmo fogo, Beethoven falara ao
contemplar os que choravam, ouvindo a sua música. \emph{Ah, loucos, não
são artistas! O artista é de fogo e não chora.} Não obstante fosse uma
página de dor, era mais uma meditação; algo que convidava a considerar a
própria dor--- \emph{o homem deve ser forte e valente em tudo.} Era
triste deixar-se vencer--- \emph{fazem falta almas livres que prefiram a
morte} à adulação\emph{, a pobreza à servidão... e de tais almas, a
minha não será a última.}

Veio o quarto movimento; era a ressurreição da alegria. Já não era
possível entregar-se assim... seria como a morte num dia de sol!

Não, Bárbara não morreria, VIVERIA para sentir as vozes humanas que em
triunfo, cantariam a mesma VIDA. E quando o coro iniciou o cântico,
lembrou-se de Álvaro que mais de uma vez lhe traduzira aqueles versos. E
um deles, que fizera parte de Leonora, vinte anos antes de ser incluído
no coro da sinfonia, veio-lhe ao pensamento: \emph{Quem conquistou o
carinho da mulher amada, una-se à nossa alegria.} Às últimas notas,
Bárbara, transportada, repetiu os versos do poeta brasileiro:

"... E a obra, por fim, resplandece acabada:

Mundo que as minhas mãos arrancaram do nada!"

\chapter{Capítulo 68}

Pelas dificuldades que surgiram, Álvaro fazia a sua viagem
interrompendo-a muitas vezes. Assim, pois, naquele cair de tarde, estava
no tombadilho do navio, contemplando os últimos raios de sol; os mesmos
que iluminavam para Bárbara, a vista panorâmica da baía. Ao longe, os
raios de fogo a refletirem na grande extensão verde da superfície do
mar. O ouro do sol, o verde do mar, as cores da bandeira do Brasil;
berço natalício de Álvaro. Esta terra grande, desconhecida ainda para a
maioria dos brasileiros. Terra em que Álvaro vira a luz pela primeira
vez; terra onde crescera, vivera e amara. Um rincão abençoado; onde as
terras devolviam em alimentação farta as sementes que lhe lançavam; onde
os subsolos férteis das mais raras riquezas universais, guardavam as
suas jazidas, receosos de que os filhos de outras terras, impulsionados
pela cobiça, viessem lançar aqui as suas vistas, num olhar de quem
descortina ``o futuro do mundo''. Álvaro sentia-se brasileiro; e amara
mais o Brasil, quando o vira através de uma estrangeira, que
compreendera, de forma ímpar, a grandeza de terra hospitaleira.

Contemplando aquele cair de tarde, os últimos restos de um dia que se
vai, Álvaro sentia, no recôndito de sua alma, a saudade docemente
dolorida da mulher que se radicara na sua existência. Pelos dias que
estivera longe dela, tivera consciência da impossibilidade de afastá-la
de seu espírito.

Estendeu o olhar naquela amplidão imensa e, de um céu imaginário,
Bárbara apareceu em visão:

--- A imaginação, Álvaro, é também um recurso para perpetuar as coisas
belas.

A visão aproximava-se e Álvaro parecia sentir Bárbara cada vez mais
perto:

--- Estarei a seu lado, sempre, enquanto precisar de mim.

E agora não a necessitava mais que nunca? Era no auge das suas
necessidades que Bárbara lhe faltava.

E Bárbara não precisaria dele? Seria assim tão forte a ponto de poder
dominar uma paixão? Deixara-se beijar por ele; ainda mais...
correspondera àquele beijo apaixonado! Sim, Bárbara o amava! Paulo tinha
razão; Bárbara era capaz de amar. Mas depois, fugira... Álvaro não lhe
pôde ir ao encalço --- temeu enfrentá-la.

E temeu por quê? Porque não tinha um nome livre para oferecer-lhe.
Maldito passado! --- disse entre dentes. Ah, o passado, o passado! Nunca
esse passado lhe pesara tanto! Ah, se pudesse voltar atrás! Desmanchar
aqueles anos infelizes e construir uma vida melhor. Quem o fizera viver
assim? Que teria realmente a culpa do que passou?

Numa revolta amarga, cerrou os punhos e ergueu os braços. Quisera
quebrar as colunas do mundo como Sansão as do Templo! Antes dele,
também, um outro homem, cujo exemplo passara à história, fora vítima de
uma Dalila. Deixou cair o braço, vencido pela fragilidade humana. Ele
não era Sansão e o mundo não tinha colunas. Uma onda de ódio sacudiu
todo o seu ser. Seria capaz de matar para se ver livre daquela mulher.
Uma vez que os homens não lhe faziam justiça, ele a faria por si mesmo.
Quanto ódio a lhe invadir a alma! Álvaro desconhecia a sua capacidade
para odiar e, percebendo-a, contemplou-se horrorizado.

Uma onda de música lhe chegou ao ouvido. Conheceu-a, sentiu-a... e
aproximou-se então do rádio que a transmitia. Aquela música pareceu
trazer-lhe uma mensagem diferente; teve quase certeza de que Bárbara a
ouvia também.

Do Teatro Municipal do Rio de Janeiro, irradiavam o concerto de
Beethoven, e Álvaro somente o apanhou já no final da Nona Sinfonia.

A frase de Beethoven voltara-lhe numa promessa: \emph{Desejo} a
\emph{redenção de toda a miséria, para quem penetre o sentido da minha
música.}

\chapter{Capítulo 69}

Quando Helena voltou à casa de Bárbara, o seu casamento já estava
marcado e certo para breve. Atormentada na casa paterna, a ideia de
retirar-se deu-lhe esperança no futuro. Assim, o seu primeiro gesto foi
revelar à amiga a sua intenção.

--- Sabe, Bárbara, vou mesmo casar. Tenho pensado muito e tirei minhas
conclusões. Como disse, antigamente, os casamentos eram feitos sem
liberdade de escolha e davam certo. O amor é como um negócio: questão de
tentativa. Luiz sabe que não morro de amores por ele, nem por isso se
importa. Naturalmente, conta que acertemos a vida mais tarde --- relatou
Helena com um desânimo doentio.

Bárbara ouvia-a atentamente; sabia que uma situação de desespero é que
levava Helena a tal loucura. E uma vez que ela resolvera pelo mais
fácil, no momento, teve medo de piorar as circunstâncias. Assim, Bárbara
nada respondeu.

--- Minha amiga --- reclamou Helena com tristeza --- não proceda assim
comigo. Diga alguma coisa, mesmo que seja para censurar-me.

--- Helena --- tornou a outra indecisa --- eu tenho receio de dizer-lhe
as coisas.

--- Por quê? Não é você a minha maior amiga?

--- Por isso mesmo. Não quero vê-la infeliz e receio colocá-la em
choque. Digo o que penso, e você me compreende; depois, não resiste a
seus pais e entra pela vida com maior certeza de ser infeliz. Você deixa
que a empurrem para o mal, conhecendo o próprio mal; isto fará com que
duvide de si própria. Então, se acovardará e será eternamente infeliz. A
verdade é difícil de se saber, Helena.

--- Bárbara, diga tudo o que você pensa; não sei que demônio me impele a
torturar-me assim, mas eu quero, eu preciso que você diga tudo.

--- Você leu o livro de Álvaro, não foi? Lembra-se de um trecho em que
ele diz \emph{não existe o sol de inverno para uns e o de verão para
outros.} Pois bem, minha querida amiga, é isto o que você vai fazer;
colocar-se propositadamente sob os raios de um \emph{sol de inverno..}.
Lembro-me de um pensador argentino que disse coisa semelhante: ``um
pensamento não iluminado pela paixão é como o sol de inverno; ilumina,
mas sob os seus raios, pode-se morrer de frio''.

Helena ouvia com um interesse religioso. Bárbara prosseguiu:

--- Bem, Helena, uma vez que você quer ouvir, eu direi tudo o que penso;
mas esteja certa, a verdade nem sempre é agradável.

--- Mas é a verdade...

--- Primeiramente, eu só poderia reprovar a atitude de seus pais. Não
sei como podem ser tão indiferentes à sua felicidade. Criar uma filha,
dando-lhe apenas assistência material, nem os animais o fazem. Agora que
se acham cansados, procuram para você um marido seguro. Isto, você bem o
percebe, não é por afeto, mas, por comodidade. No outro dia em que
estivemos lá, criticaram o casamento russo, dizendo que o ato consistia
apenas em uma troca de cartões. Não chego a ponto de acreditar nessas
coisas; esta troca de cartões poderá ser um divórcio legal como o da
minha terra ou dos Estados Unidos e tantos outros países onde o divórcio
existe. Mas, se admitíssemos que fosse uma simples troca, o que tanto
reprovaram, não estarão eles agora agindo de maneira pior? Os cartões,
se é que existem, podem ser trocados entre pessoas que se amam, sem a
influência do dinheiro; aqui no seu caso, é a mulher em troca do
dinheiro. É a mulher mercadoria, sem um gosto pessoal ou uma orientação
própria; é a mulher não preparada para a sua subsistência, colocando-se
sob a pseudoproteção de um casamento circunstancial. E o casamento sem
afeto, Helena, é uma prostituição legal. Agora, você me diga, não estão
os seus pais agindo de maneira pior que os tais cartões da Rússia? Eles,
porém, não percebem isso. O homem é assim mesmo: pronto a condenar tudo
nos outros, até no que não sabe ao certo; para de outro lado, esconder
as suas imperfeições próprias. Helena, este assunto é de uma aspereza
profunda, mas eu quero dizer tudo para ajudá-la a fugir do abismo. Tenho
encontrado muitas mulheres, que por uma falsa orientação, destruíram-se
para sempre.

Helena sorriu com um desprezo estranho; parecia a mulher gasta e
experimentada que nada mais tem para si. E pelo olhar da amiga, Bárbara
lhe adivinhava os pensamentos; era a mulher cansada, vencida, que se
entregava. Seu corpo ainda era virgem, mas na sua alma, Helena sucumbira
pelo contingente das circunstâncias. Bárbara, conhecendo os sentimentos
da amiga de juventude, tentou, num esforço supremo, reerguer aquele
caráter, cuja nobreza estava sendo atingida de maneira cruel e sutil.

--- Helena --- chamou com firmeza --- este desânimo é temporário; você
vai sair desse torpor, minha amiga. Conheço-a bem para fazer tal
afirmativa. Lembre um pouco, Helena, que pelo casamento você irá
entregar-se a um homem; e se não o ama, como poderia suportar semelhante
coisa? A brutalidade sexual não a dominará como a tantas outras
mulheres; você poderá ceder algumas vezes, cultivando a sua
insensibilidade, como o fazem as mulheres da vida. É isto mesmo, Helena,
não haverá diferença; e quando você perceber isto, tentará reagir. Mas
ao reagir, suscitará também reações; primeiramente seu marido, depois
sua família e, por fim, a sociedade. Nem um deles compreenderá que nem
mesmo você é proprietária de seus sentimentos; uma vez que não pode
impor-se a gostar do Luiz Rocha ou de Paulo Machado.

--- Mas se as coisas não derem certo, Luiz não poderá queixar-se.
Percebe que eu não tenho amor por ele; a seu lado sou apenas delicada,
pois, não tenho intenção de ofendê-lo. Há quinze dias passados, estive
no cassino e a mandado de minha cunhada, convidei Luiz para ir também.
Uma vez lá, esqueci-me dele por completo; dancei a noite toda com o Rui.
Luiz acabou saindo antes; e depois disso, ainda pediu-me a papai. Você
acha que ele não percebe?

--- Helena --- tornou Bárbara compassadamente --- cuidado com os homens!
Ainda mais com os que têm dinheiro. Pelo que se passa, ele não terá
direitos de queixas; ainda mais num casamento apressado como vai ser o
seu. E isto é de propósito; você não terá, assim, tempo para refletir.
Poucos homens são realmente homens! Mais tarde, quando você perceber
todas essas coisas e sofrer por isso, ninguém lhe estenderá a mão.
Quando a sua personalidade se impuser, forçada pelo seu mundo interior,
encontrará o mundo exterior contra você. No Brasil não existe ambiente
para a mulher de personalidade formada que não se dobra às
circunstâncias. Você vai lutar muito, Helena; mais do que agora, esteja
certa. Os homens não conhecem a caridade; tornam-se muitas vezes
assassinos, dentro de uma religião cristã que eles dizem praticar.
Helena, pense na sua vida de casada com o Luiz Rocha; pense em todos os
pormenores... sem receio de pecado, e veja se você conseguirá chegar a
isso. Sei que você não tem com quem trocar ideias, aconselhar-se, pois
você vive sem amparo na sua própria família; é por isso que eu insisto
em trazer à luz todo esse problema. Agora, Helena, posso fazer-lhe umas
perguntas?

--- Claro, Bárbara --- disse Helena profundamente agitada.

--- Como você acha que Luiz Rocha encarará esse casamento se não der
certo?

--- Não lhe passará pela mente que uma mulher possa não entregar-se
quando ele a sustenta perante a sociedade. Que engraçado, Bárbara ---
continuou a moça pensativa --- sabe de quem eu me lembrei agora?

--- Não.

--- Do Rhett Butler.

--- Do ``E o vento levou''? Por quê? --- indagou a outra.

--- Não sei; lembrei-me de repente. Talvez, seja porque Luiz o criticou.
Tachou-o de cínico, sem vergonha etc. Coitado do rapaz; ainda bem que
viveu apenas na cabeça da romancista.

--- Que coisa interessante, Helena. Luiz criticou o cinismo, a
sem-vergonhice do rapaz etc., mas, não se lembrou que este mesmo rapaz
não chegou ao ponto de aceitar para os braços dele uma mulher que não o
amasse. Veja, você, minha amiga, como os homens são cegos para as
qualidades alheias e cegos para os seus próprios defeitos. O tal Rhett
não deixava de ser um personagem interessante; era um homem que pouco se
importava com a opinião dos outros --- recordou Bárbara. --- Na verdade,
era mais homem que a maioria, embora com todos os seus defeitos; ficou
mal visto porque não se deu ao trabalho de esconder o que era. E afinal
--- disse Bárbara voltando ao assunto anterior --- como irá você,
Helena, encarar a sociedade depois do seu casamento?

--- Esta, na verdade, pouco me importa; o caso, porém, é que ela irá
dificultar-me a vida...

--- É isto mesmo, Helena. E restringindo essa sociedade aos seus
parentes, aos seus pais, sabe o que lhe dirão\textsuperscript{?}

--- Imagino...

--- Embora você o imagine, não é demais repetir. Dirão que vocês estão
casados e é preciso a sua submissão; a mulher deve submeter-se sempre...
Para a harmonia, para a paz, para a felicidade de todos deve haver
sempre uma sacrificada. E essa sacrificada, minha amiga, é, como vê,
sempre a mulher. Sua mãe, como religiosa que é, apontará a sua
responsabilidade perante Deus, por seu marido abandonado entregar-se a
outras mulheres. Mas a religião não é isto, Helena, e eu espero que você
a compreenda um dia. Religião é vida interior; portanto, não é
escravizar uma consciência, mas libertá-la. Benedetto Croce fala na sua
obra de estética que a religião exercita o homem a não meditar sobre o
destino humano, e cita a sua pátria como exemplo. Mas eu não entendo
assim. Pela religião, penso no meu destino aqui na terra, uma vez que
Deus na terra me colocou.

--- É verdade, Bárbara.

--- Religião é viver a vida na terra e não uma busca contínua de um céu
--- disse ainda Bárbara.

Helena olhou tristemente para a amiga. Bárbara parecia disposta a dizer
tudo o que pensava. Correspondeu ao olhar de Helena e, voltando-se para
ela, continuou:

--- Seu irmão, sua cunhada, seus parentes hão de falar de você com
menosprezo, colaborando para arruinar a sua vida. Na destruição, todos
colaboram, Helena. Virão então aconselhá-la a submeter-se; e talvez nem
o perceberão que estarão aconselhando a sua prostituição legal. Pelo
lado de seu marido, também haverá prostituição, pois, ele não contará
com o seu afeto. E que fará a sociedade? Dessas duas prostituições
cantará uma virtude.

Helena estava imóvel; não perdia uma sequer das palavras de Bárbara.

--- Continue --- disse ela --- diga tudo o que você pensa.

--- Quem me dera, Helena, poder falar todas estas coisas às mocas, que
como você, estarão agora ameaçadas de um casamento irrefletido. É
verdade que alguns dão certo, mas a maioria dos resultados não tem sido
satisfatória. E aqui não se trata de um jogo, mas de uma decisão
fundamental na vida de uma pessoa.

Bárbara sacudiu a cabeça, pessimista. Previa de antemão tudo o que
poderia acontecer a amiga; os anos que vivera por si mesma, os penosos
dias da sua orfandade, deram-lhe uma experiência segura, da qual ela
servia-se avidamente. A Helena não acontecera o mesmo; iria pagar com
altos juros a segurança social de um lar que na realidade não existia.
Bárbara sentia-se presa ao problema como se fora seu. Olhou com firmeza
para a amiga e insistiu nos mesmos pontos:

--- Todas as coisas contrárias, ele as esquecerá, Helena; uma atenção
delicada será tomada como prova de afeto. Os homens, no geral, são
assim; têm-se como objetos raros e como tais, acostumam-se a ser
tratados. Imagine só, tomar-se uma preciosidade, simplesmente por ser um
homem --- disse rindo, procurando descansar um pouco o espírito de
Helena.

Helena, porém, não riu. Estava esmagada sob o peso da própria fraqueza.
Compreendia a verdade das palavras de Bárbara, mas sentia-se
amedrontada. Se alguém, de força na família, a auxiliasse a sair daquela
situação, como agradeceria. Mas, os que impediram outros dos seus
namoros, estavam silenciosos agora. Helena não cria mais em religião
alguma, mas naquela hora, sentia que se poria de joelhos, para implorar
uma desgraça de menos ao Deus que tantas outras lhe dera na vida.

--- E agora, Helena, uma última pergunta --- tornou Bárbara,
observando-a --- se Paulo voltasse e a quisesse, embora a encontrasse
casada, você enfrentaria a sociedade e iria para ele?

Helena cerrou os olhos para depois abri-los com vagar:

--- Creio que sim, Bárbara; eu enfrentaria tudo por ele, se ele me
quisesse, é verdade.

--- Eu tinha certeza disto, Helena; então, por que se deixar arrastar
assim?

--- Acha então que cometo um erro irremediável? --- repetiu Helena
impressionada.

--- Uma loucura, Helena! Uma loucura! Seus pais deveriam ser os
primeiros a impedir tal casamento. E você já pensou como ficaria em
evidência com este casamento? Casar-se com o herdeiro da maior fortuna
do Brasil! A sua vida seria pública; já não teria mais descanso. Não
faltariam espias até na sua vida íntima. Com que prazer comentariam os
acontecimentos mais sem importância. E se em algum momento precisar da
caridade humana, aí então encontrará a maldade e a insídia. Os homens
não conhecem a caridade; lembre-se disso. Pensam que ser caridosos é dar
esmola, fazer doações e ter o nome no jornal. Raros são os que se
lembram, mesmo entre os cristãos, que a caridade se baseia no amor.
Disse outrora o comediógrafo brasileiro que a esmola cria o mendigo;
pois é isto que o homem quer: criar uma nova classe que receba os
benefícios de sua falsa superioridade. A caridade cristã faria dos
homens, irmãos; mas a caridade social os divide em duas classes: esmoler
e mendigo. Não quero torná-la pessimista quanto aos outros; o que digo a
você é o resultado de muitos anos de experiência. Eu vivi sozinha,
Helena, e aprendi a procurar nas coisas o verdadeiro sentido.

--- Não sei o que fazer! Meus Deus, que desespero! --- balbuciou Helena
com a voz entrecortada.

Bárbara contemplou a situação difícil de Helena, mas uma vez que
começara, resolveu ir até o fim. Era preciso prevenir a amiga; e
voltando-se para ela, disse mais:

--- Helena, se você conta com o auxílio da sua família no futuro, para
alguma dificuldade que surja, esteja certa de que este auxílio não virá.
A atitude de agora é uma profecia para mais tarde. O que a sua família
poderá fazer, será escravizá-la ainda mais, nunca, porém, apoiá-la. Há
tanta coisa a dizer num caso como o seu, que eu nem sei o que é mais
importante. Helena, arrede-se desse precipício enquanto é tempo.

--- Mas eu dei a minha palavra, Bárbara. Este casamento parece uma
responsabilidade de que eu não posso mais livrar-me. Além disso, meus
pais, minha família, não me ajudariam a voltar atrás; e eu às vezes
tenho medo deles.

--- Volte por você mesma, Helena --- suplicou Bárbara aterrada pela
perspectiva daquele casamento desigual. --- Se for infeliz mais tarde
--- continuou ela --- terá que agir pelas suas próprias forças, pois, os
outros só lhe aumentarão o peso. E uma vez que a perspectiva é única, o
caminho também é único! Você não se adaptaria a um homem por ele lhe dar
joias, dinheiro, vida confortável e uma posição privilegiada
socialmente; você sabe que não pertence a esta classe servil de
mulheres. Helena, você tem o espírito livre; mais hoje ou mais amanhã,
ele reclamará pela sua liberdade.

Pela primeira vez, Bárbara falava à amiga com todas as forças da sua
alma. Conhecia bastante a sociedade humana; as circunstâncias da vida
puseram-na em contato com toda espécie de gente. Frequentara e
frequentava ainda a alta roda, pelas condições de seu nascimento;
conhecera a camada média, aspirando a posições sociais; a sua beleza
atraíra a muitos homens de mentalidades diferentes; a sua carreira
colocara-la lado a lado com pessoas espiritualmente elevadas; e no final
das coisas, a sua orfandade, obrigando-a a tomar um partido próprio de
defesa entre os homens... e entre os fatos.

\chapter{Capítulo 70}

Manhã ainda; Bárbara, de calças a jardineira, cuidava das plantas. Os
heliotrópios que Álvaro semeara, apareciam no verde úmido das suas
primeiras folhas; não tardariam as flores. Com a minúscula enxadinha,
revolveu a terra cuidadosamente e dela tirou todo o mato invasor.
Terminado o trabalho, dirigiu-se ao terraço da frente e acomodou-se no
balanço de madeira. Olhando para o lado, viu ``COMPARAM-SE OS HOMENS", o
livro de Álvaro que estivera lendo na véspera. Tomou-o e pôs-se a
folheá-lo ao acaso. Esquecera-o no terraço; estava frio da umidade da
noite. Mas os pensamentos de Álvaro eram quentes e não tardou que ela se
deixasse contaminar por eles. ---''Se a vida fosse isto``, concluía
Álvaro numa série de cogitações,''a exemplo de Sócrates, eu sacrificaria
um galo a Esculápio, toda vez que ele, por meio da morte, nos livrasse
dos males da vida" --- Bárbara lembrou-se da cena do Pão de Açúcar ---
``pois o próprio Cristo disse: basta a cada dia o seu mal''. Passou a
mão pela fronte molhada e perguntou-se com desespero --- estarei assim
tão ligada a este homem, a ponto de não mais poder libertar-me? Os meus
pensamentos hão de viver sempre numa simbiose indestrutível com os
pensamentos dele?

Ouviu que alguém a chamava pelo nome e voltando-se, deparou com Lia e
Sérgio Augusto no portão. Levantou-se e foi até eles; convidou-os para
um refresco. Sentaram-se ali no terraço e a conversação caiu sobre
assuntos diversos. Vendo o livro de Álvaro, Sérgio Augusto perguntou:

--- Conhece este camarada, Bárbara? É o mesmo de: \emph{A flor,} \emph{o
perfume e a mulher.}

Ela sorriu:

--- Conheço.

--- Gostei muito deste livro --- continuou o rapaz. --- Gostei tanto, a
ponto de dá-lo também a Lia. É um sujeito de ideias, não resta dúvida; e
faz tal romance das ideias que a leitura se toma agradável.

--- É verdade --- tornou Lia --- Sérgio Augusto presenteou-me com um
exemplar. Li alguma coisa e fiquei pensando que para escrever assim,
deve ser alguém que já tenha sofrido um bocado na vida.

--- Deve ser --- tornou o namorado com olhares ternos --- todos os
homens que deixaram à humanidade um legado precioso, foram rigorosamente
experimentados.

Pela maneira como o rapaz falava e pelas atenções espontâneas que
demonstrava para com a menina, Bárbara percebeu logo que Sérgio Augusto
estava preso aos encantos de Lia. Desejou com todas as forças que eles
fossem felizes e tivessem consciência dessa felicidade. Depois, falando
em outras coisas, perguntou de Ivete.

--- Ivete --- respondeu Lia com simplicidade --- está esperando um bebê.
De saúde vai bem; todavia, anda um\textsuperscript{-}pouco nervosa, e
Roberto não é muito paciente. Enfim, mamãe diz que isso passa.

--- Tome o refresco, Lia --- disse Sérgio Augusto, está muito gostoso.

Ela começou a tomar aos pouquinhos e, de longe em longe, levantava os
olhos para o namorado. Quando terminou, Sérgio Augusto tirou o lenço do
bolso e enxugou as mãos úmidas de Lia; pôs ele mesmo o copo na bandeja e
depois passou-lhe o braço sobre o ombro, puxando-a para si. Numa atitude
simples e natural o rapaz parecia externar o seu carinho; e Lia
deixava-se envolver, criando assim entre eles uma deliciosa intimidade.
Sérgio Augusto, conquanto atencioso e afetivo, distanciava-se
visivelmente daqueles casais melosos, cujo procedimento parecia ser mais
para impressionar os outros. Bárbara notara também com que simplicidade
Lia lhe contara que Ivete esperava um bebê, embora na frente do
namorado. Era esquisito como as outras moças faziam rodeios e punham
malícia numa revelação dessa natureza; Lia, porém, parecia livre de tais
preocupações. Falou entusiasmada sobre os preparativos para a chegada do
herdeiro. Depois contou a Bárbara que eles pretendiam casar-se no fim do
ano. Sérgio Augusto receberia o diploma, e iria exercer a profissão em
São Paulo, numa firma especializada em Fundações. A firma, organização
rara no gênero, progredia espantosamente e São Paulo mostrava-se
propício às iniciativas particulares. Prosseguiam na conversa, quando o
telefone tocou, e Lia lembrou-se, então, de avisar a mãe que ela estava
em casa de Bárbara. Pediu licença e afastou-se.

--- A minha sogra me detesta --- disse Sérgio Augusto num sorriso meio
cômico.

--- Como? --- perguntou Bárbara espantada.

--- É sim --- tornou ele --- prefere ver o diabo na sua frente, ao pobre
Sérgio Augusto. Você pensa que ela sabe dos nossos encontros? Qual o
quê.

--- Mas que vida difícil! --- exclamou Bárbara.

--- Para você ver. Quer que Lia se case com outro Roberto Bastos. E às
vezes eu fico pensando que d\textsuperscript{a}. Alda não desconhece o
genro que tem; então, francamente, não compreendo tal insistência.

--- Não entendi, Sérgio.

--- Roberto, Bárbara, continua a frequentar como em solteiro; não deixou
os clubes, cassinos etc. Em uma noite encontrei-o num cabaré, na mesma
vida de antes. Estávamos numa roda grande e quando eu me retirei,
deixando-os à mesa para um novo brinde, ele fazia sinais compreensivos
para a cantora do jaz. E não pensava em esconder; no intervalo entre uma
e outra canção, foi até ao palco tirá-la para dançar.

--- Imagine... e com quatro meses de casado! Ivete saberá disso?

--- Vejam só --- gritou Lia do interior da casa --- liguei já quatro
vezes e quatro vezes ocupado. Você poderá emprestar-me um pente, vou me
arranjar enquanto isso.

--- Pois não --- volveu Bárbara --- entre no meu quarto, lá encontrará o
que quiser.

Sérgio Augusto continuou no assunto:

--- Não creio que Ivete saiba a quanto o Roberto chega; mas
d.\textsuperscript{a} Alda, tenho cá minhas dúvidas... Este casamento
vai acabar como a maioria dos grã-finos; toleram-se em casa, mas cada
qual arruma a vida do seu lado. É um tipo moderno de divórcio --- disse
o rapaz com ironia. --- E os filhos já vêm aprendendo desde o berço,
pois têm os pais por primeiros mestres.

Bárbara olhou intrigada para o rapaz; sabia que Sérgio Augusto poderia
ter um ou dois anos mais que Carlito; logo, uns vinte e dois ou vinte e
três. E como estava enfronhado daquela outra vida! Percebendo adivinhar
o seu pensamento, ele tornou a falar.

--- Já andei por esse meio; e não fosse o encontrar Lia, talvez já
estivesse emaranhado na teia de alguma aranha venenosa. Olhe, Bárbara,
muitas senhoras que se apresentam como modelos e com um julgamento
implacável para com os outros, ih!... --- e o rapaz fez um trejeito
expressivo. --- As esposas de uns amigos meus têm tido a honra de
censurar a mim e a Lia porque andamos de mãos dadas no cinema, ou na
rua. Se você soubesse o que já fez esta gente?! Mas como se esquecem
depressa, na hora de comentar o procedimento alheio.

Sérgio Augusto disse algumas coisas mais; e, embora interrompesse um
assunto com outro, Bárbara compreendeu. Lia chegou com uma expressão
tristonha e aproximando-se do namorado, disse como numa censura própria:

--- Menti a mamãe, Sérgio Augusto, e vou ficar com você o resto do dia.

--- Se eu já fosse formado, ficaria com você quer sua mãe quisesse ou
não, mas...

--- Como me aborrece mentir assim --- redarguiu Lia. --- Que bom se eu
pudesse dizer tudo a mamãe.

--- E se ela apresentar dificuldades sérias, como irão se arranjar? ---
indagou Bárbara, lembrando-se de Helena.

--- Eu não desistirei, Bárbara --- tornou a menina com frieza.

Sérgio Augusto envolveu-a num olhar apaixonado e levantando-se,
convidou-a para sair.

Despediram-se e afastaram-se. De longe, Bárbara contemplou-os; eram
jovens ...e juntos iriam começar a vida.

\chapter{Capítulo 71}

Onze horas da noite. Todos se retiraram; Bárbara andou pela casa vazia.
A vacuidade em que se sentira, de repente, deu-lhe a impressão de que
nada ficara das pessoas que estiveram ali. Parecia-lhe que somente ela
entrara naquela casa, e invadiu-a uma sensação de abandono... Uma
bandeja de café, largada sobre o aparador da sala, tinha três xícaras
usadas; ainda há pouco, ela, Helena e Lia haviam enchido o recinto com o
calor das próprias ideias. Naquela casa onde parecia haver o vácuo, três
mulheres diferentes conversavam sobre o amor. Todas as três amavam e
todas as três desejavam ser felizes.

Se Stendhal as visse, confirmaria a ideia de que ``o amor é como se
chama no firmamento a Via Láctea, um conglomerado brilhante, formado por
milhares de pequenas estrelas, onde cada uma delas é, muitas vezes uma
nebulosa''. E com essa nebulosa o destino de cada uma dessas três
mulheres parecia divertir-se.

Lia, na sua ingenuidade de menina que começa a viver, esperava por um
futuro certo ao lado do rapaz a quem amava. Na sua imaginação infantil,
formava-se um quadro colorido, onde Sérgio Augusto doente --- porém nada
grave --- recebia os seus desvelos de esposa terna e apaixonada.

Helena, falara de um Paulo que a protegia contra as intempéries;
confiara na sua força e, pela sua força, dera vazão à sua fraqueza. O
apoio que sempre lhe faltara na família ela encontraria ainda naquele
Paulo que conhecera aos quatorze anos de idade. Também, como mulher que
era, não pensara somente em receber; teria ele todo o seu cuidado e seu
carinho. Helena, acostumada a todas as riquezas, encararia sorrindo as
dificuldades financeiras do rapaz. E a intuição lhe dizia que embora
parecesse frágil, ela suportaria os trabalhos mais pesados se estivesse
ao lado de Paulo.

E Bárbara? Mulher forte que era, acostumada a enfrentar a vida por si
mesma, e conhecendo o amor aos vinte e quatro anos, não pensava em
Álvaro como um apoio à fraqueza feminina ou fazendo parte de um quadro
colorido onde as cores gritantes seriam excluídas com cuidado. Não que
excluísse Álvaro das suas fraquezas ou que não as tivesse... mas, no
momento em que amou, um novo ser veio unir-se ao seu, formando uma nova
personalidade, inseparável, quer na força quer na fraqueza. Tinha
consciência do seu amor e dele não se envergonhou; estava entregue
àquela fraqueza que os filósofos denominam o atributo das almas fortes.

E diante daquelas três mulheres, como agiria a sorte? Lia esperava que
lhe fosse favorável. Helena estava descrente. E Bárbara? Era a
interrogação sem resposta definitiva. Com o coração ferido, tratava
ainda dos problemas de Helena; e como Álvaro lhe dissera uma vez, que em
tudo na vida há o lado menos mau, Bárbara, preocupada com as
dificuldades da amiga, pôde diminuir um pouco a intensidade das suas.

Bárbara sabia que se Helena fosse feliz em sua casa, não pensaria mais
em se casar. E Bárbara estava pronta a receber a amiga e com ela
partilhar os seus haveres. Isto, porém, afetaria as opiniões
convencionais da sociedade, uma vez que Helena tinha pais; embora fossem
eles indiferentes ao sofrimento da filha. O casamento com Paulo iria
ferir os costumes da família; as regras da delicadeza e do bom-tom
impediriam uma união de nomes desiguais.

As pessoas de nascimento deveriam moldar-se a um padrão preestabelecido
onde as fases do pseudo-amor estavam já antecipadamente delineadas. Eram
incapazes de amar, por isso sujeitavam-se às regras daquele conjunto de
cerimônias que chamavam de etiquetas sociais. Tomaram-se produto das
circunstâncias, como potências negativas ou positivas num cursor de uma
régua de cálculo. A régua, porém, tinha finalidade objetiva e certa,
pensou Bárbara, pois um cálculo exato, o é pelas leis da matemática;
mas, as leis sociais, já intrinsecamente duvidosas e vulneráveis, eram
postas de lado por aquela gente, desde que fossem salvas as aparências.
Bárbara comparou tais pessoas a buracos de roupas interiores, abrigados
de olhares públicos. A distorção não importava existir, mas esconder.

Por quê? --- E ser poderoso não representava nada na sociedade?

Desconhecer a ânsia do homem pelo poder, seria desconhecer a própria
natureza humana. A hierarquia não se instituiu até mesmo nos sistemas
religiosos?

Barbara aproximou-se da janela aberta e contemplou a cidade iluminada.
Nunca a cidade lhe parecera tão artificial. Ao lado de riquezas
imensuráveis, uma pobreza também imensurável. Bárbara conhecia os dois
extremos; vivia entre a opulência e a miséria. Frequentava a alta roda
social que lhe abria as portas pela sua linhagem, e seu interesse pela
humanidade colocou-a ao lado de apóstolos da caridade. Era visitante
assídua das instituições beneficentes; nelas tomava contato não só com
os colaboradores, mas também com os infortunados que ali vinham minorar
as suas desgraças. Quantas vezes ao regressar da colônia de
Pirapitingui, Bárbara não pôde citar as dificuldades dos doentes porque
os outros não lhe queriam ouvir. Alegavam a sua sensibilidade para não
ouvirem narrativas desastrosas; mas aquela gente de sensibilidade
\emph{apurada,} nunca mandara a Bárbara, dez tostões que fosse, para
festejar o Natal das crianças leprosas. ``Crianças nascidas com os
corpos marcados'', como dissera uma destas, agradecendo a quatro pessoas
que ``no dia de Natal, deixavam seus lares para levarem conforto aos
meninos e meninas de Pirapitingui''. Ela observava em suas experiências,
que o auxílio entre os homens pelo verdadeiro espírito da caridade,
representava-se por grandezas infinitesimais nas estatísticas. Era mais
fácil vender um convite de baile por cinquenta mil-réis que angariar
cinco para uma ação de beneficência. E enquanto os infelizes cresciam
numa progressão geométrica. E enquanto os infelizes cresciam numa
progressão geométrica, as instituições mal os poderiam acompanhar em
progressão aritmética. E como visitadora assídua dessas instituições
aprendera muito. Tomara contato com belíssimas organizações católicas;
trabalhos incansáveis de evangélicos quase sem recursos, e finalmente,
os que dentre todos levavam a palma: o Exército da Salvação. Bárbara não
conhecera ainda nada mais abnegado que a luta desses soldados do
exército do ``bom combate''. Aos cuidados dessa corporação, fora
encontrar as mulheres decaídas e as crianças das sarjetas. Era incrível
que tudo o que vira fizesse parte de um mundo só.

* * *

Cansada, Bárbara adormeceu. Sonhos confusos embaralhavam imagens diante
dela. Moveu-se na cama como a desviar de alguma coisa, e numa posição
menos cômoda, continuou a dormir. Álvaro veio-lhe em sonho:

--- Bárbara...

--- Que é?

--- Voltei; vê, minha querida? Eu não posso viver sem você.

E achando que ela não estava bem, ajeitou-lhe o travesseiro, cobriu-a;
pois, a noite esfriara.

--- Álvaro --- chamou Bárbara.

--- Que é, meu amor?

--- Voltou porque eu não posso viver sem você.

Sentiu que ao seu rosto se uniu o rosto do rapaz e que ela adormecia
suavemente. E quando no sonho sentiu que adormecia, acordou. Olhou em
volta, a procurar\ldots{} impossível! Álvaro deveria estar ao seu lado.
Sentiu-o tão próximo, quase a respirar com ela. Ouvira o seu chamado, e
tinha um desejo grande de responder-lhe. O relógio bateu duas horas.
Bárbara virou-se na cama; sentia-se perturbada e perdia o sono aos
poucos. Acendeu o quebra-luz que tinha sobre o criado-mudo, com uma
lampadazinha fraca e azulada; ficou longo tempo a pensar. Bárbara
sofria; sim, sofria; sofria e não se entregava. Estava disposta a lutar
até que vencesse.

Levantou-se e foi até a saleta próxima, tomando cautela para não acordar
a governanta. Chegou-se ao vitral e no escuro da saleta, distinguia a
semiobscuridade da rua. Por uma fresta, pôs-se a olhar para fora sem
procurar coisa alguma. Viu um vulto que passou e pareceu olhar para sua
casa. Ia a passas lentos e tinha o chapéu enterrado na testa. Bárbara
prosseguiu nas suas meditações indefinidas... O vulto tornou a passar,
tomou a olhar; diminuiu o passo, e acabou encostando-se a uma árvore,
próxima ao portão de entrada do carro. De frente para a casa, riscou o
fósforo e acendeu um cigarro. Ela pôde ver o seu rosto avermelhado pela
chama --- era Álvaro!

Não obstante a hora adiantada da noite, veio-lhe a ideia de ir até ele.
Álvaro, porém, aceso o cigarro, afastou-se devagar. Bárbara permaneceu
ali, olhando-o numa angústia silenciosa, até que o viu desaparecer.

\chapter{Capítulo 72}

Passaram-se os dias e Bárbara continuou sem notícias de Álvaro. Setembro
começava e com ele a primavera. Eram ainda, cinco e meia da manhã quando
Bárbara abriu os olhos; a temperatura elevada não a deixou permanecer na
cama. Foi a primeira a levantar-se, mas, meia hora depois, já havia
movimento na casa. Com as janelas abertas, os cômodos recebiam os
primeiros raios solares tudo ali parecia ter vida. Bárbara tomou o café
da manhã e foi no jardim dar a sua volta costumeira. Parou diante dos
heliotrópios; não obstante a primavera, estavam murchando e morrendo aos
poucos. Bárbara lutava por conservá-los; redobrava em cuidados, e todo
trabalho parecia ser inútil. Dos muitos pezinhos nascidos, restavam
apenas três. E a moça, contemplando-os, lembrou-se de uma frase de
Álvaro ao indicar-lhe um banquinho --- Vê, Bárbara --- disse ele ---
três pés; a primeira condição de equilíbrio. Se tivesse somente dois,
você não poderia usá-lo. Recordou-se da resposta --- Nós temos somente
dois e o equilíbrio suficiente para nos mantermos em pé. Esperava que o
rapaz alegasse a maior superfície de contato entre os corpos; ele,
entretanto, sorrindo, disse com certa malícia --- para os seres humanos,
dois... formam o ponto de equilíbrio.

Todas essas lembranças lhe pareciam tão perto e tão longe. Algumas
vezes, ao recordar-se das ideias de Álvaro, Bárbara tinha a impressão de
ouvi-las de seus próprios lábios, tal a força com que lhe voltavam;
outras, Álvaro parecia um sonho a realizar-se. Era como uma princesa de
um conto de fadas, a esperar por alguém já predestinado a fim de
conquistar o seu amor. Feriam-na tais recordações; Bárbara, porém, não
conseguia furtar-se ao desejo esquisito de recordar. Rememorou uma a uma
as circunstâncias por que o rapaz entrara na sua vida; os fatos e as
coisas se ajustavam de tal maneira, que evitá-lo seria impossível. Foi
para a sala de música, onde tantas vezes estiveram juntos. Olhou para os
dois pianos fechados e num desejo estranho de torturar-se, abriu um
deles. Procurou no álbum a "Valsa do Amor". Seus dedos corriam pelo
teclado e, àquele contato, Bárbara interpretou, de modo mui pessoal, os
sentimentos do compositor alemão; aquela valsa passara a pertencer-lhe.
Às últimas páginas do álbum de manuscritos, Álvaro fizera arranjo para
os dois pianos, de uma outra valsa. Bárbara virou as folhas e achou ---
era a Valsa das Sombras. Sorriu e comentou --- É esquisito, mas a vida
tem cada ironia.

Uma campainha vibrou e, quase a seguir, Mrs.~Patrice veio chamá-la para
o telefone. Bárbara ergueu-se e caminhou. Do outro lado, falou uma voz
feminina:

--- Bárbara?

--- Que é, Helena? --- perguntou ao reconhecer a voz da amiga.

--- O meu casamento está marcado para o dia treze.

--- Com que intenção escolheu esse dia?

--- Com essa mesma que você adivinhou --- concluiu Helena.

--- Bem, desejo-lhe felicidades --- disse mais para não ficar
silenciosa.

--- Há tempo ainda; você poderá desejar-me no altar --- interveio a
amiga.

--- Não seria surpresa dizer-lhe que não irei ao seu casamento, não é
Helena?

--- Você faria isso, Bárbara?! Não creio.

--- Não posso agir de outra forma.

A campainha da porta soou com firmeza e logo depois um vulto masculino
desenhava-se no vitral. Bárbara continuou falando e o moço, lá fora,
pareceu esperar.

--- Não ir será menos difícil para mim --- afirmou convicta.

--- Ainda temos oito dias; você mudará de ideia.

--- Em todo o caso, está já avisada --- tornou Bárbara.

--- Até quando então?

--- Até quando você quiser.

E assim se despediram.

\chapter{Capítulo 73}

Bárbara deixou o telefone e foi atender a porta. Ao abri-la, com grande
surpresa, pronunciou o nome do visitante.

Paulo Machado cumprimentou-a, indagou da saúde da moça e entrou
silencioso para a sala. Estava abatido e tinha uma expressão que chegava
à crueldade.

--- Você não esconde a sua admiração por me ver aqui, não é? --- indagou
o rapaz.

--- É verdade --- tornou ela. --- Que tempo escolheu para vir ao Rio!

--- Pois é.

---\ldots{}

--- Você parece querer dizer alguma coisa... e, no entanto, fica a olhar
para mim, sem proferir palavra. Por que, Bárbara?

Bárbara perturbou-se visivelmente ante a observação do rapaz e começou
uma frase, não tendo coragem para terminá-la.

--- Estou admirada --- disse ela --- de vê-lo no Rio agora; ainda
mais...

--- Continue; ainda mais...

Bárbara não continuou; a palavra parecia-lhe difícil como nunca.

--- Ainda mais que Helena vai se casar agora. Não é isto o que quer
dizer?

--- Como soube?!

--- Qual \emph{é} o papel do jornal? E não é essa, uma das notícias do
dia? Maldito noticiário! --- disse o moço com um tremor na voz.

--- Paulo --- chamou Bárbara --- e você, o que acha de tudo isso?

--- Eu? Que hei de achar? Maldita sorte! Tudo para mim é maldito ---
acrescentou, expandindo-se na sua revolta.

Levantou-se da poltrona de couro, andou sem direção. Olhou algum tempo
pela janela aberta e acabou voltando-se para Bárbara:

--- É amarga esta vida!!! Há cousas que eu não compreendo... não
compreendo, Bárbara.

Pôs a mão no bolso precipitadamente; e logo após, retirou-a com o punho
cerrado.

--- Bárbara --- tornou ele, pronunciando marcadamente as palavras ---
você já notou como a vida dificulta tudo para os que não lhe procuram
tirar proveitos desonestos?

Ela não sabia o que responder. Na hora de sofrimento, caíam por terra
todas as filosofias; e pensar em torno daquele sofrimento já não seria
próprio da Bárbara de hoje. A razão intelectual dos fatos ia aos poucos
substituindo-se pelo sentimento. E Bárbara era honesta perante si mesma,
para dizer ao rapaz o que não pensasse ou não sentisse; era-lhe
impossível, a exemplo da maioria, apoiar-se a frases de falso consolo,
simplesmente porque estas mesmas frases poderiam parecer sensatas
perante a razão. Era certo, e Bárbara sabia por experiência própria, que
``o amor encontra razões que a própria razão desconhece''. Além disso,
no estado de espírito em que o rapaz se achava, contrariar os seus
pensamentos seria ampliar a sua revolta.

--- Que pensa você de tudo isso? --- indagou Paulo, esquecido de que ela
lhe fizera esta pergunta.

Antes que respondesse, ele soltou um riso de escárnio:

--- Você também deveria rir de mim, Bárbara.

--- Você já não me conhece, Paulo?! Acha que seria possível rir de você?

--- Já não conheço mais ninguém; afinal... Quem sou eu no rol das
cousas? --- comentou com desapego. --- Antes não tivesse voltado; não
reavivaria em mim os sintomas de um mal incurável. Eu, ainda hoje, acho
que só poderia me casar com Helena! Casar-me-ia com outra por razões
biológicas e uma simpatia de momento; mas por afeto... isto já é coisa
morta para mim.

Paulo continuava a considerar; nessas considerações expunha-se sem
cuidado. Bárbara poderia compreendê-lo; assim... baseado na compreensão
da amiga de Helena, não se preocupava sequer com a ordem de suas
exposições. Frases desencontradas levavam-no ora a um assunto ora a
outro; tal qual um psicanalista, Bárbara deixava-o falar. Por fim, ele
sentou-se desanimado e, passando a mão pelo cabelo, repetiu a mesma
pergunta de há pouco:

--- Afinal... tenho que conformar-me; pois quem sou eu no rol das
coisas?

Curiosas circunstâncias! Álvaro também lhe dissera isto uma vez! A
convivência levava os dois amigos a pensamentos e expressões
semelhantes. Assim, a presença de Paulo trazia à Bárbara uma lembrança
mais viva de Álvaro. Bárbara tinha a certeza de que Paulo falaria nele,
e nessa certeza, esperou. Paulo tirou um cigarro do bolso e acendeu-o
num gesto brusco.

--- Sabe, Bárbara, eu não compreendo esta vida. Já disse isso repetidas
vezes e quase sempre nas mesmas circunstâncias. É o \emph{sol de
inverno} que brilha para mim.

O rapaz fumou vagarosamente todo o cigarro e depois perguntou:

--- Você, algum dia, compreendeu a vida, Bárbara?

--- Ninguém a compreendeu ainda, Paulo. Porque teria eu este privilégio?

--- É verdade... afinal, que é a vida.

--- Faria esta pergunta a todos; ninguém lhe responderia
convincentemente.

--- No entanto, todos vivem. Vive-se a vida e ninguém sabe o que ela é.
Num outro dia, um cristão disse que nós temos vida para que se externe a
glória do criador. Bela concepção de Deus! Imaginar-se alguém criando
bilhões de criaturas sofredoras, só para externar-se na sua Glória. É
até uma ironia --- tornou ele numa risada sarcástica.

--- Contudo, você não pode responsabilizar os cristãos pela crença de um
deles. Aqui no Brasil há cristãos de toda a espécie; é curioso notar
como pensam diversamente, ou na verdade, como não pensam. A maioria
deles não sabe no que crê; têm uma religião por herança como um
patrimônio de família passando pelas diversas gerações.

--- E deixando de parte a religião, que eu não tenho, qual o filósofo
que você acha apresentar razões plausíveis para a vida?

--- Não sei Paulo; para mim, quando a fé religiosa não resolve estes
problemas, a filosofia também falha.

--- Estou pensando numa coisa engraçada; Bárbara, você já observou como
a vida está ao nosso alcance e como não temos direito sobre ela?

--- Como?

--- Damos a vida de nós mesmos, e podemos liquidá-la num segundo; o que
não podemos é dar a vida perfeita ou saber onde ela continua depois
daqui. Por nosso intermédio vem ao mundo um monstro ou um santo... que
coisa horrível! Eu não mais teria coragem para pôr alguém no mundo.
Daria, aos que aqui viessem por meu intermédio, o mesmo presente de
grego que eu recebi. Qual, Bárbara, eu estou começando a crer na
predestinação. Uns nascem para serem aleijados no corpo; outros para o
serem na alma. Não sei qual dos dois será o pior.

Bárbara ouviu-o com atenção; as considerações do rapaz levantavam
dúvidas no seu espírito e ela, então, aventurou uma pergunta:

--- Paulo --- disse pausadamente --- serão justas essas queixas? Que fez
você até agora?

--- Procurei não ser desonesto para com a vida. E tal atitude entre os
exploradores que crescem dia a dia, não é coisa fácil de se manter.

--- Todavia, não se pode permanecer na atitude negativa: não sendo... ou
não fazendo... É preciso que de outro lado, seja... e faça. Você toma a
atitude passiva: sabe o que quer, porém, não luta porque outros não
querem aquilo que você quer.

--- Não a. entendi, Bárbara...

--- Você quer Helena, não é verdade?

--- É querer o impossível, mas quero.

--- Impossível, por quê? E esperará você pelo possível? Que vem a ser
esse possível? Uma Helena casada, gasta pelo sofrimento, quase um
farrapo, que você irá contemplar em poder de outrem.

--- Não sei a que quer chegar --- disse ele intrigado.

--- Helena está à beira, de um precipício; você está à espera de que ela
tombe, numa atitude de angústia, mas nada faz para salvá-la. Você sofre
mais pelo seu próprio sofrimento que pelo de Helena. Paulo, custa-me
dizer a verdade; mas, eu o acho tão covarde quanto Helena. Ambos sabem o
que querem, porém esperam que isto lhes caia do céu. É como se
precisassem ir à cidade e fossem esperar o ônibus onde ele não passasse.
Paulo, pense nisso, você vai mandar Helena para o inferno.

--- Eu?!...

--- Você sim. Haveria outro Paulo de quem ela tivesse gostado?

O rapaz baixou a cabeça entre as mãos. E Bárbara disse ainda:

--- Queixa-se da vida... pois queixe-se antes de você mesmo.

--- Por quê? --- indagou sem levantar a cabeça.

--- Eu lhe respondo com outra pergunta --- tornou a moça --- que fez
você por esse amor?

--- Não o abandonei Bárbara, você sabe. Afinal, os pais têm mais
direitos.

--- Direitos sobre o coração, ninguém os tem, Paulo; nem mesmo o próprio
dono desse coração. É paradoxo, mas há muito paradoxo verdadeiro.

Paulo acendeu novo cigarro; escoava por aquele gesto nervoso uma força
incontida que se agitava dentro dele. O telefone tocou; Bárbara estendeu
a mão e apanhou-o quase a seguir. Ouviu a voz do outro lado e
conhecendo-a, foi com dificuldade que pronunciou o nome da amiga.

--- Helena...

Paulo ergueu-se precipitadamente da poltrona e chegou-se a Bárbara.
Continha a respiração, temendo prejudicar alguma coisa que ele mesmo não
sabia ao certo. Bárbara colocou-lhe o fone ao ouvido; com o coração aos
saltos, ele ouviu do outro lado:

--- ... sinto um medo inexplicável.

--- Helena...

--- Quem fala? --- tornou a voz aflita de Helena. --- Não era Bárbara?!

--- Sou eu, Helena; não reconhece a minha voz?

Houve um intervalo pesado; nem um nem outro dizia uma palavra... parecia
que ambos tinham perdido a voz. Paulo sentia os nervos como cordões de
aço a puxar-lhe todos os músculos; no seu íntimo parecia existir um
dínamo gerando corrente que se encontravam e se repeliam. Num gesto
brusco e inexplicável, bateu o fone com força.

\chapter{Capítulo 74}

Entre eles caiu um silêncio de angústia. Paulo voltou a sentar-se na
mesma poltrona; e, como nestas ocasiões o cigarro é um companheiro
apreciável, não o dispensou. Ficou ali a fumar... À sua frente, Bárbara
também silenciosa. Muita coisa acontecera em curto espaço de tempo. Não
estavam preparados para recebê-la.

Bárbara recordou a sua vida tranquila de antes, nesta atitude humana
que, em horas difíceis, nos transporta a cenas felizes do passado.
Sabia, porém, que as belezas do passado seriam apenas um refúgio
transitório para o presente. Recostou-se na poltrona e deixou cair para
trás a cabeça, esgotada pelas emoções. De olhos cerrados, procurou
descansar um pouco. Mas o descanso brincava com os homens; nunca lhes
vinha no momento desejado. Era algo precioso demais para se buscar em
ocasiões determinadas.

Bárbara abriu novamente os olhos; os cílios longos e negros pesavam
horrivelmente. Notou então que Paulo contemplava alguma coisa, e,
acompanhando o seu olhar, Bárbara encontrou Beethoven. Observou-o.
Aqueles frontais amplos, salientes e marcados; aquele olhar firme e
seguro, aquela boca rasgada e de linhas entrecortadas, os cabelos
revoltos, \emph{a} altivez da atitude quase arrogante, pareciam
proclamar a fraqueza de Paulo e de Bárbara, --- Era o homem força à
frente de todas as raças.

Por um paradoxo inexplicável, talvez inspirado no contraste da sua
própria música, viera mentalmente a Bárbara o segundo movimento da
sétima sinfonia. Aquele ritmo de marcha fúnebre caiu-lhe ao ouvido como
se fora executado por forjas e martelos. Depois, pelo mesmo contraste,
quem sabe... cantaram no seu íntimo trechos da ``sinfonia jocosa'', esse
portal magnífico de um novo mundo sinfônico --- o mundo
\emph{beethoveniano}. Bárbara pôs-se a recordar as sinfonias; achara
sempre que estas constituíam a melhor biografia do mestre, até hoje,
vinda à luz. Pensou até então na primeira, a jocosa, onde ainda se
encontravam traços do mundo já construído de Haydn, Mozart; mas, fazendo
fundo à peça sinfônica, na penumbra dos sons, através de lampejos
profundos e vigorosos, surgia a força de Beethoven com o seu novo mundo.
E Bárbara pensou na segunda sinfonia --- a escadaria marmórea --- dando
acesso ao novo mundo. Sentiam-se ainda as luzes, embora longínquas, de
Haydn, Mozart. A terceira, o arauto --- a apresentação deslumbrante,
toda ela Beethoven, desse novo mundo das sinfonias. A seguir, as demais
até a oitava inclusive a construção decisiva, firme, de uma plenitude
divina, desse novo mundo de que a terceira é o projeto absoluto. E então
a nona --- coroa suprema --- a síntese genial, não só artística, mas
também filosófica, daquele que ao fim de sua vida maravilhosa em dor e
alegria, nos ensina --- o homem foi criado para a alegria de viver ---
Beethoven.

***

--- Bárbara --- chamou Paulo com voz cansada. --- Você vai desculpar o
meu egoísmo; já é tempo de falar sobre o assunto que me traz aqui.

--- Que é Paulo --- perguntou Bárbara, acordando do seu sonho sinfônico.
Há alguma coisa difícil para dizer?

--- Há ocasiões em que tudo é difícil.

Paulo bateu a cinza do cigarro; Bárbara olhou para ele numa atitude de
expectativa.

--- Você sabe, Bárbara, Álvaro está apaixonado por você.

Ela fechou os olhos num gesto de assentimento, e perturbou-se no íntimo
pelo agradável da notícia.

--- Ele está quase louco, Bárbara. A vida o experimenta com um cinismo
revoltante; lhe põe tudo na mão para torná-lo criminoso.

--- Que lhe pôs na mão a vida? --- indagou a moça sem esconder a sua
aflição.

--- A própria vida --- concluiu Paulo.

--- Suicídio?! Não, não é possível --- tornou repelindo a ideia.

--- Homicídio... --- tornou ele com a voz entrecortada.

Bárbara baixou a cabeça ante o peso daquela revelação, compreendera todo
o alcance das palavras do amigo. Agora, só lhe restava saber as
circunstâncias.

--- E depois, Paulo? --- indagou.

--- E depois? Nesse depois há muita coisa.

--- Gostaria que fosse a elas sem preâmbulos --- observou Bárbara.

Animado, Paulo fez-lhe a pergunta embaraçosa:

--- Bárbara... você sabe que Álvaro é casado?

--- Sei.

--- E sabe que ela vive ainda? --- indagou sem pronunciar nome algum.

--- Também.

--- Este é o problema do momento. Ela está doente e internada num
hospital aqui no Rio. Conseguiu encontrar Álvaro e chamou por ele.

--- Mas como?! --- indagou Bárbara custando crer no que ouvira. ---
Chamou-o para uma reconciliação?

--- Não sei; quem pode saber o que quererá um diabo daquele. Talvez a
perspectiva da carreira de Álvaro, talvez a falta de dinheiro, e mesmo a
solidão. Informei-me do seu estado e sei que não apresenta gravidade;
apenas requer cuidados. Apelo à sua compreensão, Bárbara; qualquer
iniciativa de sua parte seria certa.

--- Não confie assim, Paulo; sou tão humana como os demais. O fato de eu
desejar a perfeição não exclui as minhas falhas; e, dessa forma, os meus
erros passam a ser maiores que os dos outros.

--- O fato, Bárbara, é que a situação é penosa. Álvaro teve impulsos
malignos... entende? Era isto o que eu tanto queria dizer.

--- Entendo... você quer dizer que Álvaro pensou em livrar-se, por si
mesmo, de uma situação em que não encontra apoio nas leis. Mas,
tranquilize-se Paulo; Álvaro jamais seria um criminoso. Conheço-o muito,
para fazer convicta, tal afirmativa.

--- Nem eu duvido de Álvaro; de seus sentimentos... mas, Bárbara, quando
um homem faz uso do álcool, perde a noção das coisas.

--- Álvaro está fazendo uso do álcool? --- indagou com acentuada
tristeza.

--- Algumas vezes, no auge do desespero. Para esquecer... você
compreende.

Ela não respondeu. Pensamentos desencontrados atravessavam-lhe o cérebro
como setas cortantes. Passou a mão pela testa molhada, e seus lábios
contraíram-se numa agitação nervosa; dentro em pouco, chorou com
amargura. Paulo contemplou-a, calado; sabia que procurar consolá-la,
seria inútil. Era a primeira vez que via Bárbara chorar; não acreditaria
que ela um dia pudesse chorar diante de alguém.

Um vulto masculino desenhou-se à porta da sala; e, antes que Paulo
dissesse alguma coisa, viu Bárbara levantar-se e caminhar para ele.
Álvaro ali estava, olhando-a apaixonadamente. Quando Bárbara se
aproximou, tomou-a nos braços e estreitou-a com força. Encostou o seu
rosto no rosto umedecido de Bárbara e, como um louco, passou-lhe os
lábios nas faces molhadas, bebendo aquelas lágrimas que ele fizera
correr. Esqueceram-se de Paulo que, paralisado pela emoção, permaneceu
imóvel na sua poltrona. Tornara-se a testemunha viva de uma cena, à
margem da qual também tinha o seu papel.

\chapter{Capítulo 75}

Chegou o dia treze. Era o dia do casamento de Helena. Bárbara começara o
dia perturbada; e a medida que as horas avançavam, a sua ansiedade
crescia. Estava a ponto de não caber mais em si mesma. Aproximara-se a
hora do casamento e, sentia perder aos poucos, toda a esperança dos
últimos dias. Ouviu bater a campainha e foi abrir a porta. Não chegou a
olhar para fora e já o telefone chamou. Atormentada, Bárbara voltou à
saleta sem fechar a porta que abrira; talvez tivesse confundido o som
das campainhas.

--- Alô --- disse ela.

Do outro lado, Helena:

--- Bárbara, por favor, venha até aqui; quero vê-la ainda uma vez.

--- Pode ser covardia, Helena; mas, não posso.

Helena parou um instante, falou alguma coisa, ao lado, que Bárbara não
compreendeu; depois, chamou a amiga novamente:

--- Bárbara, você me espere, ouviu? Quando sair para o casamento,
passarei antes por aí. Irei abraçá-la ainda solteira, pois o civil e o
religioso serão seguidos.

--- Mas venha, só, Helena, por favor --- suplicou Bárbara.

--- Irei sim, Bárbara, nem que para isso tenha que chorar terrivelmente.

Bárbara depôs o fone e ouvindo ruído, voltou-se:

--- Paulo, você?! Será possível? Toda a vez que falo com ela, você me
aparece aqui.

--- É a sina da gente --- respondeu cumprimentando-a.

--- Sente-se; esteja a gosto.

Paulo parecia não sofrer como da outra vez, mas aparentava um nervoso
indiscreto.

--- Então, Paulo, que há?

--- A primeira coisa é que você me abriu a porta e me deixou lá fora;
entrei por minha própria conta.

--- Fiz isso? --- disse a rir. --- Que indelicadeza!

--- Bárbara --- tornou o rapaz com a voz alterada --- posso saber o que
ela dizia?

--- Vai passar por aqui antes mesmo do civil. Você vai esperá-la? ---
indagou admirada.

--- Não.

--- Ainda bem; pois seria embaraçoso.

--- Bárbara --- volveu o rapaz desviando o assunto --- na última vez que
a procurei, não cheguei a dizer tudo; a presença de Álvaro impediu-me de
continuar. Eu vou viajar e estou seriamente preocupado com ele. Álvaro
tem bebido nestes dias e desde que a viu, não faz outra coisa senão
mergulhar-se na inconsciência. Hoje, está bem e prometeu-me vir
procurá-la à tarde. Convenci-o de que você vai separar-se de Helena e
precisa dele. Quem sabe, então, poderiam ir juntos ao hospital ...
Criar-se-ia assim um ambiente onde Álvaro, tendo estado com você, não
introduziria uma ideia criminosa.

Paulo falava apressado. Tinha formado planos de expor calmamente a sua
missão; todavia, gestos sem controle e palavras precipitadas traíam seus
desígnios. Após curta introdução, voltou-se para Bárbara e articulou um:

--- É...

--- É o quê? --- perguntou Bárbara.

--- É isso mesmo --- e pareceu encerrar o assunto. --- Bem, Bárbara, até
por lá.

Ela estendeu-lhe a mão e comentou:

--- Tenho vontade de conhecer o norte; talvez seja mesmo: até por lá. E
felicidades, Paulo.

--- Obrigado.

O rapaz saiu à pressa, deixando Bárbara para trás. De repente, regressou
e olhando para ela, perguntou:

--- Você não viu se eu vim de chapéu?

--- Penso que não; eu nunca o vi de chapéu --- respondeu examinando em
redor.

--- Oh, mas é verdade! --- exclamou batendo na testa; eu nunca uso
chapéu.

E na mesma pressa que veio, tornou a sair. Bárbara não o acompanhou;
sentou-se numa poltrona próxima e ficou a pensar no estranho
procedimento de Paulo. Bem, pensou, naturalmente vai à igreja para vê-la
de longe; se Helena o encontra, que balbúrdia... E ela não pareceu
disposta a persistir na ideia; mesmo porque, Álvaro interrompeu o curso
dos seus pensamentos. Nisto, ouviu uma voz que a chamava:

--- D.\textsuperscript{a} Bárbara.

--- Que é, Maria? --- perguntou reconhecendo a voz da empregada.

--- Trouxe um cafezinho para a senhora. Vi que não almoçou nada ---
ajuntou a mulher com cuidado.

--- Obrigada --- respondeu, tomando a xícara que a empregada lhe trazia.

Tomou vagarosamente o café e, quando ficou só na sala, sentiu-se tomada
de uma tristeza imensa. Havia uma pressão interior que ao mesmo tempo
causava um vácuo no seu espírito; sentimentos desencontrados agitavam e
perturbavam a sua alma. Mais uma vez, sentia-se desnorteada em face dos
problemas da vida. O casamento de Helena doía-lhe como um crime, e ela
não o pudera impedir. Álvaro regressara, mas, às portas do vício e
bafejado com ideias criminosas. Há momentos em que se acumulam as
desgraças, como se um magnetismo as atraísse entre si. Pelos seus
pensamentos, perdeu a noção das horas. Quando Helena assomou à porta em
seus trajes nupciais, Bárbara teve um choque tremendo. Contemplou a
amiga, no vestido de cetim branco, cuja riqueza e gosto o tornavam
elegantíssimo. Helena, porém, tinha uma expressão triste que lhe dava um
aspecto misterioso; lembrava mais a uma santa que a uma noiva, na pureza
das suas vestimentas brancas.

--- Helena --- exclamou a amiga --- é tudo isto possível?

Abraçaram-se demoradamente sem que mais se ouvisse uma palavra. Somente
ao despedirem-se, Bárbara falou comovida:

--- Que Deus a acompanhe, Helena.

Não quis ir com ela até a porta. Deixou-a sair só e, logo que a porta
bateu, jogou-se à mesma poltrona, como quem não sabe o que fazer.
Veio-lhe ao pensamento a frase da agonia: ``Minha alma está triste até a
morte''; mas, a agonia messiânica precedeu à ressurreição. E naquelas
circunstâncias, haveria ainda uma ressurreição? Bárbara deixou pender a
cabeça para trás e cerrando os olhos, procurou descansar o espírito.

O tempo ``passou adiante''... correram os minutos e as horas e ela não
saiu do lugar; sabendo que precisava agir, mas ignorando por onde
começar. O telefone despertou-a; Bárbara tomou-o:

--- Alô...

--- Alô --- disse uma voz aflita do outro lado.

Bárbara reconheceu ser a de d.\textsuperscript{a} Hermínia, mãe de
Helena.

--- A senhora deseja alguma coisa? --- perguntou com delicadeza.

--- Não sabe de Helena, Bárbara?

--- Não senhora; esteve aqui, mas há bastante tempo.

--- Pois até agora não chegou; estamos todos à espera --- comentou
d.\textsuperscript{a} Ermínia numa tortura indescritível.

--- Não estivemos juntas cinco minutos, d.\textsuperscript{a} Ermínia
--- inteirou Bárbara.

--- É estranho! --- tornou a outra agressiva.

Bárbara desligou o telefone; e a fisionomia de Paulo voltou-lhe à mente.
A sua atitude mudara; algo estava planejado. Pensou então em
d.\textsuperscript{a} Ermínia diante dos convidados, apresentando
desculpas pelo ocorrido; pretextando talvez algum desastre. Que palavras
encontraria ela para explicar!

E o Luiz Porto da Rocha? O milionário, cuja ideia de uma recusa não lhe
atravessara o espírito! Este mesmo milionário, abandonado na hora do
casamento. E depois de tanto tempo, Bárbara riu gostosamente:

--- Que peça bem pregada!

Levantou-se e andou pela casa mais aliviada. Seria este o começo da
ressurreição?

\chapter{Capítulo 76}

Helena atravessara a sala de música. Separando-se de Bárbara,
dirigira-se à porta, sem olhar para trás. Sabia que Bárbara ficara lá;
não a acompanhara para não partilhar da sua sorte.

Segurando a pesada cauda, Helena atravessou, com dificuldade, o
pequenino jardim de frente. Seus lábios tremiam; teve ímpetos de gritar
por Bárbara ainda uma vez. Passou pelo portão de ferro, fez menção de
entrar no carro que a esperava; a porta, porém, permaneceu fechada.
Junto à porta, um vulto masculino estava de frente para ela. Helena
arregalou os olhos e duvidou da sua visão:

--- Paulo!...

Ele a contemplou vestida de noiva. Era como se viesse para ele; pois,
não a queria para esposa? Paulo estava transbordante de amor e
achegou-se de Helena num impulso apaixonado:

--- Você acha, Helena, que eu poderia ser indiferente?

Numa agitação incontida, ela pronunciou o nome do rapaz outras vezes:

--- Paulo... Paulo...

--- Helena --- perguntou achegando-se mais da moça --- você ainda me
quer?

--- De todo o coração --- respondeu admirando-se da sua própria firmeza.

Na simplicidade dessa resposta, pareceu formar-se um mundo novo. E
aquela paixão, recalcada por quase sete anos, surgia num ímpeto de força
indomável. Contemplando Helena, Paulo teria perdido a noção do tempo, se
Álvaro, à direção do carro, não o interpelasse.

--- Paulo --- chamou apressado.

Não foi preciso dizer mais; ele compreendia tudo o que poderia seguir-se
àquele chamado. E olhando para Helena, abriu com firmeza a porta do
carro. Ela deu um passo à frente e os dois entraram quase ao mesmo
tempo. Álvaro ligou o motor e o automóvel desceu pelas ruas de Santa
Tereza. No velocímetro subiram os números; todavia, os três passageiros
não davam conta dos perigos a que se expunham. Álvaro contornou os
caminhos centrais e atravessando a cidade, quase que de ponta a ponta,
foi estacionar o carro à porta de uma casinha humilde numa rua
solitária. Helena desceu, embrulhada na comprida capa de Paulo,
abrigando ao olhar de algum curioso anônimo o seu traje nupcial. Quanto
ao véu e a grinalda, Paulo os jogara pelo caminho. Num trecho deserto,
onde o mato crescia à boa altura, foi atirado um embrulho de papel
amarelo, amarrotado pela pressa. Tudo parecia resolver-se.

Helena entrou num quartinho simples, cuja aparência indicava ser de
rapaz; um guarda-roupa de madeira comum, duas camas emparelhadas tendo
ao centro um criado-mudo, e perto da janela uma escrivaninha de várias
gavetas. Não havia cortinas e os móveis estavam limpos de objetos, a não
ser um cinzeiro sobre o criado-mudo e papéis à escrivaninha. Paulo,
embaraçado, tirou do guarda-roupa algumas caixas e, abrindo-as, mostrou
a Helena alguns vestidos de mulher.

--- Comprei-os numa loja de arrabalde; são de três tamanhos e quanto à
qualidade, você sabe... eu sou neutro nisso.

Helena sorriu e examinou os três vestidos; um deles, de seda vegetal cem
por cento e duro como um papel, foi o que lhe pareceu melhor. Tinha o
fundo branco e umas rosinhas estampadas. A padronagem seria mais para
uma camisola, mas Helena achou-o ótimo.

Quando Paulo a deixou, para falar com Álvaro na porta, ela iniciou a
substituição dos trajes. Mas que desastre! Paulo não se lembrara da
combinação. Helena olhou através da seda do vestido e notou a sua
transparência, era impossível usá-lo assim. E a que trazia no corpo lhe
ia até os pés. De repente, uma ideia: abriu as gavetas em busca de uma
tesoura; mas parecia não haver tesouras naquela casa. Na gaveta do
criado-mudo estava o aparelho de barba de um deles; Helena tomou a
gilete e, aos talhes da lâmina afiada, cortou fora o excesso da sua
própria combinação. Num vestidinho de seda ordinária e com um sapato
prateado de alto preço, Helena deixou o quarto, onde vinte minutos
antes, entrara com vestimentas luxuosíssimas. Iam já para o carro,
quando Helena falou alguma coisa baixinho para Paulo. O rapaz sorriu
numa alegria transbordante e acompanhou a moça até a loja de armarinho
da esquina --- Helena ia comprar batom, ``rouge'' e pó de arroz para a
viagem.

\chapter{Capítulo 77}

Bárbara estava no terraço, quando Álvaro abriu o portãozinho de ferro.
Aproximou-se dela, um tanto esquivo e com um desaponto visível. O fato
de ter bebido naqueles dias inferiorizava-o de tal forma perante si
mesmo, que ele se acovardava nas atitudes. Não bebia pelo vício do
álcool, estava consciente disso; procurava entorpecer-se para perder a
consciência dos fatos. Bárbara acolheu-o carinhosamente; estendeu-lhe a
mão que ele apertou de maneira significativa.

Aproveitando o interesse de Bárbara por Helena e Paulo, Álvaro começou a
narrar o incidente com todas as minúcias possíveis; numa evasão às suas
próprias dificuldades.

--- E assim, Bárbara, uma frase sua mudou o destino de tudo. Quando você
disse a Paulo que ele ia mandar Helena para o inferno, um outro inferno
surgiu dentro dele. A viver entre dois infernos, concluiu ser o melhor
viverem juntos na terra.

--- Formidável! --- exclamou a moça.

--- Formidável em toda a acepção da palavra --- ajuntou o rapaz.

--- É mesmo --- concluiu Bárbara --- formidável de causar medo! E como
conseguiram avisá-la?

--- Não a avisamos. Paulo sabia que ela viria à sua casa; entretanto não
confiamos nas circunstâncias. Seguimos o carro e numa curva mais
camarada aos nossos desígnios, passei à frente; fiz tal manobra que o
coitado do motorista não teve outro recurso senão ir para cima da
calçada. Furou um pneumático e julgamos então ser a hora; mas passou um
taxi e Helena tomou-o quase a seguir. Meio descontrolados pela
oportunidade perdida, viemos até aqui. E quando ela saiu da sua casa, já
não encontrou o carro que a trouxera. Procedemos como dois assaltantes;
estivemos sempre a espreitar. Se você soubesse, Bárbara, como o tempo
não passa numa hora dessas!

--- E Helena, quando viu Paulo? --- indagou Bárbara.

--- Oh, ela perturbou-se muito!

--- Conheceu-o logo?

--- No mesmo instante. E como o seguiu sem vacilar! Já não parecia a
Helena tímida que eu conheci. Ele parou um instante e depois disse a
Bárbara --- mas houve também coisas engraçadas. Imagine você, nós dois
numa loja de arrabalde, a calcular um vestido para Helena.

--- Calcular um vestido? --- perguntou rindo deliciosamente.

--- Isto mesmo! Escolher um vestido que sirva, é um verdadeiro cálculo.
Álvaro desandou a rir sem parar.

--- Por que ri tanto assim?

--- Comecei a calcular o porte de Helena pelo da moça que nos serviu. À
certa altura, ela desconfiou de nós e tornou-se mais esquiva. Paulo
estava tão alheio que não o percebera e ainda lhe perguntou: este
vestido serviria para a senhora? A moça fechou a carranca e
interpelou-o: mas, afinal, que é que o senhor pretende?

--- E daí? --- interrogou Bárbara.

--- Eu fui logo em auxílio. Disse que pretendíamos uma surpresa e nunca
tínhamos comprado vestido de mulher.

--- E compraram, algum dia, vestido de homem?

--- Também não --- volveu com humor --- mas a ideia da surpresa deu
resultado. A moça trouxe um vestido com um elástico na cintura... e que
cinturinha, Bárbara! --- exclamou Álvaro. --- Depois, quando a gente
esticava, tomava todos os tamanhos. Você entendeu, não é? A coisa ficava
do tamanho que se queria. Não vimos nada mais acertado. Assim mesmo, a
mulher disse que era melhor levar mais dois; mulher sempre entende mais,
e Paulo gosta muito do número três, acabou levando os três vestidos. Mas
como é prático o tal cordão na cintura --- rematou entusiasmado.

Bárbara ria a mais não poder. Fora tudo tão engraçado e Álvaro ainda
vinha contar-lhe com os termos mais atrapalhados possíveis. Fazia gestos
para tornar inteligível a narrativa e, de vez em quando, parava a
procura de uma palavra que explicasse convincentemente as coisas.

--- Por que não se lembraram de mim? --- indagou Bárbara com interesse.

--- Não quisemos envolvê-la. Você fez muito, pedindo a Helena que viesse
só à sua casa.

--- Como sabe disso?!

--- Paulo não estava aqui quando você falou com ela?

Bárbara lembrou-se então que ele poderia ter ouvido; no íntimo comentou:
há coincidências inexplicáveis... não resta dúvida.

--- Bárbara --- chamou o rapaz --- você acha que mesmo assim, a família
de Helena poderá responsabilizá-la por isso? Fizemos tudo para que tal
não acontecesse.

--- Acho que sim; mas isso não tem importância. Se eles forem felizes,
essa pseudo responsabilidade deixará de existir.

--- Penso que serão, Bárbara; não obstante o risco desse casamento
depois de quase sete anos de separação. Paulo nunca se esqueceu da moca
e parecia desiludido para um novo casamento. Quanto a Helena...

--- Helena foi feita para ele; passou toda a adolescência moldando-se
aos gostos, e até caprichos de Paulo; Helena é capaz de grandes
realizações, mas tem necessidade de que alguém a impulsione. Além disso,
Paulo chegou na hora exata; se soubesse, ainda um pouco antes, Helena
poderia ter medo, mas foi tudo tão ajustado que parece um plano
superior.

--- E assim, Bárbara, dentro em pouco serão eles marido e mulher.

Com aquelas palavras, Álvaro calou-se repentinamente. Esgotara todo o
assunto e chegara então a hora difícil. Abaixou a cabeça e pareceu
profundamente exausto; cessara todo o seu poder narrativo de há pouco.
Se estivesse só, afogaria a cabeça em vapores conhecidos; mas Bárbara
era um ímã no lado oposto; e ele a limalha de ferro em campo magnético.
Bárbara interpelou-o com todo afeto.

--- Álvaro --- chamou comovida.

--- Que é, Bárbara? --- perguntou sem levantar a cabeça.

--- Precisamos conversar...

--- É verdade, mas eu não sei o que dizer. Ontem pensei muito e não
encontrei uma solução. Francamente, estou desnorteado --- respondeu em
profundo abatimento.

--- Vamos procurar o nascente e nortearmo-nos por ele --- redarguiu no
mesmo tom afetuoso.

--- Ou desaparecer no poente? --- contestou o rapaz. --- A vida é feita
de extremos, Bárbara; um céu agora, um inferno daqui a pouco.

--- Tenha ânimo Álvaro; muita coisa entra por um inferno em nossa vida.

--- E depois?

--- E depois? --- repetiu ela desconcertada.

--- Sim; o que fazemos nós com esse inferno ou com essas coisas?

--- Álvaro, você está descrente porque sofreu muito e foi apanhado
desprevenido.

--- Agora, Bárbara, estou mais prevenido e nunca sofri com tanta
intensidade.

--- O inferno na terra não é eterno, você mesmo disse; vamos então
esperar este céu de daqui a pouco.

Ele sorriu com amargura:

--- Eu desejo lutar, Bárbara; mas percebo que só possuo força nos
braços.

--- E nós precisamos tratar de assunto delicado --- começou Bárbara. ---
Paulo falou-me a respeito, sabe?...

--- Se é delicado, não sei --- respondeu adivinhando o pensamento dela
--- mas, para mim, é já decidido.

Bárbara estremeceu ante aquela resposta; custou-lhe dominar os nervos, e
perguntar:

--- Como decidido, Álvaro?

--- Quando se odeia, Bárbara, o melhor partido a tomar é fugir ao objeto
do nosso ódio --- volveu com aspereza.

--- O ódio é como o fel, Álvaro, traz amargura à nossa vida --- ponderou
ela em resposta.

--- Quando uma solução está supersaturada pelo açúcar --- acrescentou o
rapaz acentuando as suas palavras --- e nós a provamos, chegamos a dizer
que está amargo de tão doce. Pois, Bárbara, o ódio é isto mesmo, porém,
no sentido inverso. Sente-se a sua amargura, mas no fundo há um
adocicado que acaricia.

--- Não é possível! --- exclamou a moça numa atitude espontânea --- você
não é dessas pessoas.

--- Você tem razão para me julgar covarde; à minha covardia, entretanto,
não chega faltar forças para odiar.

--- Escute Álvaro, quando estamos dominados pela paixão, geralmente não
conseguimos pensar. É o que se passa com você, agora. Não quis chamá-lo
de covarde e nem o julgo tal; apenas quis referir-me a um sentimento
mais nobre que existe em você. Não me tome pela parte negativa; assim
você não me encontraria.

Ele silenciou; no seu íntimo os sentimentos confundiam-se. Á moça
fitou-o com ternura e continuou a falar:

--- Entre as pessoas más... costuma-se considerar ainda as que se
mostram capazes de odiar; seja como for, é uma atitude definida. O ódio
torna-se uma responsabilidade visível, pois, dá ao adversário
oportunidade de defesa. Não disse Ingenieros que os maledicentes são
incapazes até para odiar? Por quê? Simplesmente, pela irresponsabilidade
das suas atitudes; o ódio seria muito forte para eles. Mas, você não
poderia partir do sentido inverso? Não considerando o menos mau do que é
mau, mas o bom do que é melhor; uma vez que a nossa imperfeição não nos
deixa alcançar o melhor do melhor.

--- Tudo isto é verdade... mas que quer você que eu faça?

--- O que você puder fazer --- volveu com firmeza.

Álvaro pareceu abalado pelas considerações de Bárbara; ela julgou ser
então o momento propício para usar de maior franqueza. Embora agitada no
íntimo, sua voz pareceu calma ao perguntar:

--- Álvaro, você sabe o que ela quer?

O nome de Dalila não fora pronunciado na tragédia; todos, porém,
entendiam quando o assunto se referia a ela. Álvaro mostrou-se
constrangido, e tentou encerrar o assunto com uma negativa categórica;
mas a moça parecia decidida a ir até o fim.

--- Talvez fosse melhor saber...

--- Para quê? --- interrompeu ele com voz cortante.

Meio descontrolada, Bárbara respondeu:

--- Muita coisa nos vêm de onde não imaginamos.

--- Você quer ver-se livre de mim? --- inquiriu bruscamente.

--- Álvaro... você não está sendo injusto nesta sua pergunta?

--- Então, porque insinua que eu vá ver esta mulher? Bem sabe que é
impossível --- comentou irritado.

--- Por que impossível?

--- Você quer razões para tudo. É porque a odeio --- declarou com raiva.
--- Essa mulher me arruinou a vida; eu não a suportaria diante de mim.
Se fosse vê-la, seria para vingar-me.

--- E como se vingaria?

--- Não sei. Não sou homem para planejar uma vingança; seguiria o
impulso do momento. Você se engana comigo, Bárbara; sou arrojado e
rancoroso.

--- Não; você é apenas impetuoso --- interveio a moça.

--- Seja como for, não a suporto e não desejo que você me fale nela.

--- Eu desejo apenas ajudá-lo. Esta é a única razão da minha
insistência.

--- Aqui você não pode ajudar-me em coisa alguma. Foi uma desgraça
irremediável, e eu não tenho culpa.

--- E quem tem a culpa? --- indagou Bárbara.

--- Quem?! --- repetiu espantado. --- Quem mais há de ser, senão ela?

--- A sua atitude é muito humana --- ponderou em resposta à acusação do
rapaz. --- Você já notou, Álvaro, que o conhecimento do imperfeito nos
vem sempre através do próximo? E tudo, por quê? Porque o homem se nega a
reconhecer a sua própria imperfeição. Isto é o que você está fazendo,
Álvaro; toda a culpa para os outros e toda a razão para si. Foi
envolvido por ela, mas considerou também que se deixou envolver?... Ela
foi forte para arrastá-lo e você, fraco para se deixar arrastar.

--- Bárbara --- disse ele irritado --- o momento não é para filosofar.
Sou mesmo humano e por ser humano não posso fazer considerações numa
hora dessas. Você não é mulher; é como eu pensava, apenas a vida de uma
ideia.

Álvaro levantou-se da cadeira, acendeu o primeiro cigarro e foi para o
outro extremo do terraço. Bárbara saiu da cadeira onde estava e
sentou-se no balanço ao lado; naquele embalo suave ficou tempo
silenciosa. No seu íntimo, todavia, todos os sentimentos gritavam
descontroladamente.

A noite caminhava. Álvaro continuava fumando num desespero abafado. Não
tivera coragem para retirar-se; conquanto se irritasse com Bárbara,
sabia ser impossível deixá-la. Falara-lhe numa aspereza desconhecida e a
ideia de tê-la magoado punha-o em choque consigo mesmo. Desejou odiá-la,
por um instante que fosse; o necessário apenas para afastar-se. Buscou
razões para o ajudarem e chegou a acreditá-la desumana. Sabia,
entretanto, que isto não era verdade. E apegar-se a uma ideia falsa,
tratando-se de criatura tão verdadeira, seria confundir-se, imiscuir-se
na sua própria indignidade. Além disso, como caluniar a mulher que
amava? Não; gritou dentro de si mesmo, tudo isto é uma loucura! Acendeu
novo cigarro e aproximou-se vagarosamente da moça. Olhou para ela,
vencido, prostrado. Apertou o cigarro entre os dedos, e o seu olhar
alternava-se entre a brasa acesa que lhe acariciava os nervos e a
fisionomia triste e expressiva da mulher que lhe acariciava a alma. Não
pensou em sair; pois tinha consciência de que não o faria. E se saísse,
mal chegaria ao portão \emph{e} já não resistiria ao desejo de voltar.
Respirou longamente e com voz cansada, dirigiu-se a ela:

--- Bárbara, você me desculpa? Não sei como pude ser tão rude assim ---
falou sacudindo a cabeça.

Ela sorriu, e ele achegou-se mais.

--- Que acha que eu devo fazer? Em mim só fala o sentimento; a razão
está calada. Além de tudo --- disse abaixando a voz --- eu tenho medo,
Bárbara.

--- Medo de quê, Álvaro?

--- De tentar a justiça por minhas próprias mãos, uma vez que não
encontrei apoio nas leis.

--- Não seria a justiça, Álvaro, seria o crime... então seria entregue
às leis. Não quero dizer com isso que o crime quando não diante das
autoridades jurídicas, deixe de ser um crime; há crimes horrorosos que
as leis deixam passar. Não é isto. Se você pensasse em matar, estaria
violando as leis do direito humano. Poderia confundir as autoridades,
mas, não poderia fugir a si mesmo. É diante de si próprio que cada um
tem que julgar os seus crimes. E você sabe que não tem direitos sobre a
vida dessa mulher; é uma personalidade independente da sua. Não se
tornou mercadoria ou escrava adquirida, embora você a tivesse pelo
dinheiro. Esse seu egoísmo é que o torna assim infeliz. Viva a sua vida
e deixe-a viver a seu modo.

--- Que me aconselharia? --- indagou pensativo.

--- Você poderia achar-me desumana --- ponderou a moça --- mas a razão
indica um só caminho.

Ele a fitou num gesto espontâneo, como a esperar que falasse.

--- Você conhece muito pouco das misérias humanas, por isso dá grande
valor às suas. O dia em que andar por aí, não é preciso ir longe, aqui
nos arrabaldes do Rio, vai ver então como a miséria existe e em ampla
escala. Quando sair desses arrabaldes, não poderá conceber esses
edifícios públicos cujo conforto tanto delicia a uma determinada classe;
ao passo que de outro lado, homens, mulheres e crianças, feitos com a
mesma carne e o mesmo sangue, vivem privados de tudo, até da água que
lhes mate a sede. Amanhã, vou fazer uma visita a duas crianças doentes
aqui numa barroca de Santa Tereza. Você quer ir comigo?

--- Naturalmente; mas como soube isso?

--- Um médico de São Paulo, velho conhecido meu, telefonou-me hoje
pedindo que fosse vê-las.

--- E está clinicando aqui, agora?

--- Não; regressou a São Paulo pelo cruzeiro desta noite. Ele está muito
ligado a essa gente; é médico e diretor de uma instituição de caridade
na capital paulista. Sob os seus cuidados há crianças, velhos e
tuberculosos. O Dr.~L. é um apóstolo da medicina, quando não da
humanidade. Vê, Álvaro? Não se pode viver egoisticamente.

--- Sei disso, Bárbara, mas que posso fazer. Quando penso nessa mulher,
cresce em mim um sentimento de revolta a ponto de não mais contê-lo.

--- Considere a má ação dessa mulher, apanhando-o desprevenido aos
dezoito anos de idade; mas considere também a sua fraqueza. Você poderia
responder que a educação recebida de seu pai o predispôs a isso,
retardando a sua personalidade. Teria razão na sua resposta, pois os
pais também são responsáveis pelos filhos que põem no mundo. Mas desta
forma, estender-se-ia a responsabilidade não só à família como à raça e
você, produto das circunstâncias, tornar-se-ia por sua vez um fator das
mesmas. É paradoxal, não é? Mas como responsabilizar, separadamente, uma
raça da qual você faz parte também? E voltando a essa mulher... Álvaro,
você precisa lembrar-se de que ela é humana, portanto, sujeita a erros
como toda, a gente, inclusive você.

--- Nesse caso, que me resta fazer? --- indagou na sua grande
prostração.

--- Subtrair-se à dependência dos erros dela. Tornar-se independente ---
repetiu com mais ênfase.

Ele pareceu não compreender o significado dessas últimas palavras e
Bárbara, percebendo, insistiu.

--- Se pensar em vingança ou não deixar de odiá-la, estará na
dependência dessa mulher, tanto ou mais, como se lhe estivesse ao lado.
Além de tudo, essa mulher está fraca e cansada; não seria jamais a
oportunidade para uma vingança.

Álvaro ouviu quase submisso; Bárbara tinha o dom de falar-lhe
diretamente à alma. Pensativo, tirou o cigarro do bolso. Dispunha-se a
acendê-lo, quando sentiu que a mão da moça estava sobre a sua.

--- Não fume mais; deixe este cigarro.

Ele vacilou àquele contato e abaixando a cabeça apertou aquela mão
contra os seus lábios quentes.

\chapter{Capítulo 78}

Na manhã seguinte, Álvaro e Bárbara foram à praia de Copacabana.
Passaram horas agradáveis, nadando, expondo-se ao sol e comendo bananas
ou chupando laranjas. Agora, de volta, estavam no bonde de Santa Tereza,
atravessando os arcos. Álvaro usava calça branca e uma camisa de gola
esporte; estava queimado e tinha um aspecto sadio. Bárbara também se
trajava de branco, seu vestido era de linho irlandês, todo abotoado na
frente, e sem mangas.

No banco da frente, quieto e cabisbaixo, um homem de cor fumava um
cigarro de palha. A sua atitude cansada e pensativa chamou a atenção de
Álvaro e Bárbara. No segundo ponto após os arcos, entrou para o bonde um
moço da mesma raça que mostrou logo ser conhecido do primeiro.

--- Alô, homem... --- cumprimentou o moço num tom matreiro --- e come
vá? disse num tom de caçoada, como se imitasse alguém.

--- Mal; muito mal --- respondeu o segundo que aparentava ser pouco mais
velho que o primeiro.

--- Então, o seu negócio não deu certo? --- indagou assumindo atitude
complacente.

--- Não, a coisa vai mal, meu caro.

--- Ora essa, e não acendeu você duas velas pro santo?

--- Mas o santo falhou, rapaz.

--- Viu? Bem que eu disse que o seu santo não era milagroso. O meu, é
batata, homem! Nem queira saber. Pois duma feita, eu me vi numa
embrulhada daquelas --- começou ele aproximando-se mais do amigo. ---
Encrenquei com a Benedita e logo... pá, uma criança. Quando ela me disse
que o bichinho vinha mesmo, acendi uma vela pro santo e esperei passar
os nove meses. Você sabe, fazer qualquer coisa pro inocente é pecado, e
eu não tinha outro recurso. Nem estava casado ainda e já com boca pra
alimentar. Pois no fim dos nove meses, veio o milagre.

O companheiro olhou espantado e o outro concluiu respeitoso:

--- O menino nasceu morto.

Bárbara e Álvaro ouviram toda a conversa e no íntimo concordaram com o
epílogo. Melhor fora morrer para aquela criança a ter nascido em
semelhantes condições. Se estivessem na Grécia antiga, seria a
oportunidade para sacrificar o galo a Esculápio.

***

Almoçaram juntos em casa de Bárbara; e, terminada a refeição, combinaram
fazer a visita prometida. Bárbara foi buscar sua bolsa de passeio a
tiracolo e encheu-a com pó de café, açúcar e biscoitos. Fez ainda alguns
embrulhos, entre os quais um litro de leite, que Álvaro prontificou-se a
levar.

E saíram a pé pelo bairro a procura da rua indicada e sob o sol
escaldante das três horas. Subiram uma escada mais além e foram dar num
trecho circular que os desorientou. Pediram informações aos transeuntes
e, com dificuldade, puderam descobrir a abertura da rua que os levaria à
casa procurada. Quando o rapaz viu o caminho batido, com degraus
escavados no chão duro, teve impressão de que seria impossível avançar
mais. Bárbara, porém, ajeitando a bolsa ao ombro, entrou no caminho sem
uma observação. Álvaro segurou os embrulhos todos; e passando à frente,
estendeu-lhe a mão, amparando-a naquele trajeto perigoso. Mais embaixo,
retirada para a esquerda, viram uma casinha de tabique e zinco. A moça
observou:

--- Deve ser aquela.

Só restavam alguns degraus e o rapaz notando a dificuldade em se
carregar embrulhos, perguntou:

--- Bárbara, dar o dinheiro para essas compras, não seria mais fácil?
Como é penoso para você, descer tais escadas; e ainda por cima, vir
carregada de volumes.

--- Você é que está levando --- interveio a moça.

--- E se eu não estivesse com você?

--- Nesse caso, eu os traria.

Nisto ouviram uns ruídos fortes. Voltaram-se para trás. Um menino, que
descia correndo a tal escadaria, pedia caminho, imitando buzina de
automóvel. Recuaram espantados perante a agilidade do garoto. O que fora
para os dois uma peregrinação tornara-se um esporte para ele. O talzinho
passou como um vento, mostrando uma perícia inimitável para descer as
serras.

--- Como até as coisas difíceis podem tornar-se um hábito --- comentou o
rapaz.

--- É verdade --- concordou Bárbara.

--- Afinal, você não respondeu a minha pergunta: por que não dar
dinheiro em vez de trazer esses embrulhos? Pois não é muito mais fácil?

--- Para mim, seria; mas o resultado é duvidoso. O dinheiro nem sempre é
empregado nas necessidades diretas do doente. Uma das minhas
experiências neste sentido, é não trabalhar com dinheiro, mas, com
gêneros alimentícios e roupas.

Chegaram diante da casinha e bateram à porta. Ninguém lhe respondeu. Lá
de dentro pareceu-lhes vir um chorinho fraco. Tornaram a bater e ficaram
sem resposta como da outra vez. Uma voz de mulher perguntou de longe:

--- Querem alguma coisa daí?

Ela se aproximava; tinha uma lata de água na cabeça e veio falando desde
longe.

--- Queríamos saber --- tornou Bárbara --- se aqui moram duas crianças
que o Dr.~L. visitou ontem.

--- Ah, é isso mesmo! --- respondeu a mulher. --- A senhora é a moça que
ele falou?

--- Sou sim; vim fazer uma visita para as crianças.

A mulher empurrou a porta de tábua e entrou na frente; logo depois,
Bárbara e Álvaro. A um canto, numa esteira de fitas de bambu, viram as
duas crianças seminuas. Uma delas mordia um pedaço de pão duro; a outra,
completamente largada, mal dava sinal de vida. Um cheiro acre parecia
desprender-se de todas as peças.

--- Que idade têm essas crianças? --- indagou Bárbara.

--- Essa pequenina fez um ano. O outro vai fazer dois.

A mulher deixou a lata em cima de um caixão e Bárbara notou que era
jovem. Desejando saber mais alguma coisa, voltou-se para ela.

--- Como a senhora é moça!

--- Tenho vinte e seis anos, e já perdi dois filhos.

--- Dois filhos?! Como foi isso?

--- Não sei. As crianças nascem, mas, não vingam. A senhora não vê agora
esses dois?

--- Quem sabe, esses dois vão sarar... Vamos cuidar deles.

--- Eu já estou esperando outro, tornou a mulher com desânimo. Não sei
no que vai dar; essa vida de pobre é um inferno!

--- Seu marido não ganha bem?

A mulher atrapalhou-se um pouco e perceberam que ela não era casada.
Bárbara desfez os embrulhos; só então Álvaro compreendeu a razão de
terem trazido um pequenino fogareiro e uma panelinha. Pôs o leite para
esquentar e, vendo lá fora um fogão improvisado, pediu à mulher que
aquecesse água. Aproximou-se das crianças e examinou-as detidamente.
Tomou o pão duro da maiorzinha e deu-lhe um biscoitinho em troca.

Mais para trás, não acreditando nos seus próprios olhos, Álvaro a
contemplava, abismado. Não era possível que Bárbara fizesse todas essas
coisas! Como se ligava às misérias humanas e compreendia o sofrimento de
seus semelhantes. Viu-a procurar roupas das crianças e fazer com que a
mãe as banhasse e deixasse em condições mais higiênicas. Passou um
algodão molhado na boca ressequida da criança de um ano; depois,
alimentou-as. Fez recomendações especiais quanto ao tratamento e,
finalmente, despediu-se, prometendo voltar no dia seguinte. Deixaram a
casa, e o rapaz permaneceu silencioso por largo espaço do caminho.
Auxiliou Bárbara a subir o morro-escada e quando falou alguma coisa foi
para demonstrar a sua descrença.

--- Bárbara --- disse com voz abafada --- não seria preferível deixar
morrer essas crianças?

--- Eu também me sentiria satisfeita se as visse morrer; mas, morrerão?
E enquanto tiverem vida, precisam ser cuidadas para viver.

Durante o trajeto de regresso, Álvaro sentiu-se perturbado; olhava para
Bárbara com maior respeito e, embora a quisesse muito, sentia-se
esquisitamente distanciado. Um sentimento novo, misto de vergonha,
respeito e amor, parecia debater-se no seu íntimo. Quis afastar-se, mas
percebeu logo que o seu desejo era de aproximação. Duas naturezas agiam
dentro dele, e pareciam ser igualmente fortes. Álvaro não quis
analisar-se, estava confuso para isso. Teria feito todo o trajeto em
silêncio se Bárbara não lhe dirigisse a palavra.

--- Hoje de manhã --- observou a moça --- tanta vida, sol, saúde e
alegria; era como estava a praia. Agora, a miséria, a doença e a
tristeza! Vê, Álvaro, como os extremos podem ser vistos num só dia.

Ele articulou algumas palavras em resposta e continuou pensativo. Por
uma associação de ideias lembrou-se da outra mulher. Estava doente e
chamara-o com insistência. Não só não a atendera como tivera ímpetos de
abreviar-lhe a vida. Quando viu Bárbara, de joelhos, diante daqueles
dois fragmentos de criaturas, teve horror das suas intenções passadas.
Sentiu que não mais odiava aquela mulher e pelo seu temperamento
sentimental, desejou auxiliá-la. Pensou em ir ao sanatório, mas, ao
ver-se mentalmente diante dela, estremeceu. Ele não a podia conceber
fraca e doente; a sua última lembrança era de uma Dalila desdenhosa,
zombeteira e insensível. E se o chamasse para depois zombar da sua
fraqueza? Sentiu o sangue afluir-lhe ao cérebro e suas mãos se fecharam
como se estrangulassem alguém; mas quando olhou para a moça, a seu lado,
sentiu-se tranquilo. A expressão de Bárbara, retribuindo o seu olhar,
era uma rocha segura, onde poderia abrigar-se dos males que o assaltavam
continuamente. E numa entonação suplicante, dirigiu-se a ela:

--- Bárbara, eu iria ver essa mulher, se você fosse comigo...

--- Como, Álvaro?! Eu não a conheço --- retorquiu admirada.

--- É uma ousadia pedir-lhe isso; mas, eu preciso de você.

--- Estou pronta a ajudá-lo; porém, no que for possível. Não conheço
essa mulher --- repetiu ela.

--- Também eu não a conheço; uma vez que não seja a que eu odiava ontem.

--- \ldots{}

--- Bárbara, você me ajudaria muito se me acompanhasse. Tenho duvidado
de mim ultimamente e nessa dúvida me torturo tanto! --- exclamou
pesaroso.

--- Como eu desejaria que você se conhecesse, Álvaro. Você se supõe
mau\ldots{}

--- Bárbara --- interrompeu o moço --- é estranho como você se aproxima
dos homens; nunca apontando falhas, mas compreendendo as suas fraquezas.
Bárbara --- chamou o rapaz num impulso apaixonado --- é impossível,
impossível deixar de ir comigo...

\chapter{Capítulo 79}

Era noite. Bárbara pôs o casaquinho sobre os ombros e saiu com Álvaro.
Que novas emoções teria ainda naquele dia? Que se iria passar mais na
sua vida? Entregue a si e aos seus pensamentos deixou passar a
distância. Sentiu-se perturbada quando ele fez parar o ônibus. Desceram
em silêncio e subiram a escadaria do hospital. Álvaro pediu informações
na portaria; e, guiados pela enfermeira, chegaram ao quarto designado. A
enfermeira deixou-os em frente à porta. Bárbara fez a sua última
tentativa:

--- Entre só Álvaro; esperá-lo-ei aqui.

--- Por favor --- suplicou nervoso o rapaz --- não me deixe.

Ver aquela mulher tornara-se quase uma necessidade para ele; temia o
primeiro encontro, mas, ao lado de Bárbara sentia certa confiança. Sem
chegar ao ponto de confessar para si mesmo, no íntimo, desejava vê-la,
porém, em companhia de Bárbara, para desfazer a impressão dos seus
impulsos malignos. Ao lado de Bárbara, estaria abrigado de um sentimento
criminoso, nascido num momento de revolta. Nesta segurança fora em busca
da experiência nova --- encontrar a mulher que desfizera a sua vida.

Bárbara bateu delicadamente à porta. Uma voz feminina gritou do
interior:

--- Entre.

Álvaro deu volta ao trinco e, abrindo a porta, passou primeiro. Esperou
até que Bárbara entrasse também para fechá-la novamente. Bárbara parecia
calma; seu sistema nervoso, porém, trabalhava com intensidade.
Aproximaram-se calados da cama da doente. Viram ali uma mulher deitada,
recostada em muitos travesseiros; tinha o aspecto cansado e aparentava
mais de trinta anos. Os cabelos oxigenados caíam-lhe em desalinho e a
sua pele parecia gasta pelo excesso dos cremes e pinturas. As olheiras
eram marcadas por um roxo escuro e saliente, pelo acúmulo de tecidos que
se contraíram ali; os lábios descorados confundiam-se no seu rosto fino
e comprido; uma pinta escura, próxima ao lábio superior, quebrava a
monotonia do semblante abatido, e seus olhos pardos tinham a expressão
do cansaço. Olhou para eles com pretenciosa indiferença e, voltando-se
para Álvaro, falou-lhe com certa ironia:

--- Apresente-me à sua senhora.

--- Não é minha senhora --- redarguiu ele, disfarçando o seu embaraço.

--- Que é então?

--- Uma amiga dedicada.

--- Só?

Dalila olhou para eles detidamente. Num sorriso quase imperceptível,
observou:

--- Não me parece que seja somente esta a ligação.

Álvaro sentiu o sangue afluir-lhe ao cérebro; num esforço desesperado
procurou conter-se. Encarou-a com firmeza e, num gesto brusco,
dirigiu-se àquela mulher como se estivesse dando uma ordem.

--- Seja como for, é um assunto sagrado que eu não desejo abordar.

--- Como queira --- disse num sacudir de ombros --- eu pouco entendo de
coisas sagradas.

Voltando-se para Bárbara, apontou-lhe uma cadeira:

--- Sente-se ali. Não é delicado deixar uma visita em pé.

--- Obrigada --- retorquiu Bárbara com acentuada polidez. --- Não me vou
demorar.

--- Em todo o caso, sente-se um pouco.

--- Não se incomode, vou sair agora.

--- Que pressa! Por que veio então?

--- Para saber se a senhora precisa de alguma coisa? Observei,
entretanto, que nada tenho a oferecer-lhe --- notou Bárbara sem elevar a
voz.

--- Não me enganei, então, concluindo que à senhora devo esta visita...

Bárbara silenciou.

--- É certo o que eu disse, Álvaro? --- indagou Dalila, olhando para
ele.

--- Julga que eu viria visitá-la ainda? --- retorquiu Álvaro com
seriedade.

--- Eu pensei que nos tivéssemos separado sem rancor um do outro ---
acentuou Dalila.

Álvaro não respondeu e dirigindo-se à Bárbara, perguntou-lhe:

--- Quer retirar-se?

Ela curvou levemente a cabeça, aquiescendo àquele convite. Álvaro
aproximou-se e, tomando-a pelo braço, caminharam juntos em direção à
porta. Mal tocou na maçaneta, ouviu a voz de Dalila:

--- Álvaro...

Ele quis prosseguir; titubeou indeciso por alguns segundos e acabou
olhando para trás:

--- Mais alguma coisa? --- inquiriu.

--- Não se despedem de mim?

--- Não julguei necessário --- volveu em tom decidido.

--- Álvaro --- começou Dalila embaraçada --- não tive intenções de
ofendê-los.

--- Contudo, vamos retirar-nos --- insistiu.

--- Por favor, fiquem mais um pouco --- pediu Dalila, mudando o
tratamento.

Álvaro olhou para Bárbara como a esperar que ela decidisse.

--- A senhora poderia desculpar-me? --- interrogou Dalila, já noutro tom
de voz.

Bárbara percebia as lutas íntimas daquela mulher. Analisando-a, deduziu
como lhe custara aquela atitude de submissão. E diante da mulher
vencida, Bárbara não teve um gesto sequer de vencedora. Assim,
dirigiu-lhe a palavra, ainda em pé, à porta.

--- Poderia ser-lhe útil em alguma coisa!

--- Venha até mais perto --- pediu Dalila.

Bárbara atendeu-a; achegou-se à cama e esperou que ela falasse. Dalila,
porém, desviou o olhar; numa agitação nervosa, moveu-se nos
travesseiros.

Álvaro permaneceu silencioso; tinha a mão ainda no trinco da porta e,
semi-voltado para trás, não perdia um pormenor sequer. Naquele silêncio
prolongado, sobressaía a atitude da outra mulher. Inquieta, Dalila
passou a mão pela testa empalidecida. Seus lábios tremiam
descontrolados; de repente, tapou os olhos com as mãos, voltou-se para o
lado e, enterrando a cabeça no travesseiro, chorou nervosamente.

\chapter{Capítulo 80}

Eram quase dez horas quando deixaram o hospital. A noite agradável
convidava-os a andar. Movidos pelo mesmo desejo, saíram a pé pelas ruas
da cidade.

--- Tudo é tão estranho, não acha Bárbara? --- indagou Álvaro cortando o
silêncio que reinava entre eles.

--- Eu estava pensando nisso --- rematou Bárbara.

--- Sabe que me arrependi de tê-la trazido. Não obstante, reconheço que
seria pior sem você.

Bárbara olhou para ele e não respondeu.

--- A vida tem cada surpresa! --- disse novamente o rapaz. --- Bem,
Bárbara, não sei como agradecer.

--- Foi tudo tão bem, Álvaro; não há o que agradecer --- comentou
discretamente. --- E, no íntimo, acho que fizemos uma boa ação.

--- Imagine... quando eu havia de pensar que ainda atenderia a uma
mulher a quem odiei com tanta força.

--- Eu já disse uma vez e repito: como você se conhece pouco --- afirmou
a moça com ternura. --- Foi bom que a visse, pois, agora dissipar-se-á
toda a impressão antiga.

--- Você sabe, Bárbara, que eu senti em mim um ódio descomunal? ---
asseverou convicto.

--- Foi um impulso de momento, notou \emph{a} moça.

--- Está sendo indulgente para comigo --- replicou ele.

--- Não. Pensa que eu não vejo também os seus defeitos? Mas é que ao
lado deles, há os bons predicados. Através dos seus impulsos maus, nunca
se esconde o Álvaro que eu conheço.

--- Que criatura estranha é você, Bárbara! --- exclamou o rapaz,
esquecido de que já dissera isso muitas vezes.

--- Eu?! Por quê?

--- Diz coisas que eu nunca ouvi de outra mulher.

--- Você conhece muitas?

--- Algumas... --- declarou com certo embaraço. --- Eu tenho vivido pelo
mundo e à semelhança da maioria. Às vezes, sentindo-me só, procurei
companhia feminina; não só nos salões como em lugares menos seletos.
Esta liberdade social de que o homem faz uso, permite-lhe conhecer mais
as mulheres que estas aos homens.

--- E que acha você das mulheres?

--- Que pergunta difícil! Porém, a tese que afirmei a princípio,
sustento; a, sua compreensão, como mulher, é invulgar.

--- Você me elogia assim, eu fico convencida. Isso é mau.

--- É mau? --- volveu ele sorrindo.

--- É sim. Uma mulher convencida é intolerável. Você já observou isso
quando se conversa num grupo de mulheres?

--- Penso que não.

--- Pois observe quando tiver oportunidade.

--- Por quê?

--- Porque as mulheres muito facilmente se convencem da perfeição.
Observe um grupo delas a palestrar. Você ouve isto a toda hora; ``eu por
exemplo...'' e lá vem um desfilar interminável de qualidades. Quando a
narradora expõe seus predicados, está ciente de que a sua interlocutora
não os possui. Mas acontece que mal uma termina, já a outra está
desejosa de contar também os seus casos exemplares. Como conclusão,
dizem um ``eu sou assim'' tão enfático, que seria impossível não se
persuadirem a si próprias desta pseudo-perfeição que todas as mulheres
imaginam possuir. Todos os termos passam a ter naquela hora um sentido
diferente pela entonação de voz.

--- Felizmente --- objetou Álvaro a sorrir --- as mulheres não fazem
dicionários...

Bárbara não foi indiferente à observação do companheiro e de bom humor,
continuou:

--- Deste modo, às vezes com bastante ingenuidade, as mulheres medíocres
passam ao rol das perfeições; por sua própria conta... é verdade. E no
final de tudo isso, você pensa que houve interesse recíproco nos famosos
exemplos de cada, uma? Não; isto percebe-se logo. Pois, no geral, não se
escutam, apenas se falam. Observe estas reuniões sociais de aniversário,
danças, visitas etc., essas em que não há um objetivo predeterminado a
ser discutido.

--- Mas se quiser que as mulheres ouçam com interesse --- continuou
Álvaro com ironia --- fale mal de alguém.

--- Os homens também são faladores, Álvaro.

--- Não tanto como as mulheres --- observou ele.

--- Bem, nesse ponto só me resta concordar. Mas eu tenho visto alguns
homens que, francamente, levam a palma nos falatórios. Às vezes, homens
de negócios, de atitudes másculas, que não se coadunam com os assuntos
que conversam. Usam até os termos comuns aos falatórios das mulheres.
Para falar a verdade, o hábito dos comentários está se estendendo muito.
Parece-me que apenas os intelectuais estão mais afastados dessas coisas.

Nesse rumo a conversa foi tornando-se mais leve; e o incidente do
hospital não mais foi lembrado. Passaram a relatar casos de suas
observações. O assunto variou continuamente. Lembraram-se de Helena; uma
pergunta saudosa veio aos lábios de Bárbara: onde estaria Helena àquelas
horas?

Álvaro então contou à moça alguma coisa da sua vida com Paulo. Os tempos
em que ambos eram estudantes na capital paulista, já iam longe, ele,
porém, se recordava como se fora há uns dias apenas. Paulo chegava
cansado do Mackenzie, estudava e trabalhava no mesmo estabelecimento. Às
vezes, vendo Álvaro a pensar na eleição do grêmio, zombava dele sempre
com as mesmas palavras: --- Estudante de direito tem a política no
sangue; vocês mais pensam nas eleições que nas obrigações --- frisava
Paulo. Álvaro ria e concordava; entretanto, achava divertido as cenas
políticas em miniatura. Contou à Bárbara como pediam auxílio dos
contínuos para terem presença nas aulas. Diziam na cidade que certa vez
os bedéis duplicaram o preço das célebres ``presenças'', e os
estudantes, reunidos, levaram o caso à Diretoria. Pois era o verdadeiro
mercado paralelo dentro da faculdade... Imagine, justo onde se acham os
representantes das leis!!

Lembrou-se também de um seu colega, cujo pai era fazendeiro, e que por
uma fatalidade qualquer viera a cair na dependência dos cambistas
universitários. Pediram-lhe uma novilha gorda. Lá veio da fazenda do
rapaz a preciosa vaquinha, representante oficial de um ano de faculdade.
Entretanto, que desilusão para o contínuo! O fazendeiro mandara-lhe uma
novilha... gorda, sim, porém, brava, brava como um touro das arenas.

Paulo achava graça de todas essas coisas, embora vivesse num ambiente
pesado de trabalho e de cálculos matemáticos. Álvaro concordava com o
amigo que os engenheiros trabalhavam mais e divertiam-se menos. E quando
Paulo lhe dizia que jamais vira uma propaganda eleitoral de grêmio no
Mackenzie, respondia sorrindo --- por isso os engenheiros não são
eleitos; desta forma não haverá proteção à classe. Finalmente, o rapaz,
concluiu:

--- Naqueles tempos, iam para a engenharia somente os vocacionados, por
isso, a política não tinha lugar entre os logaritmos ou a resistência
dos materiais. Agora Bárbara, nessa iminência de guerra na Europa e
outros povos com vistas comerciais para o Brasil, a engenharia vai
tornar-se a carreira do futuro. Você verá, então, os turistas das
faculdades de direito a mudarem o seu curso de veraneio. Teremos então
as eleições pleiteadas e discutidas, e, quem sabe, se paraninfos
políticos ou figurões nos quadros de formaturas. Será um mal para a
classe considerada, até hoje, a mais culta do país; mas será um bem para
nós, estudantes de direito, que nos livraremos, em parte, desses
parasitas profissionais. Bem diz o ditado, que a sorte de uns é o azar
para outros.

\chapter{Capítulo 81}

Álvaro e Bárbara não se encontravam com a mesma frequência de antes. Há
onze dias, tinham estado juntos no hospital; e, ele, ao acompanhá-la até
a sua casa, despedira-se, sem referência a um novo encontro. Bárbara
jogando a última parada dos seus sentimentos, expunha-se a toda prova,
lutando por si e por Álvaro.

Por seu lado, Álvaro reconhecia a sua fraqueza. Sabia ser impossível a
resolução definitiva, deixando Bárbara para sempre. Todavia, a dedicação
daquela mulher o levou ao auge da renúncia. Lutava na medida das suas
forças; sabia que era pouco, mas quando um homem dá o que pode, esse
pouco passa a ser tudo. Álvaro compreendia a extensão deste sacrifício
mútuo e, nos seus protestos íntimos, não se iludia quanto às
possibilidades de realização. Era como se andasse à beira de um
precipício; ele estava na estrada marginal que, ao menor descuido, o
levaria para baixo. E se ainda fosse só? --- A tentativa de arrastar
Bárbara consigo era-lhe uma ideia insuportável. Movido em seus
sentimentos, procurava manter-se firme; se não fora para ela um apoio,
pelo menos quisera não lhe ser um peso. Entregou-se ao trabalho
intelectual e, numa compensação justa, conseguiu, ainda que raras vezes,
sobrepor-se ao presente sem esperança.

Assim estava ele à sua escrivaninha, trabalhando febrilmente. Era manhã
ainda. Dessas manhãs de setembro, cujo excesso de luz chega a perturbar.
A conversa com Bárbara, as observações sobre Dalila, a volta repentina a
cenas particulares do passado, trouxeram-lhe emoções novas. E Álvaro,
impulsionado por estas emoções, dava ao mundo a sua contribuição
pessoal; analisava, nos seus trabalhos, a sociedade confusa do seu
tempo.

À certa altura, Álvaro deteve-se para pensar. Pousou o lápis
distraidamente. Numa abstração completa, pôs-se a olhar ao longe, como
se diante dele já não existisse uma parede. Nessas horas, seu pequenino
quarto passava a ser uma área ilimitada. Apanhou um cigarro e,
acendendo-o, começou a fumar involuntariamente. As ideias tornavam-se
mais claras. Num último esforço, ele as reuniu mentalmente em palavras.
Das palavras passou à forma literária. O seu pensamento apareceu então
nos caracteres morfológicos e expressivos --- era a palavra e o vocábulo
finalizando a tarefa do escritor. Álvaro viu escrito o seu pensamento! E
por mais que o apreciasse, reconheceu não ser claro e belo como o fora
na sua imaginação. Aquele caminhar pelo pensamento, sem consciência das
palavras em uso, era a emoção em clímax, e o escritor jamais a
transmitiria ao público na sua plenitude semântica. Às almas
semelhantes, somente, seria dada uma experiência igual.

Álvaro tomou novamente o lápis e prosseguiu na elaboração do seu segundo
livro. Uma dúvida ortográfica surgiu-lhe; neste gesto automático de
apanhar o dicionário, abriu-o descuidadamente. Com o lápis entre os
dedos, correu a mão pela página explicativa. Não estava próximo ao que
procurava; uma palavra, porém, fê-lo parar repentinamente.
\emph{Heliotrópio - planta, cuja flor muito aromática se volta para o
sol.}

O perfume suave de uma flor que se volta para o sol! Era ela! Era como
Bárbara lhe aparecera sempre. No sofrimento ou na alegria, a sua alma
estava sempre voltada para o sol. Não para o \emph{sol de inverno}, de
que ele falara um dia, mas para o sol abundante e tropical, cujos raios
se prolongavam sem limites. Levantou os olhos do dicionário, e só então
notou uma réstea de luz incidindo no canto esquerdo da sua mesa de
trabalho. Colocou ali a sua mão; a transparência de sua pele, o tom
avermelhado que tomou, levaram-no a pensar na ação vitalizadora do sol.

Era por isso que Bárbara tinha a alma tão intensa. Bárbara... tinha a
alma voltada para a luz.

\chapter{Capítulo 82}

--- Bárbara --- chamou a governanta, entrando na sala.

Sem levantar-se da poltrona de leitura, a moça voltou-se, e ouviu-a
perguntar:

--- Precisa de mim?

Ela calou-se por alguns instantes; depois, olhando firme para a
governanta, concluiu:

--- Preciso mais de mim mesma, Mrs.~Patrice.

--- Eu compreendo, Bárbara; todavia, não se esqueça, minha filha ---
tornou a reservada governanta --- que estarei a seu lado em todas as
circunstâncias.

Bárbara comoveu-se ante aquela solidariedade aos seus sentimentos. Mrs.
Patrice não era expansiva; Bárbara, porém, contava com ela.

Teriam prolongado a conversa, se o telefone não a interrompesse
bruscamente. A campainha insistia e Mrs.~Patrice, atendendo-o, procurou
falar. Bárbara recostou a cabeça para o outro lado da poltrona; numa
abstração, voou longe dali. Mrs.~Patrice chamou-a duas vezes; somente na
terceira percebeu que o telefonema era para ela.

--- Alguma coisa? --- indagou numa consternação indefinida.

--- Chamam-na do hospital --- informou a governanta.

--- Está bem --- completou Bárbara,

Levantando-se da poltrona, discou o telefone para Álvaro. Informaram-na
de que ele não estava no momento. Bárbara sentou-se novamente; embora
desligassem do outro lado, continuou com o fone na mão. Ficou ali a
pensar, sem saber o caminho a seguir. Se a chamaram, é porque
necessitavam dela. Como não atender, depois daquela cena do hospital?
Não era possível abandonar uma mulher fraca e doente. --- Afinal, ela
nada exigira de Álvaro e nem se colocara como sua vítima. Pedia um
auxílio em condições naturais, como o faria uma desconhecida; embora,
esse auxílio lhe tivesse custado alguma luta. Bárbara soube apreciar o
gesto altivo da outra mulher; ela não estava tirando proveito desonesto
da situação. Sentiu no seu íntimo sadio a necessidade imperiosa de
atendê-la.

Levou a mão à testa; só então percebeu que segurava ainda o fone.
Deixou-o sobre o gancho; levantou-se da poltrona, resolvida. Tomou o
carro e dirigiu-se ao hospital. O caminho era longo; o automóvel, porém,
era um transporte rápido. Não tardou que Bárbara reconhecesse a
escadaria ampla da casa de saúde. Subiu à portaria, no segundo andar e
solicitou a ordem de visita. A enfermeira em serviço informou-a de que a
doente esperava por ela. Bárbara tomou o corredor central e foi direito
ao quarto.

Tudo como da outra vez. Bateu; a mesma voz falou do interior. E,
entrando, viu Dalila em repouso. Ela olhou tristemente para Bárbara, e
seu primeiro gesto foi justificar a sua atitude:

--- Desculpe-me por tê-la incomodado; esta solidão prolongada\ldots{}

--- Não me causou incômodo algum --- interveio Bárbara

--- Além de tudo, não pude vencer o medo --- explicou a doente. --- É um
medo esquisito, vago; não sei ao certo do que seja.

--- É o cansaço físico. Mas isto passa --- animou Bárbara.

--- É também a falta de uma pessoa sã --- rematou a outra.

Dalila assumiu uma atitude respeitosa. Mudara muito desde a última vez;
o que viera confirmar as impressões de Bárbara.

--- Posso chamá-la pelo nome? --- perguntou embaraçada.

--- Naturalmente. Sabe como me chamo?

--- A senhora não me disse, mas eu ouvi dizer. É Bárbara, não é?

--- É sim.

--- Gostaria de falar somente Dalila? Acha que posso ser tratada com
mais intimidade? --- interrogou numa atitude quase infantil.

--- Sim, Dalila --- tornou Bárbara com brandura.

--- Então, Bárbara\ldots{}

--- \ldots{}

--- Já esteve doente alguma vez?

--- Coisas ligeiras. Sim, estive algumas vezes.

--- A minha moléstia não é ligeira; é até bem demorada. Mas eu não quero
viver assim.

--- Qual é a sua moléstia? --- indagou Bárbara.

--- Coração. Requer repouso e muito cuidado. Quem conhece a vida ativa,
não aceita estas imposições --- declarou Dalila desanimada.

--- As imposições, às vezes, são prenúncio de algo melhor.

--- Melhor? O que é melhor e o que é pior neste mundo? Ninguém sabe ---
tornou aquela mulher descrente.

E antes que Bárbara respondesse, continuou:

--- Quando me diverti, pensei que era o melhor. E veja no que acabei: no
pior.

--- Mas pode voltar ao melhor --- interveio Bárbara.

--- De que jeito? Isto é fantasia de quem procura consolar doentes.

Bárbara olhou para ela e calou-se.

--- Desculpe, mas sou muito prática. Gosto de resolver as coisas da
melhor forma; isto, porém, não impede que caia no pior. Bem, vamos
deixar esta conversa.; quem sabe há música por aqui, disse ligando o
rádio de cabeceira.

A É I Ó Urca! --- falou o locutor.

--- Urca! Quantas vezes estive lá. Diverti-me tanto! Joguei, dancei,
fumei e bebi; agora... agora tudo isso me é proibido. Oh, por favor,
arranje-me um baralho para matar o tempo. A que vida me condenaram! ---
exclamou com desespero.

Bárbara desligou o rádio e achegando-se a Dalila procurou acomodá-la.

--- Se descansar, ajudará a cura. Depois, fará o que quiser.

--- Depois?

--- Quando sarar, naturalmente --- concluiu Bárbara.

--- Já faz um mês que estou aqui --- suspirou Dalila --- nem por isso
tenho apresentado melhoras.

--- Talvez pelo repouso incompleto; a medicina de hoje tem como base o
descanso do organismo --- comentou Bárbara.

--- Se eu ainda gostasse de ler, gostasse de alguma coisa que enchesse a
vida. Tudo é tão vazio sem um afeto --- falou desanimada.

Bárbara silenciou.

--- Estou pagando os meus pecados --- asseverou convicta. --- Agora
conheço bem o Deus que expulsou a chicote os vendilhões do templo.

--- E desconhece o Deus de Madalena! --- lembrou Bárbara.

--- Madalena? Eu, porém, não sou Madalena --- afirmou com ironia. ---
Sabe de uma coisa? --- perguntou numa atitude estranha.

--- ?

--- Se existir inferno, eu irei para lá. Este inferno de demônios de
chifres, de tachos escaldantes, de feras monstruosas, descrito por aí,
talvez não seja pior que o próprio mundo --- considerou Dalila. O mundo
tem tudo isso, apenas mais encoberto. Se os sentimentos humanos fossem
visíveis, tudo o que há no inferno seria visto no próprio homem. Às
vezes, chego a pensar que o inferno seja uma cópia da imaginação do
homem. Você não sabe como se chegou a tais ideias?

--- Não --- respondeu Bárbara. --- Talvez a descrição de Dante tenha
sido o primeiro passo. Se ele deu origem a tais ideias, causou um grande
mal à humanidade; cultivou no homem a tendência de se afastar do
inferno, quando o verdadeiro, seria aproximá-lo de Deus.

Sem desviar-se da sua reserva habitual, Bárbara procurou despertar a
energia daquela mulher gasta e vencida. Já na primeira vez, observara a
inteligência de Dalila. Bárbara estava certa de que ela compreendera o
passo que dera; no fim de suas conclusões, portanto, não a culpou.
Dalila fora forte para fazer a sua vontade, e fora franca para não
esconder o que realmente era.

--- Você acha que o inferno é realmente assim? --- perguntou novamente
com a sua voz cansada e difícil.

--- Francamente --- redarguiu Bárbara --- quase nunca penso no inferno.

--- É --- suspirou Dalila --- você não tem o inferno dentro de si mesma.

E recostando-se nos travesseiros, calou-se.

--- Agora vou retirar-me --- volveu Bárbara. --- Desejo as suas
melhoras.

--- Obrigada. Novamente, queira desculpar-me.

--- Se precisar, faça a mesma coisa --- observou, procurando ser
natural.

Despediu-se atenciosamente e afastou-se com vagar. Dalila ficou só outra
vez.

\chapter{Capítulo 83}

Álvaro terminava o seu segundo trabalho. A convivência com Bárbara e o
impulso da paixão que o consumia, levaram-no a pensamentos profundos em
torno dos problemas sociais. Tornara-se mais humano, e esse Álvaro mais
humano é que aparecia nos livros, analisando a própria Humanidade. Um
impulso nervoso impelia-o ao íntimo das criaturas, rebuscando nos seus
semelhantes a razão das suas fraquezas ou a força das suas vitórias.
Considerou problemas diários, trouxe o ser humano desde o ventre materno
até a sociedade. Após um estudo intenso sobre os problemas do homem
moderno, terminou o seu novo livro com a frase que lhe serviu de título:
``TEM A SOCIEDADE MODERNA DIREITOS DE CRÍTICA AO DIVÓRCIO?''.

Deixou nas suas páginas o convite à reflexão; procurou abalar as bases
falsas em que se estribavam os contemporâneos na busca de soluções
momentâneas. Mostrou aos orientadores, com uma argumentação convincente,
que pouco importava a inclusão do divórcio nas leis brasileiras, uma vez
que ele existia independente delas. Finalmente, que era o pronunciamento
jurídico senão apenas a verificação de um fato já consumado?! E numa
comparação entre a morte física e a morte conjugal, Álvaro perguntava:

Que faz a medicina diante de um cadáver? Reconhece-lhe a morte física,
embora, contra todos os desígnios de seus princípios científicos. E
retira então o cadáver de entre os seres vivos; pois a sua permanência
entre estes, na putrefação desagregadora dos tecidos seria a medida
anticientífica, cujos males consequentes, tornar-se-iam incalculáveis.
Prevenindo esses males, a medicina declara a morte clínica; e o que está
morto é separado do que tem vida. Se a medicina não se declarasse a
respeito, a morte deixaria de existir? --- A força decompositora do
cadáver não mudaria em intensidade, direção e sentido, apenas pelo não
pronunciamento das autoridades médicas. --- Vê-se, portanto que os
desejos pessoais não são fatores determinantes da integração das
energias; o sopro de vida que animaria o cadáver ultrapassava já o poder
da ciência humana.

E Álvaro, que conhecia as amarguras de uma união desajustada,
considerava o casamento indissolúvel a permanência do cadáver entre os
vivos. O fato de as autoridades jurídicas não se pronunciarem a
respeito, não traria vida a um lar que já estava morto. E o cadáver do
lar, se desagregando entre os cônjuges, era a moléstia infecciosa,
destruindo a saúde da alma. Ninguém poderia calcular a que proporções
mortíferas chegaria a força de tais micróbios.

E não resolvendo o problema da morte, como agia a medicina? Construía
condições científicas que permitiam aos homens, usufruir, no mais alto
grau, os bens da vida. Preceitos de higiene física eram instituídos e
proclamados, para a formação de um ambiente sadio.

A sociedade deveria fazer o mesmo. Educar os homens em princípios
sadios, para lhes cultivar a higiene da alma. Fortalecendo-se os corpos,
prevenir-se-iam moléstias futuras; e o trabalho da medicina tenderia
mais para a profilaxia que para a cura de emergência. Fortalecendo-se as
almas, prevenir-se-iam os desajustes futuros; o trabalho dos dirigentes
sociais tenderia mais para a construção de um povo espiritualmente forte
que para as soluções dos tribunais executivos.

Os problemas, porém, jamais teriam soluções definitivas. As instituições
humanas estão sujeitas a forças mais altas. A medicina veria o perecer
dos corpos, não obstante o esforço incansável da ciência de curar.
Diante dela, dos seus princípios e das suas condições, das suas teorias
e dos seus trabalhos, dos seus estudos e seus ensinamentos, da
profilaxia ou da cura, passariam os cadáveres de todos os homens! Dos
que morreram naturalmente, após vida longa e sadia, como dos que
sucumbiram às forças insustentáveis das moléstias imprevistas e
insolúveis. Todos eles beneficiados pela ciência médica; uns pelos
recursos profiláticos, outros pela terapêutica.

Assim, a sociedade veria o perecer dos lares, não obstante o esforço
incansável da arte de educar. Alguns se desfariam normalmente, pela
morte física de uma das partes; outros, pelas forças insustentáveis de
um desajuste imprevisto e insolúvel.

A vida espiritual sem afeto é como a vida de um corpo sem saúde. Um
espírito bem formado correria o risco da dissolução, num ambiente
contínuo de desentendimentos; como o organismo forte se exporia a pestes
mortíferas num campo de micróbios infecciosos.

A sociedade persistiria no seu trabalho profilático e terapêutico; e
todas as almas seriam beneficiadas pelos recursos das ciências sociais.
O seu pronunciamento diante de um lar que não existe é o mesmo que a
medicina diante de um cadáver. O divórcio e a morte clínica são, pois,
meras verificações de uma realidade insofismável.

\chapter{Capítulo 84}

A entrega ao editor, do seu segundo livro, permitiu a Álvaro a procura
de melhores acomodações. Por outro lado, as suas condições melhoravam no
jornal; tinha a seu cargo toda a parte artística da redação. Publicara
ainda alguns artigos separadamente, o que lhe vinha aumentar os
rendimentos mensais. O último trimestre de faculdade, ele já o cursava
sem empecilhos financeiros; comprava os livros necessários, andava bem
arrumado, sacrificando apenas o comparecimento a algumas aulas. Quando
saía com Bárbara, entregava-se ao prazer de lhe custear algumas
extravagâncias. Convidava-a para almoçar ou jantar fora, levando-a de
taxi para casa. Outras vezes, passeavam no carro de Bárbara; Álvaro saía
à direção e, passando pela bomba, punha gasolina por sua conta. A vida
melhorava financeiramente.

Naquela tarde de outubro, Álvaro transferia-se para uma pensão melhor.

A casa era dirigida por alemães; embora arrogantes, traziam tudo em tais
condições de higiene que se tornava agradável morar ali. Álvaro iniciou
a acomodação de suas coisas. Pendurou os dois ternos novos. Ao voltar-se
para a mala, viu o terno marrom-claro que o aguentara por dois anos sem
um intervalo. Olhou detidamente para o paletó lustroso e gasto; enfiou a
mão nos bolsos furados. Segurando o terno, aproximou-se do espelho. Que
diferença! Como usara ele semelhantes roupas?! Um enchimento quebrado ao
meio, costuras em desalinho, um talhe arredondado e, por fim, uma
casimira áspera e grossa.

Álvaro usava aquele terno quando conhecera Bárbara. Recordou-se da sua
displicência, caminhando ao lado da moça como um mendigo. E lhe falara
ainda no seu perfume! Como era engraçado, agora, recorda-se daquele
sujeito boêmio, tão mal-arranjado, a falar sobre perfume suave. E que
coisa curiosa! Não se sentira intimidado diante dela; e nem um gesto de
menosprezo ele observou. Bárbara o tratara como um igual embora a sua
aparência desagradável de homem sem recursos. Era estranho recordar e,
recordando, as atitudes de Bárbara o deixavam perplexo.

Os homens, e mais ainda as mulheres, guiavam-se muito pelas aparências!
Lembrava-se como os antigos conhecidos fingiam não o ver quando o
encontravam. E ele trazia sempre o mesmo terno da ocasião em que
recebera as atenções de Bárbara. Tudo aquilo lhe parecia estranho. Se um
amigo lhe contasse ou visse escrito em romance, teria dúvidas a
respeito. Mas, o caso se dera com ele e há dez meses apenas. Era pouco
tempo para esquecer-se.

De fato, pensou Álvaro, coisas extraordinárias existem pela vida; os
homens é que perderam o hábito de vê-las. Acostumam-se à
superficialidade rotineira das coisas; quando vem algo admirável, passa,
sem ser notado, para o terreno da fantasia. Álvaro riu do próprio
mundo... e acendendo um cigarro, tirou a primeira baforada. Olhou bem a
marca do cigarro: ``Continental''. Continuando a rir, lembrou-se dos
``Castelões'' que ele fumara tanto tempo.

Num gesto nervoso, levantou-se, jogou o terno para um canto. Abriu os
embrulhos. Sobre o terno marrom-claro, outras velharias foram atiradas:
roupas interiores do mais grosseiro listado, meias esburacadas, sapato
furado, gravatas retorcidas, suspensório e liga amarrados com barbante,
lenços sem barra, e uma carteira velha de fino crocodilo, última
reminiscência dos tempos de dinheiro.

Abriu as janelas com força; a luz da tarde ainda iluminou o quarto.
Álvaro não se satisfez; acendeu todas as lâmpadas. Havia três para seu
uso: a do forro que era forte em claridade, a do criado-mudo e uma
especial em difusão sobre a escrivaninha. A cama era de madeira boa e o
guarda-roupa também; o espelho não deformava as imagens e o chão
brilhava na sua limpeza. Álvaro andou pelo quarto; como era agradável
ter comodidades. Depois olhou as velharias amontoadas e uma pergunta lhe
veio à mente; que diriam os donos da pensão? Sentiu um desejo imediato
de livrar-se daquelas coisas. Chamou o empregado e pediu-lhe que
retirasse tudo. O empregado atendeu e Álvaro, então, contemplou a sua
nova morada.

\chapter{Capítulo 85}

Bárbara sentou-se à escrivaninha e diante de si, pôs o trabalho
datilografado de Álvaro. Começou a ler --- era a nova declaração dos
direitos do homem. E o autor analisava, com profundo alcance, a confusão
da moderna sociedade brasileira.

O mundo todo estava abalado por novas concepções políticas e sociais: a
Alemanha em sua nova ordem, desafiava o poderio inglês nos continentes;
também ela queria colônias além de suas fronteiras. A Rússia era o
segredo comunista; isolada pelos demais estados ou dos demais estados,
suas notícias chegavam ao exterior de forma pouco acreditável. Uma
coisa, porém, era certa --- fora a primeira nação moderna a romper a
estrutura política, que preestabelecia direitos para classes
privilegiadas. O postulado da igualdade entre os homens se radicava na
Rússia, eliminando progressivamente os antagonismos de classes. A Itália
proclamava Mussolini o deus dos italianos; os jornais filmados mostravam
um povo delirante a aplaudir o homem sobrenatural, condutor de uma nação
secular. A Itália, a exemplo da Alemanha, também alargava as suas vistas
pelos territórios vizinhos num sonho imensurável de conquistas. Era o
povo cristão a invadir a terra dos abissínios para levar àqueles homens
de cor a \emph{moderna} civilização europeia. A França, embora
permanecesse república, atravessava período difícil; choques políticos e
sociais estremeciam as bases do seu sistema de organização. Sociólogos,
filósofos, educadores, jornalistas e outros ainda, levantavam-se,
predestinando à França uma derrota, se ela não reunisse as suas forças
dispersadas. Desde 1918, houve escritores que concitassem a França à
disciplina do espírito e dos costumes. Não por uma inferioridade do
próprio espírito nacional, mas por uma precaução a interesses alheios
orientados para a terra dos franceses. O povo que modernamente se
descuida está fadado a desaparecer na espada ou no sorriso diplomático
de uma nação vizinha.

E entre as velhas e novas concepções políticas, balançava-se o Brasil.
Influenciado pelas concepções modernas, e após a declaração do Estado
Novo, o Brasil debatia-se indeciso, sem forças para estabelecer-se. Como
``um vasto hospital'', porém, desprovido de médicos, o Brasil recebia em
seus territórios, estrangeiros refugiados, espiões e aventureiros. As
vistas comerciais do mundo voltavam-se para a terra brasileira; as
maiores fontes de riqueza não pertenciam aos seus homens; olhares
gananciosos de povos mais distantes alcançavam e atravessavam o subsolo
do Brasil; e para terminar, uma propaganda subterrânea de
desnacionalização levava brasileiros inconscientes a preferirem as
bandeiras de outras pátrias. Atravessava-se um período de completa
descrença dos homens do Brasil. E o povo, temendo o serviço do
funcionalismo público, prestigiava as instituições estrangeiras. À
sombra dessa confusão de sistemas e costumes, o povo brasileiro, falho
de formação individual, constituía uma sociedade indecisa, sem
princípios definidos para seus alicerces.

E Bárbara, que acompanhava o caminhar da nação brasileira, sentia-se
tomada de uma tristeza profunda, como se ela tivesse nascido no Brasil.
Tinha visto nos últimos tempos, debates sobre a questão petrolífera, e,
pela marcha dos fatos, concluiu desanimada que ainda caberia aos
estrangeiros a conquista do petróleo do Brasil. --- Ah o Brasil, o
Brasil --- repetiu Bárbara --- que terra estranha esse Brasil! ---
Convivia com os brasileiros, e nestes via a falta de brasilidade. Ouvira
certa vez do tio Maurício, que era fácil ter negócios no Brasil. As
correntes políticas, dissera ele, não se formam pelo interesse de servir
a pátria, mas pelo interesse objetivo de seus chefes. A desonestidade do
povo e de seus governantes facilita negócios de estrangeiros no Brasil.
A experiência comercial manda que se aproveite, enquanto não surge
dentre eles um reformador intransigente.

Álvaro passara por aqueles problemas, procurando fazer voltar o Brasil
para os brasileiros; não com palavras de belos discursos, mas concitando
os homens a constituírem-se individualidades pensantes, pois, este seria
o primeiro passo para o descortino de problemas mais graves.

A observação honesta, portanto, desapaixonada dos fatos, e um raciocínio
lógico em torno destes mesmos fatos, seriam o cimento e o ferro da
estrutura de concreto desse grande edifício que ainda poderia ser o povo
brasileiro.

***

Bárbara desviou o curso de seus pensamentos. Naquelas circunstâncias tão
sérias da leitura, a ideia de um raciocínio lógico transportou-a a cenas
do passado, quando ela estudava ainda. Sorriu. Era interessante, embora
triste, recordar alguns fatos da sua classe de primeiro ano
pré-jurídico. Desenhou-se na sua imaginação aquela sala de aulas, onde
um aluno respondia gaguejando à pergunta do professor:

--- Sim, professor, o raciocínio lógico é a base.

--- Mas a base de quê? --- insistiu o professor.

--- Ora --- tornou o rapaz de vinte e um anos --- a base de tudo.

--- De tudo?! Muito bem. Mas que é para você esse tudo? Ou antes, que é
para você a lógica?

--- A lógica... a lógica\ldots{} a lógica é a lógica personificada ---
tornou o discípulo com ênfase.

--- Com os característicos de pessoa? --- indagou o professor rindo a
não mais conter-se.

Bárbara tão admirada ficara que o riso lhe faltou. Somente agora, sete
anos depois, é que o caso lhe pareceu engraçado. Lembrava-se de que o
professor, percorrendo os componentes da classe, fez a pergunta de um
por um --- que era a lógica? Tomados de improviso na aula, os alunos
assustaram-se.

Um dos colegas, Bárbara recordava-se dele, levantou-se o calmamente
respondeu --- a lógica é a mãe de todas as ciências. Houve um murmúrio
na classe; aquela informação de algibeira fez brotar respeito por parte
dos companheiros. O professor, que conhecia a deslealdade intelectual
das pessoas, fez uma nova pergunta:

--- Como sabe você que as outras ciências são filhas da lógica?

Bárbara não se esqueceu da situação angustiosa em que o rapaz se
envolvera. Voltaram-lhe à mente as palavras do velho professor, doutor
em filosofia --- não sejam meros repetidores de fórmulas.

E este colega, de informações de algibeira, havia dito a Bárbara que
iria cursar direito e, a seguir, tomar capelo; isto é, obter o grau de
doutor em borla e capelo. Faria depois os discursos, sentado, e imporia
respeito aos demais.

Era assim, pensou Bárbara, que a classe intelectual do país se via
invadida pelos que faziam o curso de borla e agiam com a falsidade de
capelo.

Juntou novamente as folhas do trabalho de Álvaro e iniciou a leitura.
Mrs.~Patrice, porém, entrou na sala e, interrompendo-a, entregou-lhe uma
carta cuja letra Bárbara reconheceu de imediato. Abriu-a e leu:

\emph{Bárbara querida,}

\emph{Somente hoje você vai receber uma carta da pessoa mais feliz do
mundo. Venho sempre pensando em escrever-lhe, mas o deixar para o amanhã
retardou notícias nossas.}

\emph{Imagine, minha querida amiga, que quem lhe escreve é Helena
Machado. Não poderia pensar jamais que, ao deixar a sua casa,
encontraria Paulo para sempre. Estou tão feliz, Bárbara, que não sei
como dizer-lhe.}

\emph{Paulo está no trabalho e eu preciso terminar antes da sua volta;
estando ele em casa eu não poderia mandar-lhe uma linha sequer. Desculpe
o nosso egoísmo, mas a felicidade, algumas vezes, desnorteia a gente.
Ele deixou a construção da estrada e conseguiu um emprego por concurso.
Além disso, pegou alguns trabalhos extras para fazer em casa. Já aprendi
a manejar a régua de cálculo e auxilio Paulo em várias operações. Você
ainda se lembra dos logaritmos que aprendemos no ginásio?} \emph{E a
trigonometria? Pois estou recordando todas estas coisas, e muito breve
irei assustá-la, com os meus conhecimentos. Paulo acha que a matemática
é a base de todas as ciências; e, que para um povo ser intelectualmente
forte, é indispensável o raciocínio da mais perfeita criação humana. Por
isso, imagine você, eu estou entrando no caminho, e no caminho do
raciocínio! Eu, minha querida amiga, aquela caixa de sentimentalismo! No
outro dia, disse a ele que eu ia me tornar uma sábia e, depois, rumo à
faculdade. Ele riu e respondeu: pode tornar-se sábia, se quiser; e
depois, rumo a Paulo novamente. Enfim, a minha vida é uma alegria,
Bárbara.}

\emph{Com isso, estou sendo muito egoísta; até agora só falei de mim. E
para você como vão as coisas? Estive pensando na minha insistência
quanto} à \emph{sua mudança para o Rio e, agora, estou aqui, e você
nessa distância! Que resolveu da sua vida, vai ficar por aí mesmo?}

\emph{E o coração, Bárbara, está no seu ritmo regular? Paulo contou-me
que Álvaro está, apaixonado por você. Isto, naturalmente, você já o
percebeu. Não estou certa quanto aos seus sentimentos por ele, mas
suponho algum amor da sua parte também. Assim sendo, como pretendem
acertar a vida? Você é inglesa e na Inglaterra existe o divórcio; não
seria possível uma solução no consulado? Isto, sem lembrar o recurso
nosso que, modernamente, dizem ser o México. E como sabe, é apenas
questão de dinheiro. Não tenho pretensões, Bárbara, de aconselhá-la
neste sentido; apenas, desejo, de coração, vê-la, feliz, muito feliz;
feliz como eu sou nesse momento em que lhe escrevo.}

\emph{E do meu pessoal, você soube alguma coisa? Escrevi-lhes um
cartãozinho, justificando o meu procedimento; todavia, não obtive
resposta. A minha velha Babá mandou-me uma carta dizendo que mamãe e
papai irritaram-se terrivelmente comigo. E só não tomaram providências a
respeito, porque não me receberiam mais em casa. Na verdade, Bárbara,
estou ciente que o aborrecimento maior foi a humilhação social a que eu
os submeti; quanto à minha pessoa, era o objeto de menor interesse na
questão. Babá} \emph{contou-me que papai disse com raiva na sala:} ---
\emph{Helena deu um pontapé na sorte; pois, agora, que se aguente, mesmo
que vá para o inferno. -}

\emph{Escute, Bárbara, Babá quer vir comigo e eu estou pensando em
trazê-la. Gosto muito dela, você sabe. Ensinei-a a ler, escrever e fazer
tricô; lembra-se? Respondi-lhe dizendo que viesse; caso ela precise de
alguma coisa, peço a sua intervenção com o fim de auxiliá-la. Paulo está
plenamente de acordo.}

\emph{Bem, Bárbara, hoje fico por aqui. De outra vez, mandar-lhe-ei uma
catilinária.}

\emph{Paulo envia abraços a você e Álvaro.}

\emph{Um beijo da amiga afeiçoada,}

\begin{quote}
Helena
\end{quote}

\emph{31 de Outubro de 1938.}

Que distância entre as duas amigas? Por que insistia Paulo na sua
permanência pelo norte? Continuaria lá para sempre? Bárbara sentiu
saudades de Helena; as visitas da amiga na casa de Santa Tereza,
tornaram-se inesquecíveis. Helena vinha sempre triste, sob a influência
de uma desilusão que julgava irremediável; hoje, ela lhe escrevia, e da
sua carta transbordava uma alegria inconfundível. Mas as pessoas não
eram todas felizes ao mesmo tempo. Agora que Helena estava bem, Bárbara
sentia crescerem as suas dificuldades. Deixando cair sobre a mesa a
carta da amiga distante, Bárbara fitou o retrato do pai. Um sentimento
imperioso levava-a para ele nas circunstâncias difíceis. Ela teve desejo
de falar, de contar o que se passava na sua alma. Que felicidade
incalculável, ter alguém para ouvir as nossas confidências!

Mal pensara isso, surgiu-lhe outra ideia em que não se detivera ainda.
Não confiara a Álvaro as suas dificuldades e jamais lhe falara do seu
amor. Não existiria entre eles, liberdades para todos os assuntos? ---
Assim, Álvaro o fizera também. Não se referira à situação de ambos com
um plano determinado; parecia entregue às circunstâncias e receoso de
contrariá-las. Não lhe propusera um casamento no México, como Helena o
lembrara com tanta simplicidade, nem lhe pedira que o acompanhasse na
vida. Contudo, não deixava de procurá-la e em todos os seus passos
parecia estender-lhe a mão.

Era como se estivessem num barco em alto mar, à espera de um navio que
os recolhesse. A presença da ex-esposa, numa atitude submissa e um tanto
indiferente, vinha concorrer para o ambiente místico e esquisito que se
formava entre eles. Bárbara, por temperamento e por educação,
subtraía-se ao misticismo; mas, havia ali algo inalcançável, superior às
suas forças, e ela não conseguia furtar-se à singularidade daqueles
fatos. No íntimo, não perdera a confiança, e, nesse estado de ânimo, era
incapaz de uma deliberação definitiva. Sem saber ao certo por que ou em
quê, Bárbara confiava ainda; uma esperança longínqua, indefinida,
prognosticava um acerto das coisas.

Sentia cada vez mais a intensidade desse amor; não fora cedendo aos
poucos, por fraqueza ou por circunstâncias adequadas. Ao lado dele
estiveram sempre outros, cantando a sua beleza e implorando a sua
atenção; mas Álvaro sobrepujara a todos, e Bárbara compreendeu que ele
havia entrado na sua vida de forma segura e incontestável.

Depois de muito pensar, Bárbara guardou carinhosamente a carta de Helena
e encaminhou-se para a vitrola. Naquele estado de espírito, só a música
lhe caberia na alma. Tomou um álbum, ao acaso abriu-o --- \emph{Sonata
ao Luar} --- \emph{Beethoven.} Pianista --- Wilhelm Kempff.

Bárbara começou a ouvir a mensagem; era Beethoven que lhe falava. E como
Kempff se identificava com Beethoven! Em algumas passagens pareceu que o
pianista ultrapassava a correção da forma, e atingia, assim, a própria
criação. O apelo das notas repetidas com insistência transportava
Bárbara a um mundo diferente, compreendido apenas por alguns eleitos, a
quem a natureza dera a faculdade de senti-lo. Pelas mãos e pela
sensibilidade artística de Kempff, a moça ouviu toda a sonata. A beleza
profunda da paixão no \emph{Presto Agitato}, trouxeram-lhe emoções
estranhas; sob a influência dessas emoções, permaneceu algum tempo. Um
estado de alma nostálgico, perturbava-a, acariciava-a. Agora, que
Bárbara amava, qualquer coisa de novo viera juntar-se à emoção da
música. Parecia compreender mais Beethoven, cujo coração nunca estivera
vazio de amor.

Pensando em Álvaro, Bárbara compreendeu, sentiu uma afinidade
surpreendente entre a emoção da música e a do amor. A sonata ao Luar
deixara o seu coração à mesma altura em que ela, tantas vezes, o sentira
ao lado do homem a quem amava.

\chapter{Capítulo 86}

A campainha do portão vibrou com força; e logo a seguir entrou na sala
uma pessoa que Bárbara não via há muito tempo.

--- Alô --- disse com a sua voz alegre e jovial.

--- Carlito! --- tornou admirada. --- Que bons ventos o trazem?

--- Na verdade, não foi o vento; foi o acaso. Passei por aqui agora e
resolvi saber o que anda fazendo. Desde o concerto, você desapareceu;
parece incrível!

Bárbara convidou-o a sentar-se um pouco. A alegria inconsciente de
Carlito parecia contagiosa e as suas risadas francas soavam
agradavelmente pela casa. Bárbara sentia prazer na companhia do rapaz;
tinha por ele uma terna afeição e gostava do seu ar despretensioso e um
tanto indiferente. Conhecia os sentimentos de Carlito com relação à sua
pessoa; todavia, não se preocupava por isso. Carlito interessava-se por
muitas, e não seria difícil a Bárbara controlar a direção dessa
preferência do rapaz. Se ela o correspondesse, ou se colocasse num plano
inatingível, talvez ele fosse ao auge da paixão; mas Bárbara brincava ao
seu lado, tornando-se igual e não alimentando coisa alguma. Uma ou outra
vez, num ambiente mais propício, o rapaz tinha um impulso apaixonado;
contudo, isso passava e a camaradagem um tanto amorosa se restabelecia.
Assim, ela sentiu-se alegre quando Carlito entrou rindo na sua casa, e
foi sentar-se ao lado dele, comentando os acontecimentos da vida de todo
dia.

--- Então, Carlito, como vai o pife-pafe?

--- Mal --- tornou ele --- tenho perdido a valer.

--- E não desiste de jogar?

--- Mas que outra coisa se tem para fazer?

Bárbara riu e perguntou:

--- E a dança, a praia, as reuniões por aí; a faculdade?...

--- Oh, Bárbara! Sempre me esqueço que vou ser advogado. Esta ideia de
estudar em Niterói é um tanto desajeitada, sabe? Uma barca que mais
balança do que anda... você já andou nas barcas de Niterói?

--- Já.

--- Então; não tem dó de mim, a fazer aquele percurso todos os dias? Ah
tenha a paciência! Isto é um pouco muito.

--- Em que ano está, agora?

--- Eu? Estou no... segundo. É isto mesmo --- respondeu convicto ---
devo passar para o terceiro no fim do ano.

--- Quais são as matérias do segundo ano?

--- Bárbara! ---volveu espantado --- mas que pergunta! Você não quer
falar de outra coisa. E por falar nisso, tenho uma surpresa para você;
espere um pouco.

Carlito levantou-se e foi até ao carro. Bárbara notou a coincidência do
acaso com a surpresa, mas, nada disse a respeito. Não tardou que ele
regressasse com uma revista na mão. Achegando-se à moça, entregou-lhe
num gesto pomposo:

--- Veja; é uma distração de verdade.

Bárbara leu o título: ``Cena-Muda''; e começou a folheá-la.

--- Aí na terceira página --- comentou alegremente o rapaz --- diz que o
Boyer é um ótimo marido; além disso, não se importa que a esposa
trabalhe no cinema. Que coisa atrapalhada esses artistas; vivem a beijar
um a mulher do outro e no fim dá tudo certo. Faz lembrar o tal
estrangeiro, se não me engano, economista inglês; eu tenho um medo de
afirmar as coisas na sua frente.

--- Medo! Por quê?

--- Hum, se tenho! Você anda muito a par das coisas.

--- Acha? Mas, afinal --- interrogou ela --- que disse o economista
inglês?

--- Olhe, eu não afirmo que seja inglês; ponha de lado essa
responsabilidade. Mas ele disse que no Brasil todos mandam, ninguém
obedece e no fim tudo dá certo. E agora, veja aí mais adiante; as
artistas estão na praia. Cada maiô! Verdadeiras maravilhas do sexo... do
século --- corrigiu desapontado, pela palavra que pronunciara
inconscientemente.

Ficaram a conversar e ambos se esqueceram das horas. Quando Carlito
despediu-se de Bárbara, ela o acompanhou ao portão, num gesto grato pela
sua alegre palestra.

Escurecia. Bárbara estendeu-lhe a mão ainda uma vez. O rapaz
contemplou-a. Bárbara pareceu-lhe um ornamento na tarde morna que
esmaecia. Tomou a sua mão e, sem poder conter-se, beijou-a
precipitadamente.

A seguir, retirou-se, como o fizera naquela noite do concerto, lá na
entrada do teatro. Bárbara viu sair a barata e encaminhou-se para a
porta de entrada. Uma pressão nervosa deteve-a no caminho; sentiu-se
magoada sem saber por quê. Voltou-se para a rua; olhou ao redor e, não
vendo nada, entrou em casa, fechando a porta atrás de si.

Bárbara não vira que, a cinquenta metros da sua casa, alguém presenciara
aquela cena de maneira tão diferente.

\chapter{Capítulo 87}

Álvaro passara parte da noite no jornal. Inaugurara-se uma exposição de
pintura e ele deveria noticiá-la, tecendo breve comentário, não
propriamente crítico, sobre as diversas telas. Após admirar e observar
detidamente os trabalhos avançados que os modernistas apresentavam,
dirigiu-se à redação.

Álvaro não apreciava até há pouco este gênero de pintura; mas, como que
de repente, abriu-se a ele uma nova visão e um campo imenso estendeu-se
diante deste. Começou a sentir no impressionismo dos pintores atuais,
uma pincelada representativa dos sentimentos e emoções do homem da sua
época. A capacidade criadora do modernista não se enquadrava mais na
estrutura pictórica dos acadêmicos. Surgiu de maneira imperiosa a
renovação técnica, frente à renovação do mundo hodierno. E a
sensibilidade do artista o capacitara, como através de todos os tempos,
a intuir o presente da humanidade. Assim, um mundo vertiginoso, como o
atual, determinou a criação de uma técnica artística alucinante; não por
ser alucinante a arte, mas, por assim o ser o mundo moderno. E essa
técnica completamente nova, em que se sente maior liberdade quanto às
formas devido à loucura da nossa vida atual, é uma repetição na qual se
espelha a vida subjetiva do artista; pois toda obra de arte, como diz o
filósofo, é a flor eterna que assomou sobre as paixões do próprio
artista.

Voltemos, disse Álvaro, um minuto nossas vistas para Benedetto Croce em
seu ``Breviário de Estética'' que, por sua vez, é uma obra de arte;
pensemos esse minuto com ele. Mostra-nos o insigne pensador italiano
dois aspectos históricos da arte --- num, a elevação da forma como
elemento preponderante da obra artística; noutro, a superestimação do
conteúdo em oposição à forma. Numa clarividência verdadeiramente genial
de filósofo, refuta Croce tanto a Estética da Forma como a Estética do
Conteúdo, estabelecendo sua conclusão, diante da qual passamos a palavra
a ele mesmo --- ``A verdade é precisamente esta: que conteúdo e forma
devem distinguir-se perfeitamente na arte, porém, não podem
qualificar-se separadamente como artísticos, precisamente por ser
artística somente sua relação, por assim dizer, sua unidade, entendida
não como unidade abstrata e morta, senão como a concreta e viva da
síntese \emph{a priori.} A arte é uma verdadeira síntese \emph{a priori}
estética, de sentimento e imagem na intuição, da qual pode repetir-se
que o sentimento sem a imagem é cego e que a imagem sem o sentimento
está vazia.'' Esta conclusão magistral ajuda-nos a penetrar o sentido da
arte moderna, tantas, vezes alvo das mais injustas críticas contrárias
ao seu valor; justamente porque, mais do que nunca, se evidencia nas
obras dos nossos artistas modernos, a absoluta impossibilidade de
separar conteúdo e forma como elementos artísticos em si.

No Brasil, onde a arte não alcançou ainda o seu desenvolvimento
necessário, o espectador superficial sente-se confuso diante de um
trabalho desse gênero, porque perde de vista a relação entre o desenho e
a realidade; isto é, sente-se desnorteado no caminho que o levaria a
estabelecer unidade entre os elementos imagem e objeto. E assim a
liberdade de composição, pela nova construção plástica ou idealização
temática, confunde desde o início o espectador que, sentindo-se à margem
no seu ponto de partida, não encontra o caminho para prosseguir.

Mas a arte é intuição. E estão dentro da arte, tanto Michelangelo com
\emph{os} seus homens fortes e corpulentos, como El Greco nas suas
figuras ascéticas em cujas formas finas e alongadas sobressai a palidez
das cores. Rembrandt na sua distribuição de luz e sombra, onde as cores
parecem estendidas por pinceladas ligadas, como Van Gogh em sua
composição cromática de bruscas transições tonais, onde as cores parecem
batidas por pinceladas soltas e interrompidas. Poussin com as suas
criações mitológicas, cuja imaginação romântica animou o espaço por
visões exaltadoras, o que levou alguém \emph{a} dizer que a sua pintura
não é um espelho da natureza, mas a afirmação do espírito de um homem
como as \emph{Meditações} de Descartes ou as cantadas de Bach; assim,
Matisse, Picasso, Portinari, nas suas criações representativas de
emoções e sentimentos do mundo da atualidade. Em suas composições
avançadas encontra-se a afirmação de espíritos também avançados, como no
\emph{Capital} de Marx ou nas sinfonias de Beethoven.

A arte não é apenas um manancial de prazeres; se o fosse, não valeria a
pena ocuparmo-nos seriamente dela. E todo o prazer, quando é legítimo,
está ligado a atividades mais profundas, mais úteis, mais construtivas,
mais edificantes. É, pois, também um aparelho de precisão para o
conhecimento e, pela sua penetração e maleabilidade, é a primeira que se
modifica frente à evolução das civilizações; ``é, de algum modo, o
sismógrafo da história''.

E a pintura moderna começava já a encontrar adeptos fervorosos no
Brasil. Esses, sem dúvida, seriam os pioneiros desta arte nova, cujo
valor inestimável o crítico apressado de jornal jamais poderia expor na
sua totalidade.

Terminado o seu comentário, foi levá-lo a Caio Neiva para a revisão.
Caio Neiva estava à sua mesa, concentrado como um faquir hindu.

--- Que é isso? --- perguntou Álvaro de longe --- está hipnotizando
algum mosquito?

Caio Neiva levantou os olhos:

--- Olá, rapaz, em que boa hora!

--- Boa hora, o quê! --- exclamou o outro. --- Vou para a cama; tenho
aula amanhã às oito.

--- Não seja por isso --- volveu o companheiro de bom humor --- você
pode sair daqui diretamente para a aula.

--- Não brinque assim, amigo; o sono é precioso para o homem. Tenho um
colega de faculdade que comprou um colchão de molas porque na cama,
segundo ele, passa-se três quartos da vida.

Caio Neiva deu uma gargalhada que ecoou pela sala e aproximando-se de
Álvaro, comentou:

--- Estava lendo o seu trabalho. Sim senhor, você é brasileiro de fato!

--- De nascimento e de coração --- tornou Álvaro.

--- Percebi isso desde as primeiras linhas. E o que eu aprecio no seu
trabalho, é o equilíbrio de quando fala no Brasil. Não o coloca como uma
joia preciosa; como um país de maravilhas, onde tudo é riqueza e
encanto. Por outro lado, não ressalta apenas as más qualidades do nosso
povo.

--- Birau, ao citar os defeitos do povo francês, fazia-o pelo desejo de
aperfeiçoamento dos homens da sua pátria. Esta é a minha intenção,
Neiva.

--- Um trabalho como o seu, eu o chamaria de realista --- tornou o
revisor do jornal. --- Não se prende a teorias imaginárias, e não se
delicia com o lado apodrecido das coisas. Entra pelos fatos, unindo a
narrativa à reflexão. É isto mesmo --- continuou sorrindo --- no seu
trabalho não se encontra o Brasil maravilhoso, cujas árvores nativas dão
frutos especialmente para os homens; e nem as minas preciosas, cujas
jazidas guardam joias prontas ou aço preparado. E agora, meu amigo ---
acentuou com malícia --- você que apregoa um Brasil grande pelo trabalho
honesto do brasileiro... Você acha mesmo, que, no Brasil, vence o
honesto trabalhador?

Álvaro sorriu:

--- Em que capítulo está?

--- No terceiro.

--- Logo adiante, você encontrará resposta à sua pergunta.

--- Você, não concordaria comigo, de que no Brasil vence o trabalhador
esperto? Nós ainda não chegamos ao ponto da justa valorização do
trabalho; a esperteza \emph{ainda} precede a competência --- e o rapaz
piscou maliciosamente.

--- Bem, meu amigo --- tornou Álvaro --- veja o que está mais à frente,
e depois venha fazer-me as suas considerações. Gostaria que opinasse
sobre o meu trabalho, pois, você é um homem que pensa por conta própria
e tem lá as suas experiências da vida.

Caio Neiva não se perturbou com o elogio, embora se mostrasse satisfeito
pelas palavras no amigo. Logo mais, Álvaro despediu-se.

Era já madrugada. Chegando à pensão, o rapaz dormiu ainda as três horas
de que dispunha. De manhã, levantou-se da cama com bastante dificuldade;
parecia embriagado pelo sono. Foi à faculdade e esforçou-se por
acompanhar as aulas e arranjar \emph{os} seus pontos de exame. Quando
regressou para o almoço, sob o sol do meio dia, o cansaço físico e
mental o perturbava muito. Não sentiu vontade de alimentar-se; embora
estivesse à mesa, mal tocou na refeição. Bebeu um copo de água, uma
xicrinha de café e levantou-se a seguir. Tinha feito planos de estudo,
mas o seu estado de fadiga não permitiria executá-los. Sentiu-se
descontrolado e, como sempre nessas horas, dirigiu-se à casa de Bárbara.

Santa Tereza apresentava o seu aspecto calmo de costume. Álvaro desceu
do bonde e veio caminhando devagar a quadra que lhe restava.
Aproximou-se da casa branca de esquina. Vendo ao portão a linda barata
avermelhada, reconheceu-a, e passou adiante. Lembrou-se perfeitamente da
única vez em que a vira. Conduzia-a um rapaz alegre e jovial, como
Álvaro o fora no passado; Bárbara, junto dele, distraíra-se tanto das
coisas que passara por Álvaro sem o ver. Na primeira vez, sentiu-se
desapontado, vendo-a sair com o outro; mas, agora, um choque nervoso e
violento agitou-o, ampliando o seu descontrole. Seu coração parecia
comprimido por uma carga compacta e pesada: uma dor aguda, profunda, no
lado esquerdo, parecia a reação do órgão que continuava a trabalhar.

Álvaro dirigiu-se ao bar da esquina e comprou cigarros. Estava ainda
muito cansado para avaliar o seu abalo moral. Esperou algum tempo e
vendo que a, barata permanecia abandonada, ficou ali a contemplá-la de
longe. No íntimo, sentia um vácuo, uma incerteza... parecia estar
alimentando alguma coisa falsa; todavia, era-lhe impossível afastar-se
ou aproximar-se. Permaneceu nesta atitude de falsa contemplação, que
tanto mal causa às pessoas. Pôs-se a fumar sem interrupção, e no seu
espírito torturado travou-se dolorosa luta.

A força da desgraça foi vencendo a da fadiga e Álvaro, parecendo
reanimar-se, apenas entregou-se à tortura com maior intensidade. Chegara
a pensar com certeza que Bárbara o amava; mas, se o amava, não teria
perdido o direito de encontrar-se com o outro? Bárbara nunca lhe falara
desse outro e Álvaro, confiante no seu amor, confiou também na sua
lealdade. Bem dissera Castilho na sua versão de Misantropo.

\emph{"A dúvida a quem ama é dor de tal braveza}

\emph{que excede à própria dor da última certeza."}

E diante da dúvida de um minuto, esqueceu-se de um passado de dez meses.
A destruição é sempre mais forte. Desnorteado, Álvaro apagou a certeza
do passado para entregar-se a um presente que nem ele compreendia ainda.
Queria pensar, adquirir a certeza de uma conclusão justa; mas, uma coisa
era encaminhar ideias nos livros, considerando problemas gerais; outra,
bem diferente, era sob o impulso da paixão.

O rapaz acendeu novo cigarro e afastou-se. Andar, espairecer... era
preciso fazer alguma coisa. Saiu pelo bairro sem direção determinada.
Andou ao acaso; contornou várias ruas, tomou um cálice de ``Porto'' num
armazém do largo e, sem o sentir, regressou ao ponto de partida. Perdera
a noção do tempo; aquela espera pareceu-lhe uma eternidade.

Encostou-se a uma árvore próxima, e nunca o crepúsculo lhe parecera tão
triste. Era alguma coisa que atingia o auge da beleza, para depois
morrer aos poucos, todos os dias; alguma coisa que Álvaro não encontrava
definitivamente. Depois, viu abrir-se a porta do terraço; sair o mesmo
rapaz que, embora tivesse visto uma vez, reconheceu de imediato. Bárbara
acompanhava-o sorrindo; parecia satisfeita como há muito Álvaro não a
vira. E para transbordar a taça, contemplou de longe a cena da
despedida.

Quando Carlito beijou impetuosamente a mão de Bárbara, Álvaro cerrou as
mãos com força, como se estivesse a quebrar os laços que uniam Bárbara
ao outro rapaz. Não sentiu a brasa do cigarro a lhe queimar os dedos
rijos e contraídos; por um momento parecia ter enlouquecido!
Atravessou-lhe o ímpeto de partir alguma coisa, arrebentar, torturar os
outros como ele se torturava agora.

Viu Bárbara encaminhar-se novamente para a casa. E quando ela se voltou
no jardim, quis ir ao seu encontro; mas uma força demoníaca reteve-o
preso, na contemplação de sua própria amargura. No paroxismo da sua dor,
e terminado o espetáculo que tanto o abalou, Álvaro sentiu-se prostrado
como se de repente lhe faltassem as energias.

Desencostou-se da árvore e, com as mãos no bolso, desceu a pé para a
cidade. Andava devagar, com passos lentos e pouco firmes; mas, sua
imaginação doentia trabalhava ainda. Ele não era livre; Bárbara era moça
e bonita. Nada mais natural que se prendesse a outros, que se deixasse
admirar, e, até mesmo estivesse com algum casamento resolvido. Mas, se
assim era, como permitia continuar aquele estado de coisas entre eles?
Era... era... tudo aparecia na sua mente no tempo imperfeito; sentia uma
incerteza terrível em torno das coisas. De repente, parou na rua e
perguntou para si mesmo:

--- Mas o que que é?

Queria, desejava resolver os fatos pelo presente ou pelo perfeito. Mas a
cena de há pouco derrubou tudo o que foi entre eles. E a incerteza de
Álvaro tecia lembranças no imperfeito, como ações inacabadas. Era a
autodefesa, esquivando-se, num gesto inconsciente, a conclusões erradas.
O rapaz, porém, estava perturbado e, nessa perturbação, duvidou de si
próprio.

Na tela reprodutiva da imaginação, reviu o outro na sua barata
avermelhada. Por uma associação de ideias, voltou-se para o passado;
quando ele também tinha uma barata e era livre de compromissos. Ah! ---
pensou com amargura, por que não encontrara Bárbara naquela época? Era
amigo de Paulo, e Bárbara era amiga de Helena. Tudo se dera de forma tão
ignorada e eles só foram se encontrar quando tudo estava perdido.

E se encontrasse Bárbara no passado, tê-la-ia notado? Poderia ser
atraído pela beleza da moça; mas, e o espírito? O espírito não estaria à
altura do Álvaro do passado. A Bárbara de agora era a Bárbara para ele.

Sentiu um desejo imenso de regressar, de voltar para ela. Contudo, o seu
demônio interior insistiu nas lembranças más. Álvaro não era livre; como
poderia dar-se a Bárbara?! Como oferecer-lhe um amor que não encontraria
acolhida perante as leis do país? Se pudesse sair do Brasil, ir para
algum lugar onde se reconhecesse o direito do verdadeiro amor? Outros
povos, espiritualmente elevados, de princípios firmes e hábitos
honestos, concediam-se a si mesmos o direito de reparar um erro da
mocidade; portanto, o que ele desejava era um ambiente sadio, onde
pudesse oferecer, sem receio, um amor que lhe enchia a alma e passara a
fazer parte da sua própria vida. E por não poder viver ao lado da mulher
que amava, imiscuía-se com muitas outras, imbuído pela naturalidade da
educação viciada e comum, que a sua sociedade cultivava. Propor a
Bárbara que o acompanhasse e formar com ela um novo lar, seria um
ultraje à moral; mas sair à procura de outras mulheres, lhes explorando
os corpos e os sentimentos, seria apenas a satisfação da carne, muito
natural, tolerada e até oficializada por essa sociedade, constituída
juiz diante dos bons costumes. Lembrou-se, então, de Birau, o sociólogo
francês, que colocara a capital brasileira entre os três maiores centros
da escravidão branca.

Álvaro continuou andando; uma série de pensamentos procedeu-se na sua
mente. Certa hora, parou na rua, outra vez, e perguntou-se:

--- Teria eu coragem para falar a Bárbara no futuro? Propor um casamento
no México, seria absurdo, uma vez que o meu dinheiro não chega para
isso...

Mas recordou-se de como sentira-se intimidado para falar a Bárbara de
uma união entre eles. Nunca lhe tocara no assunto. E nisto, uma ideia
luminosa --- ela não era inglesa? Na Inglaterra não existia o divórcio?
Poderiam tentar uma solução, entretanto... por que pensara nisso somente
agora? O outro rapaz não seria um compromisso de Bárbara?

Álvaro não queria pensar mais; todavia, não conseguia furtar-se ao não
mais pensar. Rememorou a deliciosa camaradagem que existia entre eles;
só então percebeu a falta de uma intimidade completa que lhe permitisse
ponderar as suas dificuldades amorosas. Sem saber por que, lembrou-se
das crianças doentes que fora ver na companhia da moça. Custara-lhe crer
nos seus olhos quando presenciara a cena da choupana. Reproduziu-se na
sua imaginação aquele ato de caridade em que Bárbara, no seu vestido
simples de linho branco, ainda sobressaía mais na imundície do casebre.
Era realmente uma alma caridosa. Fora a primeira pessoa que encontrara
que compreendia a caridade. Os que lhe recebessem o auxílio, jamais se
humilhariam diante dela como o mendigo diante da sociedade. Pôs-se a
pensar nas miseráveis crianças, quando uma ideia extravagante lhe
atravessou o espírito. Não estaria Bárbara sendo caridosa para com ele?
Na maneira como o tratava, não estaria havendo um gesto semelhante ao
que tivera para com as crianças? Não se expusera ela ao ambiente infecto
da cabana para alimentar a vida dos pequeninos, como se expunha perante
a sociedade para restituir a Álvaro a confiança de viver? Qual seria o
sentimento a impulsioná-la para ele --- o amor, ou a caridade? Não
quisera nunca, ter inspirado um sentimento maternal a uma mulher a quem
amava com todas as forças de seus vinte e nove anos!

Deprimido pela fadiga mental e física teve a ilusão de sentir, naquela
hora o fim de todas as lutas a que a vida \emph{o} expusera. Passou a
mão trêmula pelo cabelo em desalinho e, perturbado no seu íntimo,
renunciou Bárbara para sempre. Achava-se nesse estado de espírito em que
o homem, vencido pelas forças contrárias, chega a uma falsa abdicação de
seus direitos e sentimentos.

\chapter{Capítulo 88}

Quando Álvaro regressou à pensão, já havia passado a hora do jantar.
Como não lhe viesse à mente a ideia de comer alguma coisa, o rapaz foi
direito ao quarto. Acendeu a luz. O quarto amplo e arejado apresentou as
condições agradáveis que ele tanto apreciara desde a sua mudança.
Entretanto, Álvaro estava esquecido das comodidades conquistadas.
Atirou-se à cama e continuou a fumar sem interrupção. Bom companheiro é
o cigarro, pensou, vendo a fumaça subir. E, observando-a, notou como
subia em linhas firmes, embora o cigarro se agitasse, inseguro, nas suas
mãos trêmulas. O seu sistema nervoso perdera o ritmo de trabalho;
deixara de ser um sistema para tomar-se um amontoado de nervos. A
atividade na redação, a noite mal dormida, o esforço intelectual, na
faculdade, a escassa alimentação daquele dia, foram coroados por um
abalo psíquico que o desnorteou em face das coisas.

Era já bem tarde, quando suas pálpebras pesadas foram cedendo ao sono.
Duas horas depois, Álvaro parecia dormir profundamente. Mas, quando o
homem não sente o descanso em plena consciência de suas atividades, na
inconsciência dá-se o mesmo. A separação do mundo exterior deu livre
curso aos seus instintos; não havia forças em reserva para os reprimir.
Tal como Narciso, o adormecido voltou-se para o próprio eu. E sonhando,
Bárbara apareceu-lhe. Álvaro estava longe de perceber que Bárbara saíra
dele mesmo. Viu-se na praia, como há dias passados; chegou a reconhecer
o maiô vermelho e o sapatinho de cordas que a moça usara em sua
companhia. Entretanto, o que lhe parecia claro a princípio, foi aos
poucos tornando-se confuso. Estava ainda, na areia, quando viu Bárbara
nadando em lugar fundo e perigoso. Dirigiu-se para ela; mas, antes que
se aproximasse, viu-a submergir entre as águas. Tomado pelo desespero,
Álvaro sentiu-se preso ao lugar, como se tivesse os pés enterrados no
fundo. Custou-lhe vencer os músculos enrijecidos e, quando o conseguiu,
pôs-se a mergulhar numa busca aflitiva e nadar em direções diversas.
Estava já a ponto de perder o fôlego pelo esforço; então, viu Bárbara na
praia, em trajes de passeio e olhando para ele sem compreender. Álvaro
foi ao seu encontro e mal tocou terra firme, já a tinha perdido de vista
novamente --- Bárbara nunca estava onde Álvaro, em sonho, procurava
encontrá-la.

Ao afastar-se dali, percebeu que chamavam por ele; voltou-se. Deparou
com uma moça de branco, cuja tristeza não lhe passou despercebida.
Sentiu-se constrangido, tomado de uma vergonha inexplicável; quis ver-se
livre da mulher desconhecida. Ela disse alguma coisa que o rapaz não
conseguia compreender; no íntimo, conhecia o seu desejo de não a
compreender. E ao sair, precipitado, tropeçou no paredão da praia.
Estremeceu na cama e acordou.

Olhou as horas; o mostrador luminoso indicava cinco. Álvaro estava ainda
de roupa, como chegara de Santa Tereza. Pegou o cigarro de sobre o
criado-mudo e acendeu-o; tinha adquirido o hábito de fumar no escuro.
Enquanto fumava, as cenas do sonho se compunham na sua mente. Quem era a
mulher de branco? Por que lhe falara, na hora em que procurava por
Bárbara? E Bárbara, por que estaria na praia quando a procurava com
loucura entre as águas?

Álvaro levantou-se e pareceu ainda mais cansado. Mudou as roupas com
vagar; olhou para a rua deserta àquela hora da madrugada e tornou a
deitar-se. Era preciso descansar. Os exames estavam a sua frente;
precisava preparar-se. Fazia matérias de outras séries, pois, as
faculdades não se ajustavam na distribuição das disciplinas. E, embora
desejasse encaminhar seus pensamentos, Álvaro terminava regressando ao
ponto de partida. Bárbara... Bárbara... sempre Bárbara!

Que se passara na sua vida para se ver, repentinamente, tão ligado a uma
mulher!! Passou a reconstituir os fatos, como se neles encontrasse a
solução. Lembrou-se perfeitamente da sua indiferença ao separar-se da
ex-esposa. Como se mantivera alheio à mulher com que participara a vida;
a sensação de liberdade após o seu desquite. O fato de não mais possuir
um lar, como compromisso público, foi-lhe o bálsamo refrigerante naquela
onda de desgraças contínuas. Na verdade, Álvaro nada recebia desse lar,
a não ser o veneno diário que, gota a gota, lhe destruía a alma.
Lembrava-se, com amargura, dos últimos dias do mês, em que era obrigado
a saldar um número incontável de despesas esparsas. A família de Dalila
vivia também do seu dinheiro. As contas das costureiras, bordadeiras,
peleterias, joalherias iam além dos seus rendimentos; sem contar, ainda,
a manutenção do luxuoso apartamento de morada com a sua criadagem
excessiva, e as estações de veraneio, facilitadas pelo belo ``Packard''
que os servia. Começou a chegar o tempo em que essas despesas lhe
pesavam e Álvaro foi empenhando os seus objetos pessoais, enquanto
Dalila conservava os bens de casal como bens de família.

Naquele tempo tudo isso lhe parecera indispensável e hoje, que coisa
estranha, tudo era supérfluo! Desligara-se desses aparatos mais depressa
do que supunha. Tais necessidades, ele as criara em sua mente,
determinado pela posição social daquela época. Era verdade que apreciava
muitos objetos que possuíram; havia algumas raridades, cuja arte o
impressionara. Eram objetos adquiridos pela esposa, mais pela mania das
coisas caras que pelo valor real de cada um deles. Dalila não possuía
gosto artístico e, não fora o dinheiro a permitir-lhe extravagâncias,
não distinguiria um trabalho de Saxe de uma cerâmica americana. Suas
joias eram grandes e vistosas; impressionava-a mais o tamanho de um
brilhante que o trabalho fino e delicado de uma joalheria francesa.

Que distância enorme os separava! E entre eles não existia o amor,
eterno bálsamo a amenizar os defeitos pessoais de cada um. Quanto mais
viviam juntos, mais se distanciavam um do outro! Para ele, como homem, a
esposa não se destacara entre as demais mulheres que, na sua amoralidade
inconsciente, experimentara na vida livre, a não ser pela continuidade e
o compromisso do seu sustento material.

Álvaro conheceu depois os seus sentimentos. Conhecendo-os, encarou-os
mais serenamente porque começara a transpor o caminho da verdade. E
quando, finalmente, adquiriu a consciência dos fatos, foi-lhe
impossível, por respeito próprio, agir em direção diversa. E ao iniciar
a vida sozinho, a solidão nunca lhe fora tão agradável. A sua riqueza
interior transbordou libertada dos diques de seguranças sociais. Álvaro
pôde compreender, uma vez que compreender é sentir, a riqueza
imensurável que acompanha a alma dos filósofos, dos artistas e dos
cientistas.

Mas Álvaro não possuía a força de um Beethoven nem a tranquilidade de um
Sócrates. Embora cultivasse a vida do espírito, não pôde acalmar a do
sexo. Por esse lado, então, resolveu --- iria viver a vida como era
costume por aí, e isento dos compromissos de um lar. Correria o risco de
apodrecer a alma; e, se apodrecesse, perderia todo o tesouro há tão
pouco descoberto. Lutou a princípio. O que tanto temia, porém, era um
fenômeno normal da sociedade; e, em breve, passou a sê-lo para ele
também. Conhecera toda sorte de mulheres. Desse conhecimento resultou
uma sensível indiferença para com elas. Faltara-lhe o amor materno; a
mulher que o trouxera ao mundo, pagara esse tributo com preço da própria
vida. Nem o embalo de uma mãe-preta o fizera sentir a doçura de uma alma
feminina. E quando seduzido pelos encantos de uma moça ambiciosa,
experimentou os arroubos da paixão; nunca, porém, a ternura delicada de
uma companheira.

Após essa via-sacra de experiências e conhecimentos, Álvaro imaginou-se
inatingível; supunha ter perdido para sempre a faculdade de querer bem.
Julgara-se incapaz de amar, como outrora o percebera em Camilo. Embora
contrário à ética do romancista português, considerava esta faculdade
sentimental um traço de união entre suas almas diferentes. A vacina
contra sentimento, porém, não se conserva em ampolas de vidro; o rapaz,
ao julgar-se imunizado, fora atingido ``in totum'' pelos micróbios do
cupido. E os fatos de um passado remoto, tão indiferentes em outros
tempos, revestiam-se de um sabor amargo como se o seu calvário começasse
agora.

Álvaro levantou-se precipitado. Chega. Não quero mais recordar, gritou
com raiva no seu íntimo. Tocou a campainha para o aviso do café e
escancarou as janelas com força. Uma luz intensa invadiu o quarto, e
Álvaro desejou abrir a própria alma para sentir os efeitos daquela luz.

Baterem à porta; logo depois o empregado entrava no quarto com a
primeira refeição do dia e a correspondência da véspera. Entre os
jornais havia uma carta. Álvaro, pela primeira vez em vinte e quatro
horas, desviou a atenção de seus problemas amorosos para ler as notícias
de uma carta. Abriu e procurou logo o signatário: Samuel Dalha. Um tanto
desapontado pelo nome desconhecido, passou ao conteúdo.

\emph{Prezado amigo,}

\emph{Por intermédio do sr. X... recebemos a importância de 500\$000
(quinhentos mil-réis) para a ornamentação do jazigo de} sua
\emph{família.}

\emph{Temos} \emph{em} \emph{grande apreço o satisfazê-lo plenamente; e
esperamos que o amigo, nosso distinto conterrâneo} e \emph{famoso
escritor, continue nos dando a honra da sua confiança e preferência.}

\emph{Inteiramente ao seu dispor}

\begin{quote}
Samuel Dalha.
\end{quote}

Álvaro iniciou o café com um sorriso de zombaria nos lábios. Lembrou-se
de Samuel Dalha, o dono da floricultura cuja loja ficava na rua
principal da sua terra. Samuel Dalha é que não se lembrava dele, quando
o vira malvestido e, provavelmente, sem um vintém no bolso, na avenida
central do Rio de Janeiro. Usava ainda o mesmo terno marrom de casimira
grossa com que conhecera Bárbara e, tantas vezes, a acompanhara nos
passeios; até mesmo num concerto no municipal. Agora, porque dispendia
quinhentos mil-réis na ornamentação de um túmulo, o esperançoso
floricultor supunha outras reservas de cifras elevadas. Estava longe de
pensar que a ação de Álvaro fora motivada antes por uma independência
psíquica que monetária. O rapaz que dispendera quinhentos mil-réis na
ornamentação de um túmulo, não tinha reserva alguma em dinheiro.

Por uma associação de ideias, lembrou-se de um seu conterrâneo, cujo
casamento os outros diziam ter sido por dinheiro. Contava-se que o rapaz
escolhera a noiva no cemitério. Anotara os jazigos de maior preço; e
saíra à procura de uma moça casadoira que estivesse ligada à frieza
daquelas pedras, mas, também ao calor das marmorarias raras. Por fim
encontrara o seu objetivo, embora os sogros vivessem ainda. Concluíra,
por fórmulas matemáticas, que o capital morto do cemitério deveria ser a
sobra de outros bens com raízes mais profundas. A matemática não o
enganara. A riqueza existia e na mesma abstração, para ele, que as leis
da ciência que o guiara. Ambas se achavam, pois, fora de alcance
material para o ambicioso rapaz. Os antigos colegas costumavam chamá-lo
de "o milionário abstrato. Álvaro riu dessas lembranças e pensou --- ai
de quem me fosse escolher pelo jazigo de família ou por quinhentos
mil-réis de flores num dia de finados!

Terminado o café, voltou a ler a carta --- famoso escritor. A sua fama
havia chegado ao comerciante de flores de uma cidade longínqua do
interior. Publicara apenas um livro e já o diziam famoso; quando
aparecesse o segundo, como o adjetivariam então? Isto, a não ser que o
bom Samuel Dalha houvesse tomado o partido de São Tomás de Aquino ---
``Timeo hominem unius libri!''

\chapter{Capítulo 89}

Álvaro conseguira vencer os dias sem retornar a Santa Tereza. A cena do
portão ficara-lhe na mente; e nela o rapaz alimentava a sua decisão. Era
preciso excluir Bárbara da sua vida; neste sentido ele se esforçava.
Pelo muito que a amou e amava ainda, não seria motivo justo para uma
renúncia por parte de Bárbara. Custara a Álvaro compreender todas essas
coisas; o seu egoísmo gritava e a abdicação parecia não caber na sua
natureza humana. A ligação entre eles, porém, fora construída em bases
leais. Por essa lealdade, o direito de não mais se quererem existiria
sempre para qualquer uma das partes. Preferia perder a mulher, a viver
das sobras de seus sentimentos; a compaixão jamais encontraria ali o seu
lugar. Álvaro não chegara, ainda, à estupidez de aceitar um sacrifício,
mesmo em pensamento, no qual estivesse incluída a alma da mulher que
amava. A distorção sentimental era para ele um crime como o roubo ou o
assassinato. A alma era o elemento livre que os homens possuíam;
subordiná-la, seria o crime mais hediondo que a justiça humana poderia
conhecer.

Porque sabia de todas essas coisas, Álvaro tentava orientar-se noutro
sentido; embora lhe chegasse algumas vezes a dúvida de que Bárbara não o
amasse.

Naquela manhã, levantara-se à hora do costume e dirigia-se para a
faculdade. Atravessou as ruas movimentadas e finalmente chegou ao seu
destino.

Os alunos mostravam-se despreocupados com os exames do dia; contavam
certo com a aprovação. Colegas que há muito Álvaro não via,
aproximaram-se dele. Alguns tinham lido o seu trabalho, e a publicação
do seu segundo livro despertara comentários sobre a família e o
divórcio. Embora alguns estudantes quisessem orientar a palestra para
ataques pessoais, Álvaro conseguiu sobrepor-se a eles, fazendo triunfar
a ideia. Insistiu com os colegas para que, numa apresentação de
problemas gerais, não se detivessem nos problemas particulares, ou na
vida difícil ou fácil de quem os apresentava. À história, e a própria
humanidade, estariam sempre ao alcance dos bem-intencionados para o
contexto das ideias em questão. Discutiam acaloradamente quando o sinal
chamou os estudantes para as salas de aula.

Foi sorteado o ponto de Filosofia do Direito:

\emph{O problema da culpabilidade na guerra constitui um problema da
Ética? Por quê?}

\emph{Em um tribunal internacional como argumentaria o examinando o
direito da guerra?}

\emph{Quais as três atitudes do nosso espírito em face da guerra? Citar
e desenvolver.}

Antes de terem início as provas, o professor da cadeira dirigiu-se aos
alunos:

--- Gostaria de lembrar aos senhores as palavras de Gustav Radbruch: ``É
precisamente ao jurista que ficaria pior resignar-se perante a guerra,
como perante uma fatalidade inevitável''. Podem incluir na tese as
guerras religiosas se quiserem; como o problema religioso, porém, é
bastante desconhecido no Brasil, torna-se aconselhável prudência na
exploração do terreno.

O professor correu o olhar pela sala, verificou as horas e disse:

--- Podem começar. Às onze horas e vinte minutos, as provas deverão ser
entregues.

Álvaro iniciou as questões, satisfeito pela matéria sorteada. De
repente, lembrou-se dos colegas de veraneio; que estariam dizendo eles
diante de tais perguntas?

O exame correu normalmente, embora as ``colas'' trabalhassem com
eficiência. O próprio Álvaro não se negou a fornecer dados auxiliares
com referência às questões. O colega ao lado, um judeu de cabelo
vermelho, escrevia sem cessar. Por duas ou três vezes, Álvaro viu a
prova do colega e admirou-se pela semelhança com a sua. Embora intrigado
pelo fato, continuou o desenvolvimento da tese, citando as guerras
religiosas do ocidente. O judeu fizera o mesmo. Não podendo conter-se
por mais tempo, perguntou baixinho ao outro:

--- Está certo isso?

--- Não sei --- volveu o rapaz muito sério --- estou copiando tudo de
você.

O primeiro movimento de Álvaro foi abrir-se numa risada franca; mas a
ocasião não lhe permitia liberdades. Conteve-se; e olhando para o
colega, acrescentou:

--- Mude um pouco as coisas; não siga as minhas ideias como se fossem
doutrinas.

--- Aqui não se trata de ideias nem de doutrinas, meu caro, mas de
exames apenas.

Álvaro continuou a sua exposição. O colega da frente parecia sossegado
quanto a informes; estava absorvido nas últimas notícias que lhe
chegaram. Aquele era conhecido como o artista da ``cola''; uma só frase
assoprada, resultava uma página inteira da questão. Diziam até que, pela
sua esperteza, conseguira fazer provas clandestinas na secretaria da
escola. Os estudantes mostravam-se alarmados diante de um colega sem
conhecimento e com notas tão altas.

Embora interrompido pelos sinais de S. O. S. que lhe vinham de todos os
lados, Álvaro terminou a prova satisfeito com os resultados. Logo
depois, bateram o sinal de entrega. Os examinandos deixaram a sala,
comentando os incidentes de exame.

Lá fora estava um número incontável de rapazes. Formaram-se as rodinhas
costumeiras: enquanto uns discutiam as perguntas, outros contavam
piadas. Por outro lado, ouviam-se as risadas sonoras da turma jovem
dispersada por ali. De repente, Álvaro estremeceu. Acabava de ver o
rapaz da barata vermelha.

--- Olá Carlito --- gritou um dos moços --- que faz você aqui?

--- Venho à procura de um amigo; ele me prometeu vários pontos.

--- De que ano? --- indagou o outro.

--- Do segundo --- informou.

--- Ah... e por falar em pontos, você é um Casanova, meu caro. Em três
vezes que o encontro, três mulheres diferentes. Por sinal que a morena é
um monumento! --- exclamou o rapaz com entusiasmo.

--- Nem me diga! E por falar nisso, você já conhece o ``If''?

--- Não. O que é isso?

--- Ela diz que é uma poesia.

--- Poesia? Pois olhe, está aqui o Álvaro que poderá ajudá-lo. Ele
entende dessas coisas.

Álvaro sentiu-se pregado ao lugar; mal pôde balbuciar uma palavra.
Conhecia a preferência de Bárbara pela poesia de Rudyard Kipling e
juntos tinham comparado as traduções em língua portuguesa. Carlito
aproximou-se e perguntou:

--- Não seria possível copiá-la para mim? Dou-lhe o meu endereço; muito
grato ficaria se pudesse me mandar.

--- Eu sei de cor --- conseguiu articular Álvaro com palavras duras.

--- Que foi isso? --- perguntou o conhecido de Carlito. --- Você foi mal
nas provas, com certeza. Não se incomode; no fim, sempre dá para o
diploma.

Notando a palidez de Álvaro e a respiração ofegante, o rapaz disse
ainda:

--- E a prova foi um espeto! O professor insistiu num só assunto. Devido
à atual situação europeia, achou oportuna a questão sorteada e não saiu
dali. Guerra, guerra, só guerra. Quem não conhecia a coisa, devia ter-se
engasgado a valer. E este diabo não cola --- falou apontando para
Álvaro.

Ficaram um minuto em silêncio. Carlito renovou o seu pedido.

--- Tome nota --- respondeu Álvaro --- vou dizê-la toda.

Encostou-se à parede e, com as mãos no bolso, começou:

\emph{``SE''}

\emph{Se és capaz de manter a tua calma quando}

\emph{Todo o mundo ao redor já a perdeu e te culpa;}

\emph{De crer em ti quando todos estão duvidando;}

\emph{E para esses no entanto achar uma desculpa.}

\emph{Se és capaz de esperar} sem \emph{te desesperares}

\emph{Ou enganado, não mentir ao mentiroso}

\emph{Ou sendo odiado sempre ao ódio te esquivares,}

\emph{E não parecer bom demais nem pretensioso;}

\emph{Se és capaz de pensar} --- \emph{sem que a isso só te atires;}

\emph{De sonhar} --- \emph{sem fazeres dos sonhos teus senhores}

\emph{Se encontrando a Desgraça e o Triunfo conseguires}

\emph{Tratar da mesma forma a esses dois impostores}

\emph{Se és capaz de sofrer a dor de ver mudadas}

\emph{Em armadilhas as verdades que disseste,}

\emph{E as coisas por que deste a vida, estraçalhadas,}

\emph{E refazê-las com o bem pouco que te reste.}

\emph{Se és capaz de arriscar numa única parada}

\emph{Tudo quanto ganhaste em toda a tua vida}

\emph{E perder, e ao perder, sem nunca dizer nada}

\emph{Resignado, tornar ao ponto de partida,}

\emph{De forçar o coração, nervos, músculos, tudo,}

\emph{A dar seja o que for que neles} \emph{ainda existe}

\emph{E a partir assim quando, exaustos, contudo}

\emph{Resta a vontade em ti que ainda ordena:} "\emph{Persiste".}

\emph{Se és capaz de entre a plebe, não te corromperes}

\emph{E entre reis não perder a naturalidade,}

\emph{E de amigos, quer bons quer maus, te defenderes}

\emph{Se a todo}s \emph{podes ser de alguma utilidade}

\emph{E} se \emph{és capaz de dar, segundo por segundo,}

\emph{Ao minuto fatal todo o valor e brilho}

\emph{Tua é a terra com tudo o que existe no mundo}

\emph{E} --- \emph{o que mais} --- \emph{tu serás um homem, ó meu
filho!}

Entre confuso e admirado, Carlito anotou os versos. Álvaro recitou-os
sem um comentário sequer, embora conhecesse os da preferência de
Bárbara. Carlito estendeu-lhe a mão e agradeceu; depois, retirou-se sem
mesmo lhe perguntar o nome.

Sob a influência da tristeza íntima que o consumia, Álvaro deixou a
Faculdade.

\chapter{Capítulo 90}

As manhãs de dezembro estavam esplêndidas. Bárbara levantou-se muito
cedo e foi dar a volta costumeira pelo bairro. Quando atravessou o
largo, uma mulher aproximou-se dela, chamando-a pelo nome. Voltou-se
admirada e reconheceu a mãe das duas crianças doentes que fora visitar.

--- Como vão os meninos? --- indagou com interesse.

--- O Antônio, que estava melhor, a senhora lembra? Esse morreu. O outro
estava muito mais doente e ainda está vivo.

--- E melhorou?

--- Mais ou menos --- respondeu a mulher desanimada.

--- O que a senhora tem feito para ajudar a cura?

--- Fiz um chá de musgo. E ontem levei o menino para um homem; mas a
coisa não deu resultado.

--- Que homem?

--- Um homem que ia tirar o encosto do coitadinho; mas não adiantou
porque a coisa ruim está em casa.

Bárbara compreendeu logo do que se tratava e prontificou-se a acompanhar
a mulher. Novamente, fizeram juntas aquele caminho difícil que dava
acesso à cabana. A mulher abriu a porta, presa apenas por um arame
grosso, e logo apareceu a esteira com a criança no chão. Bárbara
aproximou-se e, teve um mau pressentimento; o menino estava inerte.
Tomou-lhe o pulso e, não o encontrando, procurou ouvir o coração. Nada.
O menino estava morto. O seu corpinho quente ainda parecia animado do
calor da vida, mas Bárbara compreendeu que a criança deixara o mundo
para sempre. Olhou para a jovem mãe, um pouco distanciada e percebeu que
ela estava ao par de tudo, embora se conservasse num silêncio obstinado.
Bárbara procurou falar-lhe alguma coisa; alguma coisa que orientasse
aquela pobre criatura. E perguntou com voz séria, porém, afetuosa.

--- Por que a senhora não evita esses filhos?

Ela pareceu espantada pela pergunta e arregalou os olhos:

--- Pois a senhora não sabe? Isso é um pecado daqueles.

--- Por que é pecado?

--- Porque quem faz um erro, precisa pagar nesse mundo, senão, no outro
a coisa é muito pior.

--- Você sabe alguma coisa do outro mundo?

--- De mim, não; mas o vigário já contou uma porção,

--- Fique tranquila que ninguém sabe nada do outro mundo com esta
certeza que a senhora imagina.

--- Cruz credo --- disse a mulher. --- Tenho medo do que a senhora fala.
Essas coisas não presta dizer.

Bárbara sorriu complacente:

--- Ponha esse outro mundo de lado e pense em viver bem, agora.

--- Viver bem?! Se o homem aqui ganha pouco e eu morro de trabalhar.

--- Como veio parar nessa vida? --- indagou Bárbara com muito jeito.

Era difícil conceber aquela moça de pele bronzeada, com um filho mulato
e outro um pouco mais claro, vivendo pelos morros de Santa Tereza.

--- É o destino da gente --- respondeu conformada --- cada qual vive o
seu.

--- É verdade --- tornou Bárbara --- mas cada qual pode também melhorar.

--- Isso não é pra todos. O destino da gente não muda.

Ela olhou para a criança morta e disse com voz amarga:

--- A senhora pensa que eu me incomodo de ele ter morrido? Não. É melhor
levar o destino em baixo da terra do que em cima. A vida de pobre é uma
praga, é melhor desiludir logo de uma vez.

Bárbara fitou demoradamente a moça e insistiu no mesmo assunto:

--- Desde quando mora aqui no morro?

--- Desde que deixei o emprego.

--- Você trabalhava antes?

--- Antes e agora; a minha vida é de trabalho pesado, o que a senhora
pensa?

--- Eu sei que trabalha agora; mas, e antes, onde estava?

--- Numa casa de família. Depois que eu desmanchei com o meu noivo, tive
de sair de lá.

Falou numa atitude pensativa. Bárbara compreendeu que a sua queda
provinha dessa época. Como se calasse a seguir, Bárbara dirigiu-lhe a
palavra:

--- Quando a vida estiver muito difícil, pode ir ao armazém do largo;
ele é o meu fornecedor e eu darei ordem para uma, retirada de 100\$000
em víveres, todos os meses.

--- Eu ainda não estou pedindo esmola --- tornou a mulher com voz
nervosa.

--- Eu sei --- volveu Bárbara --- mas, neste mundo, todos nós somos
irmãos.

--- Irmãos? --- repetiu na sua amargura incontida. --- Irmãos que morrem
do trabalho e irmãos que nadam em dinheiro!? Eu não compreendo isso. E a
senhora pode ver, os que trabalham no duro, não são os que têm dinheiro.

--- É verdade --- concordou. --- Eu não trabalho e tenho do que viver.
Por isso mesmo que gostaria de ajudar a senhora que trabalha tanto!

Bárbara deixou ali o dinheiro que por acaso trouxera no bolso. Antes de
sair, olhou uma vez mais para o corpo sem vida que jazia na esteira, e
contemplou aquela figura que mais parecia um farrapo de mulher.
Vieram-lhe à mente os versos de Guerra Junqueiro:

\emph{O horizonte é infinito e o olhar humano é estreito}

\emph{Creio que Deus é eterno e que a alma é imortal.}

\emph{Toda alma é clarão e todo corpo é lama.}

\emph{Quando a lama apodrece inda o clarão cintila:}

\emph{Tirai o corpo --- e fica uma língua de chama. . .}

\emph{Tirai a alma} \emph{---} \emph{e resta um fragmento de argila.}

Toda alma é clarão e todo corpo é lama, repetiu mentalmente Bárbara ao
afastar-se dali. E não fora ``o horizonte infinito e o olhar humano
estreito'', como se explicaria tanta miséria dentro de um só mundo?

* * *

Bárbara caminhava pensativa. Desceu as ruas que há pouco subira menos
preocupada; ao avistar de longe a sua casa, sentiu uma satisfação
diferente. Contemplou-a com prazer e disse para si --- tenho uma casa.
Uma casa que me traz o descanso e onde eu me sinto abrigada do sol e da
chuva. Teve desejo de elevar uma prece a Deus, agradecendo aquela dádiva
que, na verdade, não era o fruto do seu trabalho. Pensou que talvez
pudesse doar aos menos afortunados os bens que ainda possuía. --- Mas,
adiantaria isso alguma coisa? Não solucionaria o problema e se veria em
situação lamentável, como outras mulheres diante dela. Nada lhe restava
fazer, senão contribuir, em mensalidades, com uma parte ainda maior dos
seus rendimentos.

Nessas conjecturas, aproximou-se mais. Uma barata avermelhada passou,
com velocidade, indo estacionar à frente da sua casa Bárbara viu Carlito
tocar a campainha, falar com Mrs.~Patrice e, finalmente, retirou-se.
Poderia tê-lo alcançado, mas, naquele dia, não desejou a alegria de
Carlito. Andou devagar. A barata já havia desaparecido ao longe quando
Bárbara chegou ao portão. Parou ali como se desejasse contemplar a sua
morada; as flores do jardim, a cor branca da pintura, o terraço espaçoso
de frente, nunca lhe pareceram tão apreciáveis. Como era bela e
agradável a sua casa!

Alguém se aproximou por trás e a moça ouviu o seu nome quase num
murmúrio:

--- Bárbara\ldots{}

Voltou-se:

--- Álvaro!... É você, Álvaro? --- repetiu admirada.

O rapaz olhou para ela, como se pelo olhar pudesse transmitir-lhe a
alma.

--- Você desapareceu --- disse em tom de afetuosa censura. --- Há mais
de um mês que eu não o vejo.

--- E sentiu falta de mim, nesse um mês?

--- Muito, Álvaro! --- exclamou com espontaneidade.

Ele sorriu com os lábios e com a alma. Depois abriu o portão de ferro e
ambos entraram para a casa. Na sala de música, Bárbara ocupou a poltrona
de sempre, e Álvaro pôs-se a andar entre as peças. Como era agradável
voltar àquele ambiente de paz e serenidade. O próprio ar que se
respirava era mais puro e perfumado. Tudo estava como antes; a mesma
vista da janela, os móveis em seus lugares e Beethoven a fitá-la lá de
cima da estante.

--- Que maravilha!

--- O que, Álvaro?

--- Voltar...

Ele desejou dizer para você; mas conteve-se e veio sentar-se perto dela.

--- Onde você foi esta manhã? --- indagou o rapaz.

--- À casa de tabique lá no morro. Você lembra das crianças? Morreram as
duas, Álvaro.

--- Quando?

--- Uma, já faz tempo; a outra nesta manhã, enquanto a mãe não estava em
casa.

--- Foi melhor para elas --- volveu convicto.

---?...

Álvaro não se cansava de contemplá-la.

--- E depois? Você parecia tão pensativa na rua --- observou.

--- Você me viu?

--- Quando perto daqui. Não me aproximei logo, porque pensei que poderia
atrapalhar a sua visita.

Bárbara lhe percebeu a intenção e continuou impassível ante a disfarçada
referência a Carlito. Álvaro teve ímpetos de fazer-lhe perguntas a
respeito, mas receou alguma cena desagradável que viesse perturbar o
agradável de estarem juntos. Aproximou-se mais e, sem saber como,
descansou a sua mão sobre a de Bárbara. Apertou-a subitamente e ia
levá-la aos lábios quando lembrou novamente de Carlito. Perturbou-se um
instante e, olhando para Bárbara, segurou-a pelos braços num desejo de
unir-se a ela. Ouviram-se passos no corredor, e a governanta pediu
licença já à porta da sala. O rapaz afastou-se para a janela. Mrs.
Patrice entrou com a bandeja; colocou-a no aparador e retirou-se a
seguir.

Bárbara serviu o café. Álvaro sentiu a xícara a tremer-lhe na mão e,
voltando-se para a moça, falou, afetuoso:

--- Eu estava com saudades de você.

--- Eu também, Álvaro. Estou encantada pela sua visita.

--- Eu me senti muito só e magoado pela sua ausência, Bárbara. Você
esteve no centro no dia de finados?

--- Estive; por quê?

--- Viu as flores pisadas das calçadas? Eu me sentia assim, Bárbara.

--- Mas as flores esmagadas intensificam o seu perfume ...

--- É isso mesmo. E o perfume chega a tornar-se acre pela intensidade. É
assim mesmo, Bárbara --- repetiu ele.

Conquanto sentissem ambos o desejo de externar o carinho que lhes enchia
a alma, permaneceram numa atitude inquieta, como à espera de alguma
coisa que ainda lhes viria ao encontro. Na verdade, estavam num barco em
alto mar, na esperança de um navio que os recolhesse.

\chapter{Capítulo 91}

Álvaro tinha deixado a casa de Bárbara, naquela manhã, com uma esperança
nova na vida. Não sabia ainda em que, mas, esperava; e a intuição lhe
dizia que esperava por alguma coisa que ainda haveria de vir. Não se
referiu a Carlito, embora a lembrança do rapaz lhe perturbasse muitas
vezes a serenidade.

Naquela tarde, Álvaro dirigiu-se à faculdade; estava de bom humor e ia
em busca dos resultados finais. Entrou satisfeito e passou pelos
estudantes com direção ao quadro. O seu nome era logo o primeiro e ali
estavam todas as suas notas. Um colega aproximou-se e, reconhecendo-o,
dirigiu-lhe a palavra.

--- Ei, Álvaro. Você ainda não pegou os seus convites? Estão prontos
desde antes dos exames. Você não é nada apressado, hein?

--- É de mau agouro, imprimir convites antes dos exames --- respondeu
galhofeiro.

O colega afastou-se e Álvaro dirigiu-se à secretaria. Pouco antes da
porta, parou em busca de cigarro; riscou o fósforo para acendê-lo quando
ouviu o seu nome. Uma curiosidade instintiva fê-lo ficar a escuta. A
conversa prosseguiu:

--- Dizem que é casado --- ajuntou um deles --- e eu o vi em companhia
de uma mulher lindíssima!

--- Eu também. Estava com ela num concerto do Municipal. E a moça não
tem jeito de mulher livre --- comentou outro com certa ponderação.

--- Eu vi esta mesma moça com Carlito --- informou um terceiro. --- Que
encrenca! Talvez seja por isso que ele nem se lembrou de vir apanhar os
convites.

Álvaro sentiu o coração perder o ritmo, comprimido pelos músculos em
tensão. Afastou-se precipitado. Alcançava já a porta de saída, quando o
judeu de cabelos vermelhos o puxou pelo braço:

--- Que pressa é essa?

--- Ah, é você? --- perguntou voltando-se para o colega. --- Que foi que
aconteceu?

--- Você viu as notas? Que injustiça! Fiz a prova igualzinha à sua; você
teve nove e eu seis.

--- Não sabia --- retrucou Álvaro.

--- Escute, o pessoal diz aí que você não procurou os convites. Você não
pretende usá-los?

--- Por quê?

--- Porque eu estou com falta. E até pedi hoje os seus na secretaria. Ia
procurá-lo...

--- Pode ficar com eles --- interrompeu Álvaro.

Ia já afastar-se quando o outro lhe estendeu a mão:

--- Pelo menos leve um --- disse enfiando-o apressadamente no bolso de
Álvaro.

--- Está bem, e até logo.

--- Até logo.

Mal chegou à esquina, passou um ônibus vazio; apressando-se, Álvaro o
apanhou. Pelo caminho, pareceu tornar-se mais calmo; chegou a abrir o
convite e ler nele o seu nome. Decidiu levá-lo a Bárbara. Sentiu prazer
em entregar-lhe o único convite que possuía.

* * *

Quando Álvaro se aproximou do portão de ferro, viu Bárbara no balanço do
jardim. Quase de costas, deixando ver parte do seu perfil, tinha a
cabeça reclinada num visível abandono. Álvaro contemplou-a; teve certeza
de que o amava. Pela intuição, penetrou nos sentimentos de Bárbara como
nos seus próprios. Sentiu que ela sofria e, uma vez mais, sofreu com
ela. Apertou na mão direita o convite da colação de grau e, lembrando-se
das palavras dos colegas, disse para si --- é verdade que mais um posto
foi escalado, porém, tendo como suporte o sacrifício incompreendido de
uma mulher. Não fora a força de Bárbara, jamais teria concluído a
carreira. Doía a Álvaro a sua fraqueza; mas a recordação dos dias
vividos longe de Bárbara tornava-o impotente ante a força do seu
egoísmo. E aceitou, com revolta, a dedicação completa de um amor que o
auxiliava a viver. Não podia dispensá-la; teve consciência disso, e
pressentiu o sofrimento intenso que o seu amor acarretaria à vida de
Bárbara. Ele, que tanto a amava, pesava-lhe na vida como um fardo! No
entanto, que coisa estranha! Era a pessoa a quem mais desejava suavizar
a vida.

--- É você, Álvaro?

--- Sim. Em que estava pensando?

--- Em tudo e em nada.

--- Bárbara, querida --- volveu com acentuada tristeza --- isto é o que
eu tenho sido na sua vida: tudo e nada.

Bárbara olhou para ele em silêncio. Álvaro estendeu-lhe o convite num
gesto lento de desânimo:

--- Falta apenas três dias...

--- Para a colação de grau? --- completou a moça. --- E este é meu? ---
indagou, tomando o envelope.

Álvaro fitou-a com amor:

--- Que há na minha vida que não seja seu?

Bárbara perturbou-se. Correu os olhos pelos nomes dos formandos, e ouviu
Álvaro dizer:

--- Sabe de uma coisa, Bárbara? Não irei à colação de grau; não mereço
esse diploma.

Deu uns passos agitados em torno dela e declarou com acentuada revolta:

--- Não posso continuar assim. Tenho que solucionar a minha vida; tenho,
preciso. E nem sequer sou capaz de uma deliberação; preciso vir magoá-la
antes. Oh Bárbara, não sei como você pode querer uma pessoa assim!
Bárbara\ldots{} Bárbara --- com uma voz seca e difícil continuou --- eu
sou um covarde. Vou embora, Bárbara, não posso arrastá-la na correnteza
das minhas paixões.

Irritado consigo e com as circunstâncias retirou-se precipitadamente,
sem atender ao chamado de Bárbara. A moça correu até a grade e viu-o
dobrar a esquina do outro lado. Desejava que ele não fosse; e só não o
seguiu porque, no íntimo, sabia que Álvaro voltaria para ela.

\chapter{Capítulo 92}

Era o dia da colação de grau. Vinte e dois de dezembro de mil novecentos
e trinta e oito. Bárbara entrou no teatro, já repleto, e ocupou a
cadeira que lhe foi destinada. Os bacharelandos tomaram seus lugares no
palco. Logo a seguir, os Professores e o Reitor da Universidade
mostravam-se prontos para iniciar a solenidade. Após a abertura da
sessão, prestou juramento o primeiro aluno chamado. Muitos jovens se
diplomavam --- e entre eles, algumas promessas para a justiça do Brasil.

Na terceira fila do palco havia uma cadeira vazia. Vendo-a, Bárbara
sentiu-se comovida, pois, adivinhou logo quem deveria ocupá-la. Nisto,
houve um silêncio mais prolongado, e o Reitor repetiu o nome de Álvaro
Prado. Correu um murmúrio pela assistência, mas, ninguém apareceu.
Bárbara cerrou os olhos, procurando conter a sua emoção; no íntimo,
compreendeu e apreciou aquele gesto de desprendimento. Na ocasião
oportuna de ser aplaudido em público, como bacharel e como escritor,
Álvaro retirou-se para não receber as honras que ele julgava imerecidas.
Poderia estabelecer-se maior contato entre ele e os seus leitores, e
aquela era a oportunidade; mas rejeitou a palma, uma vez que não
conquistada pelo esforço próprio. Era um gesto peculiar ao caráter de
Álvaro, e Bárbara não se surpreendeu.

Assim, esperou a chamada interminável dos formandos em direito, para
depois ouvir os discursos.

Um professor, que Álvaro dissera possuir uma inteligência brilhante,
paraninfou a turma de mil novecentos e trinta e oito. Bárbara admirou-o
frase por frase na sua culta exposição. Era um filósofo que falava ao
povo; e só mais tarde Bárbara veio a saber que ele ocupava a cátedra de
filosofia do Direito. Ao contrário dos hábitos \emph{discurseiros}, o
professor falou pouco; em menos de vinte minutos, disse tudo. Um só
segundo, porém, não foi tomado em afirmações vãs; nem uma sequer das
suas palavras deixou de ter um significado real.

Considerou o direito. Lembrou aos formandos e ao público que o conceito
de direito é geral, universal e necessário, portanto, eficiente e
fecundo. Assim, não sendo casual o conceito de direito, torna-se
duplamente criminoso o advogado mal-intencionado das entrelinhas legais.
Pesa ao advogado a responsabilidade da justiça e o apostolado da
verdade. Em tudo e por tudo, ele representa a lei no seu sentido direto
e verdadeiro; nunca nas suas falhas inafastáveis. Nós não somos, disse o
orador, por profissão, o contornador de códigos, embora muitos até hoje
o tenham ignorado. Lembrou os ânimos exaltados do momento e a
possibilidade, cada vez maior, de uma guerra europeia. Contra a
expectativa de todos, não condenou povos nem exaltou reais ou pretensos
benfeitores da humanidade. Sabia que cada povo procurava o seu benefício
próprio, a humanidade estaria sempre em segundo plano ou em belos
discursos. Só os ingênuos podem acreditar que uma nação se sacrifica,
conscientemente, pelo bem-estar de outra. Apontou como causa da guerra a
violação do direito. Enquanto existisse um povo escravizado, existiria
um escravizador; existindo um escravizador, existiria a violação do
direito; e existindo a violação do direito, não existiria a paz. Um só
país que dominasse, levaria aos demais a ideia da conquista --- ou todos
seriam escravos ou todos seriam livres. Logo, ou existiria a paz em todo
o mundo, ou não existiria em lugar algum. Um só olhar bastava --- a quem
sabe ver as coisas --- para verificar como ainda existiam escravos, não
obstante, o apregoado culto das democracias. Há quem considere a
democracia um sistema político em agonia. É oportuno lembrar, agora, que
o maior perigo para as democracias não provém daqueles que dizem seus
pensamentos. Prosseguindo na sua exposição, não apelou para o
sentimentalismo dos homens, mas procurou despertar o bom senso natural.
Não fez poesia em torno das crianças, velhos e mulheres que desapareciam
sob os bombardeios; mas pôs em relevo os crimes anteriores que,
inevitavelmente, levariam os homens às barbaridades da guerra. Se na
guerra, pois, exercita-se os sentimentos maus, na paz deve-se exercitar
os bons. Terminou, convidando os novos bacharéis a se tornarem cônscios
da sua responsabilidade para com os homens e para com a pátria.

Quando o paraninfo concluiu o seu discurso, o público o aplaudiu com
frieza. Geralmente os que dizem a verdade não são bem recebidos.

O bacharelando que falara primeiro recebera mais aplausos. Como a guerra
era o assunto da época, tinha posto em poesia as barbaridades da própria
guerra. Tecera uma lenda sentimental para adormecer os homens. Não
apresentara argumentos; como defensor das leis, esquecera-se da lógica.
E cheio de belas palavras, ele dirigira-se ao público como se lhe
contasse histórias irreais de um conto de fadas. Era a criança faminta e
miserável, ouvindo a história de um príncipe encantado!

Terminadas as solenidades, Bárbara saiu logo do teatro. Tomou o carro e,
de regresso para a casa, veio pensando nas palavras do paraninfo. Aquele
homem não tinha senso crítico para com o Brasil; tinha-o para com o
mundo. Na sua oração, expôs-se a um público duvidoso, que não alcançara
o verdadeiro sentido das suas palavras. Finalmente, os próprios
bacharelandos que o elegeram paraninfo, não se mostraram entusiasmados
com a sua participação na solenidade.

Assim pensativa, Bárbara aproximou-se de casa. Estacionou o carro em
frente ao portão grande e tocou a buzina para que Mrs.~Patrice lhe
viesse em auxílio. A porta do terraço abriu-se e a governanta veio
avisar que do hospital chamavam com urgência. Bárbara pensou em
comunicar-se com Álvaro e, deixando o carro, entrou em casa, apressada.
Mal procurou o número da pensão, o telefone deu alarme. Não foi preciso
segundo sinal; Bárbara atendeu imediatamente. Do outro lado, falou uma
voz masculina:

--- Alô...

--- Alô, Álvaro? --- perguntou a moça.

--- Você recebeu algum chamado, Bárbara? --- perguntou sem mesmo
cumprimentá-la.

--- Recebi sim, e com insistência; todavia, não sei o que se passa.

--- Nem eu, Bárbara; mas vou fazer-lhe um pedido: não vá ao hospital.

--- Por quê? Já não a atendi uma vez, Álvaro? E não me arrependo de o
ter feito.

--- Por favor, Bárbara --- suplicou o rapaz --- não posso, não quero
vê-la ao lado dessa mulher.

--- Poderei satisfazer a sua vontade, mas, não estou satisfazendo a
minha consciência.

--- Bárbara --- tornou ele humilhado --- há pouco estive pensando...
recordando, sabe? E senti um ódio tão violento! Oh Bárbara, é impossível
descrever, e eu não desejo vê-la, agora.

--- Você está criando outras circunstâncias. Seria melhor que
enfrentássemos as já existentes --- insistiu ela ainda.

--- Eu não aguentaria --- acrescentou revoltado.

Bárbara compreendeu de momento a atitude de Álvaro.

Ele se recordara com ódio daquela mulher e, assim, não desejava vê-la
doente e prostrada. Álvaro enfrentava uma Dalila forte nos seus rancores
pessoais e, numa defesa instintiva, quase subconsciente, fugia à
oportunidade de constatar a fraqueza da adversária. E por essa razão,
insistiu:

--- Álvaro --- pediu com afeto --- você me deixaria ir só?

--- Nunca! É um absurdo! Escute, você quer mesmo ir?

--- Quero, sim. Venha buscar-me.

--- Exige isso de mim? --- replicou quase vencido.

--- Não exijo, peço. Álvaro, você vai comigo?

--- Ah! Bárbara --- suspirou do outro lado --- vou agora mesmo.

* * *

Bárbara e Álvaro chegaram ao hospital meio confusos. Logo à portaria,
informaram-nos de que o estado da doente era bastante grave. --- Também
--- disse uma enfermeira --- não era para menos.

Álvaro notou o traje branco da moça. Sem saber por que, pareceu
impressionado pela alvura daquela roupa. Era qualquer coisa ligada ao
seu subconsciente, da qual sentia uma impressão vaga, esquisita.

Dirigiram-se para o quarto, na previsão de que acontecera algum fato
grave. Bateram à porta e a assistente veio atendê-los. Entraram. Viram
Dalila prostrada, quase inerte. O olhar tinha uma expressão triste e a
respiração difícil parecia fazê-la sofrer muito. Dalila não disse nada,
mas, a maneira como os fitou, indicava tê-los reconhecido. Bárbara
chegou-se a ela, silenciosa, arrumou-lhe as cobertas e sorriu. Embora
calada, a doente parecia compreender que Bárbara estava ali com o desejo
de auxiliá-la.

A enfermeira assistente fez sinal a Bárbara que a acompanhou para fora
do quarto. Uma vez no corredor, a enfermeira lhe narrou os últimos
acontecimentos:

--- Esta mulher é louca! --- exclamou. --- O seu estado não apresentava
gravidade, mas, agora, é perdido.

--- E aqueles balões?

--- Para aliviar somente; não há mais cura. Enfim, foi vítima de si
mesma.

Bárbara não quis fazer perguntas, receando ser indiscreta; mas a
enfermeira estava disposta a dizer tudo.

--- Essa moça é sua parenta? --- indagou.

--- Não --- respondeu --- tem os parentes no interior.

--- Ah, esses não chegarão a tempo!

--- Mas é assim tão grave?

--- Depois do que ela fez?! Imagine a senhora, coisa que nunca se deu
num hospital. Uma noite destas saiu e foi ao cassino, divertir-se.

--- Onde?! --- perguntou Bárbara, julgando não haver entendido bem.

--- Ao cassino! Ao cassino, sim senhora! --- afirmou a mulher. --- Bebeu
quando estava lá; e, não satisfeita, ainda trouxe uísque para aqui.
Chegou um pouco alcoolizada; ligou o rádio de cabeceira, dançou, fez
coisas incríveis. E moléstia de coração, a senhora sabe, exige muito
repouso. Esta mulher suicidou-se. Bem, acho melhor a senhora entrar.
Precisando alguma coisa é só tocar a campainha.

Quando Bárbara abriu a porta, encontrou Dalila atormentada pela
dispneia. No olhar suplicante, percebeu que ela desejava a sua
aproximação. Chegou-se à beira da cama e ficou ali, a olhar para a
doente numa atitude afetuosa, tão habitual em Bárbara para com os que
lhe pediam auxílio. Mais para trás, Álvaro parecia estarrecido. Uma
sensação estranha, incoerente com as suas determinações pessoais, abria
novo caminho no seu íntimo. Sentia-se desnorteado, vendo morrer aquela
mulher que julgara invencível, por sua força e astúcia.

Subitamente, num gesto de desespero, Dalila ergueu-se na cama e,
agarrando-se a Bárbara, debateu-se em convulsões tremendas. Seus olhos,
arregalados, transmitiam o terror da asfixia. Bárbara pediu o balão de
oxigênio que, embora providenciado a seguir, não chegou a prestar
auxílio.

Dalila lutou com a morte até o último alento; mas, como a morte é
invencível, foi-se entregando aos poucos... deu o seu último suspiro nos
braços quentes de Bárbara. Estava agarrada a Bárbara, como se dela
pudesse vir-lhe a vida; era o derradeiro contato dos que se vão... era a
despedida, temporária, para os que ainda ficam aqui.

\chapter{Capítulo 93}

Bárbara e Álvaro voltaram para a casa, silenciosos. A morte de Dalila
fora um sulco profundo na alma de ambos; parecia uma resposta às ações e
pensamentos anteriores. Agora estavam livres; poderiam desfrutar o seu
amor. Nem um nem outro, porém, tocava no assunto. A resposta viera
depressa demais; deixara-os aturdidos.

Pelo caminho, enquanto andavam vagarosamente nas ruas de Santa Tereza,
vieram à mente de Álvaro os versos de Rudyard Kipling. Sem uma
explicação consciente, começou a recitá-los em voz alta. A seu lado,
Bárbara ouvia em silêncio. Disse as duas primeiras estrofes, e a alma do
poeta parecia estar presente nas palavras do rapaz. Ao começar a
terceira estrofe, porém, sentiu que a memória lhe faltava; por mais que
se esforçasse, os versos rebeldes não voltaram. Bárbara veio em seu
auxílio:

\emph{Se és capaz de arriscar numa só parada}

\emph{Tudo quanto ganhaste em toda a tua vida}

\emph{E perder, e ao perder, sem nunca dizer nada}

\emph{Resignado tornar ao ponto de partida.}

Não foi preciso mais, Álvaro recordou-se dos outros versos e continuou a
recitá-los com a mesma entonação de voz. No seu íntimo, todavia, sentiu
uma alteração inexplicável. E, assim, chegaram à casa de Bárbara.
Despediram-se no portão e o rapaz esperou até que a moça entrasse e
fechasse a porta do outro lado. Depois, continuou a pé, rumo à sua
pensão. Pelo trajeto, lembrou-se novamente da poesia e considerou a
estranheza do esquecimento. Conhecia-a tão bem; nenhuma palavra lhe
faltara uma vez sequer. Agora, quatro versos inteiros de uma única
estrofe! Era estranho, muito estranho! Mentalmente, pôs-se a dizer o
poema outra vez. Mas, ao iniciar a terceira estrofe, aconteceu-lhe a
mesma coisa. Insistiu, obrigou o cérebro a trabalhar; repetiu a poesia
desde o início, e não logrou êxito nos seus esforços. Somente quando
chegou à pensão, os versos lhe apareceram, inseguros:

\emph{Se és capaz de arriscar toda a tua vida,}

\emph{E se perder não dizer nada}

\emph{E regressar ao ponto de partida. . .}

Não era isso. O cérebro recusava tais lembranças, na certeza de que não
eram verdadeiras. Ao entrar no quarto, seu primeiro cuidado foi a
verificação do texto. Admirou-se, então, de ver como o deturpara. Nisto,
uma ideia atravessou-lhe o espírito. --- E Carlito? Não teria cometido o
mesmo erro, ao dizer-lhe de cor a poesia, na faculdade? Rememorou as
cenas e, com uma clareza surpreendente, os fatos se reconstituíram na
sua imaginação. Viu-se em frente a Carlito, declamando o poema; teve
certeza de que o fizera certo. A par dessa lembrança, ocorreu-lhe uma
reminiscência desse encontro. Fora nesses versos que Álvaro observara a
expressão pueril de Carlito e, fazendo uma pausa, pensara consigo. ---
Como poderia eu perder o único bem da minha vida, e regressar depois ao
ponto de partida? E perder resignado, diante desse rapazinho bonito, que
não sabe apreciar nem sequer uma poesia! --- O esquecimento dos versos
estava ligado a um fato desagradável, que muito lhe perturbara o ânimo.
Deu uns passos pelo quarto; embora lhe faltasse o sono, resolveu
deitar-se. Acendeu o quebra-luz do criado-mudo e só então observou que,
apoiado ao despertador, estava um telegrama para ele. Abriu-o. Era de um
amigo de S. Paulo; o que substituíra no jornal, por ocasião do seu
primeiro artigo. --- Um convite para um estudo social na capital
paulista. Álvaro sentia por ele uma certa gratidão, pois fora por seu
intermédio que se lhe abriram as portas da carreira. Além disso achou
oportuno tal oferecimento e decidiu imediatamente pela aceitação. O mais
difícil seria comunicar a sua resolução a Bárbara. Iria deixá-la só, num
momento em que ela, talvez, necessitasse do seu auxílio. Mas, ele não
\emph{se} sentia em condições de enfrentar a situação; e, cônscio da sua
fraqueza, deliberou afastar-se. Como sempre acontece à mulher forte,
frente às coisas difíceis, Bárbara enfrentaria o problema mútuo, apenas
com as suas próprias forças.

Era já madrugada. Teria tempo para arranjar as malas e providenciar,
generosamente, uma passagem para S. Paulo, com os guichês ainda
fechados. Iniciou as arrumações, e o tempo foi-lhe suficiente para os
passos necessários. Dispôs tudo da melhor forma possível, enquanto
passavam as horas. Evitava pensar nos últimos acontecimentos. Os
arranjos apressados da ocasião --- foram o analgésico momentâneo de que
ele se serviu avidamente. Assim, aos primeiros sinais de atividade
comercial, Álvaro estava de posse da referida passagem e pronto para
viajar.

Conferiu, mais uma vez, as suas coisas todas e, deixando o quarto, foi
até ao café da esquina. Enquanto esperava, adquiriu jornais e correu os
olhos sobre a primeira e última páginas. Num deles, porém, viu uma
fotografia da solenidade de sua formatura. Prestando atenção, percebeu
uma cadeira vazia na terceira fila. E como havia indicação de uma
notícia na sétima página, Álvaro foi a ela. De fato, ali estava um
trecho de coluna, contando a solenidade em linhas gerais. Trazia o nome
do Reitor e dos professores. Fez alusão ao paraninfo; a seguir, uma a
Álvaro Prado, escritor já bastante conceituado nos meios intelectuais e,
que por motivos ignorados, não compareceu à colação de grau. No final da
notícia, rememorou o discurso do formando que, segundo o comentarista,
``era um linguajar plácido para acalentar bovinos''. Quanto ao
paraninfo, a sua oração, ou excedeu à capacidade intelectual do ambiente
ou não foi apreciada nas proporções devidas. Friamente aplaudido, o
professor não pareceu agradar, nem mesmo, aos seus afilhados. E, só nos
bastidores do teatro, a reportagem conseguiu saber a justa razão de
tanta frieza. O paraninfo havia sido escolhido pela dificuldade da
cadeira; o professor de Filosofia do Direito era severo na sua
disciplina. Como grande parte dos alunos o considerassem a ``peninha''
do curso, prestaram-lhe essa homenagem, em testemunho de alta
consideração e receio de mais um ano na faculdade. Ao que se permitiu
apanhar, o professor aceitou o honroso convite e reprovou igualmente a
grande parte dos seus discípulos. Quem sabe, teria feito um bem à
justiça do Brasil. Além do mais, a ausência dos reprovados não fora
observada, devido ao número incontável dos formandos em direito. Pode o
mui digno catedrático observar, bem a tempo, que os seus diletos alunos
iniciam a carreira, como bons advogados, sabendo inserir os seus desejos
nas ``entrelinhas da lei''.

Álvaro achou curioso o humor do articulista, embora brincasse com um
assunto sério; e, não sendo alvejado pela crítica, sorriu sem mais
preocupar-se. Sabia já da combinação entre os alunos para semelhante
convite; fora consultado a respeito e pediram-lhe ainda o seu apoio.
Álvaro, que apreciava o professor, apoiou a indicação. No mais, seria
atingido em linha geral; mas, sabia a matéria e o restante não lhe
interessava. Dobrou os jornais. Enquanto tomava o café, viu, pouco mais
adiante, um suíço que morava na sua pensão. O homem aproximou-se de
Álvaro e, com um sorriso zombeteiro, perguntou:

--- É a sua turma, a de ontem?

--- Parece... --- respondeu compreendendo a alusão.

--- Aqui está um fato que exemplifica a atitude do brasileiro --- tornou
o outro --- brincar, brincar sempre, até diante das coisas sérias.
Acontece algo muito grave e o povo encontra piadas para fazer humor;
converter em riso o que deve ser chorado com lágrimas de sangue.

Álvaro calou-se. Embora compreendesse a veracidade daquelas palavras,
doeu-lhe, até a alma, ouvir um estrangeiro criticar o Brasil. Deixou o
café. No curto trajeto para a pensão, imaginou quantos formandos não
sairiam naquele mês, pelas escolas do país. Mas, no Brasil, quando se
pensava em títulos, interessava mais a quantidade que a qualidade
destes.

Chegando ao quarto, Álvaro mandou retirar as bagagens. Verificou as
horas --- eram dez. Chamou um carro e saiu com destino à casa de
Bárbara.

* * *

Ela o recebeu na sala. Tinha a fisionomia cansada pelas emoções da
véspera; no seu olhar, entretanto, transparecia tranquilidade. Álvaro
fitou-a perturbado, e custou-lhe pronunciar o nome:

--- Bárbara.

--- Que foi, Álvaro? --- indagou, pressentindo uma novidade.

--- Eu... eu --- começou o rapaz --- estou com as coisas prontas para
uma viagem. Sigo para São Paulo --- contou abruptamente.

Bárbara não se mostrou surpreendida. Olhou para ele com certa tristeza e
perguntou:

--- Vai ocupar-se nalgum trabalho?

Álvaro entregou-lhe o telegrama do amigo. Bárbara leu-o atenciosamente e
devolveu:

--- Desejo-lhe feliz desempenho.

--- Você não precisa de mim, agora?

Ela sorriu.

--- Vai Álvaro; esta saída é oportuna para você.

--- É verdade, Bárbara --- tornou precipitado. --- Não posso conter-me,
preciso perder a impressão de...

Bárbara não completou e Álvaro, nervoso, sacudiu a mão.

--- ... de que fui eu que a matei. Cheguei a ter-lhe tal aversão no dia
de ontem! Pensei mesmo que eu desejasse a sua morte; mas oh! Bárbara,
percebi com certeza, que nunca se deseja a morte a alguém.

--- Eu sei disso, Álvaro. Conheço-o bastante para não o julgar
criminoso.

--- Quando vi morta aquela mulher, que impressão horrível! Parecia que
eu a tinha matado com o meu pensamento. Entrei em choque comigo mesmo e
não sabia o que fazer. Desejei ardentemente ver-me livre dela e de
repente... fiquei desconcertado. O homem não sabe o que realmente quer
--- concluiu pensativo.

Vendo que Álvaro parecia mais aliviado por ter-se expandido, Bárbara
lembrou-lhe as horas:

--- O trem não sai às onze?

--- Deve ser; não verifiquei com exatidão.

--- Distraia-se, Álvaro; esta mudança de atividade vai ser-lhe benéfica.

--- E você, Bárbara?

--- Esperarei.

Ele estendeu-lhe a mão e saiu precipitadamente, sem olhar para trás.
Quando a caminho, teve ímpetos de fazer voltar o carro; mas, sentia
necessidade de afastar-se. Assim, inquieto, querendo alguma coisa
indefinida, chegou à estação.

Antes de deixar a moça, sentia já uma saudade triste a lhe invadir a
alma. Apressado, entrou no trem prestes a partir. Logo que a máquina se
pôs em movimento, cerrou os olhos, e a imagem de Bárbara voltou-lhe com
insistência. Pela imaginação, reviu a despedida. Transportou-se a Santa
Tereza, e ouviu novamente as suas palavras. --- Vá, Álvaro, esta saída é
oportuna para você. Perguntara se necessitava do seu auxílio; ela lhe
respondera pensando exclusivamente nele. Que egoísta fora! Como
repararia todas estas coisas?

\chapter{Capítulo 94}

Álvaro chegou a São Paulo e dedicou-se inteiramente às suas atividades.
Fez uma palestra como abertura aos seus trabalhos, no auditório do
jornal. A seguir, publicou-a. Abriu depois a sua série de artigos, com
um ensaio nacionalista sobre a arte. --- O que é a arte na atualidade,
no mundo e no Brasil, o desenvolvimento artístico nacional,
principalmente da pintura; como S. Paulo se colocava no terreno
artístico; finalmente, as perspectivas futuras de um ambiente seguro,
criado e desenvolvido pelas aptidões artísticas do nosso povo.

Lembrou aos paulistas que em S. Paulo se reuniam várias expressões da
arte brasileira. Conquanto houvesse ainda concertos de piano onde
figurassem o Guarani ou outras peças do domínio da orquestra, aparecia
já, de outro lado, o impulso a orientar o povo para a arte verdadeira
que como um sol nascente espalharia os seus raios numa área infinita.

O Brasil, disse ele, não tem ainda a sua cultura própria definitiva;
mas, possui verdadeiros gênios, cuja contribuição à ciência e à arte já
aparece na virtuosidade criadora de um insigne matemático como o fora
Manuel Amoroso Costa, ou no pincel arrojado e vigoroso de um Cândido
Portinari. Não obstante, os brasileiros, infelizmente, não têm sabido
eleger os seus verdadeiros valores. Os gênios só têm reconhecida a sua
genialidade, quando confirmada primeiramente em solos estrangeiros.
Existe por aí vários talentos jovens iniciados na arte da pintura,
escultura, dos bailados e da música. Não esperemos que a sua carreira se
firme lá fora, para que depois os aplaudamos aqui. Muitos deles, e quase
sempre como acontece ao verdadeiro artista, não têm recursos para irem
colher as palmas de outros povos. Honra, pois, ao mérito, e que os
valores brasileiros se façam no Brasil!

Os artigos de Álvaro espalhavam-se no meio artístico e literário.
Estudantes, de várias faculdades, disseminavam as suas ideias aos quatro
cantos da cidade.

Não tardou que os leitores reclamassem à Direção do jornal um maior
contato com o escritor que já se ia tornando famoso, uma vez que ele se
achava em S. Paulo e escrevendo para o conceituado periódico. Álvaro
iniciou, então, uma série de conferências, que os jornais da tarde
publicavam diariamente. Houve quem apreciasse, pelo lado de fora, a
atitude do considerado periódico, classificando-a de comercial. Mas,
importaria ao povo que aquele gesto pró-arte fosse a título de
propaganda do jornal? Um anúncio seleto indica o nível do seu
anunciante; graças, portanto, ao periódico, cujos planos favoreciam o
desenvolvimento cultural.

E Álvaro ia granjeando amizades, não só pela sua presença agradável,
como pela simplicidade do seu espirito. Criou um círculo de amigos que
permaneciam parte da noite na redação, conversando e discutindo
singelamente as suas ideias.

Certo dia estava à sua escrivaninha, na redação, quando recebeu um
telegrama. Abriu-o. Era uma comunicação distante:

\emph{Parabéns, amigo}

Helena Paulo

Helena? Paulo juntara ao seu, o nome da esposa. Como deveria estar
feliz! Pensou neles por um momento; mas, Bárbara veio substitui-los.
Álvaro sentiu a saudade apertar-lhe o coração. E ele, quando veria o
nome de Bárbara ligado ao seu? Como resposta, pronunciou mentalmente,
Bárbara Álvaro. E teria, então, tudo o que a vida lhe negaria! Seria
feliz e fá-la-ia feliz. Era um mundo que começava! Casar-se por amor!
Amar! Devia ser alguma coisa rara, pois, Álvaro admirava-se ao constatar
a realidade desse amor. Unir-se a uma mulher, não só pelas necessidades
da carne ou pelo hábito de ter uma companheira, mas, por senti-la parte
integrante de si mesmo. Como vivia nele a imagem dessa mulher que, por
modo tão original, passara a fazer parte da sua vida.

A secretária aproximou-se e deixou algumas folhas sobre a mesa. Um rapaz
que passava, vendo-o tão absorvido, procurou chamá-lo à realidade:

--- Ei... olha aí um trabalho que a moça deixou datilografado. ---
Remexendo os papéis, o rapaz comentou: --- cópia? Para quem é?

--- Para minha esposa --- respondeu o escritor.

O rapaz arregalou os olhos e Álvaro, espantado pelo que dissera,
corrigiu a seguir:

--- Minha noiva, quis dizer.

--- Hum --- resmungou o outro afastando-se.

Álvaro riu sozinho. E com saudade de Bárbara, dirigiu-se ao hotel.
Trouxera consigo o jornal da fotografia da moça; olhar para ela, através
daquele papel áspero, seria, já alguma coisa.

Tinha compromisso para mais quinze dias; depois, rumo à cidade
maravilhosa... e Bárbara para sempre. Era inacreditável! Embora lhe
desagradasse, Álvaro sentiu em si o receio supersticioso do muito
desejar alguma coisa. E quando fitou a imagem de Bárbara, exclamou num
êxtase amoroso --- é mesmo diferente!

E era diferente... a sua expressão de olhar o dizia. Era a mulher de
força, capaz de amar e de acompanhar um homem. Não era a borboleta fútil
de salão; não era a mulher bonita que baseia a sua vida nos encantos
pessoais; não era a mulher que desejava ter a seus pés um homem
subjugado; não era a mulher do domínio e nem a mulher subjugada. Era a
mulher de personalidade formada que compreendera, através das
experiências da vida, o direito da liberdade devido a um ser humano,
portanto, devido a uma mulher.

\chapter{Capítulo 95}

Mil novecentos e trinta e nove parecia cheio de promessas. Assim Bárbara
o esperava. Na passagem do ano, Álvaro lhe telefonara de S. Paulo.
Conversaram quase meia hora, embora nada houvesse de extraordinário para
ser dito. Agora, preparava-se para assistir ao casamento de Lia e Sérgio
Augusto, que deveria realizar-se dentro de uma hora. Helena já estava
casada e feliz; as notícias esparsas, vindas de Helena ou de Paulo,
transpareciam uma alegria inconfundível. Lia também vencera; ia casar-se
com o rapaz que lhe conquistara o coração.

Bárbara achegou-se ao espelho. Verificou o seu traje combinado numa arte
caracteristicamente feminina; pôs o chapéu, as luvas, e dirigiu-se à
casa dos Macedo. Todavia, ao parar o carro em frente à residência,
estranhou o silêncio do recinto. Tocou a campanha e, antes de entrar,
perguntou ao criado se a cerimônia se realizaria na Igreja.

--- Não senhora --- informou o servente. --- Faça o favor de entrar.

Bárbara acompanhou-o desapontada pelas circunstâncias. Nas salas,
comumente arranjadas, havia cerca de trinta pessoas. As flores em
cestas, chapéus e vasos, constituíam a ornamentação especial para a
cerimônia.

Vendo-a, Carlito veio imediatamente ao seu encontro e Bárbara observou
que ele estava muito comovido.

Instrumentos de corda deram início à Marcha Nupcial de Wagner e Lia
desceu a escada pelo braço do pai. Seus trajes não eram ricos como os de
Ivete, mas, tinham um encanto singelo. Sérgio Augusto avançou alguns
passos e recebeu Lia do braço paterno; conduziu-a, ele mesmo, à mesa em
que o juiz os declararia marido e mulher, perante as leis do país.

Assistindo à cerimônia matrimonial, o pensamento de Bárbara voou até a
capital paulista; carícia distante para Álvaro. Viu depois com que amor
Sérgio Augusto tomou Lia nos braços, pronunciando algumas palavras ao
seu ouvido. Bárbara adiantou-se um pouco e, sem o querer, ouviu os
comentários de uma senhora na sua frente:

--- Dizem que Alda não queria o casamento. O rapaz é modesto e
voluntarioso. Contrariou os desejos de Alda e fez realizar o casamento
com simplicidade. Este não será um genro desejável.

--- Mas --- disse baixinho a companheira --- o outro também não está
sendo.

Só então Bárbara lembrou-se de Ivete. Correndo os olhos pela sala, viu-a
mais distanciada. Percebeu que ela estava grávida e recordou-se de que
Lia já lhe havia contado.

De repente, Bárbara estremeceu. Notara que Ivete se entretinha num
namoro discreto com um rapaz cujas feições assemelhavam-se às de Sérgio
Augusto. Procurou Roberto pela sala, temendo algum desfecho perigoso.
Notou, então que o marido de Ivete não assistia ao casamento da cunhada.

--- Que foi? --- perguntaram ao lado.

--- Oh, nada, Carlito --- respondeu Bárbara voltando a si.

--- Você estava tão absorvida! Não vai cumprimentar Lia?

--- Vou, sim.

Adiantando-se, Bárbara desejou aos noivos muita felicidade. Carlito
abraçou a irmã com uma ternura visível. Bárbara sábia que os dois irmãos
eram os elementos mais ligados da família. Carlito, emocionado, veio
procurá-la; como se desejasse preencher o vazio que a saída da irmã lhe
deixava na alma.

--- Não a tenho visto, Bárbara; onde tem andado nessa temporada?

--- Por aqui mesmo.

--- Onde esteve na passagem de ano?

--- Em casa --- respondeu Bárbara, lembrando-se do telefonema de Álvaro.

--- Em casa?! Não foi a um clube, cassino, nada?

--- Desta vez nada, nada.

--- Que é isso, Bárbara? --- disse num impulso apaixonado.

--- O quê?

--- Ficar em casa num momento festivo! Eu teria dado tudo para estar a
seu lado.

Carlito estava perturbado. Procurando conter-se, desviou o assunto:

--- Bárbara, este casamento está tão triste que até parece um enterro.

--- Não seja pessimista. Eles vão ser felizes.

--- Mamãe não foi justa. Só porque Sérgio Augusto não queria pompas,
tomou o lado oposto de uma vez. Isto não parece um casamento.

--- Não importa Carlito; eles se querem e isto é o principal.

--- Bárbara --- disse Carlito em tom confidencial --- há quem diga por
aí que Sérgio veio atrás de dinheiro; você acredita nisso? Se é verdade,
foi um erro tremendo. Nós gastamos muito; porém, não temos dinheiro, a
ponto de constituir um dote apreciável para cada dos um três.
Francamente, o que nos cabe não equivale a um casamento.

--- Você está perturbado nos seus sentimentos, Carlito, e por essa razão
tem medo de tudo. Eu não acredito que Sérgio Augusto tenha se unido a
Lia por questões financeiras. Lia tem encantos e é mulher para ser
amada; fique ciente.

Naquela atmosfera de casamento e envolvido em conversas amorosas,
Carlito sentiu intensificar-se a sua tendência por Bárbara. Levou-a para
o terraço, ofereceu-lhe uma taça de champanha e olhando para ela, não
venceu os seus impulsos.

--- Bárbara --- chamou com a voz quente.

--- Que é? --- indagou perturbada.

--- Nunca percebeu nada em mim?

---?...

--- Por que me tem evitado, se eu me sinto impelido para você?

--- Carlito...

--- Não gostaria de imitá-los? --- interrompeu-a, olhando para dentro.
--- Seríamos felizes para o resto da vida.

--- Casamento não se resolve assim; você está sendo precipitado.

--- Não existe precipitação em amor. Quero-a para mim, Bárbara. Eu a amo
loucamente, você não percebeu?

--- Você não me ama Carlito --- tornou Bárbara com afeto --- está apenas
entusiasmado. Os seus vinte e dois anos é que motivam essa atitude.

--- Acha que sou criança? --- perguntou ofendido.

--- Acho que gosta de cortejar-me; mas, não vai além.

--- Você está enganada. Eu faria tudo para obter o seu amor. Bárbara ---
disse ele arrebatado --- case-se comigo, e eu serei o homem mais feliz
do mundo.

Como geralmente acontece entre os homens, Carlito pensara apenas na sua
felicidade. Não lhe atravessara o espírito que Bárbara poderia não ser
feliz com ele. Estavam sós no terraço. Achegando-se à Bárbara, o moço
tentou puxá-la para si.

--- Isto é loucura --- respondeu Bárbara com firmeza e afastando-se do
rapaz.

--- Não seja bárbara\textbf{!} --- replicou, alterado.

--- Amanhã, você esquecerá tudo --- tentou explicar a moça no desejo de
não o magoar.

--- Não posso, é tarde demais --- ponderou com desespero.

--- Tem vinte e dois anos, Carlito. Nessa idade as coisas raramente são
tardias --- observou ela com doçura.

--- Não torne a dizer minha idade --- disse o rapaz com aspereza.

--- Não tive intenção de ofendê-lo --- volveu Bárbara admirada.

Enquanto ela permaneceu numa atitude de surpresa pelo acontecido,
Carlito perguntou num tom brusco e de revolta:

--- Por que já não me desiludiu antes?

--- Não existia razão para isso. Não houve entre nós um gesto ou uma
palavra que indicasse este desenlace. Desculpe-me, Carlito, mas você
está sendo criança.

--- Já sei. O fruto proibido tem um sabor especial. Eu não sou casado
--- disse, procurando atingir os sentimentos de Bárbara.

Ela o fitou numa atitude séria e, sem desviar os olhos, inquiriu:

--- Pesou as suas palavras?

--- Não sei o que responder --- tornou abatido. --- Vai odiar-me agora.

--- Se tivesse dito intencionalmente, não o cumprimentaria mais; mas,
percebo, Carlito, o que gritou em você foi a incompreensão e não a
maldade.

Sérgio e Lia, em trajes de viagem, apareceram no terraço. Vinham fazer
as despedidas. Bárbara usou da oportunidade para encerrar a conversa.

\chapter{Capítulo 96}

Com uma conferência de despedida, Álvaro concluiu o seu trabalho na
capital paulista. Conseguiu passagem para o cruzeiro da noite seguinte;
pôs-se, então, a arranjar suas coisas para a viagem. Reuniu os preciosos
papéis, representação de muitas horas de estudo. Ia encomendar um café,
quando bateram à porta. O servente entrou e disse a Álvaro que uma
comissão de estudantes esperava por ele no salão.

--- Pois bem --- disse o rapaz --- irei já. E encomende um café para
nós.

Guardou os seus trabalhos, e desceu para ver do que se tratava.

--- Alô --- cumprimentou na sua camaradagem habitual.

Os rapazes mostraram-se alegres por encontrá-lo e foram logo ao motivo
da visita.

--- Que vai fazer agora? --- indagou um deles.

--- Buscar a minha passagem. Querem alguma coisa de mim?

--- A sua passagem não é para o noturno?

--- É sim --- respondeu. --- Para avião, só daqui a três dias e eu tenho
pressa de chegar ao Rio.

--- Para o noturno desta noite? --- perguntaram aflitos.

--- Não; para o de amanhã.

--- Ah... --- concluiu um dos moços.

--- Bem, doutor --- tornou um outro --- o senhor não vai desapontar-nos.

--- Naturalmente que não. Mas, por quê?

--- Vimos convidá-lo para uma reunião na noite de hoje; queremos
homenageá-lo antes da sua partida.

--- Oh, terei imenso prazer --- volveu Álvaro comovido pela demonstração
de apreço.

--- Muito bem --- disseram os rapazes --- então, viremos buscá-lo às dez
horas. E como estamos com pressa, vamos nos retirar agora.

Os membros da comissão despediram-se alegremente. Álvaro saiu a seguir.
Conhecia bem S. Paulo; fora estudante na terra e o traçado da cidade era
vivo na sua mente. Tomou o caminho do correio. Ia dizer alguma coisa a
Bárbara da sua completa vitória em S. Paulo. Veio descendo pela avenida
São João e, finalmente, deparou com o edifício procurado. Na praça, em
que se achava o correio, havia um posto de bondes onde muitos deles
iniciavam o trajeto. O povo reunia-se ali, na pressa agitada da vida de
trabalho.

Álvaro contemplou os transeuntes a correrem pela praça, com suas
fisionomias preocupadas pelo difícil da vida. Como aquela gente
trabalhava! O peso do esforço paulista contagiava até os visitantes. O
homem da banca de jornal anunciava os vespertinos a todos pulmões; fazia
troco numa agilidade inacreditável e, quando se afastava, vendia os
periódicos, vigiando de longe o tabuleiro armado com revistas e outras
coisas mais. Depois de olhar para aquela agitação, Álvaro compreendeu
por que S. Paulo se tornara o estado mais produtivo do Brasil. E,
entrando no edifício, pensou deixar a agitação lá fora.

O correio estava movimentado por pessoas do mesmo temperamento; só os
funcionários pareciam não ter pressa. Atendiam a todos numa atitude
passiva, como se estivessem entediados por separar cartas ou carimbar
selos. Álvaro achegou-se no guichê e pediu algumas fórmulas.
Rabiscou-as, indeciso, sem saber o que dizer. Finalmente, decidiu por
duas palavras. Olhou para os empregados, e percebeu que o do guichê da
esquerda tinha uma expressão de bondade, convidativa às mensagens
amorosas. Encorajou-se. Tomou a fórmula e escreveu: amo-a. Num arrojo,
entregou-a. O agente não escondeu um sorriso.

--- Passar um telegrama para dizer isso?

--- Sim senhor. E não acha importantíssimo?

--- Importantíssimo, mas não num telegrama.

--- Pois meu caro, hoje em dia é assim. A civilização leva aos ares o
nosso amor.

--- Vai morar em arranha-céu?

Álvaro olhou para ele e riu.

--- Cuidado, não o deixe voar muito --- disse ainda o agente.

--- Bom conselho, não resta dúvida.

--- Conselho de homem experiente.

Álvaro olhou para aquele homem de cinquenta anos. Tinha os óculos na
testa e um corpo excessivamente magro; parecia de atitude indulgente e
até mesmo compreensiva. Podia, na verdade, ser um homem de experiência
e, quem sabe, se até mesmo de grande experiência! Encerrando os seus
pensamentos, pagou o telegrama, despedindo-se cordialmente do novo
conhecido; e saiu a pé pela cidade. Era preciso fazer alguma coisa, não
podia conter todo aquele humor. {}

\chapter{Capítulo 97}

Às dez horas, já estavam os convidados no clube. Álvaro foi o último a
chegar, pois, os outros combinaram precedê-lo. O ambiente era cordial e
simples. No eixo mais longo do salão, estava armada uma mesa comprida e,
rodeando-a, várias mesinhas esparsas. Quando Álvaro apareceu na porta de
entrada, os estudantes elevaram o célebre hip... hip... hurra. A seguir,
ele sentou-se ao centro, lugar de destaque, reservado para ele. Correu
champanha e os presentes estavam pela celebridade amiga. Beberam,
brindaram e discursaram. Álvaro foi o último a falar. Levantou-se
comovido e dirigiu-se aos presentes, agradecendo aquela manifestação de
amizade e simpatia. Novos brindes; e o tinir das taças que cruzavam no
ar, era prolongado e festivo. Após as saudações, abriram-se de par em
par as cortinas do palco. O ``jazz'' iniciou uma valsa lenta. O povo
paulista mostrava grande predileção pelas valsas; não obstante inúmeras
danças modernas, a tradição de gosto estava sendo mantida. Ao som de
``Vozes da Primavera'', saíram os primeiros pares. Álvaro ficou no seu
lugar, conversando com alguns jornalistas e escritores. De quando em
quando, uma taça de champanha ou alguma outra bebida. Havia tempos em
que, calados, apreciavam os dançarinos.

Lembrou-se de Bárbara e lamentou a sua ausência. Não só a saudade, mas,
a vaidade de ostentar perante todos, como sua noiva, uma mulher
fascinante, encantadora, fê-lo voltar o pensamento para ela. Afinal,
fosse como fosse, a companhia de uma mulher bonita em um salão é coisa
necessária e indiscutivelmente agradável. Ao regressar, Bárbara veria
como se portara longe dela! Como conseguira dedicar-se com êxito a um
trabalho intelectual. Pensou mesmo que, a partir dali, tomar-se-ia um
apoio para ela; pois, vencera as dificuldades que o amarravam a uma vida
inativa.

Voltou mentalmente ao passado e, qual um filme cinematográfico, as cenas
desenrolaram-se no pensamento. Vendo o caminho vencido, sorriu
satisfeito. Tudo isso, por uma mulher! Tornando a sorrir, levou o
cigarro à boca, deu uma tragada e continuou imóvel, olhando qualquer
coisa que ele mesmo não via. À certa altura, fixando a vista no mesmo
ponto em que olhava, notou que uma moça lhe sorria. --- Estará sorrindo
de mim, ou para mim? --- pensou consigo. Olhou-a novamente e viu-a
ainda, a sorrir. --- Estranho! Eu estava vendo e não via. Naturalmente
adivinhou minhas conjecturas de há pouco. --- Desviou-se para o lado,
deu um aparte na conversa e fingiu esquecer a cena. Mas, qualquer coisa
forçava-o olhar para lá. Voltou-se em ar de desafio. Encararam-se; ela
sustentou o peso do combate e ele, aos poucos, foi cedendo. Calculou mal
a sua força; já estava sendo vítima de sua impetuosidade. Dali por
diante, conversou pouco com os que lhe faziam companhia, pois, sentia a
provocação. A moça era loira oxigenada, sobrancelhas e cílios pretos,
profusamente pintados. Era bonita e provocante. Um dos jornalistas
sentou-se mais perto e disse-lhe baixinho:

--- Cuidado.

--- Porque? --- inquiriu meio contrafeito.

--- É uma mulher...

--- E que tem isso?

--- Bastante.

Álvaro compreendeu o aviso, mas, irritou-se. Resolveu tirar uma
desforra; e, olhando para ela, fez-lhe sinal, convidando-a para dançar.
Teve resposta afirmativa.

Deixou a mesa, atravessou o salão e chegou até à moça; confiando nos
seus trinta anos de experiência e, inconscientemente, no amor
indestrutível que o prendia a outra mulher.

Às vezes, os homens pensam que o amor por uma mulher os livra,
incondicionalmente, dos perigos de outra.

Quando dançavam, Álvaro falou-lhe em tom cordial:

--- Posso fazer uma pergunta?

--- Conforme.

--- É bastante simples.

--- Em não se tratando de idade...

--- Riu de mim ou para mim? --- indagou ele.

--- De você e para você.

--- Quer explicar-se?

--- A princípio ri do seu sorriso. Depois, percebi que alguma coisa
muito longe daqui o fazia sonhar em um salão de baile. Achei curioso e
comecei a observar.

--- Psicóloga?

--- O suficiente para me distrair.

--- E tem muitas ocasiões?

--- Algumas. Cumpre observar que não as perco --- tornou com malícia.

--- Com que mais se diverte?

Ela tomou um ar provocante.

--- Não está se adiantando um pouco?

--- Desculpe a minha ousadia. Não tive intenção.

--- Confesso que, até certo ponto, aprecio a ousadia.

--- Gostaria de saber este limite; procuraria não transpô-lo.

A moça não respondeu. Continuaram dançando e, já bem depois, Álvaro
notou que estavam muitos aproximados. Quis tomar atitude mais correta;
ela, porém, o provocava.

--- Onde iremos daqui? --- perguntou o rapaz.

--- Vou dormir; já está na hora.

--- Não quer cear comigo? --- insistiu.

--- Não, hoje.

--- Mas\ldots{}

--- Mas o quê?

--- Depois, eu embarcarei.

--- Vai viajar? Quando?

--- Amanhã, pelo Cruzeiro.

--- Passe antes pela minha casa; vá despedir-se de mim.

\chapter{Capítulo 98}

Bárbara recebeu o telegrama e leu-o com alegria. Ao voltar-se, viu o
portador que, antes de tomar a bicicleta, olhava rindo para ela.
Contudo, não se perturbou; estava muito alegre para isso.

Entrou para a sala de música. Sentando-se ao piano, tocou algumas notas
da ``Valsa do Amor''. Parecia que Álvaro ali estava para acompanhá-la;
Bárbara percebeu com clareza que, desde o primeiro encontro, ele não
mais a deixaria. Involuntariamente, seus dedos procuraram o teclado.
Viu-se, mais uma vez, tocando toda a Valsa do Amor. Aquela música não
lhe enchia a alma como um Beethoven, Mozart, Haydn ou um Bach, mas,
deixara-lhe recordações.

Bárbara amava, e Moszkoweky escrevera a valsa do amor. A composição
daquela valsa tinha um encanto todo próprio; sentia nela a melodia
caprichosa e vibrante em frases expressivas para namorados. Tocara-a
primeiramente com o pai, num arranjo facilitado; depois, com Álvaro. Era
a recordação da infância, do carinho paterno, avivados pelo homem que
lhe conquistara o coração.

Bárbara tomou o telegrama e, segurando-o entre as mãos, deixou-se cair
na primeira poltrona. E sonhou acordada. Lembrou-se dos tempos em que o
deixara ir; no íntimo, porém, sabia que ele voltaria. Álvaro amava-a; e
as circunstâncias daquele amor tornaram-no indestrutível. Eram duas
almas irmãs que o destino separara, provara, para unir depois. Tinham
tudo para a vida em comum. Ele, impetuoso, apaixonado; ela, comedida,
reflexiva. Ele, mais fraco, necessitando apoio; desiludido das coisas e
dos homens, pela vida difícil que atravessara. Ela, mais forte, e pronta
para apoiá-lo sempre. Ambos não tinham pais; mas o lar, desfalecido no
espírito de Álvaro, era uma lembrança viva no de Bárbara. E ainda tinham
o gosto em comum e a alma aberta para as mesmas coisas. Era curioso
notar aqueles dois temperamentos, tão diversos, a sentirem as mesmas
músicas, pararem diante das mesmas telas ou a recitarem os mesmos
poemas. Era perfeito demais, e a vida não reunia assim duas almas, sem
que as tivesse provado intensamente. Mas, valera a pena tais provações!

Agora, a vida apresentava-se diante deles numa só perspectiva ---
vitória. Esqueceram-se, assim, que a vitória muitas vezes é o início de
uma derrota. A perspectiva do fracasso deixara de existir, pelas
atitudes firmes e positivas de Bárbara. Além do mais, ela era mulher
para impulsionar, por amor, um homem na vida. Bárbara estava mais
consciente disso que dos seus encantos pessoais.

\chapter{Capítulo 99}

A despedida não durou uma noite. Passou-se uma semana, e Álvaro
permanecia em São Paulo. Desejava partir a cada dia, mas, ia ficando
como chumbo caído ao solo. Estava no hotel, pensativo, voltando da sua
inconsciência, quando ouviu baterem à porta. Pensou que fosse o
empregado a avisar-lhe a hora já passada do jantar. O rapaz não chegou a
responder e a porta abriu-se. Admirado, pronunciou o nome da visitante:

--- Lola?! Aqui?

--- Passei por acaso e resolvi entrar para vê-lo.

Acaso? --- pensou Álvaro; imediatamente repeliu a palavra.

Uma série de recordações amargas estavam ligadas ao termo que Lola
pronunciara com displicência. Virou-se para ela, quase suplicante;

--- Não mencione esta palavra; pois, tudo na minha vida acontece por
acaso e eu não quero relembrar.

--- Está bem --- respondeu. --- E estas malas? --- indagou indicando-as.

--- São as minhas.

--- Vai partir? --- perguntou com um certo receio.

--- Não as desfiz ainda.

Embora as palavras de Álvaro aparentassem indecisão, disse envergonhado:

--- E foi bom ter deixado assim. Vou partir qualquer dia desses...
amanhã... isso mesmo: vou amanhã --- exclamou mais decidido.

--- Não, Álvaro querido --- disse Lola com uma voz quente, perturbadora.
--- Conheço-o a uma semana apenas; não posso consentir que você parta.

Ele calou-se ante o prazer agitado a lhe percorrer a espinha. Não sabia
o que se passava; sentiu-se atraído pelas formas provocantes, pelas
atitudes sensuais daquela mulher oxigenada. Percebendo o efeito das suas
atitudes, Lola usou os seus recursos de sedução:

--- Álvaro --- chamou.

O rapaz abalou-se ainda mais, ouvindo o seu nome com tanto calor.

--- Há algo que nos atrai --- continuou a mulher --- estamos um para o
outro. Tentemos qualquer coisa, uma vez que tudo na vida é tentativa ---
disse repetindo um conceito ouvido dele mesmo.

As últimas palavras fustigavam o seu espírito. Álvaro teve um
estremecimento repentino. Não dissera isto a Bárbara? Percebendo o seu
abalo, Lola aproximou-se, e atirou-se a ele, beijando-o loucamente. A
agitação dos sentidos venceu a do espírito; aturdido, apertou-a nos
braços.

--- Diga que fica --- insistiu ela.

--- Não; não posso --- resistiu ainda.

--- Fica?

E aquela mulher o provocava, escorregando entre os seus braços e
entregando-se a ele.

--- Não posso. Não... --- e a voz de Álvaro tornava-se menos firme.

--- Fica?

--- Não...

Lola sentia crescer a agitação do rapaz e dela tirou todo o partido
possível.

--- Fica? --- perguntou delirante.

--- Fico --- respondeu mais de desafio que atendendo ao apelo da mulher.

Uma impressão amarga invadiu-o. --- Devo estar louco --- pensou --- mas
não posso resistir. --- E para vencer as apreensões do espírito,
mergulhou-se nos prazeres da carne.

Era o homem velho, não de todo vencido. As cinzas revoltas do passado
acendiam uma pequenina brasa que o tempo deixara ali. As esperanças
pessoais e o desejo apareciam novamente em primeiro plano. Quanto tempo
levara ele construindo, para tão depressa destruir tudo! A paixão era
cega e instantânea; não deixava tempo para a reflexão.

* * *

Era noite adiantada quando saíram para uma pequena ceia. Sob a
influência desse delírio fizeram planos. Lola, não ousando ainda uma
proposta legal, dispôs tudo de forma conveniente para o momento.

Álvaro cedeu. Seus nervos, porém, pareciam tensos como cordas de
violino; tinha a impressão de ter vendido a alma ao demônio. No fundo, a
consciência gritava, mas, ele estava agitado demais para ouvi-la.

\chapter{Capítulo 100}

O salão do Esplanada resplandecia. Todas as luzes, acesas, refletiam-se
nas paredes coloridas. Os automóveis em fila deixavam à porta do hotel o
elemento social de São Paulo. Era um grande baile! E como se relacionava
com a imprensa, escritores e jornalistas receberam convites especiais.

Vestidos riquíssimos faziam da recepção um mostruário de manequins
vivos. Tafetás, tules, organzas, cetins, sedas fantasias, tudo aparecia
ali. As joias clássicas ainda predominavam; os colos nus das senhoras
ostentavam medalhões preciosos. Os rapazes, em traje de rigor, fumavam
na sala de entrada ou se entretinham em algum namoro oportuno. Parecia
bastante animado, e, às primeiras músicas numerosos pares saíram a
dançar. As mãos finas e tratadas das moças sobressaíam na roupa escura
dos rapazes. O ambiente era festivo.

Álvaro levava Lola nos braços; ambos rodavam no meio de toda a gente.
Era magnífico dançar assim em salão amplo e ao som de boa música. Lola
pronunciava palavras de amor ao ouvido de Álvaro; de quando em quando,
punha o seu rosto em frente ao dele, procurando atrair o seu olhar. A
certa hora, aproximou-se mais e perguntou baixinho:

--- Você me ama?

--- Por quê?!

--- Por quê?! --- exclamou admirada --- você ainda não me disse isso.

Álvaro estremeceu. Parou instantaneamente no salão; o seu olhar estava
fixo. Lola, acompanhando-o, encontrou outro olhar não menos
surpreendido. Bárbara estava no baile!

Perturbado, Álvaro não pôde esconder a sua agitação. A presença de
Bárbara acordou-o à contemplação de si próprio. Seu primeiro impulso foi
precipitar-se a ela; refletindo, porém, sentiu vergonha de aproximar-se.
Uma sensação esquisita tomara-o todo: o amor de Bárbara parecia-lhe,
repentinamente, inacessível. Em seus braços estava Lola. O contato com
aquele corpo repugnou-o, e aquela mulher tomou para ele uma forma toda
material. Ali mesmo, pareceu-lhe uma estranha. Todavia, continuou
dançando. Era-lhe imperioso fazer alguma coisa; e, o esforço físico, ao
som da música, tornava-se para ele um refúgio inconsciente.

Lola, temendo a influência da adversária que já descobrira em Bárbara e
que desde logo lhe pareceu forte, não teve mais descanso. Embora
continuasse a dançar com Álvaro, sentia que ele se tornara um autômato,
movendo-se pelas circunstâncias. Veio-lhe a lembrança de que o seu amor
começara num baile e num baile poderia terminar. Não! Ela tentaria tudo
para reter Álvaro junto a si. Sua imaginação trabalhava. Podia lembrar
que já vira a dedicatória do primeiro livro, a uma mulher. Álvaro não
quis entrar no assunto e ela, prudente, deixou para saber mais tarde.
Depois, lembrou-se do jornal que vira na mala de Álvaro; trazia uma
fotografia feminina. Por um esforço de memória, conciliou primeiramente
os nomes, o do livro \emph{e} o do jornal eram iguais. Lembrou-Se da
figura; procurou Bárbara pelo salão e, vendo-a dançar, ligou-a à
fotografia do jornal. Estava tudo explicado. Fora esperta para
descobrir, restava ser esperta para conservar as coisas em seu estado.
Com a sorte em jogo, Lola, procurando armas, não perdera um só gesto de
Álvaro ou de Bárbara. Um olhar não lhe passou despercebido.

Não menos chocada ficara Bárbara diante das circunstâncias. Pensava
surpreender Álvaro com a sua presença, mas foi Álvaro quem a
surpreendeu. Imaginou, sorrindo, como a veria no baile onde ele, como
escritor, teria convite especial. Até a hora em que o avistou, recusara
todas as contradanças, reservando a primeira para ele. Quando começara a
dançar, notara, com indiferença, como atraía a atenção dos rapazes.
Todavia, na sua imaginação, não ocorreu a lembrança de um namoro
momentâneo. A ideia de uma vingança não lhe parecia justa, ainda mais
apoiada em sentimentos alheios. Embora a maneira de se curar um amor
fosse a de substitui-lo por outro, Bárbara permaneceu no baile, isolada
de qualquer namoro. Se tivesse que lutar, lutaria só, apoiada nos seus
sentimentos antigos e contra os sentimentos precários, nascidos de
ímpetos momentâneos. A consciência da sua força impediu-a de que
desviasse para os outros, o que deveria fazer por si mesma. Por fim,
cansada, abatida pela emoção, Barbara retirou-se à saleta reservada e
ali jogou-se numa poltrona.

De todos, Álvaro parecia o mais infeliz. Envergonhado, desejara sumir;
um ímã, porém, o retinha, na contemplação nervosa de um bem que perdera.
Seus olhos buscavam Bárbara; seu pensamento andava com ela. Como pudera
portar-se assim? Teria coragem ainda de aproximar-se?

* * *

Na poltrona da saleta, Bárbara procurava descansar. Tinha a cabeça
reclinada e os olhos numa abstração do entra e sai das moças que vinham
retoucar o penteado, a pintura, ou ainda repetir segredos de namorados.
Certo momento, sentiu-se perturbada no seu descanso. Uma influência
externa, muito próxima, e contrária aos seus sentimentos, atingiu-a com
uma aspereza a que Bárbara não resistiu. Abriu vagarosamente os olhos e
viu diante de si a mulher que dançava com Álvaro. Tinha uma expressão de
rancor e assumia a atitude de desafio. Mediu Bárbara de alto a baixo e
com a voz alterada, os lábios trêmulos, mal pôde pronunciar:

--- Veio procurá-lo?

Fez-se um silêncio prolongado. A atitude de Bárbara não indicava
explicação.

--- Vamos --- tornou ela --- diga alguma coisa. Que veio fazer aqui?

A princípio Lola ia chamá-la por senhora, mas, vacilou e usou a terceira
pessoa sem pronome para no fim, tratá-la por um você ostensivo e
rancoroso.

Bárbara observou-a; pressentiu, sob o batom, os seus lábios roxos de
cólera. Notou a sua expressão alterada pelo ódio e compreendeu que na
atitude daquela mulher grosseira de espírito se patenteava a
inferioridade dos sentimentos. Não podendo lutar em atitude digna,
gritava, debatendo-se nas malhas rasas de uma ofensiva baixa. Não sabia
ainda que os seus gestos, palavras, e todo o ódio, jamais atingiriam o
alvo; seriam como a poeira grossa que ao chão voltava antes de atingir
altura.

--- Por que não me responde? --- perguntou Lola com voz rouca. ---
Julga-se muita coisa e nada mais é do que eu.

As moças em volta estavam já alarmadas com a atitude de Lola. Bárbara,
fitando-a, falou sem irritação:

--- Por acaso, devo-lhe contas dos meus atos?

--- Escute --- volveu a outra --- tome nota do meu aviso; desista de
qualquer tentativa. Você o perdeu para sempre.

---?...

--- Perdeu-o; não se iluda. Ele jamais se afastará de mim; dou-lhe tudo
o que você, no seu egoísmo, negou.

Deu uns passos pela saleta vazia e, falando para si própria, mostrava-se
nervosa e descontrolada:

--- Foi um erro eu me apaixonar. Mulheres como eu não se apaixonam.

Olhou-se ao espelho, afofou o cabelo, e, voltando-se para Bárbara,
repetiu:

--- Saiba que não medirei esforços para conservá-lo. Além disso, eu nada
tenho a perder...

Como Bárbara silenciasse, Lola pôs-se à sua frente e falou numa
irritação crescente:

--- Acho que ainda sou digna da sua palavra. Ele não só fala comigo,
como ainda se entrega todo a mim.

Puxou um cigarro da bolsa; acendendo-o, continuou a falar sem
interrupção.

--- Foi isso --- concluiu finalmente --- quis vender-se muito caro;
agora, perdeu-o para sempre.

--- A senhora joga? --- perguntou Bárbara com calma.

--- Naturalmente --- respondeu admirada, sem compreender a intenção de
Bárbara.

--- Conheci há tempos atrás o famoso pôquer. A vida parece-se com ele;
apresenta, paradas, e estas sobem até um ponto em que o jogador
sustenta. Mas há também as enganosas; pois, as cartadas ora são boas e
ora são más. Está-se sempre na iminência de um \emph{full hand}, mas
também de um parzinho de figuras. Eu é que digo, cuidado! Pôs o seu
objetivo num ponto muito material e, se falha, vai-se o seu castelo.

--- Não entendo o que quer dizer com esse ponto material --- objetou
Lola.

--- Um dia entenderá.

Lola mostrou-se intrigada com as palavras de Bárbara. Não obstante, sob
o domínio da cólera, ouviu-a sem mover uma pestana. Bárbara falava sem
irritar-se, mas, uma tristeza profunda transparecia nos seus olhos
expressivos.

--- Afinal --- indagou a outra contrafeita --- que é essa coisa que eu
um dia entenderei?

Bárbara fitou-a e chegou a compadecer-se daquela mulher. Era como se
lesse o futuro nos seus olhos; tal como a cartomante o faria nas mãos.

--- Alguém afirmou, certa vez --- ponderou Bárbara --- que a beleza é o
primeiro presente que a natureza dá à mulher, mas também o primeiro que
lhe tira.. É bem verdade. É como se ao contemplar um esqueleto, ele nos
dissesse: ``já fui o que és e serás o que eu sou''.

--- Que tenho eu a ver com isso?

--- Essa é a espécie de amor que a senhora cultiva; e, creia, é o amor
que falha sempre.

--- Chega! --- interrompeu a mulher numa atitude brusca. --- Eu não
perguntei essas coisas.

Amassou o cigarro no cinzeiro do toucador e saiu precipitadamente.
Bárbara ficou só outra vez.

De que combinação era feita a vida!... Embora atingida no campo da luta,
Bárbara ainda permaneceu de pé para contemplar de onde lhe vinha o
ferimento. Uma dor profunda, indefinível e localizável magoava-a muito.
Tinha capacidade para sofrer, mas, àquele golpe, necessitava reunir
todas as forças para persistir. Antes de qualquer esperança no seu amor,
ela rebateria o lance com maior força. Agora que o destino brincava com
ela, fingindo dar, para tirar depois... Ah, isso era demais! Na flecha
que o cupido lhe atirara havia um veneno terrível. Não o veneno mortal,
instantâneo; e sim, destes que, não causando a morte imediata, deixam o
espírito lúcido para presenciar ainda as lutas da vida.

\chapter{Capítulo 101}

Se janeiro começara cheio de promessas, fevereiro não prognosticava as
mesmas coisas. O navio deixou o porto de Santos, naquela tarde morna,
quando o crepúsculo já se intensificava no horizonte.

Bárbara seguia com destino ao Rio. Há um ano atrás, fizera esta mesma
viagem; no entanto, em circunstâncias tão diversas. Um ano se passara,
e, com os dias deste ano, todo o ritmo da sua vida se alterara. Era como
se tivesse passado de uma valsa para uma outra peça de compasso
quaternário, cheio de notas sincopadas. Quando Bárbara veio ao convés, a
noite avançava em horas; olhou a amplidão escura, sobre a qual o navio
seguia a sua marcha. Sentiu-se como o navio; este, cortava as águas sem
conhecimento palpável do local, apenas orientado pela bússola. Assim
Bárbara --- estava num terreno em trevas, guiada apenas pela sua luz
íntima, ainda não extinguida de todo.

Olhou em volta; tudo lhe pareceu parado, embora a marcha do vapor não se
alterasse. Andou pelo convés como o fizera na viagem de um ano atrás.
Aquele ser que se radicara na sua alma falava-lhe em todos os seus
passos e atitudes. Contemplando as águas quebradas em torno da cidade
flutuante, Bárbara sentiu-se etérea, volátil, como as espumas brancas
que encimavam as águas. Sabia que Álvaro a amava e deixara-o em S.
Paulo, no caminho de todos os vícios. Compreendera que ele cedera à
fraqueza, companheira fiel do homem livre na sociedade. Sabia que a vida
estragada de muitos anos o deixara indiferente à posse de uma mulher sem
amor; o grito do sexo o empurrara para um corpo de encantos fêmeos. E o
deixara nesses caminhos tortuosos, afastando-se para aumentar-lhe a
confusão. Tudo isto ela percebia e tudo isto lhe doía na consciência.
Embora ele tivesse renascido a seu lado, ia permitir que morresse outra
vez. Como era difícil arrancar, com as raízes, o homem velho, para só
deixar viver o homem novo. Que aconteceria a Álvaro? Tudo isto lhe vinha
à mente, castigando-a, como se ela ainda fosse a culpada. Como é o
homem! pensou Bárbara, quando parecem vencidos todos os obstáculos, ele
mesmo empurra o fardo que, muitas vezes, o esmaga para sempre. Uma outra
Bárbara levantou-se no seu íntimo. Entrou em diálogo consigo mesma:

--- Por que o deixou? Não o amava? Devia enfrentar aquela mulher;
arrancá-lo dela não seria difícil.

--- Que adiantaria tirá-lo? Ele voltaria, eu sei, mas seria o mesmo
Álvaro que eu amei?

--- Se não fosse, você o reconstruiria. A seu lado, ele esqueceria a
outra; ou antes, a outra nem existe realmente para ele.

--- O que me impressiona não é a outra; é o que pode ficar dessa outra.

--- Não conta com a sua influência?

--- Não quero só influenciar; quero também ser influenciada.

--- Bem, deixou passar, agora aguente. Estava tudo tão na hora.

--- Estava mesmo.

--- Vai deixá-lo?

--- Vou.

--- Por ele?

--- E por mim também.

Como é a vida! pensou, encerrando o seu diálogo e unindo em si as duas
Bárbaras; o pouco que dá, ainda, algumas vezes, tira. Mas, tiraria
mesmo? Poderiam desaparecer aquelas horas de suave intimidade que os
uniu? E as suas almas, afinadas pelo mesmo diapasão, cantando as mesmas
belezas e sentindo as mesmas emoções? Não, completou Bárbara, o que se
vive no espírito jamais se perde.

E era verdade. Bárbara, tinha razão.

\chapter{Capítulo 102}

Álvaro chegava ao hotel já irritado. As mínimas coisas o faziam perder a
paciência; fosse um gesto de Lola, um objeto que não achasse na hora
precisa, a campainha do telefone... Nada o deixava em paz. A tudo Lola
procurava solução, tendo em mente agradá-lo sempre; fazia por ele o que
não fizera por homem algum. Esforçava-se por compreendê-lo, ajustá-lo a
si. Empenhada nessa luta, não alcançava até que ponto iam as lutas no
ânimo de Álvaro.

Logo que ele entrou no quarto, levou-lhe o cafezinho da hora. Depôs a
xícara a seu lado, na escrivaninha:

--- Tome meu amor, fiz agora mesmo.

O rapaz mostrou-se contrariado com aquele excesso de carinho e tomou a
xícara mais para fazer alguma coisa no momento. Sorveu o primeiro gole e
lembrou-se logo de Bárbara. A princípio, procurava escrever à custa de
excitantes alcoólicos; Bárbara, porém, o acostumara a usar o café.
Animado por ela, chegara ao auge da sua carreira; agora, como lhe
custava sustentar esse auge! Os jornais reclamavam a sua colaboração
contratada e Álvaro não dava um passo para satisfazer seus compromissos.
Seu dinheiro ia nas despesas diárias, e o regresso a uma vida de
misérias, que lhe parecera tremenda ao lado de Bárbara, era agora
indiferente; como se passar fome sozinho nada tivesse de anormal. Álvaro
rabiscava inúmeros artigos, todas as noites. Enchendo o cesto de papel,
terminava deixando para o amanhã toda a sua tarefa. Era a insígnia do
homem vencido. Tornava-se penoso para Álvaro sustentar uma situação
falsa, quando a dignidade ainda gritava.

Lola achegou-se a ele e sentou timidamente a seu lado. Mal ele a notou,
levantou-se contrariado. Tomou o café e, sem dizer nada, abriu a porta
num gesto brusco e saiu. Desceu a escada, apressado. Ao passar pela
portaria, ouviu o seu nome com surpresa. Alguém o chamava e voltando-se,
Álvaro não pôde conter a satisfação:

--- Paulo, meu velho! Você aqui!

--- Até que enfim --- exclamou o amigo numa alegria franca.

E os dois rapazes saíram abraçados pela rua.

--- Que homem difícil! Desde manhã, procuro-o por toda parte --- começou
Paulo.

--- Como soube onde eu estava?

--- Fui ao jornal. De lá, tentei falar pelo telefone; mas você não quis
atender.

--- Deu o nome? --- indagou Álvaro.

--- Não.

Álvaro compreendeu e calou-se.

\emph{---} Bem, vamos ver Helena --- disse Paulo desviando a conversa.

--- Helena --- repetiu maravilhado. --- Onde está Helena?

--- No hotel. Vamos jantar todos juntos, Álvaro. Recordar a velha
camaradagem.

--- Depois, poderemos ver um cinema? --- convidou Álvaro. --- Há bons
filmes por aí.

Os dois amigos dirigiram-se ao hotel em que Paulo se hospedava com a
esposa. Pelo caminho veio o assunto dos livros.

--- Então, Álvaro, qual será o próximo? --- perguntou o outro, alegre.

--- Que próximo?

--- O próximo livro, rapaz. Pois não estou falando com um escritor?

Ele sorriu com amargura; uma nuvem sombria cobriu-lhe a expressão.

--- Não haverá mais próximo; encerrei a carreira.

--- Está louco? --- indagou o amigo numa surpresa desconcertante.

--- Voltei ao que sempre fui --- completou. --- Às pessoas como eu, não
devem tentar coisa alguma na vida.

--- Álvaro, você a dizer isso!

Álvaro o encarou:

--- E por que não eu? As vitórias que alcancei não foram por minha
própria conta. Mas... para que pensar nisso? Estou liquidado para
sempre.

--- Bárbara? --- pronunciou Paulo.

--- É --- respondeu com tristeza --- mas como o soube?

--- A vida do escritor é pública, meu amigo.

--- Maldita vida esta, Paulo.

E a alegria de há pouco se desfazia; os dois amigos, porém, sentiam-se
mais unidos, uma vez que um deles sofria.

--- Álvaro --- chamou o outro --- volte para Bárbara.

--- Desejo isto como um louco; mas, e a vergonha da minha fraqueza? Não,
Paulo, eu não mereço coisa alguma.

O rapaz permaneceu calado, atingido pela desgraça do amigo.

--- Ela convenceu-me de que eu era bom --- continuou Álvaro --- hoje sei
que não valho nada.

--- Álvaro --- interrompeu o outro --- não seja assim pessimista.
Bárbara o ama, e você sabe disso.

--- Mas não posso abusar desse amor. Não perdoo a mim mesmo, Paulo, tudo
o que fiz. Ela é que veio pegar-me no delito; não tive jeito nem de me
confessar culpado. Agora, devo ser tratado como criminoso.

Paulo estava aflito por aquela atitude passiva do amigo. Desejava
livrá-lo das condições más do momento. Álvaro, entretanto, parecia
vencido. Estavam quase a chegar, quando Paulo perguntou inesperadamente:

--- E você, sente algum atrativo por essa mulher?

--- Nenhum --- falou sem reserva.

--- Então por que...

--- Continuo com ela?

--- Sim.

--- Francamente, não sei. Talvez seja ela que continue comigo. Na
verdade, é dessas coisas para as quais não se encontram explicações.

--- Bem, chegamos. Estamos aqui --- disse Paulo parando em frente a um
edifício.

E vendo que Álvaro continuava absorto, segurou-o pelo braço:

--- Onde vai?

--- Ah! É aqui? --- perguntou o outro surpreendido.

Entraram no hotel e, logo na sala de espera, viram Helena, folheando uma
revista.

--- Você a reconhece? --- volveu Paulo, contemplando a esposa.

Álvaro fitou a esposa do amigo. Era a mesma Helena, reconhecia-a; mas,
como mudara! Assim que Helena avistou-os, veio alegre para eles.
Cumprimentou Álvaro e lhe fez perguntas de interesse amigo. Olhando para
ela, Álvaro notou a sua expressão de felicidade. Tinha o olhar tranquilo
e as feições calmas, seguras. Circundava-a essa ternura de amante e
companheira que Paulo soubera conquistar. Álvaro percebeu, desde logo,
que não existia entre eles esses carinhos estéreis, superficiais, que
tanto enganam os casais de hoje. Paulo e Helena encontraram-se, e
estavam seguros do seu amor. Havia ali a fusão de duas vidas formando
nova personalidade. Observou como Helena estava bonita! Ela sempre
tivera encantos e agradava pelas suas feições delicadas e frágeis;
agora, tornara-se mais consciente e libertara-se dos complexos do
passado. Paulo conseguira tudo, até desenvolver a beleza da esposa,
pensou Álvaro. E por que Helena mudara assim? --- Porque a felicidade
dera-lhe mais confiança na vida. Dessa maneira, a sua fisionomia perdera
aqueles traços duros que Álvaro vira uma vez. Sim, Helena tinha mais
encantos e parecia mais jovem. Ao lado de Paulo estava Helena, sua
esposa.

\chapter{Capítulo 103}

Paulo, Helena e Álvaro foram jantar num restaurante da cidade.
Sentiam-se alegres por estarem juntos; a camaradagem existente entre os
dois amigos deixou-os logo muito a gosto. A refeição decorreu
normalmente e, logo após, saíram a pé pela cidade. Paulo era
\emph{mackenzista} e desejava ver as realizações dos colegas.

Ao dobrarem a esquina da rua Marconi, uma vendedora de flores
aproximou-se:

--- Cravos, amor-perfeito, heliotrópios, ca...

--- Quero alguns heliotrópios --- disse Álvaro interrompendo a mulher.

Ela escolheu, entre os ramalhetes, o mais bonito. E quando percebeu que
o rapaz deixava o troco, falou comovida:

--- Deus lhe pague; e lhe dê uma linda mulher.

Paulo e Helena sorriram ante as palavras da vendedora de flores,
combinando a ideia de Deus com a da mulher. Em outras circunstâncias,
teriam conversado a respeito, mas, o momento inoportuno fez com que
todos silenciassem. Álvaro contemplou o pequenino ramalhete e, as
florzinhas roxas, aveludadas, desprenderam o mesmo aroma\ldots{}

Fora no arrabalde silencioso, ele bem se lembrava, que Bárbara tirara do
bolso o lencinho perfumado. --- Em março, disse ela, uso também a flor.
--- Movido pela lembrança do passado, Álvaro voltou-se para a maior
amiga de Bárbara:

--- Permita-me, Helena?

--- Oh, muito agradecida --- respondeu a moça tomando os heliotrópios.

E não mais fez perguntas, pois, Helena compreendera o porquê do gesto do
rapaz. Álvaro tornou-se mais quieto. Dali por diante, quando lhe
dirigiam a palavra, respondia por monossílabos. De vez em quando dizia
distraidamente --- ah sei.

Em frente a um prédio grandioso, Paulo deteve-se em contemplação:

--- É uma obra de arte! --- exclamou. --- É uma peça de arquitetura!
Gigantesco! Que linhas, sóbrias e admiráveis.

--- É um arbusto --- disse Álvaro --- pelo menos foi o que me disseram.

Voltaram-se admirados para ele.

--- Você não sabe, Helena, se o heliotrópio é mesmo um arbusto? No tempo
indicado, quero adquirir alguns.

Prosseguiram o caminho em silêncio. O ambiente estava mudado; nenhum dos
três mostrava a mesma alegria inicial. O cinema fora esquecido e, não
obstante passassem por muitos, a ideia não surgiu mais. Quando chegaram
à porta do hotel, Paulo, que não aceitava a derrota do amigo, fez uma
tentativa para salvá-lo do presente da situação:

--- Vamos, Álvaro? --- perguntou a animá-lo.

--- Para onde? --- tornou o outro indiferente.

--- Para o norte. Você ainda não conhece o norte. Que boa oportunidade
agora; embarcaria conosco.

--- Vai voltar para o norte?

--- Por três meses apenas. Vou concluir um serviço e, então, virei de
mudança para S. Paulo.

--- Está já decidido?!

--- Está. Embora em S. Paulo vençam os capitalistas --- explicou o amigo
--- é o lugar onde ainda se encontra remuneração razoável para
profissionais competentes.

--- E você vem explorar o campo? --- continuou Álvaro, interessando-se.

--- Explorar, não. Já sou casado, Álvaro, e não tenho reservas para
explorações na profissão. Assinei contrato com a firma X. Vou tentar a
vida.

--- Mas você é tão seguro das suas aptidões! Foi uma pessoa que sempre
lutou; ninguém estaria tão em condições para uma tentativa, Paulo.

--- A minha segurança profissional dá para garantir um bom emprego.
Agora, Álvaro, não vou expor a minha mulher a privações, confiando num
futuro remoto. É a tal coisa: enfrentar um mal certo, na busca de um bem
incerto.

Helena apertou carinhosamente o braço do marido. Álvaro sorriu à
lembrança do velho Paulo, considerando os problemas e ligando-os a
teorias filosóficas.

--- Bem, Álvaro, vai conosco? --- inquiriu Paulo na sua voz firme e
decisiva.

Ele pensou um instante:

--- Vou; vou --- repetiu. --- Quando parte?

--- Amanhã.

--- Certo; irei com você.

--- Bravo, amigo --- bradou Paulo --- e rumo ao norte.

Álvaro despediu-se do casal e tomou o caminho para o seu hotel. Ia
preparar as coisas, e rumo ao norte. Mas que lugar do norte? Paulo
construía estradas; vivia como um judeu errante. Onde iria ele agora?
Achando graça na sua viagem sem direção, notou que tinha chegado. Entrou
sem fazer ruído e alcançando o seu quarto, deu volta ao trinco. A porta
não estava fechada a chave; percebeu, então, que Lola o esperava.
Atravessou a sala e foi sentar-se na escrivaninha, ao lado da janela.
Vendo-o entrar, Lola dirigiu-lhe a palavra. Álvaro não respondeu.

--- Álvaro --- repetiu a moça --- estou falando com você.

Ele não deu mostras de ter ouvido o chamado. Lola achegou-se por trás e,
encostando-se nele, perguntou:

--- Vamos dar uma volta? São onze horas ainda e a noite está...

--- Que perfume é esse? --- interrompeu bruscamente o rapaz.

--- Heliotrópio. Está nos seus lenços; quis usá-lo para agradar você,
meu bem.

--- Pois não me agrada nem um pouco. É suave demais; não condiz com
você.

--- Condiz com a outra --- objetou Lola na sua voz trêmula de rancor.

--- E que seja. Você tem alguma coisa com isso? --- volveu numa
irritação crescente.

--- Tenho --- gritou ela.

--- Pois não terá mais daqui por diante. Ainda tenho brio para
separar-me de você.

Conquanto a revelação lhe fosse um abalo, aquela mulher não disse mais
nada. Percebia já que tudo estava ao fim, não obstante os seus esforços
para continuar. Afastou-se dali silenciosa, e foi juntar as suas coisas.

Enquanto isso, Álvaro sentou-se à escrivaninha com a garrafa e o copo;
bebeu uma dose de aguardente de mel, pura, e tomou alguns papéis para
iniciar um ensaio.

Como penetrantes cutiladas, as palavras de Bárbara voltaram à lembrança
de Lola: ``essa espécie de amor, creia, é o amor que sempre falha''.
Atordoada, ela procurou esquecer o ocorrido; como fora inconsciente nas
suas esperanças. Lola começou a dobrar as suas roupas numa fúria
incontida; era necessário abafar as recordações do passado, para não
crescer ainda mais a sua dor. Lola, entretanto, sabia, por experiência
própria, que não se esquece determinadamente as coisas. ---``Creia,
creia...'' parecia Bárbara a falar-lhe outra vez. Andou mais depressa,
para fugir à lembrança daquele aviso; todavia, era-lhe impossível fugir
à sua própria memória. As palavras continuavam a gritar-lhe no ouvido; a
realização de um fato previsto por outrem, aumentava-lhe o peso das
desgraças. Ainda mais quando esse outrem era a sua própria rival.
Aturdida, Lola tapou os olhos com as mãos; seu corpo tremia, agitado
pelas reações nervosas insustentáveis. Sentou-se à beira da cama e,
procurando controlar-se, respirou profundamente. Depois levantou-se e
quis apressar-se ao máximo; era urgente que saísse dali.

Assim, atormentada, levou tempo para reunir os objetos que trouxera.
Procurava-os em lugar errado; e quando buscava um, deixava outro. A
muito custo, teve as suas coisas arrumadas. Embora a mala não fosse
pequena, precisou carregá-la sozinha; pois Álvaro não se moveu.

Lola abriu a porta... eram quase duas da manhã. Olhou para trás e viu-o
sentado à escrivaninha. O nível da aguardente descera muito na garrafa
e, pelo estado sonolento de Álvaro, percebeu a sua embriaguez.
Contemplando-o, involuntariamente, repetiu ainda as palavras de Bárbara
--- ambos fomos vítimas de uma parada enganosa. --- Lola compreendeu que
num jogo, onde se tem a vida por parceira, os lances são mais
arriscados; as vitórias mais incertas.

\chapter{Capítulo 104}

Quando Paulo veio à procura de Álvaro pela manhã, encontrou-o ainda do
mesmo jeito. Com os braços sobre a mesa de trabalho e deitado por cima
de papéis, Álvaro dormia. Ao lado, a garrafa vazia, e o cheiro ainda
forte da especial aguardente de mel. Folhas esparsas estavam escritas.
Paulo apanhou-as no estilo agradável de Álvaro, percebeu a tentativa de
um ensaio sobre o pessimismo humano. Mas, embora o estilo agradável,
observou também a falta de unidade. O assunto estava entrecortado;
Álvaro não conseguira manter um objetivo. Desviava-se da ideia
primordial; mais compilou citações de fatos que se prendeu a uma
argumentação de tese. De quando em quando, porém, um pensamento notável
revelava o talento perdido. Com que tristeza Paulo compreendia tudo
aquilo; e, olhando para Álvaro, comentou baixinho:

--- Estamos com a sorte invertida, Álvaro amigo. Que poderia fazer eu
por você?

--- Hein? --- respondeu o outro, levantando a cabeça. --- Disseram
alguma coisa... eu ouvi.

Paulo não chegou a responder; pela porta aberta do quarto, Helena entrou
apressada. Olhou para o relógio, a fim de contar as horas; vendo o
marido cabisbaixo, triste, aproximou-se, sem dizer palavra. Compreendeu
que não embarcariam, deixando Álvaro naquelas condições; e, silenciosa
ainda, passou a mão pelo braço de Paulo. Ele apertou a mão da esposa;
diante da infelicidade do amigo, quis ter mais certeza de que era feliz.
Estar, agora, ao lado de Helena, significava mais que estar ao lado da
mulher amada. Diante do infortúnio, o bem que nos resta parece ainda
maior.

Álvaro ergueu a cabeça e, esticando os braços, procurava tirar-lhes a
sensação dormente. Acendeu logo um cigarro; deu a primeira tragada.
Voltando-se, viu seus amigos. Profundamente envergonhado, diante de
Helena, não encontrou palavra para desculpar-se; o mais que pôde, foi
dizer um ``bom dia'' quase inarticulado. O servente bateu à porta e
entrou a seguir. Helena veio receber a bandeja e colocou-a sobre a
escrivaninha; aproveitou a ocasião para reunir os papéis e tirar a
garrafa e o copo dali.

--- Sabe --- disse Álvaro com voz cansada --- eu estava sonhando com
Bárbara. Traduzíamos um livro de física.

O rapaz parou um momento, indeciso na continuação.

--- Física? Mas você é advogado --- objetou Helena.

--- É verdade --- tornou ele rindo --- mas quando estudei filosofia,
voltei à física e à matemática. Seu marido foi o meu professor. E que
professor! --- exclamou, olhando para Helena.

Paulo sentia-se muito acabrunhado para sorrir diante dos elogios do
amigo. Helena perguntou com interesse:

--- Que foi feito do livro? Saiu boa a tradução?

--- Ah! --- disse ele --- não continuamos a tradução.

--- Não?!

--- Não.

--- Por quê?

--- Bárbara não gostou do nome do autor.

--- Quem era?

--- Não sei; não havia interesse em verificar, uma vez que não a
agradava.

--- E então?

--- Então, resolvemos dar uma volta. Saímos pelo rio afora e Bárbara ria
satisfeita. Seus cabelos eram jogados para trás pelo vento que soprava.
Notei como estava linda. Tão graciosamente feminina; fazia todos aqueles
gestos que eu conhecia muito. Trazia calças brancas de lã, presa ao
tornozelo, e uma blusinha esportiva, subida ao pescoço. Aliás ---
ponderou ele na atitude de quem recorda --- todos os seus trajes eram
assim.

--- Assim como? --- indagou Helena.

Álvaro sorriu com saudade:

--- Assim, Helena. Bárbara não usava vestido decotado. Dizia sempre que
não era do seu agrado.

Pela primeira vez, desde a sua entrada no quarto, Paulo sorriu; achou
graça em ver o amigo lembrando trajes femininos.

--- E você sabe, Helena ---continuou Álvaro --- esse era o seu traje
costumeiro. Dessa forma passava o dia lendo, escrevendo, pensando ou
ouvindo música. Eu fui encontrá-la muitas vezes de calças compridas. Eu
não gostava disso, mas em Bárbara era tudo tão diferente.

Álvaro silenciou. Helena, vendo a bandeja como viera, arrumou-lhe uma
xícara de café. Entregando-lhe, perguntou:

--- E como foi o resto do sonho?

Ele recebeu a xícara da mão de Helena e, tomando um gole, prosseguiu a
narrativa:

--- Não me lembro bem; mas notei que o pessoal não estranhava o traje
esportivo de Bárbara. Quando prestei mais atenção, vi que o lugar por
onde andávamos era tão branco como os trajes de Bárbara. Percebi, então,
que estávamos num campo de neve.

--- Neve? Você fala em neve; esteve acaso, na Europa?

--- Não. Que coisa interessante, Helena! Eu nunca vi um campo de neve;
no entanto, sonhei estar andando sobre ele. Sempre julguei não ser
possível ver nos sonhos coisas, assim, desconhecidas; mas vi a neve e
não me admirei nada por isso.

--- Talvez --- ponderou Paulo --- por uma impressão cinematográfica; ou
mesmo por descrições de Bárbara. Ela conhece a neve.

--- Talvez, mesmo --- concordou Álvaro. --- Até que ponto Bárbara
penetrou na minha vida! Fazer-me ver as coisas pela sua força
descritiva.

Parou de falar, passou a mão pelos cabelos em desalinho e começou a
tomar o café que Helena pusera na sua frente. Como se mostrasse vencido,
na atitude de quem tudo perdeu, Helena arriscou uma pergunta:

--- Álvaro --- disse com um interesse afetuoso --- já uma vez aconselhou
Paulo a lutar; por que não faz uso, agora, dos seus próprios conselhos?

Ele não reagiu; voltou-se para ela, desanimado.

--- O homem é assim mesmo, Helena; aconselha sempre, raramente age. Eu
não tenho mais coragem para vê-la --- declarou com amargura. --- Querer
Bárbara é querer demais.

--- Mas, você concluiu isso somente agora? --- inquiriu Paulo.

--- Um pouco tarde, mas ainda a tempo --- observou.

--- Tem consciência do que diz? --- insistiu o amigo.

--- Por que, Paulo? Acha que posso arrastá-la a uma escala inferior? ---
perguntou, mirando-se a si mesmo.

--- Mas pode subir, Álvaro.

--- Eu tenho medo, além de tudo --- objetou. --- Não; não quero pensar
nisso.

--- Medo de que, meu amigo? --- continuou o outro como se não ouvisse as
últimas palavras dele.

--- De que ela se sacrifique por mim. Não sou homem tão desprezível, a
ponto de aceitar o sacrifício de uma mulher.

--- Tem medo da compaixão? --- disse Paulo, completando o pensamento do
amigo.

--- Isto mesmo. Você é um homem como eu; acha possível explorar o
sentimentalismo de uma mulher?

--- Não tenha receio, Álvaro --- tornou Helena, respondendo pelo marido.
--- Conheço Bárbara. Sei que ela o desprezaria, se percebesse que você
seria homem para aceitar o sacrifício dela nesse sentido. Por compaixão,
ela não cederia, apenas teria desprezo e repulsa.

--- Helena, escute... olhe... --- e tomou a atitude de quem ia dizer
alguma coisa; mas, de repente, baixou os olhos e apenas falou: --- não
adianta. Para que voltar as vistas ao horizonte inalcançável?

Dizendo isto, levantou-se com dificuldade da cadeira que ocupara toda a
noite. Deu alguns passos pelo quarto e foi à janela. Encostou-se a um
canto do parapeito; olhando o dia claro que começava com tanta vida,
tomou uma expressão de maior tristeza. Esquecido de Paulo e de Helena e,
contemplando aquela luz, pôs a mão no bolso e começou a falar consigo
mesmo:

--- Quem sou eu, afinal? Alguém que sofreu, lutou e, certa hora, pareceu
vencer. Que vitória amarga! Talvez tenha sido construída sobre a morte
da outra mulher; talvez... não --- disse interrompendo-se. --- Estou
divagando, e, como todo o homem, procurando desculpas para as minhas
fraquezas. Mas, serão realmente fraquezas? --- Parou um pouco; olhou
para fora e tornou a perguntar-se --- não permite a sociedade tais
fraquezas? E não é, principalmente, ao sexo forte que se permite ser
fraco?

No passado, conversara com Bárbara sobre tais leis sociais. Dissera-lhe
ela, que ao sexo forte eram permitidas todas as fraquezas e ao sexo
fraco, exigidas todas as forças. Por quê? Não era uma injustiça? Teria o
homem meios de defesa mais seguros? Usando dessa pseudoliberdade, não
destruíam as leis mais sagradas da natureza? E onde estava a sua força?
Na destruição?

Álvaro calou-se, esmagado pela verdade que acabara de compreender. A
liberdade social dava ao homem a força para destruir! E os homens não
compreenderam que esta força era fraqueza.

--- Ah Bárbara --- tornou ele --- é diante de você que me julguei. Que
importa a sociedade? Enquanto vivi nela, nunca encontrei o bom caminho.
Não tenho mais coragem para vê-la e quisera que nunca me visse assim.

Silenciou novamente, olhando ao longe como quem procura alguma coisa.
Paulo e Helena viam-no de costas e, sofrendo com ele, ouviam as suas
reflexões. Estavam ali prontos para fazer tudo e não havia quase nada
que dependesse deles.

--- Álvaro --- chamou Paulo.

Ele voltou-se ao chamado. Tinha um semblante de angústia, mas tentou
ainda sorrir. Sem firmeza, deu os primeiros passos. O amigo,
aproximando-se dele tomou-o pelo braço e conduziu-o para fora. Helena
acompanhou, andando vagarosamente ao lado do marido. Álvaro deixou-se
conduzir; sentia-se cansado para lutar ainda.

***

Era um belo dia do mês de março. Uma brisa leve agitava o arvoredo.
Bárbara estava no jardim examinando as flores que surgiam e passando em
revista as suas plantas prediletas. Estendeu o olhar e viu lá em baixo a
cidade que se espraiava preguiçosamente. O mar, esbravejando como um
epilético, lançava as suas espumas brancas na praia. As montanhas de
Santa Tereza pareciam brinquedos de criança, com aquelas casinhas em
alturas diversas. Bárbara olhou uma vez mais o panorama em conjunto e
entrou. Mal chegara ao quarto, a empregada veio comunicar-lhe que uma
moça procurava ela. Olhou-se ao espelho e, julgando-se em ordem,
dirigiu-se à sala. Mal entrou, pareceu-lhe que o coração parou de
repente; tal a surpresa que lhe causou a visitante. As duas mulheres se
olharam. Antes que Bárbara pronunciasse uma palavra, a outra falou:

--- Não esperava por isso, não é?

--- Confesso que não --- respondeu.

--- Posso sentar-me?

--- Naturalmente. Desculpe a minha impolidez --- tornou Bárbara
explicando o esquecimento.

Lola tomou a primeira poltrona e, sentando-se, ergueu o véu que trazia
sobre o rosto. Parecia muito abatida, embora a sua expressão ainda
denotasse firmeza. Tirou uma das luvas e, olhando para as mãos, começou
a falar com dificuldade.

--- Compreendi em parte o que me disse, e vim para dizer-lhe que tem
razão.

O nervoso transparecia no timbre da sua voz alterada e Bárbara,
conquanto silenciasse, mostrou-se pronta para atender a adversária. Lola
sentiu-se mais animada. Levantando os olhos para a rival, procurou
explicar os seus sentimentos.

--- Não sou tão má como pareço. As mulheres como eu acabam perdendo a
sensibilidade, mas, não chegam a essas megeras que se descrevem por aí.
A senhora não acredita nisso?

--- Perfeitamente --- disse Bárbara, observando o quanto sofria aquela
mulher.

Lola olhou em volta, como a dar tempo para reunir mais forças. Depois,
continuou.

--- Dizem que nós somos as mulheres de vida fácil. Se as pessoas
soubessem como a vida para nós é difícil?... --- Fez um intervalo e
prosseguiu: há muitas mulheres na conceituada sociedade que não nos são
superiores; pois, perderam igualmente a sensibilidade. A sociedade não
as condena porque a perderam em outro sentido. Também não acha isto?

--- É verdade --- tornou Bárbara com firmeza.

--- Para mim, pois --- continuou a mulher --- não existe diferença entre
estas que continuam damas de família e as da minha classe. Ambas
embruteceram a alma; perderam a sensibilidade neste ou naquele setor.

--- Os setores podem diferir --- interveio Bárbara --- mas a
sensibilidade é uma só.

--- E a senhora concorda que existem nas suas relações mulheres assim?

--- Plenamente.

--- Assim --- repetiu Lola --- não serei indigna por vir procurá-la.

Bárbara contemplou-a; não era a mesma mulher do Esplanada. A sua
expressão ameaçadora de antes, desaparecera; no lugar desta, uma outra,
misto de astúcia e compreensão de mulher vencida e infeliz.

--- Conhece muitas mulheres? --- perguntou Lola interrompendo os
pensamentos de Bárbara.

--- Algumas --- volveu um pouco admirada.

--- Naturalmente, tem visto os tipos mais diversos --- comentou a outra.
--- Viu algum que se assemelhasse ao meu?

---?!

--- Teve contato com alguma mulher que tenha perdido a sensibilidade,
assim como eu? --- disse Lola novamente.

--- Que tenha perdido a sensibilidade, sim --- falou Bárbara.

--- E como eu?

--- É uma pergunta difícil para se responder tão depressa.

Bárbara calou-se, percebendo que a outra desejava falar.

--- A senhora notou já, com a força da sua observação, que a
degenerescência está em toda a parte, não é?

Sem esperar resposta continuou:

--- Em certas classes de senhoras que se impõem ao respeito, há o namoro
discreto; em outras, há o prazer do baixo comentário; mas a senhora vê
que nem uma nem outra se firmam em um princípio virtuoso ou em sua
própria honra. É apenas um zelo calculado pelo nome que apresentam na
sociedade. Permitam as leis que elas ultrapassem certos limites e a
honra de antes rolará por terra.

--- A que pretende a senhora chegar?

--- Que não sou tão desprezível quanto me tacha a sociedade. Outras
mulheres honradas o são tanto quanto eu.

--- Por exemplo --- disse Bárbara com seriedade.

--- Estas de que falei há pouco. Que se comprazem em assuntos baixos,
realizando, mentalmente, coisas que elas mesmas consideram de baixa
esfera moral.

--- Há nisto alguma alusão pessoal?

--- Não senhora --- respondeu imediatamente. --- Quero desfazer a
impressão de ousadia da minha parte, vindo a sua casa.

--- Desde o primeiro momento, não interpretei dessa forma.

--- Percebeu igualmente que eu não vim para enfrentá-la?

Bárbara assentiu com a cabeça,

--- A senhora não é dessas mulheres que se enfrentem. Eu mesma não sei
ao certo a que vim. Tive um desejo louco de vê-la; e, no fundo, eu sabia
que a senhora me receberia. Contava com a sua compreensão.

--- E se puder ser-lhe útil em alguma coisa...

--- Obrigada --- interrompeu Lola --- Já foi, recebendo-me e conversando
comigo.

A empregada entrou na sala e serviu o costumeiro cafezinho. Lola
devolveu a xícara, marcada pelo batom arroxeado; depois, tirou da bolsa
a cigarreira dourada e abriu-a diante de Bárbara. Vendo que esta
recusava, por não ter o hábito de fumar, pediu-lhe licença ainda antes
de acender o cigarro; como o faria uma pessoa de finas maneiras. Entre
as baforadas, Lola prosseguiu na conversa; como se descuidasse, avançou
um pouco mais:

--- Essas mulheres que me desprezam --- ponderou com certa indiferença
--- são igualmente desprezíveis. Elas é que não se conhecem, não sabem o
que realmente são. Mulheres incapazes de um sacrifício por deliberação
própria; são protegidas das circunstâncias. Que seria delas se a lei não
lhes assegurasse um marido? Quando as vejo na sua insignificância,
prefiro a minha sorte; pelo menos tenho consciência da minha força e
ainda posso lutar. Elas pensam que não, mas eu sinto que sim.

Bárbara permaneceu no mesmo silêncio diante da visão ampla daquela
mulher.

--- Nós somos como as outras; não nos entregamos a esta vida, de caso
pensado. Um rapaz que nos engana, que nos explora o sentimentalismo, nos
promete casamento; um abandono, a falta da família ou de um afeto, mesmo
uma fraqueza momentânea e, quase sempre, a falta de dinheiro; daí o
início das nossas quedas. Uma moça arrojada nos seus namoros, que se
entrega a carícias perturbadoras, age da mesma forma. As mulheres que se
comprazem nos assuntos sexuais também não estão longe da nossa classe.
Em tudo isto, porém, há as circunstâncias que beneficiam algumas e
outras atiram na lama. As mulheres que se entregam a essa vida por uma
perversão de sentimentos, portanto já com ideia predeterminada,
constituem uma porcentagem diminuta. Por isso que muitas mulheres
desvirtuadas voltam ao nível normal quando encontram alguém que as
ampare. --- Olhando nos olhos de Bárbara, ponderou com vagar --- são
dois extremos as classes a que respectivamente pertencemos; eu e a
senhora. Entre ambas fica a classe medíocre, balançando-se entre as
baixezas da minha e as alturas da sua. Não sai nunca dali; pois
falta-lhe coragem prática para participar da minha e não chega nunca a
alcançar a sua. O desprezo dói, mas nos ensina a observar as pessoas que
nos desprezam. Entre as mulheres dessa classe e as da minha, há em comum
uma alma apodrecida.

Apagando o cigarro, apertando-o de encontro ao cinzeiro, ela dispôs-se a
sair. Bárbara lembrou-se então da saleta do Esplanada onde ela tivera o
mesmo gesto, ao apagar um cigarro, mas em atitude bem diversa. Como tudo
é semelhante e como tudo é diferente!

--- Bem --- tornou a visitante --- já é tempo de continuar o caminho.

--- Lembro-lhe o meu oferecimento --- disse Bárbara animando-a.

--- Agradeço-lhe novamente. Conheço o meu lugar e sei onde colocar-me.
Vou tentar a vida, ver se começo tudo outra vez; mas quero ir para
longe. Longe do lugar das minhas recordações, dos meus fracassos. Dizem
que o Exército da Salvação acolhe as mulheres que desejem mudar de vida.
Contam por aí que é o único lugar onde se pode viver, ainda, como, um
ser humano que sente e pensa. Entre eles, pelo menos, não se tem a
impressão constante de uma sentenciada. Dizem que os oficiais dessa
esquisita agremiação não tratam as mulheres como leprosas repugnantes
que devem receber esmola à distância.

--- Eu os conheço --- disse Bárbara comovida --- e não vi, até hoje, uma
dedicação mais completa no espírito e no trabalho. Disseram-me certa vez
que essa agremiação nasceu na Inglaterra. Se é mesmo, não sei; mas, se
for, será este um dos motivos de orgulho para a minha pátria.

--- A senhora é inglesa?

--- De nascimento; mas por tudo que recebi do Brasil, tenho um coração
brasileiro.

---!!!

--- Eu não vim ao Brasil para inverter capitais, conquistando as suas
terras ou explorando a sua gente. São riquezas que deviam pertencer aos
brasileiros. Vim à sua pátria a passeio; fiquei porque gostei do Brasil.

Lola não escondeu a sua admiração ao ouvir tais coisas; por fim, tornou
ao assunto primitivo:

--- Então, a senhora acha que eu poderia procurar aquela gente? Onde a
senhora os conheceu?

--- Contribuo para muitas organizações de caridade; sempre me interessei
pelo sofrimento alheio. Um dia, eles vieram bater à minha porta; como eu
não tinha espírito preconcebido, recebi-os. Falaram-me do seu trabalho e
pediram o meu-auxílio. Desde aí, passei a ajudá-los. Algumas vezes,
aceitei seus convites para conhecer as suas organizações.

--- Mas eu tenho medo --- tornou Lola um pouco desapontada --- que uma
vez lá, me obriguem a assistir missa todos os dias.

Bárbara sorriu:

--- Não lhes conheço os rituais religiosos; apenas sei que se firmam em
princípios cristãos. E se a senhora não se der bem, poderá afastar-se
sem dificuldade.

Lola ia retirar-se; entretanto, alguma coisa a detinha como se lhe fosse
difícil sair sem ela. Bruscamente, padeceu resolver-se; voltando-se para
Bárbara, venceu o constrangimento:

--- Se o ama, não o deixe --- falou com voz alterada. --- Ele não teve
tanta culpa; foi um lapso que a senhora pode compreender e perdoar. Eu
fui a maior culpada. Portei-me como as mulheres que destroem o que há de
realmente honesto. Não pude, entretanto, nem de longe, destruir este
amor; era mais forte que eu.

Estendeu a mão a Bárbara, satisfeita pelo que dissera. Esta retribuiu o
gesto da rival vencida e acrescentou:

--- Que Deus a acompanhe.

--- Deus?! Talvez seja a primeira pessoa a dizer a uma mulher como eu:
que Deus a acompanhe.

--- Deus vive para todos --- ajuntou com uma firmeza afetuosa ---
poucos, porém, sabem disso.

--- Espero saber, ainda, se esta for a minha sorte.

Olhou tristemente ao longe e, pela última vez, dirigiu-se a Bárbara:

--- Eu não me humilharia diante de outra mulher; e, principalmente, da
espectadora da minha derrota. Se o fiz, é porque a senhora é humana,
antes de ser mulher.

***

Não obstante, Paulo e Helena precisassem partir, ainda permaneciam na
capital paulista; numa assistência contínua ao amigo que tanto
necessitava de cuidados. Álvaro passara ao hotel em que ambos se
hospedavam e, junto de Paulo, parecia inconsciente de si e de todos.
Sentado, por hábito, à escrivaninha, Álvaro olhava pela janela aberta.
Descortinava-se extenso panorama; era agradável o conjunto daquelas
diferentes formas de vida --- a vegetação, o ar, o homem. A vida lá fora
continuava; as árvores cresciam, mudavam folhas e davam frutos. As
estações sucediam-se. Todos os dias nasciam crianças. O ritmo não
mudava; só ele permanecia parado, como as bactérias em ampolas de
cientistas. O próprio ar que o cercava trazia partículas de vida em
suspensão; como permanecer inerte diante da atividade natural?

Olhou longo tempo para fora, na contemplação do horizonte infinito;
depois, detendo-se mais, pegou o jornal que trazia a fotografia de
Bárbara e colocou-o à sua frente. Lembrou-se então de uma cena do
passado --- aquela em que Bárbara dera a Paulo a imagem de Helena. Dera
porque Paulo a amava, e, já naquele dia, Álvaro também tinha certeza do
seu amor. Nisto, achou estranho que lhe faltasse uma fotografia de
Bárbara, como oferta pessoal de mulher amada. Ela não lhe ofereceu; ele
jamais pensara em apoderar-se clandestinamente de alguma. Rememorou a
cena com todos os pormenores; entregou-se conscientemente à morbidez de
recordar. Recordar um bem do passado, que se distancia pelo tempo e
pelas circunstâncias

Viu Paulo com a imagem de Helena nas mãos. Segurando com amargura, como
se ela lhe tivesse escapado para sempre. E hoje não estava ele, Álvaro,
nas mesmas condições? Naquele dia faltava Helena para o amigo; hoje,
Bárbara para ele. Aquele sorriso de Bárbara, o seu olhar, a sua
expressão de mulher que pensa; a sua atitude afetuosamente compreensiva!
Que era ele antes de conhecê-la E que chegou a ser a seu lado? Por ela,
tornou às suas atividades; compreendeu a sua participação na sociedade e
iniciou-se num caminho de conhecimento próprio, do qual julgou nunca
mais poder afastar-se. Mas, o que ele adquirira, não era matéria que se
guardasse em cofre seguro. Ao inverso das substâncias palpáveis, os bens
do espírito ou se cultivam ou morrem. Como tivera consciência daquele
amor! Como lutara por ele, desesperadamente; e, quando o conseguira,
como o fora perder de maneira tão inacreditável!

Lembrou-se, então, de quando soube do estado grave da outra mulher; do
desejo que teve de ver-se livre dela. Mais tarde, como lamentou os seus
antigos desejos. Viu-se ao lado de Paulo, já meio fora de si pelo
álcool, desejando matar aquela mulher que era um entrave na sua vida.
Lembrou-se das primeiras horas após a morte de Dalila; e como ele,
Álvaro, nos seus sentimentos, jamais desejaria realmente a morte de
alguém. A vida era coisa muito sagrada para que o homem atentasse contra
ela, mesmo no pensamento. A sua vida, como a de Dalila, a de Lola e a de
Bárbara, pertenciam a um ser superior; a quem caberia, hoje e sempre, a
direção deste mundo efêmero. Uma vez que neste mesmo mundo a passagem
dos homens é transitória.

Estranho! Como colocara Bárbara ao lado das outras duas mulheres, em se
pensando na morte! Ele mesmo não tivera plena consciência do seu
pensamento. Depois, lembrou-se de um texto de São Mateus, evangelista,
que Bárbara lhe citara uma vez --- ``não ajunteis tesouros na terra,
onde a traça e a ferrugem tudo consomem; mas ajuntai tesouros no céu.
Porque onde estiver o vosso tesouro, aí estará também o vosso coração''.

Teria Bárbara sabido de todos esses maus impulsos?

Se tivesse, não os teria compreendido?

Não o compreendera tantas vezes? Não conhecia os seus sentimentos, mais
que ele mesmo? Não saberia, então, que ele não chegaria a isso?

Sim, Bárbara o compreendia. Diante de Bárbara poderia ser o homem que
realmente era.

Quando, aos poucos, voltou à realidade, respirou demoradamente. Umas
mãos macias roçaram no seu rosto e pousaram nos seus cabelos em
desordem. Ele ficou silencioso, imóvel, temendo que a sensação
desaparecesse. Uma das mãos descansou no seu ombro, e pôde tocá-la. Como
reconhecia a maciez daquela pele! Bafejou-lhe o ouvido um hálito
próximo, e um rosto encostou-se ao seu.

--- Bárbara!!!

Segurou firme aquelas mãos; sentiu o perfume que o perturbara há pouco.
Reclinando-se para trás, encostou a cabeça ao seu seio e puxou-a mais
para si.

--- Será isto um sonho?

--- Não --- sussurrou baixinho --- eu estou bem acordada.

Através daquele contato momentâneo, Bárbara sentiu no Álvaro de hoje o
primeiro Álvaro que conhecera. Parecia estar no mesmo ponto onde o
encontrara pela primeira vez. Toda a conquista daquele ano, teria sido
um terreno perdido? --- Não. Não era possível; ele ainda estava a ponto
de continuar a caminhada em direção à eternidade. --- A eternidade, sim;
porque ao espírito não se impõem limites divisórios, numa separação
palpável entre o finito e o infinito.

--- Mas --- disse Bárbara interrompendo os seus pensamentos --- que
jornal é aquele?

Álvaro sorriu.

--- Foi por aí que eu a identifiquei --- respondeu.

--- Por isso me chamou pelo nome, naquela noite? Lembra-se? Foi na
viagem pelo mar.

--- Lembro. E agora, Bárbara, posso perguntar...

--- A que vim?

--- É\emph{.}

--- Porque nos demos mutuamente, Álvaro; e uma dádiva de amor não se
desfaz.

Comovido, perturbado no seu sentimento, apertou-a mais de encontro a si.
Deslumbrado ante a delicadeza daquele espírito feminino, tudo lhe
parecia pequeno diante dela.

--- Você, Bárbara, é o princípio, o meio e o fim de tudo o que existe
para mim. Você querida, é a minha filosofia...

FIM
